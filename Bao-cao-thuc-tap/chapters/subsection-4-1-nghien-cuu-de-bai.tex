\section*{1. Nghiên cứu đề bài, thu thập tài liệu}

Em đã tiến hành nghiên cứu toàn diện về bài toán giám sát biến động rừng, bao gồm cơ sở lý thuyết về rừng và biến động rừng, công nghệ viễn thám, học sâu, và các nghiên cứu liên quan. Các nội dung chính đã nghiên cứu được tóm tắt như sau:

\subsection*{1.1. Rừng và biến động rừng}

\textbf{Khái niệm và tầm quan trọng:} Theo Tổ chức Lương thực và Nông nghiệp Liên hợp quốc (FAO), rừng được định nghĩa là vùng đất có diện tích tối thiểu 0,5 ha, độ che phủ tán lá ít nhất 10\%, và chiều cao cây từ 5m trở lên, không phải là đất nông nghiệp hay đô thị. Rừng đóng vai trò then chốt trong việc duy trì cân bằng sinh thái, điều hòa khí hậu, lưu giữ carbon và bảo tồn đa dạng sinh học. Đặc biệt, rừng ngập mặn có khả năng lưu giữ carbon gấp 3--5 lần so với rừng nhiệt đới trên cạn.

\textbf{Tình hình mất rừng toàn cầu:} Trong giai đoạn 1990--2020, thế giới đã mất khoảng 420 triệu ha rừng với tốc độ trung bình 10 triệu ha mỗi năm (giai đoạn 2015--2020). Nguyên nhân chính bao gồm nông nghiệp quy mô lớn (chăn nuôi gia súc, cao su, cọ dầu), khai thác gỗ, và đô thị hóa. Các khu vực chịu ảnh hưởng nặng nề nhất là Mỹ La-tinh (26,9 triệu ha), trong đó riêng rừng Amazon mất 15,5 triệu ha, và Đông Nam Á với khu vực Borneo mất 5,8 triệu ha. Phá rừng đóng góp khoảng 23\% tổng lượng phát thải khí nhà kính toàn cầu.

\textbf{Tình hình tại Việt Nam:} Độ che phủ rừng Việt Nam đã tăng từ 27,2\% (1990) lên 42,01\% (2020), cho thấy xu hướng tích cực. Tuy nhiên, chất lượng rừng vẫn là vấn đề đáng lo ngại khi rừng nguyên sinh chỉ chiếm 0,25\% trong tổng diện tích 10,29 triệu ha rừng tự nhiên, phần lớn là rừng trồng (cao su, keo). Nguyên nhân mất rừng tại Việt Nam bao gồm chuyển đổi sang nông nghiệp, khai thác gỗ trái phép, đô thị hóa, cháy rừng, và đặc biệt là chuyển đổi sang nuôi trồng thủy sản ở vùng ven biển.

\subsection*{1.2. Công nghệ viễn thám và dữ liệu Sentinel}

\textbf{Nguyên lý viễn thám:} Viễn thám là kỹ thuật thu thập thông tin về bề mặt Trái Đất từ khoảng cách xa mà không tiếp xúc trực tiếp. Có hai loại chính: viễn thám bị động (sử dụng bức xạ Mặt Trời với cảm biến quang học) và viễn thám chủ động (phát xung điện từ và ghi nhận tín hiệu phản xạ như ra-đa SAR).

\textbf{Vệ tinh Sentinel-1:} Sử dụng ra-đa khẩu độ tổng hợp (SAR) hoạt động ở băng tần C (5,55 cm, 5,405 GHz). Chế độ IW (Interferometric Wide Swath) cung cấp 2 phân cực VV và VH với độ phân giải 10m, dải quét 250 km và chu kỳ 6--12 ngày. Ưu điểm nổi bật là khả năng xuyên qua mây, hoạt động cả ngày đêm, và nhạy cảm với cấu trúc vật thể cũng như độ ẩm. Phân cực VV nhạy với tán xạ bề mặt (độ ẩm đất), trong khi VH nhạy với tán xạ thể tích (cấu trúc tán lá).

\textbf{Vệ tinh Sentinel-2:} Cung cấp dữ liệu quang học đa phổ với 13 dải phổ (443--2190 nm) và độ phân giải từ 10m đến 60m, chu kỳ 5--10 ngày. Ưu điểm là cung cấp thông tin phổ phong phú cho việc tính toán các chỉ số thực vật, nhưng hạn chế bởi mây mù.

\textbf{Các chỉ số thực vật:} Em đã nghiên cứu ba chỉ số quan trọng. (1) \textbf{NDVI} (Normalized Difference Vegetation Index): $\text{NDVI} = \frac{\text{NIR} - \text{Red}}{\text{NIR} + \text{Red}}$, với giá trị cao (0,3--0,8) chỉ thị thực vật khỏe mạnh. (2) \textbf{NBR} (Normalized Burn Ratio): $\text{NBR} = \frac{\text{NIR} - \text{SWIR2}}{\text{NIR} + \text{SWIR2}}$, dùng để phát hiện cháy rừng và mất rừng. (3) \textbf{NDMI} (Normalized Difference Moisture Index): $\text{NDMI} = \frac{\text{NIR} - \text{SWIR1}}{\text{NIR} + \text{SWIR1}}$, đo độ ẩm tán lá và phát hiện stress rừng.

\textbf{Lợi ích tích hợp dữ liệu:} Kết hợp Sentinel-1 và Sentinel-2 có thể cải thiện độ chính xác phân loại từ 5--15\% so với sử dụng đơn nguồn, do cung cấp thông tin bổ sung về cấu trúc (SAR) và phổ (quang học).

\subsection*{1.3. Học sâu và mạng nơ-ron tích chập (CNN)}

\textbf{Kiến trúc CNN:} Mạng nơ-ron tích chập (CNN) là một dạng học sâu đặc biệt phù hợp cho xử lý ảnh. Các thành phần chính bao gồm: (1) \textbf{Lớp tích chập} thực hiện phép tích chập 2D để trích xuất đặc trưng cục bộ, có ưu điểm chia sẻ tham số và bất biến tịnh tiến; (2) \textbf{Hàm kích hoạt} ReLU (Rectified Linear Unit) với công thức $f(x) = \max(0, x)$ giúp giảm vấn đề vanishing gradient, và Softmax chuyển logits thành phân phối xác suất cho phân loại đa lớp; (3) \textbf{Pooling} với Max Pooling chọn giá trị lớn nhất và Global Average Pooling (GAP) tính trung bình toàn bộ feature map.

\textbf{Huấn luyện mạng:} Hàm mất mát Cross-Entropy được sử dụng: $L = -\sum y_i \cdot \log(\hat{y}_i)$. Thuật toán tối ưu AdamW kết hợp momentum và RMSprop với phân rã trọng số tách biệt. Các kỹ thuật điều chuẩn (regularization) bao gồm Batch Normalization (chuẩn hóa kích hoạt), Dropout (tắt ngẫu nhiên nơ-ron), và Dropout2d (tắt toàn bộ feature map, phù hợp cho CNN).

\textbf{Ứng dụng trong viễn thám:} CNN được áp dụng cho phân loại dựa trên patch, trong đó trích xuất patch nhỏ quanh mỗi pixel trung tâm và sử dụng CNN để phân loại. Dữ liệu cần được chuẩn hóa (Z-score) để đảm bảo thang đo đồng nhất giữa các nguồn dữ liệu khác nhau (NDVI [-1,1], quang học [0,1], SAR [-25,0] dB).

\subsection*{1.4. Các nghiên cứu liên quan}

\textbf{Phát triển phương pháp giám sát rừng:} Các phương pháp giám sát rừng đã phát triển từ khảo sát thực địa truyền thống (trước 1970), qua giải đoán trực quan ảnh viễn thám (1970--1990), đến học máy truyền thống như Random Forest và SVM (1990--2012), và hiện nay là học sâu với CNN và U-Net (từ 2012 đến nay).

\textbf{Nghiên cứu tiêu biểu:} Hansen et al. (2013) tạo bản đồ Global Forest Change sử dụng Decision Tree với dữ liệu Landsat, đạt độ chính xác khoảng 85\% ở quy mô toàn cầu. Kussul et al. (2017) áp dụng CNN 2D với Sentinel-2 tại Ukraine, đạt 94,5\% accuracy, vượt trội Random Forest (<92\%). Hu et al. (2020) kết hợp Sentinel-1 và Sentinel-2 tại Madagascar, cải thiện accuracy từ 87\% (chỉ S2) lên 92\% (S1+S2). Tại Việt Nam, Nguyen et al. (2020) sử dụng Random Forest/SVM đạt 91,2\% tại Đắk Nông; Vo et al. (2020) phân tích biến động rừng ngập mặn Cà Mau qua Landsat (2001--2019).

\textbf{Khoảng trống nghiên cứu:} Hiện còn thiếu các nghiên cứu áp dụng CNN cho rừng Việt Nam, đặc biệt là rừng ngập mặn Cà Mau. Bên cạnh đó, vấn đề bộ dữ liệu nhỏ (CNN thường cần hàng trăm nghìn mẫu nhưng viễn thám chỉ có 2.000--5.000 mẫu) và tối ưu hóa kiến trúc CNN cho dữ liệu kết hợp SAR và quang học vẫn còn là thách thức.
