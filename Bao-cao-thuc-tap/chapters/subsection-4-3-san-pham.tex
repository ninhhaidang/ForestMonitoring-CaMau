\section{Các sản phẩm hoàn thành}
Trong khuôn khổ đợt thực tập, dựa trên các nỗ lực nghiên cứu và thực nghiệm đã trình bày, ba sản phẩm chính đã được hoàn thành với chất lượng cao, đáp ứng trực tiếp nhu cầu giám sát rừng của đơn vị và địa phương.

Sản phẩm thứ nhất là \textbf{Bản đồ hiện trạng và biến động rừng tỉnh Cà Mau giai đoạn 2024-2025}. Đây là kết quả trực tiếp của việc áp dụng mô hình học sâu CNN trên toàn bộ dữ liệu ảnh vệ tinh Sentinel. Bản đồ có độ phân giải không gian 10m, cung cấp mức độ chi tiết cao hơn nhiều so với các sản phẩm giám sát rừng toàn cầu hiện có (thường là 30m). Mỗi điểm ảnh trên bản đồ được gán một trong bốn nhãn trạng thái: Rừng ổn định, Mất rừng, Phi rừng hoặc Phục hồi rừng. Định dạng dữ liệu đầu ra là GeoTIFF chuẩn hóa với hệ tọa độ VN-2000 hoặc UTM WGS84, cho phép tích hợp dễ dàng vào các hệ thống thông tin địa lý (GIS) chuyên dụng mà Chi cục Kiểm lâm đang sử dụng. Bản đồ này đóng vai trò như một cơ sở dữ liệu nền tảng, giúp các nhà quản lý định lượng chính xác diện tích rừng bị mất đi hay được phục hồi tại từng tiểu khu, từ đó hỗ trợ công tác lập quy hoạch và phân bổ nguồn lực bảo vệ rừng.

Sản phẩm thứ hai là \textbf{Ứng dụng Web GIS giám sát biến động rừng trực tuyến}. Nhận thấy việc truy cập và thao tác với dữ liệu bản đồ dạng tĩnh có thể gặp khó khăn đối với người dùng không chuyên, một ứng dụng web tương tác đã được xây dựng trên nền tảng Google Earth Engine Apps. Ứng dụng này cho phép người dùng truy cập mọi lúc mọi nơi thông qua trình duyệt web mà không cần cài đặt phần mềm phức tạp. Các chức năng chính bao gồm: xem bản đồ biến động rừng trên nền bản đồ vệ tinh hoặc bản đồ hành chính, tra cứu thông tin chi tiết tại bất kỳ vị trí nào bằng thao tác nhấp chuột, lọc hiển thị các lớp đối tượng quan tâm (ví dụ chỉ hiện lớp mất rừng), và tải xuống dữ liệu thống kê diện tích theo đơn vị hành chính. Ứng dụng đã được công bố công khai và hoạt động ổn định, là công cụ trực quan hóa hiệu quả để báo cáo và chia sẻ kết quả nghiên cứu.

Sản phẩm thứ ba là \textbf{Bộ mã nguồn và tài liệu kỹ thuật hoàn chỉnh}. Toàn bộ quy trình từ thu thập dữ liệu, tiền xử lý, xây dựng và huấn luyện mô hình đến dự đoán và tạo bản đồ đã được đóng gói thành các module mã nguồn Python/Jupyter Notebook rõ ràng, mạch lạc. Mã nguồn được tổ chức khoa học, có chú thích chi tiết, đảm bảo tính tái lập của nghiên cứu. Kèm theo đó là bộ dữ liệu huấn luyện đã được dán nhãn chuẩn hóa và tài liệu hướng dẫn sử dụng. Sản phẩm này có giá trị chuyển giao công nghệ cao, cho phép các kỹ sư tại đơn vị thực tập có thể tiếp nhận, vận hành và tiếp tục phát triển hệ thống sau khi đợt thực tập kết thúc.



\section{Các kiến nghị đề xuất}
Mặc dù hệ thống giám sát rừng đã đạt được những kết quả khả quan với độ chính xác ấn tượng, quá trình thực hiện vẫn bộc lộ một số hạn chế nhất định. Dựa trên những phân tích này, báo cáo đề xuất một số hướng phát triển tiếp theo nhằm hoàn thiện và nâng cao hiệu quả của hệ thống trong tương lai.

Thứ nhất, về mặt phương pháp luận, cần mở rộng nghiên cứu theo hướng phân tích chuỗi thời gian (time-series analysis). Hiện tại, mô hình mới chỉ so sánh dữ liệu tại hai thời điểm tĩnh (bi-temporal), do đó có thể bỏ sót các biến động ngắn hạn hoặc các thay đổi mang tính mùa vụ. Việc sử dụng các kiến trúc mạng nơ-ron hồi quy (RNN) như LSTM hoặc Transformer kết hợp với chuỗi ảnh vệ tinh liên tục sẽ cho phép mô hình nắm bắt được quy luật biến đổi theo thời gian của thảm thực vật, từ đó phân biệt tốt hơn giữa mất rừng thực sự và hiện tượng rụng lá theo mùa.

Thứ hai, để giải thích rõ hơn cơ chế hoạt động của mô hình "hộp đen", các kỹ thuật XAI (Explainable AI) nên được áp dụng. Việc trực quan hóa các vùng đặc trưng mà mạng CNN tập trung vào để đưa ra quyết định sẽ giúp các nhà khoa học hiểu rõ hơn về các yếu tố ảnh hưởng đến kết quả phân loại, đồng thời tăng cường niềm tin của người sử dụng vào hệ thống AI.

Thứ ba, về phạm vi ứng dụng, mô hình hiện tại được huấn luyện và tối ưu hóa cho điều kiện đặc thù của rừng ngập mặn Cà Mau. Để mở rộng khả năng ứng dụng sang các khu vực khác hoặc các loại hình rừng khác (như rừng khộp Tây Nguyên hay rừng nhiệt đới phía Bắc), cần thực hiện các nghiên cứu về học chuyển giao (Transfer Learning). Kỹ thuật này cho phép tận dụng tri thức đã học được từ mô hình hiện có để áp dụng cho bài toán mới với lượng dữ liệu huấn luyện ít hơn đáng kể.

Cuối cùng, việc tăng cường công tác khảo sát thực địa độc lập để kiểm chứng kết quả là vô cùng cần thiết. Cần thiết lập một mạng lưới các điểm giám sát cố định và thực hiện đo đạc định kỳ nhằm xây dựng bộ dữ liệu kiểm chứng (Ground Truth) có độ tin cậy cao hơn nữa. Sự kết hợp chặt chẽ giữa công nghệ viễn thám hiện đại và giám sát thực địa truyền thống sẽ tạo nên một hệ thống quản lý rừng toàn diện và bền vững.

