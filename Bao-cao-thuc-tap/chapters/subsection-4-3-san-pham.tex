\section{Các sản phẩm hoàn thành}
Trong khuôn khổ đợt thực tập, dựa trên các nỗ lực nghiên cứu và thực nghiệm đã trình bày, ba sản phẩm chính đã được hoàn thành với chất lượng cao, đáp ứng trực tiếp nhu cầu giám sát rừng của đơn vị và địa phương.

Sản phẩm thứ nhất là \textbf{Bản đồ phân loại biến động rừng tỉnh Cà Mau giai đoạn 2024--2025}. Đây là kết quả của việc áp dụng mô hình CNN đã huấn luyện trên diện rộng, bao phủ toàn bộ vùng quy hoạch lâm nghiệp tỉnh Cà Mau với diện tích 162.469 ha (chiếm 95,5\% diện tích ranh giới). Bản đồ có độ phân giải không gian 10m, chi tiết hơn nhiều so với các sản phẩm toàn cầu như Hansen (30m). Kết quả định lượng cho thấy bản đồ đã phát hiện 7.282 ha mất rừng (4,48\%) và 4.941 ha phục hồi rừng (3,04\%) trong giai đoạn nghiên cứu. Độ chính xác toàn cục của bản đồ đạt 98,86\%, khẳng định độ tin cậy cao của sản phẩm. Các tập tin bản đồ được xuất dưới định dạng GeoTIFF chuẩn hóa, tương thích hoàn toàn với hệ thống GIS của Chi cục Kiểm lâm.

Sản phẩm thứ hai là \textbf{Ứng dụng Web GIS giám sát biến động rừng trực tuyến}. Em đã xây dựng thành công ứng dụng trên nền tảng Google Earth Engine Apps, cho phép truy cập công khai tại địa chỉ: \url{https://ee-bonglantrungmuoi.projects.earthengine.app/view/giam-sat-bien-dong-rung-ca-mau}. Ứng dụng cung cấp các công cụ trực quan hóa mạnh mẽ, cho phép người dùng xem lớp phủ rừng trên nền ảnh vệ tinh mới nhất, tra cứu thông tin chi tiết tại từng vị trí, và theo dõi diễn biến rừng mà không cần cài đặt phần mềm chuyên dụng. Đây là công cụ hỗ trợ đắc lực cho công tác báo cáo và ra quyết định nhanh chóng.

Sản phẩm thứ ba là \textbf{Bộ quy trình công nghệ và mã nguồn mở}. Toàn bộ quy trình từ thu thập, tiền xử lý dữ liệu đa nguồn (Sentinel-1/2), trích xuất đặc trưng, đến huấn luyện mô hình và dự đoán đã được đóng gói thành các module mã nguồn Python/Jupyter Notebook khoa học. Bộ mã nguồn này đã được công bố trên GitHub (\url{https://github.com/ninhhaidang}), đảm bảo tính minh bạch và khả năng tái lập của nghiên cứu. Đi kèm với đó là bộ dữ liệu mẫu gồm 2.630 điểm thực địa đã dán nhãn, phục vụ cho việc nghiên cứu và phát triển tiếp theo. Quy trình này có tính chuyển giao cao, sẵn sàng để đơn vị thực tập tiếp nhận và vận hành.



\section{Các kiến nghị đề xuất}
Dựa trên kết quả thực nghiệm và những hạn chế đã phân tích, em đề xuất bốn hướng phát triển chính nhằm hoàn thiện và nâng cao hiệu quả của hệ thống trong tương lai:

Thứ nhất, về phương pháp luận, cần mở rộng nghiên cứu sang phân tích chuỗi thời gian (time-series analysis) thay vì chỉ so sánh hai thời điểm. Việc áp dụng các mô hình tiên tiến như Transformer hoặc LSTM trên chuỗi ảnh vệ tinh liên tục sẽ giúp khai thác hiệu quả các đặc trưng thời gian, từ đó nắm bắt tốt hơn quy luật diễn thế của rừng và phân biệt chính xác giữa mất rừng thực sự với các thay đổi theo mùa.

Thứ hai, để cải thiện hiệu suất mô hình, cần thử nghiệm cơ chế học chuyển giao (Transfer Learning) từ các mô hình đã được huấn luyện trên bộ dữ liệu lớn, đồng thời áp dụng các phương pháp kết hợp (Ensemble Learning). Việc kết hợp kết quả dự đoán từ nhiều mô hình khác nhau có thể giúp giảm thiểu phương sai, tăng độ ổn định và độ chính xác tổng thể của hệ thống.

Thứ ba, về phạm vi ứng dụng thực tiễn, kiến nghị mở rộng việc áp dụng mô hình sang các tỉnh khác trong vùng Đồng bằng sông Cửu Long có điều kiện sinh thái tương tự. Đồng thời, cần đẩy mạnh việc tích hợp kết quả phân loại vào các hệ thống thông tin địa lý (GIS) hiện có của cơ quan quản lý nhà nước, tạo nên quy trình giám sát liền mạch và tự động hóa.

Cuối cùng, cần tăng cường công tác khảo sát thực địa để kiểm chứng kết quả và mở rộng bộ dữ liệu mẫu (Ground Truth). Việc thu thập thêm dữ liệu đa thời gian và thiết lập mạng lưới quan trắc cố định sẽ giúp nâng cao độ tin cậy của mô hình, đồng thời tạo cơ sở dữ liệu huấn luyện phong phú hơn cho các nghiên cứu tiếp theo.
