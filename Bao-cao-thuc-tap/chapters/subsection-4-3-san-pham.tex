% \section*{3. Các sản phẩm khác}

Ngoài báo cáo nghiên cứu chi tiết, quá trình thực tập đã tạo ra các sản phẩm cụ thể sau:

\textbf{Thứ nhất, mô hình học sâu CNN.} Mô hình mạng nơ-ron tích chập (CNN) được xây dựng với 36,676 tham số, nhận đầu vào là patch 3×3 pixel với 27 kênh đặc trưng và xuất ra phân loại 4 lớp gồm Rừng ổn định, Mất rừng, Phi rừng, và Phục hồi rừng. Mô hình đạt độ chính xác 98.86\% trên tập kiểm tra, ROC-AUC score đạt 99.98\%, và kết quả cross-validation cho thấy độ chính xác 98.48\% ± 0.36\%.

\textbf{Thứ hai, bản đồ biến động rừng tỉnh Cà Mau.} Bản đồ phân tích biến động rừng được thực hiện trên toàn bộ ranh giới lâm nghiệp tỉnh Cà Mau mới sau quyết định sáp nhập tỉnh Cà Mau với tỉnh Bạc Liêu cũ theo Nghị quyết số 1278/NQ-UBTVQH15 ngày 24/10/2024, có hiệu lực từ 01/07/2025 (162,469 ha). Thời kỳ phân tích từ tháng 1/2024 đến tháng 2/2025 với độ phân giải không gian 10m. Kết quả phát hiện diện tích mất rừng là 7,282 ha (4.48\% tổng diện tích); diện tích phục hồi rừng là 4,941 ha (3.04\% tổng diện tích); và các khu vực rừng ổn định cùng khu vực không phải rừng. Dữ liệu được lưu trữ dưới format GeoTIFF với hệ tọa độ EPSG:32648 (UTM Zone 48N).

\textbf{Thứ ba, ứng dụng web hiển thị kết quả trên Google Earth Engine.} Ứng dụng web được triển khai tại URL \url{https://ee-bonglantrungmuoi.projects.earthengine.app/view/giam-sat-bien-dong-rung-ca-mau} với các tính năng bao gồm: (1) hiển thị bản đồ biến động rừng tương tác; (2) phân biệt màu sắc cho 4 lớp biến động; (3) cho phép người dùng phóng to/thu nhỏ, di chuyển bản đồ; (4) hiển thị thông tin metadata và chú giải. Ứng dụng được phát triển bằng Google Earth Engine Apps và JavaScript API, có thể truy cập công khai không yêu cầu đăng nhập.

\textbf{Thứ tư, mã nguồn và tài liệu.} Sản phẩm bao gồm mã nguồn tiền xử lý dữ liệu S1/S2 trên Google Earth Engine (JavaScript); mã nguồn huấn luyện mô hình CNN (Python, PyTorch); scripts phân loại và xuất kết quả; tài liệu hướng dẫn sử dụng; và báo cáo kỹ thuật chi tiết (tài liệu này).
