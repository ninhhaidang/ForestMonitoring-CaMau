\section{Thiết lập thực nghiệm}

\subsection{Khu vực nghiên cứu}

Theo Nghị quyết số 1278/NQ-UBTVQH15 ngày 24/10/2024 của Ủy ban Thường vụ Quốc hội, kể từ ngày 01/07/2025, tỉnh Cà Mau và tỉnh Bạc Liêu được sáp nhập thành tỉnh Cà Mau mới với tổng diện tích tự nhiên 7.942,38 km² và dân số khoảng 2,6 triệu người. Đồ án này nghiên cứu trên phạm vi rừng của tỉnh Cà Mau mới, bao gồm cả vùng rừng thuộc địa bàn Bạc Liêu cũ.

Tỉnh Cà Mau mới nằm ở cực Nam Tổ Quốc, sở hữu hệ sinh thái rừng đa dạng bao gồm cả rừng ngập mặn ven biển và rừng tràm nội địa. Theo số liệu trước khi sáp nhập, tỉnh Cà Mau cũ có diện tích rừng khoảng 94.319 ha và tỉnh Bạc Liêu có khoảng 5.730 ha rừng, tổng cộng khoảng 100.000 ha rừng trên toàn tỉnh Cà Mau mới \citevi{snnptntcamau2021}. Trong đó, rừng ngập mặn Cà Mau chiếm khoảng 20\% diện tích rừng ngập mặn của Việt Nam. Hệ thống rừng tại Cà Mau đóng vai trò then chốt trong việc phòng hộ ven biển (chắn sóng, chống xâm thực và bảo vệ bờ biển), bảo tồn đa dạng sinh học vì là môi trường sống cho nhiều loài động thực vật quý hiếm, cung cấp nguồn sinh kế thông qua các hoạt động thủy sản và du lịch sinh thái, và góp phần giảm nhẹ biến đổi khí hậu nhờ khả năng lưu giữ carbon cao, gấp khoảng 3–5 lần so với rừng nhiệt đới trên cạn \citeen{donato2011,alongi2014}.

Tuy nhiên, rừng Cà Mau đang phải đối mặt với nhiều thách thức. Trước hết là áp lực chuyển đổi sang nuôi tôm do kinh tế, khiến nhiều khu vực rừng bị chuyển đổi thành ao nuôi. Ngoài ra, hiện tượng xâm nhập mặn gia tăng do biến đổi khí hậu làm giảm sức khỏe rừng; đồng thời xói mòn bờ biển cũng làm suy giảm diện tích rừng ven biển; và tình trạng thiếu nước ngọt ảnh hưởng tới khả năng tái sinh tự nhiên của rừng. Giai đoạn 2011--2023, sạt lở vùng ven biển đã làm mất hơn 6.200 ha đất và rừng phòng hộ \citevi{nongnghiepmoitruong2024}.

Để hiểu rõ bối cảnh không gian và đa dạng lớp phủ bề mặt tại khu vực nghiên cứu, Hình~\ref{fig:camau_lulc} trình bày bản đồ phân loại lớp phủ/sử dụng đất (LULC) tỉnh Cà Mau. Phân loại LULC cung cấp thông tin nền tảng về các loại hình sử dụng đất khác nhau trên địa bàn, từ đó giúp xác định ranh giới giữa vùng rừng và phi rừng, cũng như các vùng có nguy cơ chuyển đổi cao \citeen{gong2013}. Bản đồ này được xây dựng từ dữ liệu viễn thám từ Esri phản ánh hiện trạng sử dụng đất phức tạp tại tỉnh Cà Mau, nơi mà sự tương tác giữa các hệ sinh thái tự nhiên (rừng ngập mặn, đất ngập nước) và các hoạt động kinh tế-xã hội (nuôi trồng thủy sản, nông nghiệp) diễn ra liên tục \citeen{renaud2015}.

\begin{figure}[H]
    \centering
    \includegraphics[width=0.95\textwidth]{img/chapter3/LULC-Ca-Mau.png}
    \caption{Bản đồ lớp phủ bề mặt khu vực tỉnh Cà Mau}
    \label{fig:camau_lulc}
\end{figure}

Phân tích bản đồ cho thấy sự phân bố không đồng nhất của các loại hình lớp phủ, với rừng ngập mặn tập trung chủ yếu dọc bờ biển phía Tây và Nam, trong khi các vùng nuôi trồng thủy sản và nông nghiệp lúa nước chiếm ưu thế ở khu vực trung tâm và phía Đông. Mô hình phân bố này phản ánh lịch sử khai thác và quản lý tài nguyên tại Cà Mau, đồng thời làm nổi bật sự cần thiết phải có công cụ giám sát biến động rừng hiệu quả để hỗ trợ các quyết định quản lý bền vững \citeen{kuenzer2011}.

Đồ án tập trung vào toàn bộ vùng quy hoạch lâm nghiệp của tỉnh Cà Mau mới. Dữ liệu ranh giới quy hoạch lâm nghiệp được cung cấp bởi Công ty TNHH Tư vấn và Phát triển Đồng Xanh — đối tác của Chi cục Kiểm lâm tỉnh Cà Mau.

Tổng diện tích ranh giới quy hoạch là 170.178,82 ha (tương đương 1.701,79 km²), bao gồm 666 polygon trong file shapefile ranh giới. Diện tích thực tế được phân loại là 162.468,50 ha (khoảng 95,5\% diện tích ranh giới); phần còn lại (~7.710 ha, chiếm 4,5\%) bị loại do mây che phủ hoặc dữ liệu không hợp lệ (nodata) trong quá trình xử lý ảnh vệ tinh. Kích thước raster là 12.547 × 10.917 điểm ảnh (ở độ phân giải 10m), sử dụng hệ quy chiếu EPSG:32648 (WGS 84 / UTM Zone 48N).

\subsection{Phân chia dữ liệu}

Bộ dữ liệu thực địa gồm 2.630 điểm, trong đó phân bố lớp gần như cân bằng: Lớp 0 (Rừng ổn định) 656 điểm (24,94\%), Lớp 1 (Mất rừng) 650 điểm (24,71\%), Lớp 2 (Phi rừng ổn định) 664 điểm (25,25\%) và Lớp 3 (Phục hồi rừng) 660 điểm (25,10\%).

\begin{table}[H]
\centering
\caption{Phân bố dữ liệu theo tập huấn luyện và kiểm tra}
\label{tab:data_split_info}
\begin{tabular}{|l|c|c|}
\hline
\textbf{Tập dữ liệu} & \textbf{Số mẫu} & \textbf{Tỷ lệ} \\
\hline
Huấn luyện + Kiểm định (5-Fold CV) & 2.104 & 80\% \\
\hline
Kiểm tra (cố định) & 526 & 20\% \\
\hline
\textbf{Tổng} & \textbf{2.630} & \textbf{100\%} \\
\hline
\end{tabular}
\end{table}

Việc chia tập dữ liệu được thực hiện như sau: 80\% dữ liệu (2.104 patches) được dành cho Train+Val để thực hiện 5-Fold Cross Validation, còn 20\% dữ liệu (526 patches) được giữ lại làm fixed test set.

Kết quả trên hoàn thành mục tiêu thứ nhất, xây dựng bộ dữ liệu huấn luyện gồm 2.630 điểm mẫu với 27 đặc trưng từ ảnh vệ tinh Sentinel-1 và Sentinel-2 đa thời gian, bao gồm các kênh phổ, chỉ số thực vật và tán xạ ngược cho khu vực quy hoạch lâm nghiệp tỉnh Cà Mau.
