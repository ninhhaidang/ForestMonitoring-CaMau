

\section{Giới thiệu chung}
Nằm ở cực Nam của Tổ quốc, tỉnh Cà Mau sở hữu hệ sinh thái rừng ngập mặn đặc thù và vô cùng quan trọng đối với sự cân bằng môi trường cũng như đời sống kinh tế xã hội của vùng Đồng bằng sông Cửu Long. Tuy nhiên, trước áp lực của biến đổi khí hậu, nước biển dâng và các hoạt động kinh tế như nuôi trồng thủy sản, diện tích rừng tại đây đang đứng trước nguy cơ suy giảm nghiêm trọng. Việc giám sát biến động tài nguyên rừng theo phương pháp truyền thống thường tốn kém nhiều thời gian, công sức và khó triển khai trên diện rộng. Trong bối cảnh đó, sự phát triển mạnh mẽ của công nghệ viễn thám và trí tuệ nhân tạo, đặc biệt là các mô hình học sâu, đã mở ra hướng đi mới đầy tiềm năng cho công tác quản lý và bảo vệ rừng. Xuất phát từ thực tiễn cấp thiết này, báo cáo thực tập tập trung vào đề tài ``Giám sát biến động rừng tỉnh Cà Mau sử dụng dữ liệu viễn thám và mạng nơ-ron nhân tạo'', với mục tiêu ứng dụng công nghệ hiện đại để giải quyết bài toán môi trường tại địa phương.

Quá trình thực tập được thực hiện tại Công ty TNHH Tư vấn và Phát triển Đồng Xanh dưới sự hướng dẫn của TS. Hoàng Việt Anh. Đây là đơn vị có nhiều kinh nghiệm trong lĩnh vực tư vấn giải pháp công nghệ quản lý tài nguyên và là đối tác chiến lược của Chi cục Kiểm lâm tỉnh Cà Mau. Trong thời gian thực tập từ ngày 15/9/2025 đến 15/12/2025, em đã trực tiếp tham gia vào dự án xây dựng hệ thống giám sát rừng tự động, từ khâu thu thập dữ liệu vệ tinh đa nguồn, xử lý ảnh, đến thiết kế và huấn luyện các mô hình học sâu tiên tiến.

Công ty TNHH Tư vấn và Phát triển Đồng Xanh (GFD) là doanh nghiệp tiên phong trong lĩnh vực công nghệ và kỹ thuật, chuyên sâu về các giải pháp GIS (Hệ thống Thông tin Địa lý), Viễn thám và Tư vấn quản lý tài nguyên thiên nhiên. Với hơn 15 năm hình thành và phát triển, GFD đã khẳng định vị thế vững chắc thông qua việc triển khai thành công hơn 100 dự án GIS và 300 dự án tư vấn cho các khách hàng lớn trong nước (Bộ Nông nghiệp và Phát triển Nông thôn, Bộ Tài nguyên và Môi trường) và các tổ chức quốc tế (World Bank, ADB, JICA, USAID). Quy trình sản xuất phần mềm và quản lý an toàn thông tin của công ty tuân thủ nghiêm ngặt các tiêu chuẩn quốc tế ISO 9001:2015 và ISO 27001:2022.

Lĩnh vực hoạt động cốt lõi của GFD bao gồm phát triển các giải pháp phần mềm GIS và Viễn thám toàn diện, cung cấp chuỗi dịch vụ khép kín từ khảo sát thực địa, số hóa dữ liệu đến thiết kế phần mềm trên đa nền tảng dựa trên các công nghệ hiện đại như Open Source GIS, điện toán đám mây và các giải pháp thương mại từ ESRI, Bentley; đồng thời, công ty còn cung cấp các dịch vụ tư vấn chuyên sâu về quản lý tài nguyên thiên nhiên như quản lý rừng bền vững, bảo tồn đa dạng sinh học, đo đạc carbon lâm nghiệp và đánh giá thiệt hại thiên tai, luôn đảm bảo tuân thủ hệ thống pháp luật hiện hành và đáp ứng các yêu cầu khắt khe của các tổ chức quốc tế.

\section{Khái quát nội dung thực tập}
Nội dung thực tập tập trung vào việc nghiên cứu và giải quyết bài toán phát hiện mất rừng và phục hồi rừng thông qua phân tích ảnh vệ tinh. Cụ thể, em đã tiến hành nghiên cứu cơ sở lý thuyết vềviễn thám, đặc điểm phổ của các đối tượng rừng ngập mặn, cũng như các kiến trúc mạng nơ-ron tích chập (CNN). Dựa trên nền tảng lý thuyết đó, quá trình thực nghiệm được triển khai với việc thu thập dữ liệu ảnh vệ tinh Sentinel-1 và Sentinel-2 trên nền tảng Google Earth Engine. Việc kết hợp dữ liệu quang học và dữ liệu radar được xem là chìa khóa để khắc phục hạn chế về mây che phủ thường xuyên tại khu vực nhiệt đới, đồng thời cung cấp thêm thông tin về cấu trúc bề mặt và độ ẩm của thảm thực vật.

Sau khi xây dựng được bộ dữ liệu huấn luyện chất lượng cao với các nhãn được kiểm chứng, em đã thiết kế và tối ưu hóa một kiến trúc mạng CNN chuyên biệt, phù hợp với quy mô dữ liệu và đặc thù bài toán. Mô hình được huấn luyện để phân loại bốn trạng thái biến động chính gồm rừng ổn định, mất rừng, phi rừng và phục hồi rừng. Quá trình đánh giá mô hình được thực hiện nghiêm ngặt thông qua kỹ thuật kiểm định chéo (cross-validation) để đảm bảo độ tin cậy và khả năng tổng quát hóa của kết quả. Cuối cùng, mô hình tối ưu đã được áp dụng để phân loại toàn bộ khu vực quy hoạch lâm nghiệp của tỉnh Cà Mau, tạo ra bản đồ biến động rừng chi tiết và trực quan.

\section{Kết quả đạt được}
Kết quả thực tập đã đạt được những thành tựu đáng kể cả về mặt khoa học và ứng dụng thực tiễn. Nghiên cứu đã chứng minh hiệu quả vượt trội của việc tích hợp dữ liệu đa nguồn Sentinel-1 và Sentinel-2 so với việc chỉ sử dụng đơn lẻ dữ liệu quang học. Mô hình đề xuất đạt độ chính xác ấn tượng lên tới 98,86\% trên tập dữ liệu kiểm tra độc lập, với chỉ số ROC-AUC đạt 99,98\%. Điều này khẳng định khả năng phân biệt chính xác các đối tượng rừng và biến động rừng của thuật toán đã phát triển.

Về mặt sản phẩm, báo cáo đã xây dựng thành công bản đồ phân loại biến động rừng tỉnh Cà Mau giai đoạn 2024 -- 2025 với độ phân giải 10m. Kết quả phân tích không gian cho thấy bức tranh tổng thể về tài nguyên rừng của tỉnh, trong đó phát hiện khoảng 7.282 ha rừng bị mất và 4.941 ha rừng được phục hồi. Bên cạnh đó, em cũng đã xây dựng một ứng dụng web trực tuyến trên nền tảng Google Earth Engine, cho phép người dùng truy cập, tương tác hiển thị dữ liệu và kết quả phân loại biến động rừng một cách dễ dàng và thuận tiện.

\section{Kinh nghiệm rút ra}
Quá trình thực tập không chỉ mang lại kiến thức chuyên môn mà còn giúp em tích lũy nhiều kinh nghiệm quý báu. Bài học lớn nhất là tầm quan trọng của việc xử lý và làm sạch dữ liệu. Trong các bài toán học sâu, chất lượng dữ liệu đầu vào quyết định phần lớn hiệu suất của mô hình. Việc hiểu rõ đặc điểm vật lý của ảnh viễn thám và thực hiện các bước tiền xử lý kỹ lưỡng là yếu tố then chốt để đạt được kết quả cao. Ngoài ra, kỹ năng tối ưu hóa mã nguồn và quản lý tài nguyên tính toán cũng được rèn luyện khi phải làm việc với lượng dữ liệu lớn trên quy mô toàn tỉnh. Cuối cùng, khả năng phân tích và biện giải kết quả từ "hộp đen" của mô hình học sâu giúp em hiểu sâu sắc hơn về cơ chế hoạt động của thuật toán và đưa ra những cải tiến phù hợp.

\section{Kết luận}
Kỳ thực tập đã hoàn thành tốt các mục tiêu đề ra, góp phần đưa ra một giải pháp công nghệ hiệu quả cho công tác giám sát rừng tại Cà Mau. Phương pháp tiếp cận kết hợp viễn thám đa nguồn và học sâu đã chứng minh được tính ưu việt và khả năng ứng dụng thực tiễn. Những kết quả đạt được không chỉ có giá trị học thuật mà còn là công cụ hỗ trợ đắc lực cho các nhà quản lý trong việc ra quyết định bảo vệ và phát triển rừng bền vững. Mặc dù vẫn còn một số hạn chế nhất định cần tiếp tục nghiên cứu, nhưng thành công bước đầu này là tiền đề vững chắc cho các hướng phát triển tiếp theo, mở rộng phạm vi ứng dụng sang các khu vực khác và nâng cao hơn nữa độ chính xác của mô hình.
\newpage
