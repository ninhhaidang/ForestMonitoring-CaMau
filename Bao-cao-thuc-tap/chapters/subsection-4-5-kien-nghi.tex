\section*{5. Kiến nghị}

Dựa trên kết quả đạt được và những khó khăn gặp phải, em xin đề xuất một số kiến nghị cho các nghiên cứu tiếp theo:

\textbf{Thứ nhất, về mở rộng nghiên cứu theo chuỗi thời gian.} Đề xuất (1) mở rộng thành chuỗi thời gian dài hạn với tần suất giám sát cao hơn (hàng tháng hoặc hàng quý); (2) phân tích xu hướng biến động rừng theo mùa và theo năm; (3) xây dựng hệ thống cảnh báo sớm về mất rừng dựa trên phân tích chuỗi thời gian; (4) áp dụng các mô hình time-series như LSTM, Transformer cho dự báo biến động rừng.

\textbf{Thứ hai, về tăng cường kiểm chứng kết quả phân loại.} Cần (1) tổ chức khảo sát thực địa để kiểm chứng độ chính xác của bản đồ biến động rừng do mô hình tạo ra; (2) đo đạc các thông số sinh thái tại các khu vực được phân loại để xác nhận kết quả; (3) so sánh kết quả phân loại trên diện rộng với số liệu thực tế để đánh giá độ tin cậy; (4) xây dựng quy trình kiểm chứng độc lập cho các sản phẩm giám sát rừng.

\textbf{Thứ ba, về cải thiện hiệu năng mô hình.} Đề xuất (1) tối ưu hóa kiến trúc mô hình để giảm thời gian dự đoán; (2) nghiên cứu các kỹ thuật model compression (pruning, quantization) để triển khai nhanh hơn; (3) sử dụng GPU/TPU mạnh hơn hoặc phân tán tính toán; (4) tăng kích thước bộ dữ liệu huấn luyện để cải thiện độ chính xác.

\textbf{Thứ tư, về mở rộng vùng nghiên cứu.} Nên (1) áp dụng mô hình cho các tỉnh ven biển khác có rừng ngập mặn (Kiên Giang, Sóc Trăng); (2) so sánh đặc điểm biến động rừng giữa các vùng khác nhau; (3) xây dựng bản đồ biến động rừng quy mô quốc gia; (4) tích hợp với các hệ thống giám sát rừng hiện có của Bộ Nông nghiệp và Phát triển Nông thôn.

\textbf{Thứ năm, về phát triển công cụ và ứng dụng.} Kiến nghị (1) phát triển ứng dụng mobile cho công tác giám sát rừng tại hiện trường; (2) xây dựng dashboard tương tác cho các nhà quản lý; (3) tích hợp tính năng báo cáo tự động và xuất số liệu thống kê; (4) kết nối với các hệ thống cảnh báo cháy rừng, thiên tai.
