\documentclass[a4paper,13pt]{report}

% --------------------------------------------------------------------
% Compile instructions:
% - To compile with XeLaTeX (recommended for Times New Roman and Unicode):
%     latexmk -pdfxe internship-report.tex
%   or
%     xelatex internship-report.tex
%
% - To compile with LuaLaTeX, use:
%     latexmk -pdflua internship-report.tex
%
% - To compile with pdfLaTeX (default behaviour):
%     latexmk -pdf internship-report.tex
%   (This uses Latin Modern and T5 encoding for Vietnamese.)
% --------------------------------------------------------------------

% Encoding and Vietnamese support
% This file supports both pdfLaTeX and XeLaTeX/LuaLaTeX. The default
% behaviour below keeps the existing pdfLaTeX setup (Latin Modern + T5
% encoding for Vietnamese). If you switch to XeLaTeX/LuaLaTeX, the
% document will use system fonts via `fontspec` (set to Times New Roman).
\usepackage{iftex}
\ifPDFTeX
	% pdfLaTeX: keep older encoding packages and Latin Modern.
	\usepackage[utf8]{inputenc} % keep for pdfLaTeX; not used with XeLaTeX
	\usepackage[T5]{fontenc}    % T5 encoding (Vietnamese) - required for vntex/vietnam
	\usepackage[utf8]{vietnam}  % Vietnamese support (vntex)
	% If you want a Times-like font with pdfLaTeX you can enable the
	% following packages, but note: T5 encoding support and some
	% Vietnamese glyphs may not be complete with these fonts. Test the
	% output before enabling.
	% \usepackage{newtxtext,newtxmath} % Times-like fonts (for pdfLaTeX)
\else
	% XeLaTeX/LuaLaTeX: use fontspec and polyglossia for Unicode & Vietnamese
	% NOTE: If your source uses TeX accent macros (e.g., `\h{a}` or `\'a`)
	% instead of raw Unicode characters, you may need to either (A) keep
	% the `vietnam` package loaded or (B) convert source files to Unicode
	% characters for Vietnamese. Loading both `polyglossia` and `vietnam`
	% may cause conflicts; test carefully.
	\usepackage{fontspec}
	\usepackage{polyglossia}
	% If your source contains TeX accent macros rather than raw Unicode
	% characters (e.g. `\h{a}`), uncomment the following line to load
	% the `vietnam` package which defines such macros. Keep in mind
	% that `vietnam` is designed for T5 encoding and may sometimes
	% overlap with `polyglossia` macros.
	\usepackage[utf8]{vietnam}
	\setmainlanguage{vietnamese}
	% Set Times New Roman as the main font. You can change the name if
	% your OS has a differently named Times family (e.g., "Times").
	\setmainfont{Times New Roman}[
		Ligatures=TeX,
		Script=Latin
	]
	% If you want math fonts to match Times, you can add e.g.:
	% \usepackage{unicode-math}
	% \setmathfont{TeX Gyre Termes Math}
	% Optional: specify sans/mono fonts to better match system defaults
	\setsansfont{Arial}
	\setmonofont{Courier New}
\fi
\usepackage{amsmath, amsfonts, amssymb}
\usepackage{textcomp}

% Note: If you prefer to compile with XeLaTeX or LuaLaTeX (recommended for Unicode):
%  - remove or comment out the `inputenc` and `fontenc` lines above
%  - uncomment the fontspec lines below and set an appropriate system font
%  - compile with: `latexmk -pdfxe internship-report.tex` or `latexmk -pdflua internship-report.tex`
%
% %%% XeLaTeX / LuaTeX example (uncomment when using XeLaTeX/LuaTeX):
% \usepackage{fontspec}
% \setmainfont{Times New Roman}
% \usepackage{polyglossia}
% \setmainlanguage{vietnamese}

% set font, font size and spacing
% If compiling with pdfLaTeX, keep Latin Modern for T5 coverage. For
% XeLaTeX/LuaLaTeX the fontspec package (above) will set Times.
\ifPDFTeX
	\usepackage{lmodern}
\fi
\usepackage{microtype} % micro-typography
\usepackage{scrextend}
\changefontsizes{13pt}
\renewcommand{\baselinestretch}{1.3}

% Hình ảnh và đồ họa
\usepackage{graphicx}
\usepackage{tikz}
\usetikzlibrary{shapes.geometric, arrows, positioning, arrows.meta, calc, fit}
% PGFPlots: required for axis environment and plotting
\usepackage{pgfplots}
\pgfplotsset{compat=1.18}
\usepgfplotslibrary{fillbetween}
% Pie chart support
\usepackage{pgf-pie}
% Disable thousand separators by default for axis tick labels (e.g., 1990 shows as 1990, not 1,990)
\pgfplotsset{every axis/.append style={xticklabel style={/pgf/number format/1000 sep={}}}}

% Định dạng bảng và caption
\usepackage{array}
\usepackage{booktabs}
% Combine caption options to avoid duplicate package load and enable sub-captions
\usepackage[justification=centering,font=small]{caption}
% Sub-figures support (provides the `subfigure` environment used in chapters)
\usepackage{subcaption}
\usepackage{makecell}
\usepackage{multirow}

% Điều chỉnh lề và layout trang
\usepackage[left=3cm,right=2cm,top=2.5cm,bottom=3cm]{geometry}

% Hyperlinks và danh mục
\usepackage[unicode, breaklinks=true, hidelinks]{hyperref}
\usepackage{xurl} % Cho phép URL tự động xuống dòng

% Tùy chỉnh các phần khác
\usepackage{scrextend, enumerate, float, afterpage, sectsty, tocloft, calc, listings}

% Thêm dấu chấm (...) cho chapter trong mục lục
\renewcommand{\cftchapdotsep}{\cftdotsep}

% Định dạng tiêu đề chương, mục (phải đặt SAU sectsty và tocloft)
\usepackage{titlesec}
\titleformat{\chapter}[block]{\centering\normalfont\fontsize{13pt}{16pt}\selectfont\bfseries}{CHƯƠNG \thechapter.}{0.5em}{\MakeUppercase}
\titleformat{name=\chapter,numberless}[block]{\centering\normalfont\fontsize{13pt}{16pt}\selectfont\bfseries}{}{0pt}{\MakeUppercase}
\titlespacing*{\chapter}{0pt}{-20pt}{10pt}
\titlespacing{\section}{0pt}{12pt}{6pt}
\titlespacing{\subsection}{0pt}{12pt}{6pt}

% Định dạng chung cho các tiêu đề không đánh số: chữ hoa, 13pt, đậm, căn giữa
\newcommand{\unnumberedchapter}[1]{%
  \vspace*{-20pt}%
  \begin{center}
    {\fontsize{13pt}{16pt}\selectfont\textbf{\MakeUppercase{#1}}}
  \end{center}
  \vspace{10pt}%
}


\usepackage{pdflscape} % Gói giúp xoay trang ngang trong PDF
\usepackage{graphicx}
\usepackage{float} % Để cố định vị trí bảng
\graphicspath{{img/}{img/chapter1/}{img/chapter2/}{img/chapter3/}}
\usepackage{longtable} % Nếu bảng dài
\usepackage{lscape} % Để hỗ trợ xoay trang (tuỳ chọn)

% Gói hỗ trợ thuật toán
\usepackage[linesnumbered,ruled,vlined]{algorithm2e}
\usepackage{optidef}

% Điều chỉnh các thuật toán
\SetAlgorithmName{Thuật toán}{Thuật toán}{Danh sách các Thuật toán}
\SetKwInput{KwIn}{Đầu vào}
\SetKwInput{KwOut}{Đầu ra}

% Định dạng danh mục tài liệu tham khảo
\usepackage[numbers,sort&compress]{natbib}
\usepackage{multibib}
\newcites{vi,en}{{},{}}

% Loại bỏ chapter header của multibib
\makeatletter
\renewcommand{\bibsection}{}
\makeatother

% Định dạng lại các tiêu đề section và subsection (dùng sectsty)
% Không dùng \chapterfont vì titlesec đã định dạng chapter
\sectionfont{\fontsize{13pt}{16pt}\selectfont\bfseries}
\subsectionfont{\fontsize{13pt}{16pt}\selectfont\bfseries}

% Thêm dấu chấm sau số section và subsection (1.1. thay vì 1.1)
\renewcommand{\thesection}{\thechapter.\arabic{section}.}
\renewcommand{\thesubsection}{\thechapter.\arabic{section}.\arabic{subsection}.}

% Điều chỉnh khoảng cách giữa các đoạn
\usepackage{indentfirst}
\setlength{\parskip}{6pt}
\setlength{\parindent}{1cm}

% Tùy chỉnh chú thích
\usepackage{perpage} 
\MakePerPage{footnote}

% Khoảng cách dòng
\renewcommand{\baselinestretch}{1.3}

% Tùy chỉnh mục lục
\renewcommand{\cftchappresnum}{Chương }
\AtBeginDocument{\addtolength\cftchapnumwidth{\widthof{\bfseries Chương }}}

% Thêm tiền tố "Hình" vào danh mục hình ảnh
\renewcommand{\cftfigpresnum}{Hình }
\setlength{\cftfignumwidth}{4.5em}

% Thêm tiền tố "Bảng" vào danh mục bảng
\renewcommand{\cfttabpresnum}{Bảng }
\setlength{\cfttabnumwidth}{4.5em}

% Thiết lập tiêu đề và tác giả
\title{Giám sát biến động rừng tỉnh Cà Mau sử dụng dữ liệu viễn thám và mạng nơ ron nhân tạo}
\author{Ninh Hải Đăng} 


\newcommand{\argmax}{\arg\!\max}

\begin{document}
% Bìa
% Cover Vietnamese 1

% Cover english
\pagenumbering{gobble}
\begin{center}
	\begin{tikzpicture}[overlay,remember picture]
	    \draw [line width=3pt,rounded corners=0pt,
	        ]
	        ($ (current page.north west) + (25mm,-25mm) $)
	        rectangle
	        ($ (current page.south east) + (-15mm,25mm) $);
	    \draw [line width=1pt,rounded corners=0pt]
	        ($ (current page.north west) + (26.5mm,-26.5mm) $)
	        rectangle
	        ($ (current page.south east) + (-16.5mm,26.5mm) $);
	\end{tikzpicture}
	\\[1mm]
	%%12pt
	\textbf{ĐẠI HỌC QUỐC GIA HÀ NỘI\\TRƯỜNG ĐẠI HỌC CÔNG NGHỆ}\\[1cm]
	\includegraphics[width=0.2\linewidth]{img/uet}\\[0.3cm]
	{\fontsize{14pt}{16pt}\selectfont\textbf{Ninh Hải Đăng}}
	\\[2cm]

	{\fontsize{18pt}{22pt}\selectfont\textbf{ỨNG DỤNG VIỄN THÁM VÀ HỌC SÂU TRONG \\ GIÁM SÁT BIẾN ĐỘNG RỪNG TỈNH CÀ MAU}}
	\\[2.6cm]
	{\fontsize{14pt}{16pt}\selectfont\textbf{ĐỒ ÁN TỐT NGHIỆP ĐẠI HỌC HỆ CHÍNH QUY
			\\[2mm]
			Ngành: Công nghệ Hàng không Vũ trụ}}

	\vfill
	{\fontsize{12pt}{14pt}\selectfont\textbf{HÀ NỘI - 2025}}
	\vspace{10mm}
\end{center}

\begin{center}
	\begin{tikzpicture}[overlay,remember picture]
	\draw [line width=3pt,rounded corners=0pt]
	($ (current page.north west) + (25mm,-25mm) $)
	rectangle
	($ (current page.south east) + (-15mm,25mm) $);
	\draw [line width=1pt,rounded corners=0pt]
	($ (current page.north west) + (26.5mm,-26.5mm) $)
	rectangle
	($ (current page.south east) + (-16.5mm,26.5mm) $);
	\end{tikzpicture}
	\\[1mm]
	\textbf{ĐẠI HỌC QUỐC GIA HÀ NỘI\\TRƯỜNG ĐẠI HỌC CÔNG NGHỆ}
	\\[1cm]
	\includegraphics[width=0.2\linewidth]{img/uet}
	\\[0.3cm]
	{\fontsize{14pt}{16pt}\selectfont\textbf{Ninh Hải Đăng}}
	\\[2cm]

	{\fontsize{18pt}{22pt}\selectfont\textbf{ỨNG DỤNG VIỄN THÁM VÀ HỌC SÂU TRONG  GIÁM SÁT BIẾN ĐỘNG RỪNG TỈNH CÀ MAU}}
	\\[1.5cm]
{\fontsize{14pt}{16pt}\selectfont\textbf{ĐỒ ÁN TỐT NGHIỆP ĐẠI HỌC HỆ CHÍNH QUY
		\\[2mm]
	Ngành: Công nghệ Hàng không Vũ trụ}}
\end{center}
\vspace{1mm}
\hspace*{12mm}\textbf{Cán bộ hướng dẫn: TS. Hà Minh Cường}
\\[2.5cm]
\hspace*{51mm}\textbf{ThS. Hoàng Tích Phúc}
\vfill
\begin{center}
		{\fontsize{12pt}{14pt}\selectfont\textbf{HÀ NỘI - 2025}}
	\vspace{4mm}
\end{center}
\newpage\cleardoublepage

% Chuyển sang đánh số trang kiểu La Mã
\pagenumbering{roman}

% Mục lục
\phantomsection
\addcontentsline{toc}{chapter}{MỤC LỤC}
\unnumberedchapter{MỤC LỤC}
\makeatletter\@starttoc{toc}\makeatother
\newpage\cleardoublepage

% Lời cảm ơn
\phantomsection
\addcontentsline{toc}{chapter}{LỜI CẢM ƠN}
\unnumberedchapter{LỜI CẢM ƠN}

Trước tiên, em xin chân thành cảm ơn Công ty TNHH Tư vấn và Phát triển Đồng Xanh đã tạo điều kiện cho em thực hiện đề tài thực tập tốt nghiệp này. Môi trường làm việc chuyên nghiệp và sự hỗ trợ nhiệt tình từ công ty đã giúp em có cơ hội áp dụng kiến thức đã học vào thực tiễn.

Em xin gửi lời cảm ơn sâu sắc đến TS. Hoàng Việt Anh - cán bộ hướng dẫn tại công ty, người đã dành thời gian hướng dẫn, giải đáp các thắc mắc và định hướng cho em trong suốt quá trình thực tập.

Em xin chân thành cảm ơn TS. Nguyễn Văn Thương - giảng viên phụ trách bộ môn, cùng TS. Hà Minh Cường và ThS. Hoàng Tích Phúc - các thầy giảng viên hướng dẫn tại Viện Công nghệ Hàng không Vũ trụ, đã tận tình chỉ bảo, theo dõi và hỗ trợ em trong suốt quá trình thực hiện đề tài. Sự hướng dẫn tận tâm của các thầy đã giúp em nâng cao kiến thức chuyên môn và hoàn thiện kỹ năng nghiên cứu khoa học.

Cuối cùng, em xin cảm ơn gia đình, bạn bè và đồng nghiệp đã luôn động viên, chia sẻ và tạo điều kiện tốt nhất để em hoàn thành báo cáo thực tập này.

Do thời gian và kinh nghiệm còn hạn chế, báo cáo không tránh khỏi những thiếu sót. Em rất mong nhận được sự góp ý của quý thầy cô và các bạn để báo cáo được hoàn thiện hơn.

Em xin chân thành cảm ơn!

\vspace{1cm}

\begin{flushright}
\textit{Hà Nội, tháng 12 năm 2025}\\
\textit{Sinh viên}\\[1.5cm]
\textbf{Ninh Hải Đăng}
\end{flushright}
\newpage\cleardoublepage

% Chuyển sang đánh số trang kiểu Ả Rập
\pagenumbering{arabic}

% ===== NỘI DUNG CHÍNH =====

% Phần 1: Nhận xét và xác nhận
\phantomsection
\addcontentsline{toc}{chapter}{PHẦN 1: NHẬN XÉT VÀ XÁC NHẬN}
\unnumberedchapter{PHẦN 1: NHẬN XÉT VÀ XÁC NHẬN}

\section*{1.1. Nhận xét của cán bộ hướng dẫn nơi thực tập}
\vspace{8cm} % Khoảng trống để viết tay
\begin{center}
    \begin{tabular}{lr}
       & \textit{........., ngày ...... tháng ...... năm ......} \\
       & \textbf{Cán bộ hướng dẫn} \\
       & (Ký và ghi rõ họ tên) \\
       & \\ & \\ & \\
       & \textbf{TS. Hoàng Việt Anh} \\
    \end{tabular}
\end{center}

\newpage

\section*{1.2. Nhận xét của giảng viên phụ trách sinh viên}
\vspace{8cm} % Khoảng trống để viết tay
\begin{center}
    \begin{tabular}{lr}
       & \textit{........., ngày ...... tháng ...... năm ......} \\
       & \textbf{Giảng viên phụ trách} \\
       & (Ký và ghi rõ họ tên) \\
       & \\ & \\ & \\
       & \textbf{TS. Nguyễn Văn Thương} \\
    \end{tabular}
\end{center}
\newpage
\newpage\cleardoublepage

% Phần 2: Khái quát báo cáo thực tập
\phantomsection
\addcontentsline{toc}{chapter}{PHẦN 2: KHÁI QUÁT BÁO CÁO THỰC TẬP}
\unnumberedchapter{PHẦN 2: KHÁI QUÁT BÁO CÁO THỰC TẬP}

\section{Giới thiệu chung}
\subsection{Tên đề tài}
\textbf{Tên đề tài (Tiếng Việt):} Giám sát biến động rừng tỉnh Cà Mau sử dụng dữ liệu viễn thám và mạng nơ ron nhân tạo.

\subsection{Đơn vị thực tập}
\begin{itemize}
    \item \textbf{Tên đơn vị thực tập:} Công ty TNHH Tư vấn và Phát triển Đồng Xanh (Green Field Development - GFD)
    \item \textbf{Địa chỉ:} 14 Trần Hưng Đạo, Phường Phan Chu Trinh, Quận Hoàn Kiếm, Thành phố Hà Nội
    \item \textbf{Lĩnh vực hoạt động:} Công ty chuyên về tư vấn và phát triển các giải pháp công nghệ trong lĩnh vực quản lý tài nguyên thiên nhiên, nông nghiệp bền vững và giám sát môi trường. Công ty là đối tác của Chi cục Kiểm lâm tỉnh Cà Mau trong các dự án giám sát và quản lý tài nguyên rừng.
    \item \textbf{Giám sát viên/Người hướng dẫn:} ThS. Nguyễn Văn A - Giám đốc Dự án Giám sát Rừng
    \item \textbf{Nhiệm vụ được giao:} Nghiên cứu và phát triển mô hình học sâu để giám sát biến động rừng tỉnh Cà Mau sử dụng dữ liệu viễn thám đa nguồn (Sentinel-1 và Sentinel-2).
    \item \textbf{Thời gian thực tập:} 3 tháng (từ ngày 15/9/2024 đến ngày 15/12/2024)
\end{itemize}

\section{Khái quát nội dung thực tập}
\subsection{Mục đích thực tập}
Nghiên cứu và phát triển mô hình học sâu trong giám sát biến động rừng sử dụng dữ liệu viễn thám đa nguồn, nhằm áp dụng công nghệ trí tuệ nhân tạo và viễn thám vào bài toán thực tiễn về bảo vệ và quản lý tài nguyên rừng tại Việt Nam.

\subsection{Các nội dung đã tham gia}
Trong quá trình thực tập tại Công ty TNHH Tư vấn và Phát triển Đồng Xanh, em đã tham gia thực hiện các nội dung sau:

(1) \textbf{Nghiên cứu tổng quan về vấn đề biến động rừng:} Tìm hiểu tình hình mất rừng toàn cầu và Việt Nam, nghiên cứu vai trò của rừng ngập mặn tại tỉnh Cà Mau, và tìm hiểu các phương pháp giám sát rừng truyền thống và hiện đại.

(2) \textbf{Nghiên cứu công nghệ viễn thám và học sâu:} Nghiên cứu về vệ tinh Sentinel-1 (dữ liệu radar SAR) và Sentinel-2 (dữ liệu quang học), tìm hiểu các chỉ số thực vật (NDVI, NBR, NDMI), nghiên cứu kiến trúc mạng nơ-ron tích chập (CNN) và ứng dụng trong phân loại ảnh viễn thám.

(3) \textbf{Thu thập và xử lý dữ liệu:} Thu thập dữ liệu Sentinel-1 và Sentinel-2 trên nền tảng Google Earth Engine; tiền xử lý dữ liệu bao gồm lọc mây, tính toán các chỉ số thực vật; tạo bộ dữ liệu mẫu với 2,630 điểm ground truth thuộc 4 lớp; và trích xuất 27 đặc trưng.

(4) \textbf{Phát triển mô hình học sâu:} Thiết kế kiến trúc mạng CNN với 36,676 tham số, thực hiện cross-validation 5-fold để tối ưu siêu tham số, huấn luyện và đánh giá mô hình.

(5) \textbf{Đánh giá và triển khai:} Phân tích kết quả với độ chính xác cao; áp dụng mô hình phân loại toàn bộ vùng nghiên cứu (162,469 ha); và phát triển ứng dụng web hiển thị kết quả.

(6) \textbf{Viết báo cáo và tài liệu:} Viết báo cáo chi tiết, tổng hợp tài liệu tham khảo, chuẩn bị biểu đồ và hình ảnh minh họa.

\section{Kết quả đạt được}
Đồ án đã hoàn thành các mục tiêu đề ra với kết quả đạt được ở mức độ cao:
\begin{itemize}
    \item \textbf{Xây dựng bộ dữ liệu chất lượng:} Thu thập và xử lý thành công dữ liệu Sentinel-1 và Sentinel-2 đa thời gian (01/2024 và 02/2025), trích xuất 27 đặc trưng và 2.630 điểm ground truth phân bố cân bằng cho 4 lớp.
    \item \textbf{Mô hình tối ưu:} Xây dựng thành công kiến trúc CNN với 36.676 tham số, patch size 3x3 tối ưu, đạt accuracy 98.86\% và ROC-AUC 99.98\% trên tập kiểm tra.
    \item \textbf{Hiệu quả dữ liệu đa nguồn:} Chứng minh việc kết hợp Sentinel-1 và Sentinel-2 cải thiện accuracy 5.44\% so với chỉ dùng Sentinel-2.
    \item \textbf{Sản phẩm thực tiễn:} Bản đồ biến động rừng toàn tỉnh Cà Mau (162.469 ha) và ứng dụng web Google Earth Engine trực quan hóa kết quả.
    \item \textbf{Kết quả thống kê:} Phát hiện 7.282 ha mất rừng (4.48\%) và 4.941 ha phục hồi rừng (3.04\%).
\end{itemize}

\section{Kinh nghiệm rút ra}
Qua quá trình thực hiện đồ án và đối mặt với các thách thức, em đã rút ra được những bài học kinh nghiệm quý báu:

\begin{itemize}
    \item \textbf{Về dữ liệu thực địa:} Tầm quan trọng của việc kiểm chứng thực địa (ground-truthing) để đánh giá độ tin cậy của mô hình. Dữ liệu ground truth chất lượng cao là yếu tố quyết định đến hiệu suất của mô hình học máy.
    \item \textbf{Về tối ưu hóa tính toán:} Kinh nghiệm xử lý dữ liệu lớn (Big Data) trong viễn thám. Việc dự đoán trên toàn vùng nghiên cứu rộng lớn đòi hỏi phải tối ưu hóa code, sử dụng xử lý theo lô (batch processing) và quản lý bộ nhớ hiệu quả.
    \item \textbf{Về mô hình hóa:} Sự cân bằng (trade-off) giữa độ phức tạp của mô hình và khả năng tổng quát hóa (generalization), đặc biệt khi làm việc với bộ dữ liệu nhỏ. Các kỹ thuật như Dropout và Data Augmentation là rất cần thiết.
    \item \textbf{Về giải thích mô hình:} Nhận thức rõ về hạn chế "hộp đen" của Deep Learning và sự cần thiết của các phương pháp XAI (Explainable AI) để tăng tính minh bạch và thuyết phục của kết quả.
\end{itemize}

\section{Kết luận}
Đồ án đã xây dựng thành công quy trình giám sát biến động rừng tỉnh Cà Mau sử dụng công nghệ viễn thám đa nguồn kết hợp với mô hình học sâu CNN. Kết quả nghiên cứu khẳng định tính ưu việt của việc kết hợp dữ liệu Sentinel-1 và Sentinel-2 cũng như hiệu quả của mô hình CNN được thiết kế tối ưu. Sản phẩm của đồ án gồm bản đồ biến động rừng và ứng dụng web có giá trị thực tiễn cao, hỗ trợ đắc lực cho công tác quản lý và bảo vệ rừng. Tuy nhiên, nghiên cứu cũng mở ra các hướng phát triển tiếp theo về mở rộng chuỗi thời gian, tăng cường kiểm chứng thực địa và áp dụng các mô hình tiên tiến hơn để khắc phục những hạn chế còn tồn tại.
\newpage
\newpage\cleardoublepage

% Phần 3: Báo cáo chi tiết kết quả đạt được
\phantomsection
\addcontentsline{toc}{chapter}{PHẦN 3: BÁO CÁO CHI TIẾT KẾT QUẢ ĐẠT ĐƯỢC}
\unnumberedchapter{PHẦN 3: BÁO CÁO CHI TIẾT KẾT QUẢ ĐẠT ĐƯỢC}

\section{Nghiên cứu đề bài và sơ sở lý thuyết}
\section*{1. Nghiên cứu đề bài, thu thập tài liệu}

Em đã tiến hành nghiên cứu toàn diện về bài toán giám sát biến động rừng, bao gồm cơ sở lý thuyết về rừng và biến động rừng, công nghệ viễn thám, học sâu, và các nghiên cứu liên quan. Các nội dung chính đã nghiên cứu được tóm tắt như sau:

\subsection*{1.1. Rừng và biến động rừng}

\textbf{Khái niệm và tầm quan trọng:} Theo Tổ chức Lương thực và Nông nghiệp Liên hợp quốc (FAO), rừng được định nghĩa là vùng đất có diện tích tối thiểu 0,5 ha, độ che phủ tán lá ít nhất 10\%, và chiều cao cây từ 5m trở lên, không phải là đất nông nghiệp hay đô thị. Rừng đóng vai trò then chốt trong việc duy trì cân bằng sinh thái, điều hòa khí hậu, lưu giữ carbon và bảo tồn đa dạng sinh học. Đặc biệt, rừng ngập mặn có khả năng lưu giữ carbon gấp 3--5 lần so với rừng nhiệt đới trên cạn.

\textbf{Tình hình mất rừng toàn cầu:} Trong giai đoạn 1990--2020, thế giới đã mất khoảng 420 triệu ha rừng với tốc độ trung bình 10 triệu ha mỗi năm (giai đoạn 2015--2020). Nguyên nhân chính bao gồm nông nghiệp quy mô lớn (chăn nuôi gia súc, cao su, cọ dầu), khai thác gỗ, và đô thị hóa. Các khu vực chịu ảnh hưởng nặng nề nhất là Mỹ La-tinh (26,9 triệu ha), trong đó riêng rừng Amazon mất 15,5 triệu ha, và Đông Nam Á với khu vực Borneo mất 5,8 triệu ha. Phá rừng đóng góp khoảng 23\% tổng lượng phát thải khí nhà kính toàn cầu.

\textbf{Tình hình tại Việt Nam:} Độ che phủ rừng Việt Nam đã tăng từ 27,2\% (1990) lên 42,01\% (2020), cho thấy xu hướng tích cực. Tuy nhiên, chất lượng rừng vẫn là vấn đề đáng lo ngại khi rừng nguyên sinh chỉ chiếm 0,25\% trong tổng diện tích 10,29 triệu ha rừng tự nhiên, phần lớn là rừng trồng (cao su, keo). Nguyên nhân mất rừng tại Việt Nam bao gồm chuyển đổi sang nông nghiệp, khai thác gỗ trái phép, đô thị hóa, cháy rừng, và đặc biệt là chuyển đổi sang nuôi trồng thủy sản ở vùng ven biển.

\subsection*{1.2. Công nghệ viễn thám và dữ liệu Sentinel}

\textbf{Nguyên lý viễn thám:} Viễn thám là kỹ thuật thu thập thông tin về bề mặt Trái Đất từ khoảng cách xa mà không tiếp xúc trực tiếp. Có hai loại chính: viễn thám bị động (sử dụng bức xạ Mặt Trời với cảm biến quang học) và viễn thám chủ động (phát xung điện từ và ghi nhận tín hiệu phản xạ như ra-đa SAR).

\textbf{Vệ tinh Sentinel-1:} Sử dụng ra-đa khẩu độ tổng hợp (SAR) hoạt động ở băng tần C (5,55 cm, 5,405 GHz). Chế độ IW (Interferometric Wide Swath) cung cấp 2 phân cực VV và VH với độ phân giải 10m, dải quét 250 km và chu kỳ 6--12 ngày. Ưu điểm nổi bật là khả năng xuyên qua mây, hoạt động cả ngày đêm, và nhạy cảm với cấu trúc vật thể cũng như độ ẩm. Phân cực VV nhạy với tán xạ bề mặt (độ ẩm đất), trong khi VH nhạy với tán xạ thể tích (cấu trúc tán lá).

\textbf{Vệ tinh Sentinel-2:} Cung cấp dữ liệu quang học đa phổ với 13 dải phổ (443--2190 nm) và độ phân giải từ 10m đến 60m, chu kỳ 5--10 ngày. Ưu điểm là cung cấp thông tin phổ phong phú cho việc tính toán các chỉ số thực vật, nhưng hạn chế bởi mây mù.

\textbf{Các chỉ số thực vật:} Em đã nghiên cứu ba chỉ số quan trọng. (1) \textbf{NDVI} (Normalized Difference Vegetation Index): $\text{NDVI} = \frac{\text{NIR} - \text{Red}}{\text{NIR} + \text{Red}}$, với giá trị cao (0,3--0,8) chỉ thị thực vật khỏe mạnh. (2) \textbf{NBR} (Normalized Burn Ratio): $\text{NBR} = \frac{\text{NIR} - \text{SWIR2}}{\text{NIR} + \text{SWIR2}}$, dùng để phát hiện cháy rừng và mất rừng. (3) \textbf{NDMI} (Normalized Difference Moisture Index): $\text{NDMI} = \frac{\text{NIR} - \text{SWIR1}}{\text{NIR} + \text{SWIR1}}$, đo độ ẩm tán lá và phát hiện stress rừng.

\textbf{Lợi ích tích hợp dữ liệu:} Kết hợp Sentinel-1 và Sentinel-2 có thể cải thiện độ chính xác phân loại từ 5--15\% so với sử dụng đơn nguồn, do cung cấp thông tin bổ sung về cấu trúc (SAR) và phổ (quang học).

\subsection*{1.3. Học sâu và mạng nơ-ron tích chập (CNN)}

\textbf{Kiến trúc CNN:} Mạng nơ-ron tích chập (CNN) là một dạng học sâu đặc biệt phù hợp cho xử lý ảnh. Các thành phần chính bao gồm: (1) \textbf{Lớp tích chập} thực hiện phép tích chập 2D để trích xuất đặc trưng cục bộ, có ưu điểm chia sẻ tham số và bất biến tịnh tiến; (2) \textbf{Hàm kích hoạt} ReLU (Rectified Linear Unit) với công thức $f(x) = \max(0, x)$ giúp giảm vấn đề vanishing gradient, và Softmax chuyển logits thành phân phối xác suất cho phân loại đa lớp; (3) \textbf{Pooling} với Max Pooling chọn giá trị lớn nhất và Global Average Pooling (GAP) tính trung bình toàn bộ feature map.

\textbf{Huấn luyện mạng:} Hàm mất mát Cross-Entropy được sử dụng: $L = -\sum y_i \cdot \log(\hat{y}_i)$. Thuật toán tối ưu AdamW kết hợp momentum và RMSprop với phân rã trọng số tách biệt. Các kỹ thuật điều chuẩn (regularization) bao gồm Batch Normalization (chuẩn hóa kích hoạt), Dropout (tắt ngẫu nhiên nơ-ron), và Dropout2d (tắt toàn bộ feature map, phù hợp cho CNN).

\textbf{Ứng dụng trong viễn thám:} CNN được áp dụng cho phân loại dựa trên patch, trong đó trích xuất patch nhỏ quanh mỗi pixel trung tâm và sử dụng CNN để phân loại. Dữ liệu cần được chuẩn hóa (Z-score) để đảm bảo thang đo đồng nhất giữa các nguồn dữ liệu khác nhau (NDVI [-1,1], quang học [0,1], SAR [-25,0] dB).

\subsection*{1.4. Các nghiên cứu liên quan}

\textbf{Phát triển phương pháp giám sát rừng:} Các phương pháp giám sát rừng đã phát triển từ khảo sát thực địa truyền thống (trước 1970), qua giải đoán trực quan ảnh viễn thám (1970--1990), đến học máy truyền thống như Random Forest và SVM (1990--2012), và hiện nay là học sâu với CNN và U-Net (từ 2012 đến nay).

\textbf{Nghiên cứu tiêu biểu:} Hansen et al. (2013) tạo bản đồ Global Forest Change sử dụng Decision Tree với dữ liệu Landsat, đạt độ chính xác khoảng 85\% ở quy mô toàn cầu. Kussul et al. (2017) áp dụng CNN 2D với Sentinel-2 tại Ukraine, đạt 94,5\% accuracy, vượt trội Random Forest (<92\%). Hu et al. (2020) kết hợp Sentinel-1 và Sentinel-2 tại Madagascar, cải thiện accuracy từ 87\% (chỉ S2) lên 92\% (S1+S2). Tại Việt Nam, Nguyen et al. (2020) sử dụng Random Forest/SVM đạt 91,2\% tại Đắk Nông; Vo et al. (2020) phân tích biến động rừng ngập mặn Cà Mau qua Landsat (2001--2019).

\textbf{Khoảng trống nghiên cứu:} Hiện còn thiếu các nghiên cứu áp dụng CNN cho rừng Việt Nam, đặc biệt là rừng ngập mặn Cà Mau. Bên cạnh đó, vấn đề bộ dữ liệu nhỏ (CNN thường cần hàng trăm nghìn mẫu nhưng viễn thám chỉ có 2.000--5.000 mẫu) và tối ưu hóa kiến trúc CNN cho dữ liệu kết hợp SAR và quang học vẫn còn là thách thức.


\section{Báo cáo chi tiết nội dung nghiên cứu và thực hiện}
\section*{2. Viết báo cáo chi tiết nội dung đã tiến hành trong quá trình thực tập}

Em đã thực hiện toàn bộ quy trình nghiên cứu từ thu thập dữ liệu, xây dựng phương pháp, đến triển khai và đánh giá mô hình. Nội dung chi tiết được trình bày trong hai phần chính:

\subsection*{Phần A: Dữ liệu và Phương pháp nghiên cứu}

\subsubsection*{A.1. Thu thập và tiền xử lý dữ liệu viễn thám}

Em đã thu thập và xử lý hai nguồn dữ liệu vệ tinh chính trên nền tảng Google Earth Engine:

\textbf{Dữ liệu Sentinel-2 (quang học):} Sản phẩm S2\_SR\_HARMONIZED (Surface Reflectance Level-2A) với độ phân giải 10m tại hai thời điểm 30/01/2024 (T1 - kỳ trước) và 28/02/2025 (T2 - kỳ sau). Các kênh phổ được sử dụng bao gồm B4 (Red), B8 (NIR), B11 (SWIR1), và B12 (SWIR2). Các chỉ số thực vật được tính toán gồm NDVI, NBR, và NDMI. Quy trình tiền xử lý bao gồm lọc mây với ngưỡng 50\%, chuyển đổi sang giá trị phản xạ, tính các chỉ số thực vật, và mosaic các tiles thành ảnh liền mạch.

\textbf{Dữ liệu Sentinel-1 (ra-đa):} Sản phẩm S1\_GRD (Ground Range Detected) với độ phân giải 10m tại thời điểm 04/02/2024 (T1) và 22/02/2025 (T2). Dữ liệu sử dụng phân cực VV (tán xạ bề mặt/độ ẩm) và VH (tán xạ thể tích/sinh khối) ở chế độ Interferometric Wide (IW) trên quỹ đạo đi xuống. Ưu điểm của dữ liệu này là khả năng xuyên qua mây và hoạt động cả ngày đêm.

Dữ liệu đầu ra được xuất dưới định dạng GeoTIFF với hệ quy chiếu EPSG:32648 (UTM Zone 48N).

\subsubsection*{A.2. Bộ dữ liệu thực địa và trích xuất đặc trưng}

\textbf{Dữ liệu ground truth:} Công ty GFD đã cung cấp 2.630 điểm thực địa được thu thập qua khảo sát drone và số hóa trên QGIS, phân bố cân bằng theo 4 lớp: (1) Lớp 0 - Rừng ổn định với 656 điểm (24,9\%) là các vùng có rừng ở cả 2 kỳ; (2) Lớp 1 - Mất rừng với 650 điểm (24,7\%) là các vùng rừng chuyển thành không rừng; (3) Lớp 2 - Phi rừng ổn định với 664 điểm (25,3\%) là các vùng không rừng ở cả 2 kỳ; (4) Lớp 3 - Phục hồi rừng với 660 điểm (25,1\%) là các vùng không rừng chuyển thành rừng.

\textbf{Trích xuất đặc trưng (27 features):} Mỗi điểm dữ liệu được trích xuất dưới dạng patch 3×3 pixel (30m × 30m) với 27 kênh đặc trưng: (1) 21 đặc trưng từ Sentinel-2 bao gồm 4 band quang phổ (B4, B8, B11, B12) cộng 3 chỉ số (NDVI, NBR, NDMI), mỗi loại có 3 giá trị (T1, T2, và Δ = T2 - T1); (2) 6 đặc trưng từ Sentinel-1 bao gồm 2 phân cực (VV, VH), mỗi loại có 3 giá trị (T1, T2, Δ).

Phân tích cho thấy các đặc trưng Delta (ΔNDVI, ΔNBR, ΔNDMI) có khả năng phân tách tốt nhất giữa các lớp: Mất rừng có giá trị âm (suy giảm thực vật), Phục hồi rừng có giá trị dương (gia tăng sinh khối), trong khi Rừng ổn định và Phi rừng tập trung quanh 0.

\subsubsection*{A.3. Kiến trúc mô hình CNN}

Em đã thiết kế một mạng CNN nhẹ với 36.676 tham số, phù hợp với bộ dữ liệu nhỏ (2.630 mẫu). Kiến trúc gồm: (1) Input là patch 3×3 pixel × 27 channels; (2) Conv1 gồm 64 filters với kernel 3×3, BatchNorm, ReLU và Dropout2D 70\% với tổng 15.680 tham số; (3) Conv2 gồm 32 filters với kernel 3×3, BatchNorm, ReLU và Dropout2D 70\% với tổng 18.496 tham số; (4) GAP (Global Average Pooling) chuyển đổi thành vector 32 chiều; (5) FC1 chuyển từ 32 sang 64 chiều với BatchNorm, ReLU và Dropout 70\% với tổng 2.240 tham số; (6) FC2 chuyển từ 64 sang 4 chiều (Output) với 260 tham số.

\textbf{Lý do thiết kế:} Số filter giảm dần (64→32) thay vì tăng để tránh quá khớp với dữ liệu nhỏ; Dropout 70\% (cao hơn mức thông thường 20--50\%) để cân bằng tỷ lệ mẫu/tham số thấp (72:1); Chỉ 2 lớp tích chập vì đủ cho patch 3×3 mà không quá phức tạp; Global Average Pooling thay vì Fully Connected truyền thống để giảm tham số.

\subsubsection*{A.4. Chuẩn bị dữ liệu và chiến lược huấn luyện}

\textbf{Chuẩn hóa dữ liệu:} Áp dụng Z-score normalization với tham số (μ, σ) tính từ tập huấn luyện, sau đó áp dụng cho tập kiểm định và kiểm tra để đảm bảo tính nhất quán.

\textbf{Phân chia dữ liệu:} Dữ liệu được chia thành 80\% (2.104 mẫu) cho huấn luyện và kiểm định với 5-Fold Cross Validation, và 20\% (526 mẫu) cho tập kiểm tra cuối cùng (không dùng trong huấn luyện). Tỷ lệ lớp được bảo tồn trong cả tập huấn luyện và kiểm tra.

\textbf{Siêu tham số huấn luyện:} Các siêu tham số bao gồm: Epochs tối đa 200; Batch size 64; Learning rate khởi tạo 0,001; Optimizer AdamW với weight decay $10^{-3}$; Dropout rate 70\%; Early stopping với patience 15 epochs; Learning rate scheduler ReduceLROnPlateau.

\textbf{Quy trình huấn luyện:} (1) Khởi tạo trọng số với Kaiming/He cho Conv và Xavier cho FC; (2) Thực hiện 5-Fold Cross Validation trên 80\% dữ liệu để đánh giá độ ổn định; (3) Huấn luyện cuối cùng trên toàn bộ 80\% với siêu tham số tối ưu; (4) Đánh giá trên 20\% tập kiểm tra cố định.

\subsubsection*{A.5. Áp dụng mô hình phân loại toàn vùng}

Sau khi huấn luyện, mô hình được áp dụng để phân loại toàn bộ vùng nghiên cứu với phạm vi khoảng 16,2 triệu pixel (162.468 ha) khu vực quy hoạch lâm nghiệp Cà Mau. Quy trình thực hiện như sau: Với mỗi pixel, trích xuất patch 3×3 lân cận (mirror padding cho pixel biên), chuẩn hóa Z-score, đưa qua CNN, và chọn lớp có xác suất cao nhất (argmax). Quá trình được tối ưu hóa bằng cách xử lý theo lô 10.000 pixel và sử dụng mixed precision (FP16) để giảm bộ nhớ GPU. Output được lưu dưới dạng GeoTIFF (EPSG:32648, 10m).

\subsection*{Phần B: Kết quả thực nghiệm}

\subsubsection*{B.1. Thiết lập và thử nghiệm mô hình}

\textbf{Ảnh hưởng của kích thước patch:} Em đã thử nghiệm 4 kích thước patch khác nhau: (1) 1×1 đạt 98,23\% accuracy và 99,78\% ROC-AUC; (2) 3×3 đạt 98,86\% accuracy và 99,98\% ROC-AUC (tối ưu); (3) 5×5 đạt 98,67\% accuracy và 99,89\% ROC-AUC; (4) 7×7 đạt 98,29\% accuracy và 99,86\% ROC-AUC. Patch 3×3 cho kết quả tốt nhất do cân bằng giữa ngữ cảnh không gian và tránh nhiễu từ điểm lân cận xa.

\textbf{Ảnh hưởng của nguồn dữ liệu:} Các thử nghiệm bao gồm: (1) Chỉ S2 (delta) đạt 87,65\% accuracy; (2) S2 đầy đủ (T1+T2+Δ) đạt 93,42\% accuracy; (3) Chỉ S1 đạt 83,27\% accuracy; (4) S1 + S2 (tất cả) đạt 98,86\% accuracy và 99,98\% ROC-AUC (tối ưu). Kết hợp S1+S2 cải thiện 5,44\% so với chỉ dùng S2, chứng minh lợi ích của dữ liệu đa nguồn.

\subsubsection*{B.2. Kết quả Cross-Validation và đánh giá cuối cùng}

\textbf{Kết quả 5-Fold Cross Validation:} Accuracy trung bình đạt 98,48\% với độ lệch chuẩn ±0,36\% (dao động 98,10\%--99,05\%). Epoch tối ưu trung bình là 71 epoch với validation loss 0,0532. Kết quả ổn định cho thấy mô hình không bị quá khớp nhờ Dropout 70\% hiệu quả.

\textbf{Kết quả trên tập test cuối cùng (526 mẫu):} Mô hình đạt Accuracy 98,86\% và ROC-AUC 99,98\%. Precision, Recall và F1-Score đều đạt 98,86\% (macro average).

\subsubsection*{B.3. Phân tích ma trận nhầm lẫn}

Tổng số lỗi: 6/526 mẫu (tỷ lệ lỗi 1,14\%). Chi tiết từng lớp như sau: Lớp 0 (Rừng ổn định) đạt Precision 96,99\%, Recall 98,47\%, F1 97,73\% với 2 False Positive và 2 False Negative (nhầm với Mất rừng). Lớp 1 (Mất rừng) đạt Precision 98,44\%, Recall 96,92\%, F1 97,67\% với 2 False Positive và 4 False Negative (nhầm với Rừng ổn định). Lớp 2 (Phi rừng ổn định) và Lớp 3 (Phục hồi rừng) đều đạt Precision, Recall, F1-Score 100\% mà không có lỗi phân loại.

\textbf{Nguyên nhân nhầm lẫn giữa Lớp 0 và Lớp 1:} Cả hai lớp đều có rừng ở ít nhất một thời điểm nên có sự tương đồng về phổ. Rừng suy thoái nhẹ có thể có phổ tương tự rừng ổn định. Ngoài ra, hiệu ứng biên (pixel hỗn hợp tại ranh giới) và biến động theo mùa của rừng ngập mặn cũng góp phần vào sự nhầm lẫn.

\subsubsection*{B.4. Bản đồ biến động rừng và thống kê diện tích}

Mô hình đã được áp dụng phân loại toàn bộ 162.468,50 ha khu vực quy hoạch lâm nghiệp tỉnh Cà Mau. Kết quả thống kê diện tích:

\begin{table}[H]
\centering
\begin{tabular}{lrrr}
\toprule
\textbf{Lớp biến động} & \textbf{Diện tích (ha)} & \textbf{Tỷ lệ (\%)} & \textbf{Số pixel} \\
\midrule
Rừng ổn định & 120.716,91 & 74,30 & 12.071.691 \\
Mất rừng & 7.282,15 & 4,48 & 728.215 \\
Phi rừng ổn định & 29.528,54 & 18,17 & 2.952.854 \\
Phục hồi rừng & 4.940,90 & 3,04 & 494.090 \\
\midrule
\textbf{Tổng} & \textbf{162.468,50} & \textbf{100,00} & \textbf{16.246.850} \\
\bottomrule
\end{tabular}
\caption{Thống kê diện tích biến động rừng tỉnh Cà Mau (01/2024 - 02/2025)}
\end{table}

\textbf{Nhận xét chính:} Rừng ổn định chiếm 3/4 diện tích (120.716,91 ha), cho thấy phần lớn diện tích rừng được bảo tồn tốt; Mất rừng là 7.282,15 ha (4,48\%), tập trung ven ao nuôi tôm và kênh mương; Phục hồi rừng là 4.940,90 ha (3,04\%), chủ yếu ven bờ ao và biên giới biển; Mất rừng ròng là 2.341,25 ha (1,44\%), tương đương chênh lệch giữa mất rừng và phục hồi.

Bản đồ chi tiết cho thấy mất rừng (màu đỏ) tập trung ở các khu vực ven ao nuôi và kênh mương, phản ánh hoạt động sản xuất thực tế. Phục hồi rừng (màu xanh lam) rải rác ven bờ ao và biên giới biển, cho thấy khả năng tái sinh tự nhiên.

\subsubsection*{B.5. So sánh với các nghiên cứu khác}

So với Global Forest Watch (Hansen et al., 2013) - công cụ giám sát rừng toàn cầu: GFW đạt Accuracy khoảng 85\%, sử dụng Landsat 30m với Decision Trees ở quy mô toàn cầu; trong khi nghiên cứu này đạt Accuracy 98,86\%, sử dụng Sentinel 10m với CNN ở quy mô khu vực.

\textbf{Ưu điểm của phương pháp đề xuất:} Độ phân giải cao hơn (10m so với 30m) giúp phát hiện chi tiết ao nuôi và kênh mương; Đa nguồn dữ liệu (S1+S2) giúp khắc phục vấn đề mây và cung cấp thông tin bổ sung; CNN học sâu có khả năng tự động học đặc trưng phức tạp.

Hai phương pháp bổ sung nhau: GFW phù hợp cho giám sát toàn cầu dài hạn với tính nhất quán cao; phương pháp đề xuất phù hợp cho giám sát chi tiết ở quy mô khu vực với độ chính xác cao hơn.


\section{Các sản phẩm hoàn thành}
\section*{3. Các sản phẩm khác}

Ngoài báo cáo nghiên cứu chi tiết, quá trình thực tập đã tạo ra các sản phẩm cụ thể sau:

\subsection*{3.1. Mô hình học sâu CNN}

\begin{itemize}
    \item Kiến trúc: Mạng nơ-ron tích chập (CNN) với 36,676 tham số
    \item Input: Patch 3×3 pixel với 27 kênh đặc trưng
    \item Output: Phân loại 4 lớp (Rừng ổn định, Mất rừng, Phi rừng, Phục hồi rừng)
    \item Độ chính xác: 98.86\% trên tập kiểm tra
    \item ROC-AUC score: 99.98\%
    \item Cross-validation: 98.48\% ± 0.36\%
\end{itemize}

\subsection*{3.2. Bản đồ biến động rừng tỉnh Cà Mau}

\begin{itemize}
    \item Vùng nghiên cứu: Toàn bộ ranh giới lâm nghiệp tỉnh Cà Mau mới sau quyết định sáp nhập tỉnh Cà Mau với tỉnh Bạc Liêu cũ theo Nghị quyết số 1278/NQ-UBTVQH15 ngày 24/10/2024, có hiệu lực từ 01/07/2025 (162,469 ha)
    \item Thời kỳ phân tích: Tháng 1/2024 - Tháng 2/2025
    \item Độ phân giải không gian: 10m
    \item Kết quả phát hiện:
    \begin{itemize}
        \item Diện tích mất rừng: 7,282 ha (4.48\% tổng diện tích)
        \item Diện tích phục hồi rừng: 4,941 ha (3.04\% tổng diện tích)
        \item Rừng ổn định và khu vực không phải rừng
    \end{itemize}
    \item Format: GeoTIFF, hệ tọa độ EPSG:32648 (UTM Zone 48N)
\end{itemize}

\subsection*{3.3. Ứng dụng web hiển thị kết quả trên Google Earth Engine}

\begin{itemize}
    \item URL: \url{https://ee-bonglantrungmuoi.projects.earthengine.app/view/giam-sat-bien-dong-rung-ca-mau}
    \item Tính năng:
    \begin{itemize}
        \item Hiển thị bản đồ biến động rừng tương tác
        \item Phân biệt màu sắc cho 4 lớp biến động
        \item Cho phép người dùng phóng to/thu nhỏ, di chuyển bản đồ
        \item Hiển thị thông tin metadata và chú giải
    \end{itemize}
    \item Công nghệ: Google Earth Engine Apps, JavaScript API
    \item Truy cập: Công khai, không yêu cầu đăng nhập
\end{itemize}

\subsection*{3.4. Mã nguồn và tài liệu}

\begin{itemize}
    \item Mã nguồn tiền xử lý dữ liệu S1/S2 trên Google Earth Engine (JavaScript)
    \item Mã nguồn huấn luyện mô hình CNN (Python, PyTorch)
    \item Scripts phân loại và xuất kết quả
    \item Tài liệu hướng dẫn sử dụng
    \item Báo cáo kỹ thuật chi tiết (tài liệu này)
\end{itemize}


\section{Các kiến nghị đề xuất}
\section*{5. Kiến nghị}

Dựa trên kết quả đạt được và những khó khăn gặp phải, em xin đề xuất một số kiến nghị cho các nghiên cứu tiếp theo:

\subsection*{5.1. Mở rộng nghiên cứu theo chuỗi thời gian}

\begin{itemize}
    \item Mở rộng thành chuỗi thời gian dài hạn với tần suất giám sát cao hơn (hàng tháng hoặc hàng quý)
    \item Phân tích xu hướng biến động rừng theo mùa và theo năm
    \item Xây dựng hệ thống cảnh báo sớm về mất rừng dựa trên phân tích chuỗi thời gian
    \item Áp dụng các mô hình time-series như LSTM, Transformer cho dự báo biến động rừng
\end{itemize}

\subsection*{5.2. Tăng cường kiểm chứng kết quả phân loại}

\begin{itemize}
    \item Tổ chức khảo sát thực địa để kiểm chứng độ chính xác của bản đồ biến động rừng do mô hình tạo ra
    \item Đo đạc các thông số sinh thái tại các khu vực được phân loại để xác nhận kết quả
    \item So sánh kết quả phân loại trên diện rộng với số liệu thực tế để đánh giá độ tin cậy
    \item Xây dựng quy trình kiểm chứng độc lập cho các sản phẩm giám sát rừng
\end{itemize}

\subsection*{5.3. Cải thiện hiệu năng mô hình}

\begin{itemize}
    \item Tối ưu hóa kiến trúc mô hình để giảm thời gian dự đoán
    \item Nghiên cứu các kỹ thuật model compression (pruning, quantization) để triển khai nhanh hơn
    \item Sử dụng GPU/TPU mạnh hơn hoặc phân tán tính toán
    \item Tăng kích thước bộ dữ liệu huấn luyện để cải thiện độ chính xác
\end{itemize}

\subsection*{5.4. Mở rộng vùng nghiên cứu}

\begin{itemize}
    \item Áp dụng mô hình cho các tỉnh ven biển khác có rừng ngập mặn (Kiên Giang, Sóc Trăng)
    \item So sánh đặc điểm biến động rừng giữa các vùng khác nhau
    \item Xây dựng bản đồ biến động rừng quy mô quốc gia
    \item Tích hợp với các hệ thống giám sát rừng hiện có của Bộ Nông nghiệp và Phát triển Nông thôn
\end{itemize}

\subsection*{5.5. Phát triển công cụ và ứng dụng}

\begin{itemize}
    \item Phát triển ứng dụng mobile cho công tác giám sát rừng tại hiện trường
    \item Xây dựng dashboard tương tác cho các nhà quản lý
    \item Tích hợp tính năng báo cáo tự động và xuất số liệu thống kê
    \item Kết nối với các hệ thống cảnh báo cháy rừng, thiên tai
\end{itemize}

\newpage
\newpage\cleardoublepage

% Tài liệu tham khảo
\phantomsection
\addcontentsline{toc}{chapter}{TÀI LIỆU THAM KHẢO}
\unnumberedchapter{TÀI LIỆU THAM KHẢO}

% Tài liệu tiếng Việt
\nocitevi{*}
\bibliographystylevi{vnuvi}
{\renewcommand{\chapter}[2]{}\bibliographyvi{references_vi}}

\vspace{1em}

% Tài liệu tiếng Anh
\nociteen{*}
\bibliographystyleen{vnu}
{\renewcommand{\chapter}[2]{}\bibliographyen{references_en}}

\end{document}