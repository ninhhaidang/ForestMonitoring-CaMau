% ==============================================================================
% 3.6. KẾT QUẢ NGHIÊN CỨU VÀ PHÂN TÍCH (BÀN LUẬN) KẾT QUẢ
% ------------------------------------------------------------------------------
% Trình bày thành các chương 1, 2, 3,...; nêu các kết quả nghiên cứu đạt được
% và đánh giá về các kết quả này.
% ==============================================================================
% CHƯƠNG 1: CƠ SỞ LÝ THUYẾT CỦA BÀI TOÁN GIÁM SÁT BIẾN ĐỘNG RỪNG
% ==============================================================================

\chapter{CƠ SỞ LÝ THUYẾT CỦA BÀI TOÁN GIÁM SÁT BIẾN ĐỘNG RỪNG}

% 1.1. Rừng và biến động rừng
\section{Rừng và biến động rừng}

\section{Rừng và biến động rừng}

Rừng là một hệ sinh thái bao gồm chủ yếu là cây cối, thực vật và động vật sống cùng nhau trong một môi trường phức tạp. Theo định nghĩa của Tổ chức Nông nghiệp và Lương thực Liên Hợp Quốc (FAO), rừng là vùng đất có diện tích tối thiểu 0.5 ha với độ che phủ tán cây trên 10\%, chiều cao cây tối thiểu 5 mét khi trưởng thành, và không phải là đất nông nghiệp hoặc đô thị \citeen{fao2020}. Rừng bao phủ khoảng 31\% diện tích đất liền toàn cầu và đóng vai trò quan trọng trong việc duy trì cân bằng sinh thái, điều hòa khí hậu thông qua hấp thụ CO$_2$ và thải oxy, lưu giữ carbon, bảo tồn đa dạng sinh học, điều tiết nguồn nước và chống xói mòn đất, đồng thời cung cấp tài nguyên thiên nhiên và sinh kế cho hàng tỷ người trên thế giới. Tùy theo vị trí địa lý và điều kiện khí hậu, rừng được phân loại thành nhiều kiểu khác nhau như rừng nhiệt đới, rừng ôn đới, rừng phương bắc (taiga), rừng ngập mặn, và rừng tràm. Trong đó, rừng ngập mặn là hệ sinh thái đặc biệt quan trọng ở các vùng ven biển nhiệt đới, có khả năng lưu giữ carbon cao gấp 3--5 lần so với rừng nhiệt đới trên cạn \citeen{donato2011,alongi2014}.

\subsection{Tình hình mất rừng trên thế giới}

Tốc độ mất rừng toàn cầu vẫn đang ở mức báo động. Theo báo cáo ``Global Forest Resources Assessment 2020'' của FAO \citeen{fao2020}, tổng diện tích rừng bị phá từ năm 1990 đến 2020 ước tính khoảng 420 triệu ha. Mặc dù tốc độ mất rừng ròng đã có xu hướng giảm trong thập kỷ gần đây nhờ các nỗ lực trồng rừng và phục hồi rừng, việc chuyển đổi đất rừng tự nhiên sang mục đích nông nghiệp, chăn nuôi và phát triển cơ sở hạ tầng vẫn diễn ra với quy mô lớn và phức tạp. Đáng lo ngại hơn, phần lớn diện tích rừng bị mất là rừng nguyên sinh --- loại rừng có giá trị sinh thái cao nhất và gần như không thể phục hồi được.

Sự suy giảm diện tích rừng tập trung nghiêm trọng nhất tại các khu vực nhiệt đới và cận nhiệt đới, nơi chứa đựng hơn một nửa đa dạng sinh học của Trái Đất. Theo báo cáo ``Deforestation Fronts: Drivers and Responses in a Changing World'' của WWF công bố năm 2021 \citeen{wwf2021}, trong giai đoạn 2004--2017, hơn 43 triệu ha rừng đã bị xóa sổ tại 24 ``mặt trận phá rừng'' (deforestation fronts) trên thế giới --- diện tích tương đương với quy mô lãnh thổ Morocco. Các mặt trận này chiếm hơn một nửa (52\%) tổng diện tích rừng bị mất tại khu vực Mỹ Latinh, Châu Phi cận Sahara, Đông Nam Á và Châu Đại Dương trong cùng giai đoạn. Nông nghiệp quy mô lớn được xác định là nguyên nhân trực tiếp lớn nhất gây mất rừng trên toàn thế giới, trong đó chăn nuôi gia súc là tác nhân đơn lẻ gây mất rừng nhiệt đới nhiều nhất.

Trong số 24 mặt trận, 9 mặt trận nằm tại Mỹ Latinh và đây là khu vực chịu tổn thất nặng nề nhất với 26.9 triệu ha rừng bị mất --- chiếm gần 2/3 tổng diện tích mất rừng toàn cầu. Rừng Amazon tại Brazil dẫn đầu với 15.5 triệu ha bị phá hủy, tiếp theo là Gran Chaco (5.2 triệu ha), Cerrado của Brazil (3 triệu ha) và vùng đất thấp Bolivia (1.5 triệu ha). Rừng Amazon --- được mệnh danh là ``lá phổi xanh của Trái Đất'' --- đang bị thu hẹp với tốc độ đáng báo động do áp lực từ ngành chăn nuôi gia súc quy mô lớn, trồng đậu nành xuất khẩu, khai thác gỗ và mở rộng cơ sở hạ tầng. Hình~\ref{fig:deforestation_latin_america} minh họa các mặt trận mất rừng trọng điểm tại khu vực này.

\begin{figure}[H]
    \centering
    \includegraphics[width=0.95\textwidth]{img/chapter1/Mat-rung-My-Latinh.png}
    \caption{Các mặt trận mất rừng trọng điểm tại khu vực Mỹ Latinh. (Nguồn: WWF, 2021)}
    \label{fig:deforestation_latin_america}
\end{figure}

Không chỉ giới hạn ở Châu Mỹ, tình trạng phá rừng cũng đang diễn biến phức tạp tại các khu vực khác trên thế giới. Tại Châu Phi cận Sahara, WWF đã xác định 8 mặt trận mất rừng trọng điểm trải dài từ Tây Phi đến Đông Phi. Khu vực Tây Phi (Liberia, Bờ Biển Ngà, Ghana) đang chịu áp lực từ việc mở rộng trồng ca cao và khai thác gỗ. Lưu vực Congo tại Trung Phi --- khu rừng nhiệt đới lớn thứ hai thế giới sau Amazon, bao gồm các quốc gia như Cameroon, Gabon, Cộng hòa Congo, CHDC Congo và Cộng hòa Trung Phi --- đang đối mặt với áp lực ngày càng tăng từ hoạt động khai thác gỗ, mở rộng nông nghiệp hộ gia đình và khai khoáng. Khác với Mỹ Latinh nơi nông nghiệp thương mại quy mô lớn là nguyên nhân chính, tại Châu Phi cận Sahara, nông nghiệp tự cung tự cấp và nông nghiệp thương mại quy mô nhỏ là động lực chủ yếu gây mất rừng. Tại Đông Phi, các mặt trận mất rừng tại Zambia, Mozambique và Madagascar cũng đang diễn biến phức tạp do nhu cầu đất canh tác và sản xuất than củi.

Tại khu vực Đông Nam Á và Châu Đại Dương, 7 mặt trận mất rừng được ghi nhận với quy mô thiệt hại đứng thứ hai sau Mỹ Latinh. Borneo (Indonesia và Malaysia) là mặt trận mất rừng lớn thứ hai thế giới với 5.8 triệu ha bị phá hủy, tiếp theo là Sumatra (2.5 triệu ha), New Guinea (1.3 triệu ha) và Myanmar (1 triệu ha). Lưu vực sông Mekong --- bao gồm Cambodia, Lào và Myanmar --- đang mất rừng nhanh chóng do khai thác gỗ trái phép, xây dựng đập thủy điện và mở rộng nông nghiệp. Nông nghiệp thương mại với các đồn điền cọ dầu và cao su quy mô lớn là nguyên nhân chính gây mất rừng tại khu vực này. Indonesia và Malaysia --- hai quốc gia sản xuất dầu cọ lớn nhất thế giới --- đã mất hàng triệu ha rừng nhiệt đới trong hai thập kỷ qua. Ngoài ra, việc đốt rừng để mở rộng đất canh tác còn gây ra các đợt khói mù nghiêm trọng, ảnh hưởng đến sức khỏe của hàng trăm triệu người trong khu vực. Đáng chú ý, miền Đông Australia cũng được xác định là một mặt trận mất rừng do các đợt cháy rừng thảm khốc và hoạt động khai hoang. Hình~\ref{fig:deforestation_africa_asia} thể hiện các mặt trận mất rừng trọng điểm tại hai khu vực này.

\begin{figure}[H]
    \centering
    \includegraphics[width=0.95\textwidth]{img/chapter1/Mat-rung-Chau-Phi-va-Dong-Nam-A.png}
    \caption{Các mặt trận mất rừng trọng điểm tại Châu Phi và Đông Nam Á. (Nguồn: WWF, 2021)}
    \label{fig:deforestation_africa_asia}
\end{figure}

Xu hướng này vẫn tiếp diễn trong những năm gần đây. Theo Global Forest Watch \citeen{gfw2021}, thế giới mất khoảng 10 triệu ha rừng nhiệt đới mỗi năm trong giai đoạn 2015--2020. Việc này không chỉ làm giảm khả năng hấp thụ CO$_2$ mà còn trực tiếp phát thải khí nhà kính từ việc đốt rừng và phân hủy sinh khối. Theo IPCC \citeen{ipcc2019}, phá rừng và thay đổi sử dụng đất đóng góp khoảng 23\% tổng lượng phát thải khí nhà kính do con người gây ra, góp phần làm gia tăng hiện tượng biến đổi khí hậu toàn cầu.

\subsection{Tình hình mất rừng tại Việt Nam}

Việt Nam đã trải qua những biến đổi lớn về độ che phủ rừng trong 30 năm qua. Sau thời kỳ suy giảm nghiêm trọng (độ che phủ chỉ còn 28\% vào năm 1990 do chiến tranh và khai thác bừa bãi), Việt Nam đã thực hiện nhiều chương trình phục hồi và phát triển rừng. Nhờ các chương trình như ``Trồng 5 triệu ha rừng'' (1998--2010), độ che phủ rừng đã tăng lên 42\% vào năm 2020 \citevi{bnnptnt2021}.

Tuy nhiên, chất lượng rừng là một vấn đề đáng lo ngại. Mặc dù tổng diện tích rừng tăng từ 9.4 triệu ha (1990) lên 14.6 triệu ha (2020) chủ yếu nhờ rừng trồng (cao su, keo, thông), chất lượng rừng tự nhiên lại suy giảm đáng kể. Theo số liệu của Bộ NN\&PTNT (2020), rừng tự nhiên hiện có khoảng 10.29 triệu ha, nhưng rừng nguyên sinh chỉ còn chiếm khoảng 0.25\% tổng diện tích rừng \citevi{thanhnien2021}.

Nguyên nhân chính gây mất rừng tại Việt Nam bao gồm việc chuyển đổi sang đất nông nghiệp như trồng cà phê, cao su và điều; khai thác gỗ trái phép; phát triển cơ sở hạ tầng và đô thị hóa; cháy rừng; và hoạt động nuôi trồng thủy sản, đặc biệt tại khu vực ven biển và đồng bằng sông Cửu Long.

\begin{figure}[H]
    \centering
    \begin{tikzpicture}
        \begin{axis}[
            width=0.95\textwidth,
            height=8cm,
            xlabel={Năm},
            ylabel={Độ che phủ rừng (\%)},
            xmin=1990, xmax=2020,
            ymin=25, ymax=45,
            xtick={1990,1995,2000,2005,2010,2015,2020},
            xticklabel style={/pgf/number format/1000 sep={},yshift=-3pt},
            yticklabel style={xshift=-3pt},
            ytick={25,30,35,40,45},
            legend pos=north west,
            grid=major,
            grid style={dashed,gray!30},
            every axis plot/.append style={thick}
        ]
        \addplot[color=green,mark=*] coordinates {
            (1990,27.2)
            (1995,28.2)
            (2000,33.7)
            (2005,37.0)
            (2010,39.5)
            (2015,40.84)
            (2020,42.01)
        };
        \legend{Độ che phủ rừng}
        \end{axis}
    \end{tikzpicture}
    \caption{Biến động độ che phủ rừng Việt Nam giai đoạn 1990--2020 (Báo cáo tổng hợp)}
    \label{fig:vietnam_forest_change}
\end{figure}

Hình~\ref{fig:vietnam_forest_change} cho thấy độ che phủ rừng Việt Nam tăng từ 27.2\% (1990) lên 42.01\% (2020). Tốc độ phục hồi nhanh nhất trong giai đoạn 1995--2000 (tăng 5.5 điểm phần trăm) nhờ chính sách đóng cửa rừng tự nhiên và chương trình trồng rừng quy mô lớn, sau đó chậm dần khi quỹ đất phù hợp dần cạn kiệt. Tuy nhiên, phần lớn sự gia tăng này đến từ rừng trồng với giá trị sinh thái thấp hơn đáng kể so với rừng tự nhiên.


% 1.2. Viễn thám
\input{chapters/chapter1/section1-2.tex}

% 1.3. Học máy và học sâu
\input{chapters/chapter1/section1-3.tex}

% 1.4. Các nghiên cứu liên quan
\input{chapters/chapter1/section1-4.tex}