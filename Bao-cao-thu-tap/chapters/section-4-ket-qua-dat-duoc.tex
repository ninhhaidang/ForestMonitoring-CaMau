\unnumberedchapter{IV. KẾT QUẢ ĐẠT ĐƯỢC}

Qua quá trình thực tập 3 tháng tại Công ty TNHH Tư vấn và Phát triển Đồng Xanh, em đã đạt được các kết quả sau:

\section*{1. Nghiên cứu đề bài, thu thập tài liệu}

Em đã tiến hành nghiên cứu toàn diện về bài toán giám sát biến động rừng, bao gồm cơ sở lý thuyết về rừng và biến động rừng, công nghệ viễn thám, học sâu, và các nghiên cứu liên quan. Nội dung chi tiết được trình bày trong \textbf{Chương 1}.

Các nội dung chính đã nghiên cứu:
\begin{itemize}
    \item Tình hình mất rừng toàn cầu và Việt Nam, vai trò của rừng ngập mặn
    \item Nguyên lý viễn thám, vệ tinh Sentinel-1/2, các chỉ số thực vật
    \item Mạng nơ-ron tích chập (CNN), các kỹ thuật tối ưu và regularization
    \item Tổng quan các nghiên cứu về giám sát rừng sử dụng học sâu và dữ liệu viễn thám
\end{itemize}

\section*{2. Viết báo cáo chi tiết nội dung đã tiến hành trong quá trình thực tập}

Em đã thực hiện toàn bộ quy trình nghiên cứu từ thu thập dữ liệu, xây dựng phương pháp, đến triển khai và đánh giá mô hình. Nội dung chi tiết được trình bày trong \textbf{Chương 2} và \textbf{Chương 3}.

\textbf{Chương 2} trình bày về cơ sở dữ liệu và phương pháp nghiên cứu:
\begin{itemize}
    \item Thu thập và tiền xử lý dữ liệu Sentinel-1 SAR và Sentinel-2 quang học
    \item Xây dựng bộ dữ liệu mẫu với 2,630 điểm ground truth
    \item Trích xuất 27 đặc trưng từ hai nguồn dữ liệu đa nguồn
    \item Thiết kế kiến trúc mạng CNN với patch size 3×3
    \item Chiến lược huấn luyện với cross-validation 5-fold
\end{itemize}

\textbf{Chương 3} trình bày kết quả thực nghiệm:
\begin{itemize}
    \item Thiết lập môi trường thực nghiệm và siêu tham số
    \item Kết quả cross-validation: độ chính xác 98.48\% ± 0.36\%
    \item Kết quả kiểm tra cuối cùng: độ chính xác 98.86\%, ROC-AUC 99.98\%
    \item Phân loại toàn bộ vùng nghiên cứu 162,469 ha
    \item Phát hiện 7,282 ha mất rừng và 4,941 ha phục hồi rừng
    \item So sánh hiệu quả của dữ liệu đa nguồn (cải thiện 5.44\% so với chỉ dùng Sentinel-2)
\end{itemize}

% Subsection cho các sản phẩm
\section*{3. Các sản phẩm khác}

Ngoài báo cáo nghiên cứu chi tiết, quá trình thực tập đã tạo ra các sản phẩm cụ thể sau:

\subsection*{3.1. Mô hình học sâu CNN}

\begin{itemize}
    \item Kiến trúc: Mạng nơ-ron tích chập (CNN) với 36,676 tham số
    \item Input: Patch 3×3 pixel với 27 kênh đặc trưng
    \item Output: Phân loại 4 lớp (Rừng ổn định, Mất rừng, Phi rừng, Phục hồi rừng)
    \item Độ chính xác: 98.86\% trên tập kiểm tra
    \item ROC-AUC score: 99.98\%
    \item Cross-validation: 98.48\% ± 0.36\%
\end{itemize}

\subsection*{3.2. Bản đồ biến động rừng tỉnh Cà Mau}

\begin{itemize}
    \item Vùng nghiên cứu: Toàn bộ ranh giới lâm nghiệp tỉnh Cà Mau mới sau quyết định sáp nhập tỉnh Cà Mau với tỉnh Bạc Liêu cũ theo Nghị quyết số 1278/NQ-UBTVQH15 ngày 24/10/2024, có hiệu lực từ 01/07/2025 (162,469 ha)
    \item Thời kỳ phân tích: Tháng 1/2024 - Tháng 2/2025
    \item Độ phân giải không gian: 10m
    \item Kết quả phát hiện:
    \begin{itemize}
        \item Diện tích mất rừng: 7,282 ha (4.48\% tổng diện tích)
        \item Diện tích phục hồi rừng: 4,941 ha (3.04\% tổng diện tích)
        \item Rừng ổn định và khu vực không phải rừng
    \end{itemize}
    \item Format: GeoTIFF, hệ tọa độ EPSG:32648 (UTM Zone 48N)
\end{itemize}

\subsection*{3.3. Ứng dụng web hiển thị kết quả trên Google Earth Engine}

\begin{itemize}
    \item URL: \url{https://ee-bonglantrungmuoi.projects.earthengine.app/view/giam-sat-bien-dong-rung-ca-mau}
    \item Tính năng:
    \begin{itemize}
        \item Hiển thị bản đồ biến động rừng tương tác
        \item Phân biệt màu sắc cho 4 lớp biến động
        \item Cho phép người dùng phóng to/thu nhỏ, di chuyển bản đồ
        \item Hiển thị thông tin metadata và chú giải
    \end{itemize}
    \item Công nghệ: Google Earth Engine Apps, JavaScript API
    \item Truy cập: Công khai, không yêu cầu đăng nhập
\end{itemize}

\subsection*{3.4. Mã nguồn và tài liệu}

\begin{itemize}
    \item Mã nguồn tiền xử lý dữ liệu S1/S2 trên Google Earth Engine (JavaScript)
    \item Mã nguồn huấn luyện mô hình CNN (Python, PyTorch)
    \item Scripts phân loại và xuất kết quả
    \item Tài liệu hướng dẫn sử dụng
    \item Báo cáo kỹ thuật chi tiết (tài liệu này)
\end{itemize}


% Subsection cho khó khăn
\section*{4. Những khó khăn trong quá trình thực tập}

Trong quá trình thực hiện đề tài, em đã gặp một số khó khăn sau:

\textbf{Thứ nhất, về kiểm chứng kết quả phân loại.} Dữ liệu ground truth đã được Công ty GFD thu thập qua khảo sát drone và số hóa trên QGIS, tuy nhiên trong khuôn khổ thực tập chưa có cơ hội tổ chức thêm chuyến khảo sát thực địa để kiểm chứng kết quả phân loại trên toàn vùng nghiên cứu. Do đó, chưa có so sánh trực tiếp giữa bản đồ biến động do mô hình tạo ra với số liệu đo đạc tại hiện trường, dẫn đến hạn chế trong việc đánh giá độ tin cậy tuyệt đối của kết quả phân loại trong điều kiện thực tế.

\textbf{Thứ hai, về thời gian xử lý và tính toán.} Quá trình huấn luyện mô hình với nhiều thử nghiệm siêu tham số mất nhiều thời gian, trong khi dự đoán cho toàn bộ vùng nghiên cứu (162,469 ha) với độ phân giải 10m đòi hỏi tài nguyên tính toán lớn. Thời gian xử lý dữ liệu và chạy mô hình trên toàn vùng nghiên cứu kéo dài do giới hạn về phần cứng.

\textbf{Thứ ba, về hạn chế của dữ liệu.} Số lượng mẫu huấn luyện (2,630 điểm) tương đối nhỏ cho bài toán deep learning; chỉ phân tích 2 thời điểm (1/2024 và 2/2025), chưa xây dựng được chuỗi thời gian dài hạn; và dữ liệu Sentinel-2 bị ảnh hưởng bởi mây mù trong một số thời điểm.

\textbf{Thứ tư, về tính giải thích của mô hình.} Mô hình CNN là "black-box", khó giải thích cụ thể tại sao một pixel được phân loại vào lớp nào. Chưa có phân tích sâu về importance của từng đặc trưng trong quá trình phân loại, dẫn đến khó khăn trong việc truyền đạt kết quả cho những người không chuyên về deep learning.


% Subsection cho kiến nghị
\section*{5. Kiến nghị}

Dựa trên kết quả đạt được và những khó khăn gặp phải, em xin đề xuất một số kiến nghị cho các nghiên cứu tiếp theo:

\subsection*{5.1. Mở rộng nghiên cứu theo chuỗi thời gian}

\begin{itemize}
    \item Mở rộng thành chuỗi thời gian dài hạn với tần suất giám sát cao hơn (hàng tháng hoặc hàng quý)
    \item Phân tích xu hướng biến động rừng theo mùa và theo năm
    \item Xây dựng hệ thống cảnh báo sớm về mất rừng dựa trên phân tích chuỗi thời gian
    \item Áp dụng các mô hình time-series như LSTM, Transformer cho dự báo biến động rừng
\end{itemize}

\subsection*{5.2. Tăng cường kiểm chứng kết quả phân loại}

\begin{itemize}
    \item Tổ chức khảo sát thực địa để kiểm chứng độ chính xác của bản đồ biến động rừng do mô hình tạo ra
    \item Đo đạc các thông số sinh thái tại các khu vực được phân loại để xác nhận kết quả
    \item So sánh kết quả phân loại trên diện rộng với số liệu thực tế để đánh giá độ tin cậy
    \item Xây dựng quy trình kiểm chứng độc lập cho các sản phẩm giám sát rừng
\end{itemize}

\subsection*{5.3. Cải thiện hiệu năng mô hình}

\begin{itemize}
    \item Tối ưu hóa kiến trúc mô hình để giảm thời gian dự đoán
    \item Nghiên cứu các kỹ thuật model compression (pruning, quantization) để triển khai nhanh hơn
    \item Sử dụng GPU/TPU mạnh hơn hoặc phân tán tính toán
    \item Tăng kích thước bộ dữ liệu huấn luyện để cải thiện độ chính xác
\end{itemize}

\subsection*{5.4. Mở rộng vùng nghiên cứu}

\begin{itemize}
    \item Áp dụng mô hình cho các tỉnh ven biển khác có rừng ngập mặn (Kiên Giang, Sóc Trăng)
    \item So sánh đặc điểm biến động rừng giữa các vùng khác nhau
    \item Xây dựng bản đồ biến động rừng quy mô quốc gia
    \item Tích hợp với các hệ thống giám sát rừng hiện có của Bộ Nông nghiệp và Phát triển Nông thôn
\end{itemize}

\subsection*{5.5. Phát triển công cụ và ứng dụng}

\begin{itemize}
    \item Phát triển ứng dụng mobile cho công tác giám sát rừng tại hiện trường
    \item Xây dựng dashboard tương tác cho các nhà quản lý
    \item Tích hợp tính năng báo cáo tự động và xuất số liệu thống kê
    \item Kết nối với các hệ thống cảnh báo cháy rừng, thiên tai
\end{itemize}

