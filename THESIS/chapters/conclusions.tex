\phantomsection
\addcontentsline{toc}{chapter}{KẾT LUẬN VÀ KIẾN NGHỊ}
\unnumberedchapter{KẾT LUẬN VÀ KIẾN NGHỊ}

Đồ án đã hoàn thành các mục tiêu đề ra và đạt được các kết quả chính. Về xây dựng bộ dữ liệu, đồ án đã thu thập và tiền xử lý dữ liệu ảnh vệ tinh Sentinel-1 và Sentinel-2 từ hai thời kỳ (01/2024 và 02/2025) thông qua nền tảng Google Earth Engine. Tổng cộng 27 đặc trưng được trích xuất bao gồm các kênh phổ quang học (B4, B8, B11, B12), chỉ số thực vật (NDVI, NBR, NDMI) và dữ liệu ra-đa (VV, VH) cho mỗi thời kỳ cùng với giá trị delta. Bộ dữ liệu thực địa gồm 2,630 điểm được thu thập cho 4 lớp phân loại (Rừng ổn định, Mất rừng, Phi rừng, Phục hồi rừng) với phân bố cân bằng.

Về thiết kế và huấn luyện mô hình, đồ án đã xây dựng kiến trúc CNN với 36,676 tham số, sử dụng patches 3×3 để khai thác ngữ cảnh không gian, áp dụng các kỹ thuật điều chuẩn hiệu quả gồm chuẩn hóa theo lô, Dropout 70\% và phân rã trọng số. Phương pháp kiểm định chéo 5 phần cho kết quả CV accuracy 98.48\% ± 0.36\%, test accuracy đạt 98.86\% và ROC-AUC 99.98\%.

Về ứng dụng thực tế, mô hình đã được áp dụng để phân loại toàn vùng quy hoạch lâm nghiệp tỉnh Cà Mau (170,179 ha ranh giới, 162,469 ha phân loại thực tế), phát hiện 7,282 ha mất rừng (4.48\%) và 4,941 ha phục hồi rừng (3.04\%) trong giai đoạn nghiên cứu. Đồ án cũng xây dựng ứng dụng web trên nền tảng Google Earth Engine Apps để trực quan hóa kết quả, có thể truy cập công khai tại: \url{https://ee-bonglantrungmuoi.projects.earthengine.app/view/giam-sat-bien-dong-rung-ca-mau}. Giao diện ứng dụng được minh họa trong Hình \ref{fig:ee-app}.

\begin{figure}[H]
    \centering
    \includegraphics[width=\textwidth]{chapter3/ee-app.png}
    \caption{Giao diện ứng dụng web Google Earth Engine hiển thị kết quả phân loại biến động rừng tỉnh Cà Mau}
    \label{fig:ee-app}
\end{figure}

Đồ án đóng góp vào lĩnh vực giám sát rừng bằng viễn thám và học sâu trên nhiều phương diện. Về phương pháp luận, đồ án đề xuất quy trình tích hợp dữ liệu đa nguồn kết hợp dữ liệu ra-đa Sentinel-1 và quang học Sentinel-2, khai thác ưu điểm bổ sung của từng nguồn. Kết quả thực nghiệm chứng minh sự kết hợp này cải thiện accuracy 5.44\% so với chỉ sử dụng Sentinel-2 đơn lẻ (từ 93.42\% lên 98.86\%). Đồ án thiết kế kiến trúc CNN với 36,676 tham số phù hợp cho bộ dữ liệu nhỏ, tránh hiện tượng quá khớp. Thông qua nghiên cứu loại trừ có hệ thống, đồ án xác định kích thước patch 3×3 là tối ưu cho dữ liệu Sentinel 10m và đề xuất cấu trúc vector đặc trưng 27 chiều. Về ứng dụng thực tiễn, đồ án tạo ra bản đồ phân loại biến động rừng với độ chính xác cao (98.86\%) cho toàn bộ vùng nghiên cứu 162,469 ha, định lượng được 7,282 ha mất rừng và 4,941 ha phục hồi rừng. Quy trình được thiết kế module hóa có thể áp dụng cho các khu vực rừng ngập mặn khác. Mã nguồn được công bố trên Github tại \url{https://github.com/ninhhaidang}. Về kết quả khoa học, mô hình đạt hiệu suất vượt trội so với các nghiên cứu tương tự như Hansen và cs. (85\%), Hethcoat và cs. (94.3\%), Zhang và cs. (96.8\%). Phân tích cho thấy lỗi phân loại chủ yếu xảy ra giữa lớp ``Rừng ổn định'' và ``Mất rừng'' do sự tương đồng đặc trưng quang phổ tại vùng ranh giới. Kết quả nghiên cứu loại trừ khẳng định Sentinel-2 đóng góp chính trong khi Sentinel-1 có vai trò bổ sung quan trọng.

Bên cạnh những kết quả đạt được, đồ án vẫn tồn tại một số hạn chế cần lưu ý. Thứ nhất, thời gian dự đoán toàn bộ raster còn dài (khoảng 14.83 phút cho 16.2 triệu pixel), chưa đáp ứng được yêu cầu xử lý thời gian thực. Thứ hai, khả năng giải thích của mô hình hạn chế do tính chất "hộp đen" của CNN, khó xác định được các đặc trưng quan trọng nhất ảnh hưởng đến quyết định phân loại. Thứ ba, quy mô dữ liệu thực địa còn nhỏ với chỉ 2,630 điểm và chưa có khảo sát thực địa độc lập để kiểm chứng kết quả phân loại trên toàn vùng. Thứ tư, phân tích chỉ dừng lại ở hai thời điểm, chưa khai thác được chuỗi thời gian đầy đủ để phát hiện các biến động theo mùa hoặc xu hướng dài hạn.

Dựa trên kết quả và hạn chế của đồ án, một số hướng phát triển tiếp theo được đề xuất. Các nghiên cứu tiếp theo nên mở rộng phân tích đa thời gian bằng cách sử dụng chuỗi thời gian thay vì chỉ phân tích 2 thời điểm, đồng thời áp dụng các mô hình khác tiên tiến hơn để khai thác các đặc trưng thời gian. Để cải thiện mô hình, cần thử nghiệm cơ chế học chuyển giao từ các mô hình đã huấn luyện, và áp dụng các phương pháp kết hợp nhằm tăng độ chính xác và độ ổn định. Về ứng dụng thực tế, hướng phát triển bao gồm mở rộng phạm vi áp dụng sang các tỉnh trong vùng Đồng bằng sông Cửu Long, và tích hợp kết quả với hệ thống thông tin địa lý của cơ quan quản lý rừng. Ngoài ra, cần khảo sát thực địa để kiểm chứng kết quả, mở rộng bộ dữ liệu thực địa, và thu thập thêm dữ liệu đa thời gian để nâng cao khả năng khai thác chuỗi thời gian.
