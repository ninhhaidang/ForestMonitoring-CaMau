\begin{center}
\textbf{\large{Tóm tắt}}
\end{center}
\addcontentsline{toc}{chapter}{Tóm tắt}

\textbf{\textit{Tóm tắt:}} Đồ án này nghiên cứu ứng dụng mạng Neural Tích chập (Convolutional Neural Networks - CNN) kết hợp với dữ liệu viễn thám đa nguồn từ vệ tinh Sentinel-1 (SAR - Synthetic Aperture Radar) và Sentinel-2 (Optical Multispectral) để phát hiện và phân loại biến động rừng tại tỉnh Cà Mau (theo địa giới hành chính mới sau khi sáp nhập với tỉnh Bạc Liêu từ 01/07/2025) — khu vực rừng ngập mặn quan trọng bậc nhất tại Việt Nam. Nghiên cứu sử dụng 27 đặc trưng bao gồm các kênh phổ (Red, NIR, SWIR), các chỉ số thực vật (NDVI, EVI, NDWI, NDMI) và dữ liệu backscatter SAR (VV, VH) từ hai thời kỳ (01/2024 và 02/2025) với độ phân giải không gian 10m, cho phép phân tích biến động rừng trong khoảng thời gian 13 tháng. Kiến trúc CNN nhẹ (lightweight) với khoảng 36,000 tham số được thiết kế phù hợp cho bộ dữ liệu có quy mô vừa phải, sử dụng patches không gian 3×3 để khai thác ngữ cảnh không gian xung quanh mỗi pixel trung tâm, áp dụng các kỹ thuật regularization mạnh (Batch Normalization, Dropout 0.7, Weight Decay) để tránh overfitting trên bộ dữ liệu 2,630 điểm ground truth với 4 lớp phân loại (Rừng ổn định, Mất rừng, Phi rừng, Phục hồi rừng). Quy trình đánh giá khoa học chặt chẽ được triển khai với stratified random split kết hợp 5-Fold Stratified Cross Validation, đảm bảo phân bố cân bằng các lớp trong mỗi fold và giữ lại 20\% dữ liệu làm tập test cố định. Kết quả cho thấy mô hình đạt độ chính xác 98.86\% trên tập test với ROC-AUC 99.98\% cho thấy khả năng phân biệt xuất sắc giữa các lớp, Cross Validation cho kết quả trung bình 98.15\% ± 0.28\% với độ lệch chuẩn thấp chứng tỏ mô hình ổn định và có khả năng tổng quát hóa tốt. Ứng dụng thực tế phân loại toàn vùng quy hoạch lâm nghiệp Cà Mau với tổng diện tích ranh giới 170,179 ha, trong đó 162,469 ha được phân loại thực tế (tương đương 16.2 triệu pixel), cho thấy 74.30\% diện tích là rừng ổn định, phát hiện 7,282 ha mất rừng (4.48\%) và 4,941 ha phục hồi rừng (3.04\%) trong giai đoạn nghiên cứu, đóng góp một phương pháp tiếp cận hiệu quả cho công tác giám sát và quản lý tài nguyên rừng.

\textbf{\textit{Từ khóa:}} CNN, Deep Learning, viễn thám, Sentinel-1, Sentinel-2, biến động rừng, rừng ngập mặn, Cà Mau
