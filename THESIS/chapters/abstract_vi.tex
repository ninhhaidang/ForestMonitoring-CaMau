\phantomsection
\addcontentsline{toc}{chapter}{TÓM TẮT}
\unnumberedchapter{TÓM TẮT}

{\fontsize{12pt}{14pt}\selectfont
\textbf{Tóm tắt:} Đồ án này nghiên cứu ứng dụng mạng Neural Tích chập (CNN) kết hợp với dữ liệu viễn thám đa nguồn từ vệ tinh Sentinel-1 và Sentinel-2 để phát hiện và phân loại biến động rừng tại tỉnh Cà Mau — khu vực rừng ngập mặn quan trọng bậc nhất tại Việt Nam. Nghiên cứu sử dụng 27 đặc trưng bao gồm các kênh phổ, chỉ số thực vật (NDVI, NBR, NDMI) và dữ liệu SAR từ hai thời kỳ (01/2024 và 02/2025) với độ phân giải 10m.

Kiến trúc CNN nhẹ với khoảng 36,000 tham số được thiết kế, sử dụng patches 3×3 để khai thác ngữ cảnh không gian, áp dụng các kỹ thuật regularization (Batch Normalization, Dropout, Weight Decay) để tránh overfitting trên bộ dữ liệu thực địa gồm 2,630 điểm với 4 lớp phân loại: Rừng ổn định, Mất rừng, Phi rừng và Phục hồi rừng. Quy trình đánh giá triển khai với stratified random split kết hợp 5-Fold Cross Validation, đảm bảo phân bố cân bằng các lớp và giữ lại 20\% dữ liệu làm tập test cố định.

Kết quả cho thấy mô hình đạt độ chính xác 98.86\% trên tập test với ROC-AUC 99.98\%, Cross Validation cho kết quả trung bình 98.15\% ± 0.28\% chứng tỏ mô hình ổn định và có khả năng tổng quát hóa tốt. Ứng dụng thực tế phân loại toàn vùng quy hoạch lâm nghiệp Cà Mau với diện tích 170,179 ha, phát hiện 7,282 ha mất rừng (4.48\%) và 4,941 ha phục hồi rừng (3.04\%) trong giai đoạn nghiên cứu, đóng góp phương pháp hiệu quả cho công tác giám sát tài nguyên rừng.

\textbf{\textit{Từ khóa:}} CNN, viễn thám, biến động rừng, Cà Mau.
}
