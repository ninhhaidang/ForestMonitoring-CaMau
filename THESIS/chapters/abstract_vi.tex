\phantomsection
\addcontentsline{toc}{chapter}{TÓM TẮT}
\unnumberedchapter{TÓM TẮT}

{\fontsize{12pt}{14pt}\selectfont
\textbf{Tóm tắt:} Đồ án này nghiên cứu ứng dụng mạng nơ-ron tích chập kết hợp dữ liệu viễn thám đa nguồn từ vệ tinh Sentinel-1 và Sentinel-2 để phát hiện và phân loại biến động rừng tại khu vực quy hoạch lâm nghiệp tỉnh Cà Mau. Nghiên cứu sử dụng dữ liệu ảnh vệ tinh từ hai thời kỳ (01/2024 và 02/2025) với độ phân giải 10m, trích xuất 27 đặc trưng bao gồm các kênh phổ quang học, chỉ số thực vật (NDVI, NBR, NDMI) và dữ liệu ra-đa khẩu độ tổng hợp. Kiến trúc CNN với khoảng 36,000 tham số được thiết kế, sử dụng patches 3×3 để khai thác ngữ cảnh không gian, áp dụng các kỹ thuật điều chuẩn gồm chuẩn hóa theo lô, Dropout và phân rã trọng số để tránh quá khớp. Bộ dữ liệu thực địa gồm 2,630 điểm với 4 lớp phân loại (Rừng ổn định, Mất rừng, Phi rừng, Phục hồi rừng) được phân chia theo tỷ lệ 80-20 kết hợp kiểm định chéo 5 phần. Kết quả thực nghiệm cho thấy mô hình đạt độ chính xác 98.86\% trên tập kiểm tra với ROC-AUC 99.98\%, kiểm định chéo cho kết quả trung bình 98.48\% ± 0.36\% chứng tỏ mô hình ổn định và có khả năng tổng quát hóa tốt. Ứng dụng thực tế phân loại toàn vùng (170,179 ha ranh giới, 162,469 ha phân loại thực tế), phát hiện 7,282 ha mất rừng (4.48\%) và 4,941 ha phục hồi rừng (3.04\%) trong giai đoạn nghiên cứu, đóng góp phương pháp hiệu quả cho công tác giám sát tài nguyên rừng.

\textbf{\textit{Từ khóa:}} CNN, viễn thám, biến động rừng, Cà Mau.
}
