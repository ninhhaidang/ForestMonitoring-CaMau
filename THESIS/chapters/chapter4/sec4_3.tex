\section{Phân tích đặc tính dòng chảy tức thời xung quanh rô-to}

Trong phần này, phân bố áp suất, vận tốc, cường độ xoáy và cường độ rối xung quanh hai biên dạng cánh của tua-bin gió Savonius được phân tích để kiểm chứng sự cải thiện về công suất của biên dạng mới. Để thuận tiện, trong các hình ảnh phân bố giá trị, các vị trí đặc biệt được lưu ý như sau:
\begin{itemize}
    \item Cánh tiến là cánh đang di chuyển cùng chiều với hướng gió, nhận được nhiều lực đẩy hơn và đóng góp chính vào việc tạo ra mô-men xoắn.
    \item Cánh lùi là cánh di chuyển ngược lại với hướng gió, nhận ít lực đẩy hơn và có thể tạo ra lực cản.
    \item Do đặc tính đối xứng của thiết kế tua-bin Savonius, từng cánh đơn (Cánh 1 và Cánh 2) đều trải qua cả vùng chủ động, đóng vai trò cánh tiến và vùng bị động, đóng vai trò cánh lùi trong quá trình quay.
\end{itemize}

Đầu tiên, Hình \ref{fig:pressure05} thể hiện phân bố áp suất xung quanh và trên từng cánh tại tỉ tốc 0.5 của hai biên dạng tua-bin được khảo sát. Các dữ liệu ở góc quay $0^\circ$, $60^\circ$ và $150^\circ$ được phân tích để thấy rõ sự chênh lệch trong mô-men được tạo ra tại các vị trí góc này như quan sát được trong Hình \ref{fig:Ct360}a.

Quan sát tổng quan tại góc quay $\theta = 0^\circ$ để hiểu rõ nguyên lý chuyển động quay của tua-bin gió Savonius. Khi Cánh 1 của cả hai biên dạng đều đang ở vùng chủ động, vị trí thuận lợi đón hướng gió, đóng vai trò cánh tiến của tua-bin, vùng áp suất thấp (AST) có giá trị âm (màu xanh) xuất hiện ở mặt lồi đầu cánh và vùng áp suất cao (ASC) có giá trị dương ở mặt lõm (màu đỏ). Sự chênh lệch áp suất ở mặt lõm lớn hơn mặt lồi tạo ra lực đẩy có lợi, giúp tua-bin sinh mô-men xoắn dương có lợi và tạo ra chuyển động quay. Chênh lệch áp suất dương càng lớn thì mô-men có lợi càng lớn và giúp tua-bin tạo ra hiệu suất lớn hơn. Ngược lại, tại Cánh 2 đang đóng vai trò cánh lùi, vùng áp suất cao (ASC) xuất hiện ở giữa mặt lồi của cánh và tạo ra chênh lệch áp suất âm với mặt lõm đang có áp suất thấp hơn. Điều này tạo ra lực đẩy bất lợi, chống lại chuyển động quay của tua-bin.

Xét về giá trị trong biểu đồ phân bố trên cánh để thấy được sự chênh lệch áp suất khác nhau giữa hai cấu hình Ov01 và Fx. Nhìn chung, tại góc quay $0^\circ$, sự chênh lệch áp suất có lợi được tạo ra nhiều hơn bởi biên dạng Fx so với Ov01, được quan sát rõ ở vùng chồng cánh (-0.025 $\le x \le$ 0.025) và đầu Cánh 2. Mặc dù tạo ra sự chênh lệch áp suất dương lớn ở các vị trí khác nhau trên Cánh 1 nhưng về trung bình, giá trị mô-men do cánh này tạo ra ở cả 2 biên dạng là khá tương đồng. Ngược lại, tại vùng chồng cánh và đầu Cánh 2, sự chênh lệch áp suất có lợi giữa 2 mặt trên tua-bin Fx xuất hiện nhiều hơn Ov01. Điều này tạo nên sự ưu thế về mô-men của biên Fx so với Ov01 tại góc quay này như trong Hình \ref{fig:Ct360}a.

Phân tích tương tự đối với góc quay $60^\circ$ (Hình \ref{fig:pressure05}b), lúc này, cả 2 cánh của 2 tua-bin nằm giao thoa giữa hai vùng chủ động và bị động so với hướng đón gió. Vùng áp suất thấp xuất hiện ở đầu Cánh 1 tại góc $0^\circ$ đã tách khỏi rô-to, di chuyển về phía hạ lưu khi cánh này về phía vùng bị động. Đồng thời, vùng áp suất cao vốn nằm ở giữa Cánh 2 đã dịch chuyển lên đầu cánh. Quan sát biểu đồ phân bố, mặc dù vùng áp suất cao ở mặt lõm của Cánh 2 biên dạng Fx có sự biến động nhẹ ở gần tâm quay rô-to nhưng không tạo ra chênh lệch áp suất trung bình quá khác biệt với Ov01. Tuy nhiên, hiện tượng ngược lại có thể quan sát được ở Cánh 1 khi áp suất dương có lợi được tạo ra lớn hơn trên biên dạng Ov01. Nhờ vậy, mô-men xoắn được tạo ra do tua-bin Ov01 vẫn cao hơn tua-bin Fx.

Cuối cùng, tại góc quay $\theta = 150^\circ$, Cánh 2 của hai tua-bin đã nằm trong vùng chủ động và đóng vai trò cánh tiến. Lúc này, vùng áp suất âm bắt đầu hình thành trên đầu cánh tiến tương tự như ở góc quay $0^\circ$ nhưng với diện tích nhỏ hơn. Đồng thời, Cánh 1 ở vị trí cánh lùi, chịu tác dụng của vùng áp suất cao. Các kết quả tính toán cho thấy biên dạng Fx tạo ra sự chênh lệch đáng kể giữa áp suất mặt lõm, lồi ở đầu vùng đầu Cánh 1 và vùng chồng cánh so với biên dạng Ov01. Đây cũng là nguyên nhân tạo ra giá trị mô-men ưu thế của biên dạng Fx tại góc quay này như trong Hình \ref{fig:Ct360}a.

\begin{figure}[H]
    \centering
    \includegraphics[width=1\linewidth]{img/chapter4/0.5 Pressure.png}
    \caption{Phân bố áp suất xung quanh và trên cánh của hai biên dạng ở tỉ tốc $\lambda = 0.5$ tại các góc quay: (a) $\theta = 0^\circ$; (b) $\theta = 60^\circ$; (c) $\theta =150^\circ$.}
    \label{fig:pressure05}
\end{figure}

Lưu ý rằng, như đã phân tích ở phần 4.2, khoảng góc quay tạo ra mô-men ưu thế của biên dạng Fx trong một vòng quay vẫn còn hạn chế hơn biên dạng Ov01. Vì vậy, hiệu suất trung bình của tua-bin Fx vẫn chưa có sự cải thiện so với tua-bin Ov01 tải tỉ tốc 0.5.

\begin{figure}[H]
    \centering
    \includegraphics[width=1\linewidth]{img/chapter4/1.4 Pressure.png}
    \caption{Phân bố áp suất xung quanh và trên cánh của hai biên dạng ở tỉ tốc $\lambda = 1.4$ tại các góc quay: (a) $\theta = 0^\circ$; (b) $\theta = 60^\circ$; (c) $\theta =150^\circ$.}
    \label{fig:pressure14}
\end{figure}

Tiếp theo, xem xét tại tỉ tốc 1.4, phân bố áp suất trong Hình \ref{fig:pressure14} tiếp tục được phân tích để làm rõ sự cải thiện đáng chú ý của biên dạng Fx so với Ov01. Về tổng quan, phân bố áp suất trên mặt lồi cánh của cả hai biên dạng duy trì sự tương đồng ở cả ba góc quay. Cụ thể, vùng áp suất thấp xuất hiện ở mặt lồi đầu Cánh 1 tại góc quay $0^\circ$, sau đó, tách ra khỏi tua-bin và di chuyển xuống hạ lưu khi tua-bin quay đến góc $60^\circ$. Ngoài ra, vị trí xuất hiện của vùng áp suất cao giữa mặt lồi khi các cánh ở vùng bị động và vùng áp suất thấp gần tâm quay rô-to cũng có sự tương tự giữa hai cấu hình. Điều này có thể được lý giải là do thiết kế cánh Fx mới vẫn giữ nguyên mặt lồi là hình dạng bán nguyệt của biên dạng Ov01, dẫn đến sự giống nhau về đặc tính dòng chảy tác dụng lên vị trí này.

Điểm đáng chú ý tạo ra sự chênh lệch mô-men xoắn tại tỉ tốc này của biên dạng Fx so với biên dạng Ov01 nằm ở vùng chồng cánh và mặt lõm các cánh. Quan sát biểu đồ phân bố áp suất tại góc quay $0^\circ$, cả hai biên dạng đều có khoảng chênh lệch áp suất âm khá lớn khi áp suất mặt lồi lớn hơn mặt lõm ở Cánh 2. Đặc biệt, khoảng áp suất âm trên Cánh 1 của biên dạng Fx còn lớn hơn so với Ov01. Điều này dẫn đến giá trị mô-men âm được tạo ra bởi hai biên dạng tại góc quay $0^\circ$, trong đó, biên dạng Fx đang tạo ra mô-men bất lợi lớn hơn. Đối với góc quay $150^\circ$, lúc này Cánh 2 của cả hai biên dạng đã bước vào vùng chủ động, đóng vai trò cánh tiến có sự chênh lệch áp suất dương lớn để tạo ra mô-men xoắn dương và hỗ trợ chuyển động quay của tua-bin. Mặc dù chênh lệch áp suất dương trên cánh tiến của biên dạng Fx vẫn bé hơn biên dạng Ov01 nhưng sự chênh lệch lớn tại vùng chồng cánh đã giúp biên dạng Fx tạo ra mô-men xoắn lớn hơn tại góc quay này. Hiện tượng tương tự có thể quan sát được ở góc quay $60^\circ$. Cụ thể, tại vùng chồng cánh, biên dạng Ov01 không cho thấy sự chênh lệch đáng kể nào giữa hai mặt của Cánh 1 trong khi sự chênh lệch lớn được tạo ra ở cùng vị trí tại Cánh 1 của biên dạng Fx, thậm chí cả ở Cánh 2 ở góc $150^\circ$. Nhờ vậy, biên dạng Fx có thể tạo ra giá trị hệ số mô-men xoắn lớn hơn biên dạng Ov01 tại các góc quay này.

Ngoài ra, sự cải thiện về mô-men của biên dạng Fx được duy trì trong hai khoảng góc lớn trên một vòng quay. Điều này dẫn đến hệ số công suất của biên dạng này được nâng lên đáng kể so với biên dạng Ov01 tại tỉ tốc 1.4.

Để làm rõ hơn về đặc điểm trong thiết kế của biên dạng Fx đã tạo ra sự cải tiến về hiệu suất tua-bin gió Savonius, phân bố vận tốc của dòng chảy khi đi qua hai rô-to được thể hiện và phân tích trong Hình \ref{fig:velocity05}, \ref{fig:velocity14}.

Hình \ref{fig:velocity05} so sánh phân bố vận tốc xung quanh rô-to dạng Ov01 và Fx ở tỉ tốc gió $\lambda = 0.5$ tại ba vị trí góc $0^\circ$, $60^\circ$ và $150^\circ$. Thông thường, dòng chảy xung quanh tua-bin thể hiện những đường dòng phức tạp, bao gồm các dòng chảy khác nhau như xoáy đầu cánh, dòng xoáy trong cánh, dòng chảy ngược, dòng phục hồi, dòng qua khe và tách dòng như mô tả trong hình. Tại đây, lý thuyết Bernoulli được sử dụng để phân tích mối liên hệ giữa vận tốc và áp suất xung quanh rô-to.

\begin{figure}[H]
    \centering
    \includegraphics[width=1\linewidth]{img/chapter4/0.5 Velocity.png}
    \caption{Phân bố vận tốc của hai biên dạng ở tỉ tốc $\lambda = 0.5$ tại các góc quay: (a) $\theta = 0^\circ$; (b) $\theta = 60^\circ$; (c) $\theta =150^\circ$.}
    \label{fig:velocity05}
\end{figure}

Về tổng quan, tại tỉ tốc 0.5, cả hai biên dạng cho thấy sự tương đồng về đặc tính dòng chảy tại phần lớn các vị trí. Điểm tạo ra sự khác biệt đối với hiệu suất khí động xảy ra khi dòng chảy đi qua vùng chồng cảnh của hai biên dạng.

Quan sát tại góc quay $0^\circ$ cho thấy tổng quan hiện tượng dòng chảy khi đi qua cả 2 rô-to. Cụ thể, dòng chảy đi vào Cánh 1 của cả hai biên dạng, tạo ra vùng vận tốc cao (VTC) ở đầu cánh khi dòng chảy va chạm với bề mặt cánh có tiết diện nhỏ. Đây là hiện tượng hình thành xoáy đầu cánh và tạo ra vùng áp suất thấp (AST) như quan sát được trong Hình \ref{fig:pressure05}a. Ở mặt lồi, dòng phục hồi xuất hiện và di chuyển với tốc độ thấp, khiến cho áp suất mặt lồi tăng dần lên. Trong khi đó, tại mặt lõm, dòng chảy tác động trực diện với vận tốc thấp (VTT) và tạo vùng áp suất cao. Ngược lại, đối với Cánh 2, mặt lồi va chạm trực tiếp với dòng chảy vận tốc thấp, dẫn đến hình thành dòng chảy tách ra khỏi bề mặt cánh và di chuyển về phía đầu cánh với vận tốc cao hơn, gọi là dòng tách thành. Điều này tạo ra sự thay đổi độ lớn áp suất, cao từ giữa cánh tới thấp ở hai đầu cánh, trên mặt lồi Cánh 2. 

Điểm đặc biệt của việc sử dụng biên dạng cánh chồng là khu vực chồng cánh được tạo ra như một khe dẫn dòng từ mặt lõm Cánh 1 sang Cánh 2. Sự khác biệt trong thiết kế của biên dạng Fx đã khiến dòng chảy đi qua khu vực này có sự khác biệt đối với biên dạng Ov01. Cụ thể, đặc điểm đa độ dày của airfoil FX74-CL5-140 và sự sắp đặt điểm đầu của airfoil về phía trục rô-to đã tạo ra khe chồng cánh của biên dạng Fx có độ rộng hẹp hơn so với biên dạng Ov01. Khe hẹp này giúp tăng tốc dòng chảy từ mặt lõm cánh tiến sang mặt lõm cánh lùi, gọi là dòng qua khe. Đối với góc quay $0^\circ$ và $150^\circ$, dòng vận tốc cao sau khi ra khỏi khe chồng cánh nhanh chóng tách khỏi bề mặt mặt lõm cánh lùi để di chuyển về phía mặt lồi của cánh tiến, tạo ra vùng áp suất thấp tại đây và làm tăng chênh lệch áp suất dương ở cánh tiến, cải thiện hiệu suất của biên dạng Fx so với biên dạng Ov01. Ngược lại, tại các góc quay bất lợi với hướng gió như góc $60^\circ$, dòng qua khe này chưa tạo ra tác động tích cực khi nó di chuyển bám sát bề mặt mặt lõm của cánh lùi, tách thành chậm hơn và làm giảm áp suất tại mặt này của cánh.

Đối với vùng tỉ tốc cao như 1.4, quan sát trong Hình \ref{fig:velocity14}, dòng qua khe hình thành tại vùng chồng cánh lại tạo ra lợi thế cho tua-bin cải thiện hiệu suất khí động. Tại đây, dòng chảy qua khe hở hẹp giữa hai cánh của biên dạng Fx với vận tốc cao hơn so với dòng chảy tại vùng chồng cánh rộng của biên dạng Ov01. Do rô-to quay với vận tốc góc lớn kết hợp với độ cong tự nhiên của mặt hút airfoil, dòng qua khe này chỉ duy trì trên bề mặt lõm cánh lùi Fx một thời gian ngắn, sau đó tách ra khỏi bề mặt cánh và tiếp tục tác dụng lên đầu cánh, khiến cho vùng này có áp suất cao hơn so với đầu cánh Ov01. Dòng qua khe này đồng thời cũng làm gia tăng sự chênh lệch áp suất có lợi cho chuyển động quay của rô-to tại vùng chồng cánh Fx. Trong khi đó, biên dạng Ov01 với khe chồng cánh rộng hơn đã tạo điều kiện cho sự lưu thông xoáy giữa hai mặt của cánh và không tạo ra sự chênh lệch đáng kể giữa 2 mặt cánh tiến ở vị trí này. Điều này sẽ được phân tích chi tiết hơn khi xem xét Hình \ref{fig:vorOv01}.

\begin{figure}[H]
    \centering
    \includegraphics[width=1\linewidth]{img/chapter4/1.4 Velocity.png}
    \caption{Phân bố vận tốc của hai biên dạng ở tỉ tốc $\lambda = 1.4$ tại các góc quay: (a) $\theta = 0^\circ$; (b) $\theta = 60^\circ$; (c) $\theta =150^\circ$.}
    \label{fig:velocity14}
\end{figure}

Hình \ref{fig:vorOv01} cho thấy hiện tượng dòng chảy qua biên dạng Ov01 tại các góc quay $0^\circ$, $60^\circ$, và $150^\circ$, dựa trên phân bố cường độ xoáy (vorticity magnitude) và cường độ rối (turbulent intensity). Tại góc $0^\circ$, dòng chảy va chạm trực diện vào Cánh 1 của rô-to, tạo ra một vùng xoáy mạnh ở đầu cánh. Cường độ xoáy cao xuất hiện tại đầu Cánh 1, cho thấy sự hình thành của xoáy lớn. Khi góc quay tăng lên, xoáy này lan toả vào dòng chảy sau cánh (wake region), tạo ra sự bất ổn định cho dòng chảy, như quan sát thấy cường độ rối tăng cao quanh đầu Cánh 1 và lan rộng về phía sau. Ở góc $60^\circ$, dòng chảy có xu hướng đi qua vùng chồng cánh giữa Cánh 1 và Cánh 2. Lúc này, sự hình thành xoáy diễn ra phức tạp hơn, với cường độ xoáy cao xuất hiện tại đây. Tuy nhiên, mức độ xoáy ở đầu Cánh 1 giảm dần so với góc $0^\circ$ do xoáy đầu cánh bắt đầu tách ra khỏi bề mặt cánh và di chuyển xuống hạ lưu, trong khi Cánh 2 bắt đầu tạo ra xoáy mới ở phía sau. Tại góc này, cường độ rối phân bố rộng hơn so với góc $0^\circ$, đặc biệt là ở vùng chồng cánh, cho thấy sự tương tác phức tạp giữa các xoáy từ hai cánh. Khi rô-to quay đến góc $150^\circ$, xoáy đầu Cánh 1 đã hoàn toàn tách khỏi bề mặt cánh và dòng chảy tiếp tục tương tác trực tiếp với cánh tiến lúc này là Cánh 2. Lúc này, hai vùng xoáy nhỏ nằm ở 2 mặt của cánh tiến được thể hiện rõ và cường độ rối vẫn duy trì ở mức cao quanh vùng chồng cánh và lan rộng về phía cánh lùi. Điều này cho thấy dòng chảy tiếp tục chịu ảnh hưởng của các xoáy mạnh.

\begin{figure}[H]
    \centering
    \includegraphics[width=1\linewidth]{img/chapter4/1.4 Ov01 vor+ti.png}
    \caption{Phân bố cường độ xoáy và cường độ rối của biên dạng Ov01 ở tỉ tốc $\lambda = 1.4$ tại các góc quay: (a) $\theta = 0^\circ$; (b) $\theta = 60^\circ$; (c) $\theta =150^\circ$.}
    \label{fig:vorOv01}
\end{figure}

Tương tự, trích xuất dữ liệu của cường độ xoáy và cường độ rối với biên dạng Fx tại các góc quay tương ứng được thể hiện trong Hình \ref{fig:vorFx}. Tại góc quay $0^\circ$, cánh Fx cũng tạo ra vùng xoáy mạnh tập trung ở đầu Cánh 1 khá tương đồng với biên dạng Ov01. Cường độ rối trong cánh này có diện tích nhỏ hơn so với Ov01, do đặc điểm độ dày của cánh Fx. Biên dạng Fx thể hiện khả năng duy trì dòng chảy ổn định với cường độ xoáy thấp hơn so với Ov01 khi đi qua vùng chồng cánh, cho thấy sự tối ưu hoá trong việc giảm tổn thất động năng dòng chảy. Kích thước và sự lan rộng của xoáy trong vùng chồng cánh nhỏ hơn đáng kể, tạo điều kiện thuận lợi cho dòng chảy mượt mà qua khu vực này. Cường độ rối ở vùng chồng cánh thấp hơn so với Ov01, thể hiện khả năng giảm nhiễu loạn của thiết kế Fx. Dòng chảy không bị đứt gãy khi qua vùng chồng cánh, mà vẫn duy trì được sự ổn định, đảm bảo lực khí động được tối ưu hoá từ cánh tiến sang cánh lùi. Tại góc quay $60^\circ$, xoáy cường độ cao ở đầu cánh tiến đã cơ bản tách ra khỏi bề mặt cánh và di chuyển xuống hạ lưu. Tại vùng chồng cánh, xoáy hình thành do sự tách thành của dòng chảy trên Cánh 2 đã tác động sớm lên bề mặt Cánh 1 do độ dày lớn tại vị trí gần tâm rô-to và làm giảm đáng kể áp suất trên mặt lõm và gia tăng chênh lệch áp suất có lợi. Đồng thời, nhờ khe chồng cánh hẹp, dòng chảy được lưu thông qua mặt lõm Cánh 2 vẫn duy trì động lượng lớn và tiếp tục tác động lên bề mặt cánh để giảm hiệu ứng tiêu cực của chênh lệch áp suất âm. Ngoài ra, cường độ rối thấp hơn rõ rệt, đặc biệt trong giai đoạn dòng chảy chuyển tiếp từ cánh tiến sang cánh lùi. Dòng chảy qua cánh lùi được tái gắn hiệu quả, đảm bảo lực khí động không bị suy giảm. Điều này giúp biên dạng Fx duy trì hiệu suất cao ngay cả ở giai đoạn trung gian của chu kỳ quay. Ở góc quay $150^\circ$,  dòng chảy qua vùng chồng cánh được duy trì ổn định, không xuất hiện các xoáy lớn gây nhiễu loạn như ở Ov01. Ở vùng chồng cánh, cường độ rối vẫn được kiểm soát tốt, dù dòng chảy đã trải qua một góc quay lớn. Điều này cho thấy thiết kế Fx có khả năng duy trì hiệu quả khí động học trong suốt chu kỳ. Dòng chảy tách ra từ cánh lùi có cường độ rối thấp hơn, không bị nhiễu loạn mạnh, giúp ổn định dòng chảy ở phía sau tua-bin và giảm tổn thất năng lượng.

\begin{figure}[H]
    \centering
    \includegraphics[width=1\linewidth]{img/chapter4/1.4 fx vor+ti.png}
    \caption{Phân bố cường độ xoáy và cường độ rối của biên dạng Fx ở tỉ tốc $\lambda = 1.4$ tại các góc quay: (a) $\theta = 0^\circ$; (b) $\theta = 60^\circ$; (c) $\theta =150^\circ$.}
    \label{fig:vorFx}
\end{figure}


