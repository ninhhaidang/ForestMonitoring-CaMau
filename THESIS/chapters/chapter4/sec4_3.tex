\section{Kết quả phân loại toàn bộ vùng nghiên cứu}

\subsection{Thống kê phân loại}

\textbf{Phân bố diện tích theo lớp:}

\begin{table}[H]
\centering
\caption{Phân bố diện tích theo lớp phân loại}
\label{tab:area_distribution}
\begin{tabular}{|c|l|r|r|r|r|}
\hline
\textbf{Lớp} & \textbf{Tên lớp} & \textbf{Số pixels} & \textbf{Tỷ lệ (\%)} & \textbf{Diện tích (ha)} & \textbf{Diện tích (km²)} \\
\hline
0 & Rừng ổn định & 12,071,691 & 74.30\% & 120,716.91 & 1,207.17 \\
\hline
1 & Mất rừng & 728,215 & 4.48\% & 7,282.15 & 72.82 \\
\hline
2 & Phi rừng & 2,952,854 & 18.17\% & 29,528.54 & 295.29 \\
\hline
3 & Phục hồi rừng & 494,090 & 3.04\% & 4,940.90 & 49.41 \\
\hline
\textbf{Tổng} & & \textbf{16,246,850} & \textbf{100\%} & \textbf{162,468.50} & \textbf{1,624.69} \\
\hline
\end{tabular}
\end{table}

\begin{figure}[H]
    \centering
    \includegraphics[width=0.95\textwidth]{chapter4/Classification.png}
    \caption{Bản đồ phân loại biến động rừng tỉnh Cà Mau}
    \label{fig:classification_map}
\end{figure}

\begin{figure}[H]
    \centering
    \definecolor{stableforest}{HTML}{00734C}
    \definecolor{deforestation}{HTML}{E60000}
    \definecolor{nonforest}{HTML}{FFD37F}
    \definecolor{reforestation}{HTML}{00C5FF}
    \begin{tikzpicture}
        \pie[
            radius=3,
            text=legend,
            color={stableforest, deforestation, nonforest, reforestation},
            explode={0, 0.1, 0, 0}
        ]{
            74.30/Rừng ổn định (74.30\%),
            4.48/Mất rừng (4.48\%),
            18.17/Phi rừng (18.17\%),
            3.04/Phục hồi rừng (3.04\%)
        }
    \end{tikzpicture}
    \caption{Tỷ lệ diện tích các lớp phân loại}
    \label{fig:pie_chart}
\end{figure}

\begin{figure}[H]
    \centering
    \definecolor{stableforest}{HTML}{00734C}
    \definecolor{deforestation}{HTML}{E60000}
    \definecolor{nonforest}{HTML}{FFD37F}
    \definecolor{reforestation}{HTML}{00C5FF}
    \begin{tikzpicture}
        \begin{axis}[
            ybar,
            width=0.85\textwidth,
            height=8cm,
            ylabel={Diện tích (ha)},
            symbolic x coords={Rừng ổn định, Mất rừng, Phi rừng, Phục hồi rừng},
            xtick=data,
            xticklabel style={rotate=15, anchor=east, font=\small},
            nodes near coords,
            nodes near coords align={vertical},
            every node near coord/.append style={font=\scriptsize},
            ymin=0,
            ymax=140000,
            bar width=25pt,
            enlarge x limits=0.2,
        ]
        \addplot[fill=stableforest] coordinates {(Rừng ổn định, 120717)};
        \addplot[fill=deforestation] coordinates {(Mất rừng, 7282)};
        \addplot[fill=nonforest] coordinates {(Phi rừng, 29529)};
        \addplot[fill=reforestation] coordinates {(Phục hồi rừng, 4941)};
        \end{axis}
    \end{tikzpicture}
    \caption{Biểu đồ cột phân bố diện tích theo lớp phân loại}
    \label{fig:area_bar_chart}
\end{figure}

\textbf{Lưu ý về ước tính diện tích:} Theo khuyến nghị của Olofsson và cộng sự \cite{olofsson2014}, diện tích ước tính từ bản đồ phân loại cần được hiệu chỉnh dựa trên Confusion Matrix để đảm bảo tính không chệch. Với Accuracy cao của mô hình (98.86\%, Precision và Recall đều trên 96\% cho tất cả các lớp), sai số giữa diện tích thô và diện tích hiệu chỉnh được kỳ vọng là nhỏ. Tuy nhiên, việc thực hiện hiệu chỉnh đầy đủ theo phương pháp Olofsson sẽ là hướng phát triển trong tương lai.