\section{Kết quả phân loại toàn bộ vùng nghiên cứu}

\subsection{Thống kê phân loại}

\begin{table}[H]
\centering
\caption{Thống kê phân loại full raster}
\label{tab:raster_stats}
\begin{tabular}{|l|r|}
\hline
\textbf{Thông số} & \textbf{Giá trị} \\
\hline
Tổng số pixels được xử lý & 136,975,599 pixels \\
\hline
Pixels hợp lệ (valid data) & 16,246,850 pixels (11.86\%) \\
\hline
Pixels bị mask (nodata) & 120,728,749 pixels (88.14\%) \\
\hline
Kích thước raster & 12,547 × 10,917 pixels \\
\hline
Độ phân giải & 10m × 10m \\
\hline
Hệ tọa độ & EPSG:32648 (UTM Zone 48N) \\
\hline
\end{tabular}
\end{table}

\textbf{Phân bố diện tích theo lớp:}

\begin{table}[H]
\centering
\caption{Phân bố diện tích theo lớp phân loại}
\label{tab:area_distribution}
\begin{tabular}{|c|l|r|r|r|r|}
\hline
\textbf{Lớp} & \textbf{Tên lớp} & \textbf{Số pixels} & \textbf{Tỷ lệ (\%)} & \textbf{Diện tích (ha)} & \textbf{Diện tích (km²)} \\
\hline
0 & Rừng ổn định & 12,071,691 & 74.30\% & 120,716.91 & 1,207.17 \\
\hline
1 & Mất rừng & 728,215 & 4.48\% & 7,282.15 & 72.82 \\
\hline
2 & Phi rừng & 2,952,854 & 18.17\% & 29,528.54 & 295.29 \\
\hline
3 & Phục hồi rừng & 494,090 & 3.04\% & 4,940.90 & 49.41 \\
\hline
\textbf{Tổng} & & \textbf{16,246,850} & \textbf{100\%} & \textbf{162,468.50} & \textbf{1,624.69} \\
\hline
\end{tabular}
\end{table}

\begin{figure}[H]
    \centering
    \fbox{\parbox{0.9\textwidth}{\centering\vspace{4cm}\textbf{[PLACEHOLDER]}\\ Bản đồ phân loại biến động rừng toàn vùng nghiên cứu\\ với 4 lớp màu khác nhau và chú thích\vspace{4cm}}}
    \caption{Bản đồ phân loại biến động rừng tỉnh Cà Mau}
    \label{fig:classification_map}
\end{figure}

\begin{figure}[H]
    \centering
    \fbox{\parbox{0.6\textwidth}{\centering\vspace{2cm}\textbf{[PLACEHOLDER]}\\ Biểu đồ tròn (pie chart) thể hiện\\ tỷ lệ phần trăm diện tích từng lớp\vspace{2cm}}}
    \caption{Tỷ lệ diện tích các lớp phân loại}
    \label{fig:pie_chart}
\end{figure}