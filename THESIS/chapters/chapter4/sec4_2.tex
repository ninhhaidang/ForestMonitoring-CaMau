\section{Kết quả huấn luyện mô hình CNN}

\subsection{Kết quả 5-Fold Cross Validation}

\begin{table}[H]
\centering
\caption{Kết quả 5-Fold Cross Validation}
\label{tab:cv_results}
\begin{tabular}{|c|c|c|}
\hline
\textbf{Fold} & \textbf{Accuracy} & \textbf{F1-Score} \\
\hline
Fold 1 & 98.34\% & 98.34\% \\
\hline
Fold 2 & 98.57\% & 98.57\% \\
\hline
Fold 3 & 98.10\% & 98.10\% \\
\hline
Fold 4 & 97.86\% & 97.86\% \\
\hline
Fold 5 & 97.86\% & 97.86\% \\
\hline
\textbf{Mean ± Std} & \textbf{98.15\% ± 0.28\%} & \textbf{98.15\% ± 0.28\%} \\
\hline
\end{tabular}
\end{table}

\textbf{Phân tích kết quả CV:}

Kết quả 5-Fold Cross Validation cho thấy sự ổn định của mô hình: độ lệch chuẩn của accuracy chỉ khoảng 0.28\%, chính xác từng fold đều vượt ngưỡng 97.8\%, và điều này cho thấy không có dấu hiệu overfitting nghiêm trọng, tức CV accuracy phản ánh tốt khả năng tổng quát hóa của mô hình.

\begin{figure}[H]
    \centering
    \fbox{\parbox{0.8\textwidth}{\centering\vspace{2cm}\textbf{[PLACEHOLDER]}\\ Biểu đồ cột so sánh accuracy của 5 folds\\ với đường trung bình và error bars\vspace{2cm}}}
    \caption{So sánh accuracy giữa các folds trong Cross Validation}
    \label{fig:cv_comparison}
\end{figure}

\subsection{Kết quả trên tập test (Test Set)}

\begin{table}[H]
\centering
\caption{Metrics trên tập test (526 patches)}
\label{tab:test_metrics}
\begin{tabular}{|l|c|c|}
\hline
\textbf{Metric} & \textbf{Giá trị} & \textbf{Phần trăm} \\
\hline
\textbf{Accuracy} & 0.9886 & \textbf{98.86\%} \\
\hline
Precision (macro-avg) & 0.9886 & 98.86\% \\
\hline
Recall (macro-avg) & 0.9886 & 98.86\% \\
\hline
F1-Score (macro-avg) & 0.9886 & 98.86\% \\
\hline
ROC-AUC (macro-avg) & 0.9998 & 99.98\% \\
\hline
\end{tabular}
\end{table}

\textbf{Ma trận nhầm lẫn - Test Set:}

\begin{table}[H]
\centering
\caption{Ma trận nhầm lẫn trên Test Set}
\label{tab:confusion_matrix}
\begin{tabular}{|c|c|c|c|c|c|}
\hline
 & \textbf{Pred 0} & \textbf{Pred 1} & \textbf{Pred 2} & \textbf{Pred 3} & \textbf{Total} \\
\hline
\textbf{Actual 0} & 129 & 2 & 0 & 0 & 131 \\
\hline
\textbf{Actual 1} & 4 & 126 & 0 & 0 & 130 \\
\hline
\textbf{Actual 2} & 0 & 0 & 133 & 0 & 133 \\
\hline
\textbf{Actual 3} & 0 & 0 & 0 & 132 & 132 \\
\hline
\end{tabular}
\end{table}

\begin{figure}[H]
    \centering
    \fbox{\parbox{0.6\textwidth}{\centering\vspace{2cm}\textbf{[PLACEHOLDER]}\\ Confusion matrix dạng heatmap\\ với màu sắc và số liệu\vspace{2cm}}}
    \caption{Ma trận nhầm lẫn dạng heatmap}
    \label{fig:confusion_heatmap}
\end{figure}

\textbf{Phân tích chi tiết từng lớp - Test Set:}

\begin{table}[H]
\centering
\caption{Phân tích chi tiết từng lớp}
\label{tab:class_analysis}
\begin{tabular}{|l|c|c|c|c|c|}
\hline
\textbf{Lớp} & \textbf{Precision} & \textbf{Recall} & \textbf{F1-Score} & \textbf{Support} & \textbf{Số lỗi} \\
\hline
0 - Rừng ổn định & 96.99\% & 98.47\% & 97.73\% & 131 & 4 FP, 2 FN \\
\hline
1 - Mất rừng & 98.44\% & 96.92\% & 97.67\% & 130 & 2 FP, 4 FN \\
\hline
2 - Phi rừng & 100.00\% & 100.00\% & 100.00\% & 133 & 0 \\
\hline
3 - Phục hồi rừng & 100.00\% & 100.00\% & 100.00\% & 132 & 0 \\
\hline
\end{tabular}
\end{table}

\textbf{Phân tích lỗi phân loại:}

Tổng cộng chỉ có 6/526 mẫu bị phân loại sai, tương đương tỷ lệ lỗi 1.14\%. Trong đó, hai mẫu thuộc Lớp 0 (Rừng ổn định) bị nhầm thành Lớp 1 (Mất rừng) và bốn mẫu thuộc Lớp 1 (Mất rừng) bị nhầm thành Lớp 0 (Rừng ổn định). Đánh giá chi tiết cho thấy Lớp 2 (Phi rừng) và Lớp 3 (Phục hồi rừng) được phân loại hoàn hảo với độ chính xác 100\%.

\textbf{Phân tích nguyên nhân nhầm lẫn giữa Rừng ổn định và Mất rừng:}

Việc nhầm lẫn chỉ xảy ra giữa hai lớp Rừng ổn định (Lớp 0) và Mất rừng (Lớp 1) có thể được giải thích bởi các yếu tố sau:

\begin{itemize}
    \item \textbf{Sự tương đồng về đặc trưng quang phổ:} Cả hai lớp đều có sự hiện diện của rừng ở ít nhất một thời điểm. Các khu vực rừng bị suy thoái nhẹ (degradation) có thể có chữ ký phổ tương tự với rừng ổn định, đặc biệt khi mức độ mất rừng không rõ ràng.

    \item \textbf{Hiệu ứng biên (Edge effects):} Tại ranh giới giữa vùng rừng và vùng mất rừng, các pixel có thể chứa cả hai loại lớp phủ (mixed pixels), dẫn đến vector đặc trưng không điển hình cho một lớp cụ thể.

    \item \textbf{Biến động theo mùa:} Một số khu vực rừng ngập mặn có thể có biến động theo mùa về mật độ tán lá, tạo ra sự thay đổi NDVI tương tự như mất rừng nhưng thực tế là biến động tự nhiên.

    \item \textbf{Độ phân giải thời gian:} Với chỉ hai thời điểm quan sát (bi-temporal), một số biến động ngắn hạn hoặc phục hồi nhanh có thể không được ghi nhận chính xác.
\end{itemize}

Tuy nhiên, với tỷ lệ nhầm lẫn rất thấp (chỉ 6/526 mẫu, ~1.14\%), mô hình vẫn đạt hiệu quả cao trong việc phân biệt các lớp biến động rừng.

\subsection{Đường cong ROC}

\begin{table}[H]
\centering
\caption{ROC-AUC score cho từng lớp (Test Set)}
\label{tab:roc_auc}
\begin{tabular}{|l|c|l|}
\hline
\textbf{Lớp} & \textbf{ROC-AUC} & \textbf{Độ phân biệt} \\
\hline
0 - Rừng ổn định & 0.9998 & Xuất sắc \\
\hline
1 - Mất rừng & 0.9997 & Xuất sắc \\
\hline
2 - Phi rừng & 1.0000 & Hoàn hảo \\
\hline
3 - Phục hồi rừng & 1.0000 & Hoàn hảo \\
\hline
\textbf{Macro-average} & \textbf{0.9998} & \textbf{Xuất sắc} \\
\hline
\end{tabular}
\end{table}

\begin{figure}[H]
    \centering
    \fbox{\parbox{0.8\textwidth}{\centering\vspace{2cm}\textbf{[PLACEHOLDER]}\\ Đường cong ROC cho 4 lớp\\ với AUC values\vspace{2cm}}}
    \caption{Đường cong ROC cho các lớp phân loại}
    \label{fig:roc_curves}
\end{figure}