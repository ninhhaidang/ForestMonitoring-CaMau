\section{Rừng và biến động rừng}

\subsection{Khái niệm về rừng}

Rừng là một hệ sinh thái bao gồm chủ yếu là cây cối, thực vật và động vật sống cùng nhau trong một môi trường phức tạp. Theo định nghĩa của Tổ chức Nông nghiệp và Lương thực Liên Hợp Quốc (FAO), rừng là vùng đất có diện tích tối thiểu 0.5 hecta với độ che phủ tán cây trên 10\%, chiều cao cây tối thiểu 5 mét khi trưởng thành, và không phải là đất nông nghiệp hoặc đô thị \citeen{fao2020}. Rừng đóng vai trò quan trọng trong việc điều hòa khí hậu thông qua hấp thụ CO$_2$ và thải oxy, bảo tồn đa dạng sinh học, điều tiết nguồn nước và chống xói mòn đất, đồng thời cung cấp tài nguyên thiên nhiên và sinh kế cho hàng tỷ người trên thế giới.

Rừng bao phủ khoảng 31\% diện tích đất liền toàn cầu \citeen{fao2020}. Tùy theo vị trí địa lý và điều kiện khí hậu, rừng được phân loại thành nhiều kiểu khác nhau như rừng nhiệt đới, rừng ôn đới, rừng phương bắc (taiga), rừng ngập mặn, và rừng tràm. Trong đó, rừng ngập mặn là hệ sinh thái đặc biệt quan trọng ở các vùng ven biển nhiệt đới, có khả năng lưu giữ carbon cao gấp 3–5 lần so với rừng nhiệt đới trên cạn \citeen{donato2011,alongi2014}.

\subsection{Tình hình mất rừng trên thế giới}

Tốc độ mất rừng toàn cầu vẫn đang ở mức báo động. Theo báo cáo ``Global Forest Resources Assessment 2020'' của FAO \citeen{fao2020}, tổng diện tích rừng bị phá (gross deforestation) từ năm 1990 đến 2020 ước tính khoảng 420 triệu hecta. Mặc dù diện tích mất rừng ròng đã giảm nhờ nỗ lực trồng rừng, nhưng việc chuyển đổi đất rừng sang nông nghiệp và chăn nuôi vẫn diễn ra phức tạp.

Sự suy giảm này tập trung nghiêm trọng nhất tại khu vực nhiệt đới. Theo báo cáo của WWF, chỉ tính riêng giai đoạn 2004--2017, hơn 43 triệu hecta rừng đã bị xóa sổ tại các ``mặt trận'' nóng bỏng nhất, diện tích tương đương quy mô nước Maroc. Trong đó, Lưu vực Amazon (Nam Mỹ) là nơi chịu tổn thất nặng nề nhất do áp lực từ chăn nuôi và nông nghiệp quy mô lớn.

\begin{figure}[H]
    \centering
    \includegraphics[width=0.95\textwidth]{img/chapter1/Mat-rung-My-Latinh.png}
    \caption{Các mặt trận mất rừng trọng điểm tại khu vực Mỹ Latinh. (Nguồn: WWF, 2021)}
    \label{fig:deforestation_latin_america}
\end{figure}

Không chỉ giới hạn ở Châu Mỹ, tình trạng phá rừng cũng đang diễn biến phức tạp tại bờ bên kia đại dương. Lưu vực Congo (Trung Phi) và Đông Nam Á là những điểm nóng tiếp theo. Tại Đông Nam Á, rừng nguyên sinh đang bị thu hẹp nhanh chóng tại các khu vực như sông Mekong, đảo Sumatra và Borneo để nhường chỗ cho các đồn điền cây công nghiệp.

\begin{figure}[H]
    \centering
    \includegraphics[width=0.95\textwidth]{img/chapter1/Mat-rung-Chau-Phi-va-Dong-Nam-A.png}
    \caption{Các mặt trận mất rừng trọng điểm tại Châu Phi và Đông Nam Á. (Nguồn: WWF, 2021)}
    \label{fig:deforestation_africa_asia}
\end{figure}

Xu hướng này vẫn tiếp diễn trong những năm gần đây. Theo Global Forest Watch \citeen{gfw2021}, thế giới mất khoảng 10 triệu hecta rừng nhiệt đới mỗi năm trong giai đoạn 2015--2020. Việc này không chỉ làm giảm khả năng hấp thụ CO$_2$ mà còn trực tiếp phát thải khí nhà kính từ việc đốt rừng và phân hủy sinh khối. Theo IPCC \citeen{ipcc2019}, phá rừng và thay đổi sử dụng đất đóng góp khoảng 23\% tổng lượng phát thải khí nhà kính do con người gây ra, góp phần làm gia tăng hiện tượng biến đổi khí hậu toàn cầu.

\subsection{Tình hình mất rừng tại Việt Nam}

Việt Nam đã trải qua những biến đổi lớn về độ che phủ rừng trong 30 năm qua. Sau thời kỳ suy giảm nghiêm trọng (độ che phủ chỉ còn 28\% vào năm 1990 do chiến tranh và khai thác bừa bãi), Việt Nam đã thực hiện nhiều chương trình phục hồi và phát triển rừng. Nhờ các chương trình như ``Trồng 5 triệu hecta rừng'' (1998-2010), độ che phủ rừng đã tăng lên 42\% vào năm 2020 \citevi{bnnptnt2021}.

Tuy nhiên, chất lượng rừng là một vấn đề đáng lo ngại. Mặc dù tổng diện tích rừng tăng từ 9.4 triệu hecta (1990) lên 14.6 triệu hecta (2020) chủ yếu nhờ rừng trồng (cao su, keo, thông), chất lượng rừng tự nhiên lại suy giảm đáng kể. Theo số liệu của Bộ NN\&PTNT (2020), rừng tự nhiên hiện có khoảng 10.29 triệu hecta, nhưng rừng nguyên sinh (primary forest) chỉ còn chiếm khoảng 0.25\% tổng diện tích rừng \citevi{thanhnien2021}.

Nguyên nhân chính gây mất rừng tại Việt Nam bao gồm việc chuyển đổi sang đất nông nghiệp như trồng cà phê, cao su và điều; khai thác gỗ trái phép; phát triển cơ sở hạ tầng và đô thị hóa; cháy rừng; và hoạt động nuôi trồng thủy sản, đặc biệt tại khu vực ven biển và đồng bằng sông Cửu Long.

\begin{figure}[H]
    \centering
    \begin{tikzpicture}
        \begin{axis}[
            width=0.95\textwidth,
            height=8cm,
            xlabel={Năm},
            ylabel={Độ che phủ rừng (\%)},
            xmin=1990, xmax=2020,
            ymin=25, ymax=45,
            xtick={1990,1995,2000,2005,2010,2015,2020},
            xticklabel style={/pgf/number format/1000 sep={},yshift=-3pt},
            yticklabel style={xshift=-3pt},
            ytick={25,30,35,40,45},
            legend pos=north west,
            grid=major,
            grid style={dashed,gray!30},
            every axis plot/.append style={thick}
        ]
        \addplot[color=green,mark=*] coordinates {
            (1990,27.2)
            (1995,28.2)
            (2000,33.7)
            (2005,37.0)
            (2010,39.5)
            (2015,40.84)
            (2020,42.01)
        };
        \legend{Độ che phủ rừng}
        \end{axis}
    \end{tikzpicture}
    \caption{Biến động độ che phủ rừng Việt Nam giai đoạn 1990-2020}
    \label{fig:vietnam_forest_change}
\end{figure}
