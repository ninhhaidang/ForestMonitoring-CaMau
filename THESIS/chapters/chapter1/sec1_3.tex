\section{Tổng quan các nghiên cứu liên quan}

\subsection{Phương pháp Deep Learning}

\textbf{Convolutional Neural Networks (CNN):}

CNN đã cách mạng hóa computer vision và ngày càng được áp dụng rộng rãi trong viễn thám \cite{zhu2017}. Zhang et al. \cite{zhang2016} giới thiệu các kiến trúc CNN phổ biến và ứng dụng của chúng trong viễn thám, Kussul et al. \cite{kussul2017} áp dụng CNN cho phân loại cây trồng từ Sentinel-2 và đạt accuracy 94.5\%, và Xu et al. \cite{xu2021} sử dụng CNN kết hợp với cơ chế attention để đạt accuracy 96.8\% trên dữ liệu đa nguồn.

\textbf{Các kiến trúc CNN tiêu biểu trong viễn thám:}

Ronneberger et al. \cite{ronneberger2015} đề xuất kiến trúc U-Net với cấu trúc encoder-decoder, ban đầu cho phân đoạn ảnh y sinh nhưng sau đó được áp dụng rộng rãi trong viễn thám nhờ khả năng phân đoạn ngữ nghĩa hiệu quả. Zhong et al. \cite{zhong2018} phát triển SatCNN - kiến trúc CNN chuyên biệt cho phân loại ảnh vệ tinh. Karra et al. \cite{karra2021} ứng dụng deep learning kết hợp Sentinel-2 để tạo bản đồ sử dụng đất toàn cầu với độ phân giải 10m.

\subsection{Ứng dụng trong giám sát rừng}

\textbf{Phát hiện mất rừng:}

Hansen et al. \cite{hansen2013} phát triển Global Forest Change dataset sử dụng chuỗi thời gian Landsat và thuật toán decision tree để phát hiện mất rừng toàn cầu giai đoạn 2000–2012 ở độ phân giải 30m. Reiche et al. \cite{reiche2018} kết hợp Sentinel-1 và Landsat để phát hiện mất rừng near-real-time tại Amazon và báo cáo accuracy đạt 93.8\%. Hethcoat et al. \cite{hethcoat2019} áp dụng CNN trên chuỗi thời gian Landsat để phát hiện khai thác vàng trái phép tại Amazon và đạt F1-score 0.92.

\textbf{Tích hợp SAR và Optical:}

Hu et al. \cite{hu2020} kết hợp Sentinel-1 và Sentinel-2 để phân loại rừng ở Madagascar và ghi nhận accuracy tăng từ 87\% lên 92\% khi sử dụng cả hai nguồn dữ liệu. Ienco et al. \cite{ienco2019} ứng dụng deep neural networks kết hợp chuỗi thời gian SAR và Optical để phân loại cây trồng và đạt accuracy 96.5\%.

\textbf{Nghiên cứu tại Việt Nam:}

Pham et al. \cite{pham2019} đã sử dụng kết hợp ảnh QuickBird, LiDAR và chỉ số địa hình GIS để nhận dạng loài cây bản địa trong cảnh quan phức tạp. Nguyen et al. \cite{nguyen2020} áp dụng Sentinel-2 đa thời gian để lập bản đồ sử dụng đất tại Đắk Nông với overall accuracy 91.2\%. Bùi et al. \cite{bui2021} nghiên cứu biến động rừng ngập mặn ven biển Đồng bằng sông Cửu Long bằng chuỗi thời gian Landsat (1990–2020).

\begin{table}[H]
\centering
\caption{Tổng hợp các nghiên cứu liên quan}
\label{tab:related_works}
\begin{tabular}{|l|c|l|l|c|}
\hline
\textbf{Tác giả} & \textbf{Năm} & \textbf{Phương pháp} & \textbf{Dữ liệu} & \textbf{Accuracy} \\
\hline
Hansen et al. & 2013 & Decision Tree & Landsat & ~85\% \\
\hline
Kussul et al. & 2017 & CNN & Sentinel-2 & 94.5\% \\
\hline
Reiche et al. & 2018 & Bayesian & S1+Landsat & 93.8\% \\
\hline
Hethcoat et al. & 2019 & CNN (ResNet) & S1/S2 & 94.3\% \\
\hline
Nguyen et al. & 2020 & Random Forest & Sentinel-2 & 91.2\% \\
\hline
\end{tabular}
\end{table}