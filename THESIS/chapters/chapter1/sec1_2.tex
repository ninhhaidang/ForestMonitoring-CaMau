\section{Các nguồn năng lượng tái tạo}

Hiện nay, sử dụng năng lượng tái tạo đang được coi là một trong những biện pháp hiệu quả nhằm ứng phó với biến đổi khí hậu và hướng đến sự phát triển bền vững toàn cầu. Dựa vào nguồn gốc và phương thức khai thác, năng lượng tái tạo có thể được chia thành năm nhóm chính: năng lượng mặt trời, năng lượng gió, năng lượng nước, năng lượng sinh học và năng lượng địa nhiệt.

\begin{figure}[H]
    \centering
    \includegraphics[width=0.75\linewidth]{img/chapter1/energy types.jpg}
    \caption{Các nguồn năng lượng tái tạo chính hiện nay.\cite{vietnam_ete_renewable_energy_sources}}
    \label{fig:enter-label}
    
\end{figure}

Năng lượng mặt trời được khai thác từ các tia bức xạ ion hoá phát ra từ Mặt Trời và là một trong những nguồn năng lượng tái tạo được sử dụng phổ biến nhất trên toàn cầu. Hai hệ thống năng lượng mặt trời chính là năng lượng nhiệt mặt trời và năng lượng quang điện.\cite{ang2022renewable} Đây là hai dạng công nghệ chuyển đổi năng lượng mặt trời thường được lắp đặt và khai thác. Trong đó, công nghệ quang điện được ưa chuộng hơn và là một nguồn năng lượng tiềm năng lớn trong tương lai. Quang điện mặt trời có thể được sản xuất và lắp đặt không chỉ với quy mô lớn tại các nhà máy công suất lớn mà còn ứng dụng quy mô nhỏ tại nhiều địa điểm khác nhau như mái nhà dân sinh. Theo báo cáo của Cơ quan Năng lượng quốc tế, quang điện mặt trời đã có mức tăng kỷ lục 270 TWh (lên đến 26\%) trong năm 2022, lần đầu tiên trong lịch sử vượt qua năng lượng gió. Sự tăng trưởng được đánh giá là theo đúng với mục tiêu dự đoán giai đoạn 2023 – 2030 hướng tới Net Zero năm 2050. Đặc biệt, năm 2023, riêng quang điện mặt trời đã đóng góp 3/4 tổng công suất bổ sung năng lượng tái tạo trên toàn thế giới.\cite{iea_solar_pv_2024}

Năng lượng gió là nguồn năng lượng tái tạo phổ biến thứ hai trên thế giới chỉ sau thuỷ điện nhờ một số ưu điểm như cơ sở hạ tầng dễ lắp đặt, giá thành hợp lý, công nghệ tiên tiến. Thông thường, năng lượng gió được khai thác nhờ các trang trại tua-bin gió bao gồm hai dạng chính là trang trại gió trên bờ, với hệ thống tua-bin gió được lắp đặt trên đất liền, và trang trại gió ngoài khơi. Hầu hết các trang trại gió hiện nay được xây dựng theo kiểu thứ nhất nhưng gần đây, công nghệ xây dựng điện gió ngoài khơi đang phát triển nhanh chóng và được ứng dụng ngày càng nhiều ở các khu vực, đặc biệt là Châu Âu.\cite{ang2022renewable} Theo Cơ quan Năng lượng quốc tế, năng lượng gió được đánh giá là một trong những tiềm năng lớn để các quốc gia nâng cao sản lượng khai thác năng lượng tái tạo. Trong năm 2022, sản lượng điện gió đã có mức tăng kỷ lục đạt 265 TWh, tương đương 14\%, là mức tăng cao thứ hai trong các loại năng lượng tái tạo, chỉ sau quang điện mặt trời. Theo dự đoán, nhờ việc đơn giản hoá quy trình cấp phép và đấu giá, các dự án kết hợp điện gió – mặt trời quy mô lớn sẽ được triển khai nhanh hơn và đến năm 2025, điện gió có thể vượt sản lượng điện hạt nhân.\cite{iea_wind_2024}

Thuỷ điện là dạng khai thác phổ biến của năng lượng nước và hiện nay được phát triển mạnh mẽ ở hầu hết các quốc gia. Vốn là dạng năng lượng tái tạo được khai thác lâu đời, sản lượng thuỷ điện hàng năm thường cao hơn tổng tất cả các dạng năng lượng tái tạo khác gộp lại và mức khai thác này được dự đoán vẫn sẽ duy trì cho đến những năm 2030.\cite{iea_hydroelectricity_2024} Trong xu hướng chuyển dịch năng lượng tái tạo, tổ hợp năng lượng gió – mặt trời được đặt mục tiêu sẽ vượt qua thuỷ điện nhưng việc khai thác năng lượng nước vẫn đóng vai trò nguồn năng lượng tái tạo chủ chốt, giúp cân bằng các yếu tố không ổn định của các nguồn khác. Lý giải cho xu hướng giảm của thuỷ điện trong một thập kỷ tới chính là sự thiếu linh hoạt trong việc thay đổi các chính sách nhằm đối mặt với tình hình biến đổi khí hậu. Lượng mưa ngày càng thất thường do biến đổi khí hậu cũng đang làm gián đoạn hoạt động sản xuất thuỷ điện ở nhiều nơi trên thế giới. Thực tế hiện nay, các dự án thuỷ điện tại Trung Quốc, Mỹ Latin và Châu Âu đã có phần giảm tốc độ triển khai nhưng lại đang được đẩy mạnh tại khu vực Châu Á Thái Bình Dương, Châu Phi và Trung Đông.\cite{iea_hydroelectricity_2024} Nhìn chung, sản lượng đầu ra vẫn phần nào được duy trì ổn định.

Năng lượng sinh học đang trở thành những nguồn năng lượng tái tạo tiềm năng, có đóng góp lớn trong lĩnh vực nhiệt sưởi, giao thông và sản xuất điện sạch. Năng lượng sinh học được khai thác từ các nguồn nhiên liệu thô sinh học, hay còn gọi là sinh khối, bằng các phương pháp truyền thống hoặc hiện đại. Trong phương pháp truyền thống, sinh khối là các nguyên liệu nông nghiệp như củi, than, tàn dư cây trồng và chất thải động vật sẽ được xử lý để sử dụng cho các khu vực đô thị. Ngược lại, phương pháp hiện đại được ứng dụng để sản sinh nhiệt, điện cho công nghiệp khí sinh học, dầu sinh học, than sinh học thông qua các công nghệ chuyển đổi nhiệt từ sinh khối như quá trình các-bon hoá, sấy, khí hoá, đốt cháy và nhiệt phân.\cite{ang2022renewable}

Năng lượng địa nhiệt là năng lượng nhiệt sinh ra từ sự phân rã phóng xạ của tài nguyên khoáng sản và từ cấu trúc nguyên thuỷ của Trái Đất. Thông thường, với mỗi 1 m sâu vào mặt đất, địa nhiệt sẽ tăng $0.03^oC$ nên 99\% nhiệt độ của Trái Đất đều trên $1000^oC$. So với các nguồn tài nguyên năng lượng tái tạo không liên tục khác như năng lượng mặt trời, năng lượng gió và năng lượng thuỷ điện, năng lượng địa nhiệt bên trong Trái Đất rất phong phú và vô tận. Ngoài ra, năng lượng địa nhiệt có tính ổn định tự nhiên và không thải ra $CO_2$. Năng lượng địa nhiệt có tiềm năng kinh tế lớn ở những khu vực cùng với năng lượng thuỷ nhiệt, đặc biệt ở những nước có núi lửa đang hoạt động. Ngày nay, có khoảng 26 quốc gia sử dụng năng lượng địa nhiệt để sản xuất điện, như Hoa Kỳ, Indonesia, Philippines, Mexico, Ý, Iceland, New Zealand và Nhật Bản.\cite{ang2022renewable}