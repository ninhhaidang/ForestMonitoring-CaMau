\section{Công nghệ viễn thám trong giám sát rừng}

\subsection{Ưu điểm của công nghệ viễn thám}

Công nghệ viễn thám vệ tinh đem lại nhiều ưu thế rõ rệt so với các phương pháp điều tra thực địa truyền thống. Trước hết, ảnh vệ tinh có khả năng bao phủ diện tích rất lớn, giúp quan sát đồng thời hàng nghìn km² rừng. Bên cạnh đó, các vệ tinh hiện đại có chu kỳ lặp ngắn, thường chỉ 3–5 ngày, tạo điều kiện phát hiện kịp thời những biến động xảy ra trong rừng. Nguồn dữ liệu từ nhiều chương trình vệ tinh hiện nay còn được cung cấp miễn phí, góp phần giảm đáng kể chi phí so với khảo sát ngoài thực địa. Hệ thống lưu trữ ảnh vệ tinh phong phú theo thời gian cũng cho phép phân tích chuỗi biến động dài hạn. Đối với những khu vực khó tiếp cận như rừng sâu, vùng núi cao hay khu vực biên giới, viễn thám vẫn có thể giám sát hiệu quả. Ngoài ra, tính khách quan và khả năng lặp lại của dữ liệu viễn thám giúp hạn chế các sai lệch do yếu tố chủ quan trong quá trình điều tra trực tiếp.

\subsection{Chương trình Copernicus và vệ tinh Sentinel}

Chương trình Copernicus của Liên minh Châu Âu (EU) là một trong những chương trình quan sát Trái Đất lớn nhất thế giới, cung cấp dữ liệu miễn phí và mở. Hai vệ tinh quan trọng cho giám sát rừng là:

\textbf{Sentinel-1 (SAR - Synthetic Aperture Radar):}

Vệ tinh Sentinel-1 hoạt động ở dải sóng C-band (xấp xỉ 5.5 cm) với hai chế độ phân cực chính là VV (Vertical-Vertical) và VH (Vertical-Horizontal); độ phân giải không gian trong chế độ Interferometric Wide (IW) là 10m \cite{esa2024s1}. Về chu kỳ quay trở lại, Sentinel-1A có chu kỳ khoảng 12 ngày; khi kết hợp với Sentinel-1B (đã ngừng hoạt động từ tháng 12/2021 do sự cố nguồn điện), chu kỳ giảm còn 6 ngày. Sentinel-1C được phóng vào tháng 12/2024 để thay thế Sentinel-1B. Do là hệ thống chủ động, Sentinel-1 có ưu điểm xuyên qua mây và khói, hoạt động được cả ngày lẫn đêm, và nhạy cảm đối với cấu trúc thực vật cũng như độ ẩm.

\textbf{Sentinel-2 (Optical - Multispectral Imaging):}

Vệ tinh Sentinel-2 cung cấp 13 dải phổ từ vùng nhìn thấy đến hồng ngoại ngắn (từ 443 nm đến 2190 nm) với nhiều cấp độ độ phân giải không gian: 10m cho các dải B2, B3, B4 và B8; 20m cho các dải B5, B6, B7, B8a, B11 và B12; và 60m cho B1, B9 và B10 \cite{esa2024s2}. Chu kỳ quay trở lại của tổ hợp hai vệ tinh Sentinel-2A và Sentinel-2B vào khoảng 5 ngày, và vì có thông tin quang phổ phong phú nên Sentinel-2 rất phù hợp để tính toán chỉ số thực vật.


 \begin{figure}[H]
    \centering
    \includegraphics[width=0.95\textwidth]{img/chapter1/Sentinel-1.jpeg}
    \caption{Vệ tinh Sentinel-1. (Nguồn: European Space Agency)}
    \label{fig:sentinel1_image}
\end{figure}


 \begin{figure}[H]
    \centering
    \includegraphics[width=0.95\textwidth]{img/chapter1/Sentinel-2.png}
    \caption{Vệ tinh Sentinel-2. (Nguồn: European Space Agency)}
    \label{fig:sentinel2_image}
\end{figure}


\subsection{Chỉ số thực vật từ dữ liệu quang học}

Các chỉ số thực vật (vegetation indices) là công cụ quan trọng trong giám sát rừng, được tính toán từ các dải phổ khác nhau:

\textbf{NDVI (Normalized Difference Vegetation Index):}
\begin{equation}
NDVI = \frac{NIR - Red}{NIR + Red}
\end{equation}

NDVI có dải giá trị từ -1 đến 1; giá trị NDVI lớn hơn 0.6 thường biểu thị thực vật xanh tốt, trong khi giá trị NDVI nhỏ hơn 0.2 thường liên quan đến đất trống, nước hoặc khu vực đô thị \cite{huang2021}.

\textbf{NBR (Normalized Burn Ratio):}
\begin{equation}
NBR = \frac{NIR - SWIR_2}{NIR + SWIR_2}
\end{equation}

NBR nhạy cảm với vùng cháy; biến đổi Delta NBR (dNBR) được sử dụng để đánh giá mức độ tổn thất do cháy rừng.

\textbf{NDMI (Normalized Difference Moisture Index):}
\begin{equation}
NDMI = \frac{NIR - SWIR_1}{NIR + SWIR_1}
\end{equation}

NDMI được dùng để đánh giá hàm lượng nước trong thực vật; giá trị NDMI thấp có thể chỉ ra trạng thái stress do hạn hán.

\subsection{Tích hợp dữ liệu SAR và Optical}

Việc kết hợp dữ liệu SAR (Sentinel-1) và Optical (Sentinel-2) mang lại nhiều lợi ích thực tế. Về khía cạnh bổ sung thông tin, SAR cung cấp dữ liệu về cấu trúc, độ nhám bề mặt và độ ẩm, trong khi Optical cung cấp thông tin quang phổ và các chỉ số thực vật. Về khía cạnh khắc phục hạn chế, SAR hoạt động hiệu quả trong điều kiện mây mù — điều quan trọng trong môi trường rừng nhiệt đới — còn Optical lại cung cấp dữ liệu trực quan dễ phiên giải. Về khía cạnh nâng cao độ chính xác, nhiều nghiên cứu cho thấy việc kết hợp SAR và Optical giúp tăng accuracy từ khoảng 5 đến 10\% so với việc sử dụng mỗi nguồn dữ liệu riêng lẻ \cite{ienco2019, hu2020}. Về khía cạnh phát hiện biến động đa chiều, SAR nhạy với biến đổi cấu trúc như chặt cây, trong khi Optical nhạy với biến đổi quang phổ thể hiện sức khỏe thực vật.