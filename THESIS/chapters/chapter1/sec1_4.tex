\section{Khai thác năng lượng gió và tua-bin gió}

Khái niệm khai thác năng lượng gió đã có từ khi con người lập ra những nền văn minh cổ đại. Từ xa xưa, người Ai Cập và Hy Lạp cổ đã phát minh ra những con thuyền buồm, phương tiện vốn dựa vào sức gió để di chuyển, và tạo nên cuộc cách mạng cho giao thương toàn cầu. Vào khoảng năm 500 đến 900 sau Công nguyên, cối xay gió đầu tiên đã được phát minh tại vùng Ba Tư (Iran ngày nay) để phục vụ cho việc xay nghiền ngũ cốc và bơm nước. Dạng cối xay gió này thường bao gồm cấu trúc trục thẳng đứng với cánh quạt gỗ và cánh buồm.\cite{windcycle_wind_turbine_technology_2024} Gió làm quay các cánh buồm thẳng đứng, cung cấp năng lượng cho các cơ cấu xay ngũ cốc hoặc nâng nước từ giếng. Đến thời Trung Cổ, cối xay gió Châu Âu ra đời và được xây dựng phổ biến trên toàn châu lục này, trở thành hình ảnh biểu tượng của cấu trúc cối xay gió với trục ngang cánh gỗ. Ở Châu Á, người Trung Quốc cũng đã sử dụng cối xay gió để kéo nước biển làm muối.\cite{renewableenergyworld_history_wind_turbines_2024} Trải qua quá trình phát triển cho tới cuối thế kỷ XIX, khi thời đại mới của năng lượng gió bắt đầu, công nghệ tua-bin gió ra đời và đi cùng với sự phát triển của công nghệ điện gió.

Kể từ những năm 1880, khi các mẫu tua-bin điện gió đầu tiên được phát minh bởi Charles F. Bush ở Ohio và James Blythe ở Glasgow, cuộc cách mạng năng lượng gió đã chứng kiến hành trình phát triển với nhiều cột mốc đáng chú ý, trong đó phải kể đến sự ra đời của các thiết kế và công nghệ mới như:

\begin{itemize}
    \item Từ năm 1900, các nhà khoa học Đan Mạch đã phát triển và xây dựng hơn 2,500 công trình cối xay gió có công suất tối đa 30 MW phục vụ nghiền ngũ cốc và bơm nước. Tới năm 1908, 72 hệ thống cối xay gió phát điện đã vận hành trên toàn đất nước với dải công suất từ 5 đến 25 kW.
    \item Năm 1922, tua-bin gió Savonius được phát minh bởi kỹ sư người Phần Lan Sigurd Johannes Savonius. Đây là một dạng thiết kế nguyên bản của tua-bin gió trục đứng bao gồm 2 cánh hình bán nguyệt và được cấp bằng sáng chế vào năm 1926.\cite{solari_wind_science_engineering_2019}  Năm 1931, một kỹ sư hàng không người Pháp là Georges Jean Marie Darrieus đã đưa ra một thiết kế khác của tua-bin gió trục đứng, nay gọi là tua-bin Darrieus. Loại tua-bin gió này vẫn được sử dụng cho đến ngày nay, nhưng không rộng rãi như tua-bin gió trục ngang.\cite{renewableenergyworld_history_wind_turbines_2024}
    \item Từ năm 1931 đến 1941, những nguyên mẫu của tua-bin gió trục ngang đã được xây dựng, tiêu biểu như nguyên mẫu Soviet công suất 100 kW cao 32 m ở gần Yalta và tua-bin gió Smith-Putnam 1.25 MW ở Castletown, Vermont. Đến năm 1957, Johannes Juul, chế tạo một tua-bin gió trục ngang có đường kính 24 m và 3 cánh có thiết kế rất giống với các tua-bin gió vẫn được sử dụng ngày nay. Tua-bin gió có công suất 200 kW được ứng dụng công nghệ mới là phanh khí động học khẩn cấp.\cite{renewableenergyworld_history_wind_turbines_2024}
\end{itemize}

Tuy nhiên, sự phát triển của công nghệ tua-bin gió đã có phần giảm sút sau Chiến tranh thế giới thứ II do sự lên ngôi của các dòng năng lượng hoá thạch khiến cho nhu cầu năng lượng gió bị đình trệ. Cho tới những năm 1970, cuộc khủng hoảng dầu mỏ lại một lần nữa chứng minh tính bền vững của năng lượng gió và tái khởi động xu hướng ứng dụng nguồn năng lượng này ở nhiều quốc gia. Điều này đã thúc đẩy mạnh mẽ các nhà khoa học, kỹ sư ở nhiều khu vực trên thế giới tham gia vào các cuộc nghiên cứu nhằm cải tiến các dạng tua-bin gió và quá trình này vẫn còn tiếp diễn cho tới ngày nay.

Đối với nền công nghệ điện gió hiện nay, tua-bin gió là thiết bị chính, có khả năng chuyển đổi động năng của gió thành cơ năng, sau được chuyển hoá thành điện năng. Dựa theo đặc điểm của trục quay và nguyên lý hoạt động, tua-bin gió được chia thành 2 loại chính là tua-bin gió trục ngang (HAWT) và tua-bin gió trục đứng (VAWT). Tuy nhiên, một tua-bin gió cơ bản thường có cấu tạo bao gồm các thành phần được trình bày trong Hình \ref{fig:turbine-components}. Trong đó:

\begin{figure}[H]
    \centering
    \includegraphics[width=0.75\linewidth]{img/chapter1/VAWT and HAWT.jpg}
    \caption{Cấu tạo cơ bản của tua-bin gió \cite{islam_power_electronics}}
    \label{fig:turbine-components}
\end{figure}

\begin{itemize}
    \item Trụ (1) có tác dụng nâng đỡ các bộ phận và giúp tua-bin gió tiếp cận luồng gió.
    \item Vỏ (2) là nơi chứa và bảo vệ các thành phần chính của tua-bin bao gồm máy phát điện, hộp số và hệ thống điều khiển.
    \item Máy phát điện (3) chuyển hoá cơ năng từ chuyển động quay của tua-bin gió thành điện năng
    \item Hộp số (4) giúp tăng tốc độ quay của trục quay (rô-to) tua-bin.
    \item Cánh quạt (5) gắn với rô-to của tua-bin giúp thu gió và chuyển hoá thành cơ năng nhờ chuyển động quay.
    \item Các tham số bao gồm đường kính (6) và chiều cao (7) rô-to được tính toán để xác định hiệu suất của tua-bin gió.
\end{itemize}

\subsection{Tua-bin gió trục ngang (HAWT)}

Tua-bin gió trục ngang, là thiết kế sử dụng phổ biến nhất trong các loại tua-bin gió với rô-to cánh quạt được kết nối với 1 trục ngang và bộ sản xuất điện năng. Hiện nay, HAWT ba cánh quạt được đánh giá là có hiệu suất cao nhất nhưng tuỳ thuộc vào mục đích thiết kế, số lượng cánh có thể thay đổi thành hai hoặc một. Tua-bin trục ngang sử dụng thiết kế cánh tương tự cánh máy bay và dựa vào lực nâng sinh ra trên cánh để quay và tạo ra công suất. Những đặc điểm trong thiết kế đòi hỏi HAWT phải đặt mặt phẳng cánh quạt đúng hướng gió, được lắp đặt cảm biến xác định hướng gió và hỗ trợ chuyển động quay. Tua-bin gió trục ngang được thiết kế với dải công suất đa dạng và được sử dụng phổ biến trong việc sản xuất điện năng từ gió. Với một số đặc trưng như kích thước lớn, cánh quạt dài, HAWT có thể hoạt động trên cao nhằm thu được lượng gió lớn, mạnh và ổn định. Do đó, HAWT thích hợp cho những cánh đồng gió cỡ lớn, thường được đặt ở những khu vực có luồng không khí có hướng ổn định, hạn chế nhiễu loạn.

\begin{figure}[H]
    \centering
    \includegraphics[width=0.75\linewidth]{img/chapter1/HAWT.png}
    \caption{Tua-bin gió trục ngang (HAWT) \cite{avaada_types_of_wind_turbines}}
    \label{fig:enter-label}
\end{figure}
\subsection{Tua-bin gió trục đứng (VAWT)}

Tua-bin gió trục đứng bao gồm rô-to cánh quạt được kết nối với trục thẳng đứng. Trong vài năm trở lại đây, VAWT đang dần được chú ý nhờ những ưu điểm trong thiết kế và công nghệ. Tua-bin gió trục đứng bao gồm hai loại là tua-bin Darrieus và tua-bin Savonius. Tua-bin Darrieus hoạt động nhờ lực nâng khí động học với các cánh quạt có dạng airfoil. Trong khi đó, tua-bin Savonius hoạt động dựa trên nguyên lý lực cản khí động học, bao gồm 2 cánh bán nguyệt nửa hình trụ được gắn với 1 trục thẳng đứng. Tua-bin trục đứng là loại đa hướng, chúng hiệu quả khi gặp thời tiết gió mạnh, từ nhiều hướng. Dạng tua-bin này phù hợp với những khu vực có địa hình phức tạp, tốc độ gió không ổn định như vùng núi, vùng đô thị,…

\begin{figure}[H]
    \centering
    \includegraphics[width=0.75\linewidth]{img/chapter1/VAWT.jpg}
    \caption{Tua-bin gió trục đứng (VAWT) \cite{typma_maglev_wind_generator}}
    \label{fig:enter-label}
    
\end{figure}

Hiện nay, cư dân tại các vùng đô thị và sinh sống tại các vùng có địa hình phức tạp gặp phải những hạn chế trong quá trình tiếp cận với các nguồn năng lượng tái tạo. Đối với năng lượng gió, các kế hoạch tiếp cận này phần lớn được thực hiện bởi các công ty điện lực mua từ các cơ sở sản xuất gió ở các vùng nông thôn. Điều này dẫn đến thách thức trong việc phát triển sản xuất năng lượng gió trong môi trường đô thị.

Một trong những tiến bộ công nghệ chính làm tăng khả năng ứng dụng của năng lượng gió trong môi trường đô thị là cải tiến thiết kế tua-bin gió trục đứng. VAWT có các đặc điểm có thể khiến chúng vượt trội hơn so với HAWT trong các ứng dụng đô thị. Các đặc điểm đó bao gồm kết cấu nhỏ gọn, dễ dàng vận chuyển, lắp đặt, vận hành và bảo dưỡng. Với cách vận hành không phụ thuộc vào hướng gió, VAWT có thể hoạt động ở dải gió rộng, phù hợp lắp đặt ở không gian nhỏ hẹp. Điều này là thuận lợi do tính chất không ổn định của gió tại các địa hình khác nhau. Không giống như HAWT, VAWT không cần máy phát điện và hộp số phải lắp gần các cánh quạt và có thể được đặt ở chân đế, cải thiện cả cấu hình của tua-bin và khả năng tiếp cận để bảo trì. 

Trong các loại tua-bin gió trục đứng, tua-bin gió Savonius gần đây đang thu hút sự chú ý của các nhóm nghiên cứu và được đánh giá là mô hình tiềm năng để ứng dụng khai thác năng lượng gió vùng đô thị. Được phát minh và đăng ký bằng sáng chế bởi kỹ sư người Phần Lan Sigurd Johannes Savonius, thiết kế của tua-bin Savonius nguyên bản bao gồm chia một khối trụ thành hai nửa theo mặt phẳng trung tâm và di chuyển những nửa trụ này sang hai bên theo mặt cắt để tạo ra một hình chữ S trên mặt cắt ngang. Mặc dù so với Darrieus, tua-bin gió Savonius vẫn còn hạn chế về mặt hiệu suất nhưng lại dễ thiết kế và lắp đặt hơn. Ngoài ra, tua-bin Savonius còn có một số ưu điểm vượt trội hơn như có khả năng tự khởi động, hoạt động ở những khu vực có đặc điểm gió đa hướng và phức tạp, có thể hoạt động với tốc độ gió thấp, và có tính linh hoạt và thích ứng cao. Với những lợi thế trên, tiềm năng ứng dụng tua-bin gió trục đứng trong môi trường đô thị ở Việt Nam là rất lớn.

Tuy nhiên, tua-bin gió Savonius nói riêng hay tua-bin gió trục đứng nói chung hiện nay vẫn tồn tại hạn chế lớn so với dòng trục ngang về mặt công suất. VAWT được đánh giá là có hiệu suất tương đối thấp và có xu hướng giảm mạnh ở tỉ tốc gió cao. Chính vì vậy, nhằm tối ưu hoá khả năng ứng dụng trong môi trường gió nhiễu loạn, các nghiên cứu cải thiện hiệu suất của tua-bin gió Savonius đang được đẩy mạnh.

Tại Việt Nam, nghiên cứu và phát triển tua-bin gió trục đứng đã đạt được những thành tựu đáng kể từ các nhóm nghiên cứu và cá nhân trong lĩnh vực này. Tiêu biểu là nhóm nghiên cứu của PGS.TS. Lê Đình Anh, với các giải pháp tối ưu hóa biên dạng cánh nhằm nâng cao hiệu suất khí động học của tua-bin Savonius, cùng nhiều công bố khoa học có giá trị. Nhà sáng chế Phạm Phú Uynh cũng đã nhận bằng sáng chế về thiết kế tua-bin gió trục đứng với các cải tiến độc đáo, mở ra tiềm năng ứng dụng thực tế. Các viện nghiên cứu như Viện Khoa học và Công nghệ Việt Nam (VAST) và các trường đại học như Đại học Bách Khoa Hà Nội, Đại học Công nghiệp TP.HCM đã phát triển các mẫu tua-bin hiệu suất cao, phù hợp với điều kiện gió đặc thù của Việt Nam, đồng thời đẩy mạnh ứng dụng mô phỏng CFD để tối ưu hóa thiết kế. Những thử nghiệm tua-bin gió quy mô nhỏ tại các khu vực ven biển và nông thôn cũng đã góp phần đánh giá tính khả thi và hiệu quả thực tế của công nghệ này. Những thành tựu trên không chỉ khẳng định tiềm năng phát triển tua-bin gió trục đứng tại Việt Nam mà còn thúc đẩy việc ứng dụng năng lượng tái tạo một cách bền vững, phù hợp với điều kiện kinh tế và môi trường trong nước.
