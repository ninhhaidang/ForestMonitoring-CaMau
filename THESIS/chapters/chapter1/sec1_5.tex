\section{Khu vực nghiên cứu}

\subsection{Vị trí địa lý}

Theo Nghị quyết số 1278/NQ-UBTVQH15 ngày 24/10/2024 của Ủy ban Thường vụ Quốc hội, kể từ ngày 01/07/2025, tỉnh Cà Mau và tỉnh Bạc Liêu được sáp nhập thành tỉnh Cà Mau mới. Tỉnh Cà Mau mới nằm ở cực Nam Tổ Quốc, thuộc vùng Đồng bằng sông Cửu Long; tọa độ địa lý nằm trong khoảng 8°36'–9°40' Bắc và 104°43'–105°50' Đông, diện tích tự nhiên là 7,942.38 km², dân số khoảng 2.6 triệu người, và chiều dài đường bờ biển khoảng 300 km (bao gồm cả bờ biển Bạc Liêu cũ).

\begin{figure}[H]
    \centering
    \fbox{\parbox{0.8\textwidth}{\centering\vspace{3cm}\textbf{[PLACEHOLDER]}\\ Bản đồ vị trí khu vực nghiên cứu:\\(a) Vị trí Cà Mau trong Việt Nam\\(b) Ranh giới tỉnh Cà Mau\\(c) Vùng rừng nghiên cứu với tọa độ UTM\vspace{3cm}}}
    \caption{Bản đồ vị trí khu vực nghiên cứu}
    \label{fig:study_area}
\end{figure}

\subsection{Phạm vi vùng nghiên cứu}

Đồ án tập trung vào toàn bộ vùng quy hoạch lâm nghiệp của tỉnh Cà Mau mới. Dữ liệu ranh giới quy hoạch lâm nghiệp được cung cấp bởi Công ty TNHH Tư vấn và Công nghệ Đồng Xanh — đối tác của Chi cục Kiểm lâm tỉnh Cà Mau.

Tổng diện tích ranh giới quy hoạch là 170,178.82 hecta (tương đương 1,701.79 km²), bao gồm 666 polygon trong file shapefile ranh giới. Diện tích thực tế được phân loại là 162,468.50 hecta (khoảng 95.5\% diện tích ranh giới); phần còn lại (~7,710 ha, chiếm 4.5\%) bị loại do mây che phủ hoặc dữ liệu không hợp lệ (nodata) trong quá trình xử lý ảnh vệ tinh. Kích thước raster là 12,547 × 10,917 pixels (ở độ phân giải 10m), sử dụng hệ quy chiếu EPSG:32648 (WGS 84 / UTM Zone 48N).
