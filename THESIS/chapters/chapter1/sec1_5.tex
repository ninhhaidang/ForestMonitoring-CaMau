\section{Nghiên cứu cải thiện hiệu suất tua-bin gió Savonius}

Trong những thập kỷ gần đây, tua-bin gió Savonius đã thu hút các nỗ lực nghiên cứu với các công trình được thực hiện nhằm nâng cao hiệu suất khí động học. Nhiều sáng kiến mới lạ đã được giới thiệu và có thể được chia thành hai cách tiếp cận chính, bao gồm tối ưu hóa các tham số hình học và sử dụng các hệ thống phụ trợ.

\begin{figure}[H]
    \centering
    \includegraphics[width=0.75\linewidth]{img/chapter1/2 approaches.jpg}
    \caption{Hai hướng nghiên cứu cải thiện hiệu suất tua-bin gió Savonius. (a) Cải tiến thiết kế cánh.\cite{minh_performance_enhancement_2023}; (b) Sử dụng hệ thống phụ trợ.\cite{Augmentation_Techniques}}
    \label{fig:enter-label}
    
\end{figure}

Đối với tua-bin gió Savonius được kết hợp các hệ thống phụ trợ, tuỳ thuộc vào độ phức tạp trong cấu trúc hay vị trí lắp đặt, công suất của tua-bin có thể được cải thiện rất lớn, thậm chí có thể vượt qua ngưỡng công suất tối đa của định luật Betz. Cụ thể, ý tưởng sử dụng cánh hướng dòng của nhóm nghiên cứu El-Askary (2015)\cite{elaskary_harvesting_wind_energy_2015} nhằm hướng trực tiếp gió vào cánh chủ động của tua-bin, đồng thời, chặn gió vào cánh bị động, đã nâng cao hệ số công suất $C_P$ lên tới 136\%. Nhóm nghiên cứu Layeghmand (2020)\cite{layeghmand_improving_efficiency_2020} đã sử dụng các tấm chắn dạng airfoil để cải thiện hệ số công suất từ 0.23 lên 0.31 (khoảng 50\%). Mặc dù có tác dụng cải thiện đáng kể hiệu suất khí động của tua-bin gió Savonius nhưng cách tiếp cận này cũng tồn tại một số hạn chế trong việc duy trì các ưu điểm đặc trưng của dạng tua-bin này. Các thiết bị hỗ trợ được lắp đặt xung quanh tua-bin gió nhằm tác động vào nguồn gió đầu vào sẽ phụ thuộc vào hướng gió và làm mất đi tính đa hướng của tua-bin. Ngoài ra, một số hệ thống phụ trợ có quy mô lớn, thiết kế phức tạp sẽ làm tăng kích thước vùng hoạt động của tua-bin, khiến việc lắp đặt, bảo trì trở nên khó khăn hơn và gia tăng chi phí thi công.

Chính vì vậy, các sáng kiến thiết kế mới dựa trên việc tối ưu hoá hay kết hợp các thông số hình học đang cho thấy những ưu điểm trong việc tiếp cận và nghiên cứu. Nhờ việc cải tiến các thiết kế cánh tua-bin Savonius, các ưu điểm đặc trưng của dòng tua-bin này như độ nhỏ gọn, tính đa hướng gần như được bảo toàn. Nhiều nhóm nghiên cứu trên thế giới đã công bố nhiều công trình liên quan tới các biên dạng cánh tua-bin mới có khả năng cải thiện công suất của tua-bin gió Savonius bằng cách gia tăng mô-men dương có lợi và hạn chế mô-men âm có hại tác động lên cánh tua-bin. Một số dạng cánh mới có thể được liệt kê như cánh Overlap, cánh Bach, cánh đa độ cong, đa độ dày, cánh được thiết kế lấy cảm hứng sinh học. Cụ thể, biên dạng cánh Overlap được phát triển từ biên dạng nguyên bản của Savonius đã được thiết kế và thực nghiệm bởi nhóm nghiên cứu của Blackwell\cite{blackwell_wind_tunnel_1977}. Sự cải tiến này đem đến sự cải thiện hệ số công suất $C_P$ lên đến 25.5\% ở tỉ lệ chồng là 0.1 so với rô-to ban đầu. Thiết kế mới của V. G. Bach\cite{bach_untersuchungen_1931} đã được cải tiến và khảo sát trên nhiều tỉ tốc bằng phương pháp thực nghiệm bởi nhóm nghiên cứu Roy \cite{roy_wind_tunnel_experiments_2015} và mô phỏng bởi nhóm Kacprzak \cite{kacprzak_numerical_investigation_2013}. Hệ số công suất $C_P$ lớn nhất tăng lên đến 30\% ở tỉ tốc 0.81. Việc sử dụng các cánh đa độ cong hay cánh elip cũng được khảo sát với sự cải thiện $C_P$ lên tới 84\% tại $\lambda$ = 1.0 nhờ thiết kế ứng dụng đường cong Bezier của nhóm nghiên cứu M. Zemamou \cite{zemamou_novel_blade_design_2020} và 185\% tại $\lambda$ = 1.5 nhờ thiết kế cánh phụ elip của nhóm tác giả Anh và Minh \cite{minh_performance_enhancement_2023}. Mặc dù đã có nhiều sự cải tiến nhờ các thiết kế mới nhưng trong đó vẫn còn tồn tại một số hạn chế. Cụ thể, đa số các biên dạng đều cải thiện hiệu suất của tua-bin ở mức tỉ tốc gió thấp dưới 1.0, làm giảm khả năng ứng dụng ở các khu vực độ thị có tỉ tốc cao. Một số thiết kế có các thông số hình học phức tạp, khó sản xuất thực tế và chỉ phù hợp cho nghiên cứu lý thuyết.

Một trong những giải pháp mới trong thiết kế đang thu hút sự chú ý gần đây khi các nhà nghiên cứu có thể ứng dụng biên dạng của cánh máy bay (airfoil) để đưa ra biên dạng cánh mới cho tua-bin gió Savonius. Ví dụ, nhóm nghiên cứu của Tartuferi \cite{tartuferi_enhancement_2015} đã sử dụng airfoil SR3345 để cải thiện hiệu suất ở tỉ tốc thấp, SR5050 cho tỉ tốc cao. Nhóm Khan\cite{khan_performance_enhancement_2022} đã sử dụng bề mặt của airfoil S1048 làm thiết kế mới và cho thấy hiệu suất tốt hơn 14\% so với thiết kế ban đầu ở $\lambda$ = 1.0. Đặc biệt, trong những năm trở lại đây, nhờ cảm hứng từ thế giới sinh học tự nhiên, các biên dạng cánh độc đáo đã được thiết kế dựa trên hình dạng của cá Koi (C. Ma \cite{ma_optimization_design_2024}, I. Hashem \cite{hashem_metamodeling_based_2021}), lươn cát (Hashem \cite{hashem_performance_investigation_2022}) có khả năng nâng cao hiệu suất của tua-bin gió Savonius.