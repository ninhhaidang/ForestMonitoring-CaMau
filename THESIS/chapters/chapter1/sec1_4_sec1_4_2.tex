\subsection{Tua-bin gió trục đứng (VAWT)}

Tua-bin gió trục đứng bao gồm rô-to cánh quạt được kết nối với trục thẳng đứng. Trong vài năm trở lại đây, VAWT đang dần được chú ý nhờ những ưu điểm trong thiết kế và công nghệ. Tua-bin gió trục đứng bao gồm hai loại là tua-bin Darrieus và tua-bin Savonius. Tua-bin Darrieus hoạt động nhờ lực nâng khí động học với các cánh quạt có dạng airfoil. Trong khi đó, tua-bin Savonius hoạt động dựa trên nguyên lý lực cản khí động học, bao gồm 2 cánh bán nguyệt nửa hình trụ được gắn với 1 trục thẳng đứng. Tua-bin trục đứng là loại đa hướng, chúng hiệu quả khi gặp thời tiết gió mạnh, từ nhiều hướng. Dạng tua-bin này phù hợp với những khu vực có địa hình phức tạp, tốc độ gió không ổn định như vùng núi, vùng đô thị,…

\begin{figure}[H]
    \centering
    \includegraphics[width=0.75\linewidth]{img/chapter1/VAWT.jpg}
    \caption{Tua-bin gió trục đứng (VAWT) \cite{typma_maglev_wind_generator}}
    \label{fig:enter-label}
    
\end{figure}

Hiện nay, cư dân tại các vùng đô thị và sinh sống tại các vùng có địa hình phức tạp gặp phải những hạn chế trong quá trình tiếp cận với các nguồn năng lượng tái tạo. Đối với năng lượng gió, các kế hoạch tiếp cận này phần lớn được thực hiện bởi các công ty điện lực mua từ các cơ sở sản xuất gió ở các vùng nông thôn. Điều này dẫn đến thách thức trong việc phát triển sản xuất năng lượng gió trong môi trường đô thị.

Một trong những tiến bộ công nghệ chính làm tăng khả năng ứng dụng của năng lượng gió trong môi trường đô thị là cải tiến thiết kế tua-bin gió trục đứng. VAWT có các đặc điểm có thể khiến chúng vượt trội hơn so với HAWT trong các ứng dụng đô thị. Các đặc điểm đó bao gồm kết cấu nhỏ gọn, dễ dàng vận chuyển, lắp đặt, vận hành và bảo dưỡng. Với cách vận hành không phụ thuộc vào hướng gió, VAWT có thể hoạt động ở dải gió rộng, phù hợp lắp đặt ở không gian nhỏ hẹp. Điều này là thuận lợi do tính chất không ổn định của gió tại các địa hình khác nhau. Không giống như HAWT, VAWT không cần máy phát điện và hộp số phải lắp gần các cánh quạt và có thể được đặt ở chân đế, cải thiện cả cấu hình của tua-bin và khả năng tiếp cận để bảo trì. 

Trong các loại tua-bin gió trục đứng, tua-bin gió Savonius gần đây đang thu hút sự chú ý của các nhóm nghiên cứu và được đánh giá là mô hình tiềm năng để ứng dụng khai thác năng lượng gió vùng đô thị. Được phát minh và đăng ký bằng sáng chế bởi kỹ sư người Phần Lan Sigurd Johannes Savonius, thiết kế của tua-bin Savonius nguyên bản bao gồm chia một khối trụ thành hai nửa theo mặt phẳng trung tâm và di chuyển những nửa trụ này sang hai bên theo mặt cắt để tạo ra một hình chữ S trên mặt cắt ngang. Mặc dù so với Darrieus, tua-bin gió Savonius vẫn còn hạn chế về mặt hiệu suất nhưng lại dễ thiết kế và lắp đặt hơn. Ngoài ra, tua-bin Savonius còn có một số ưu điểm vượt trội hơn như có khả năng tự khởi động, hoạt động ở những khu vực có đặc điểm gió đa hướng và phức tạp, có thể hoạt động với tốc độ gió thấp, và có tính linh hoạt và thích ứng cao. Với những lợi thế trên, tiềm năng ứng dụng tua-bin gió trục đứng trong môi trường đô thị ở Việt Nam là rất lớn.

Tuy nhiên, tua-bin gió Savonius nói riêng hay tua-bin gió trục đứng nói chung hiện nay vẫn tồn tại hạn chế lớn so với dòng trục ngang về mặt công suất. VAWT được đánh giá là có hiệu suất tương đối thấp và có xu hướng giảm mạnh ở tỉ tốc gió cao. Chính vì vậy, nhằm tối ưu hoá khả năng ứng dụng trong môi trường gió nhiễu loạn, các nghiên cứu cải thiện hiệu suất của tua-bin gió Savonius đang được đẩy mạnh.

Tại Việt Nam, nghiên cứu và phát triển tua-bin gió trục đứng đã đạt được những thành tựu đáng kể từ các nhóm nghiên cứu và cá nhân trong lĩnh vực này. Tiêu biểu là nhóm nghiên cứu của PGS.TS. Lê Đình Anh, với các giải pháp tối ưu hóa biên dạng cánh nhằm nâng cao hiệu suất khí động học của tua-bin Savonius, cùng nhiều công bố khoa học có giá trị. Nhà sáng chế Phạm Phú Uynh cũng đã nhận bằng sáng chế về thiết kế tua-bin gió trục đứng với các cải tiến độc đáo, mở ra tiềm năng ứng dụng thực tế. Các viện nghiên cứu như Viện Khoa học và Công nghệ Việt Nam (VAST) và các trường đại học như Đại học Bách Khoa Hà Nội, Đại học Công nghiệp TP.HCM đã phát triển các mẫu tua-bin hiệu suất cao, phù hợp với điều kiện gió đặc thù của Việt Nam, đồng thời đẩy mạnh ứng dụng mô phỏng CFD để tối ưu hóa thiết kế. Những thử nghiệm tua-bin gió quy mô nhỏ tại các khu vực ven biển và nông thôn cũng đã góp phần đánh giá tính khả thi và hiệu quả thực tế của công nghệ này. Những thành tựu trên không chỉ khẳng định tiềm năng phát triển tua-bin gió trục đứng tại Việt Nam mà còn thúc đẩy việc ứng dụng năng lượng tái tạo một cách bền vững, phù hợp với điều kiện kinh tế và môi trường trong nước.
