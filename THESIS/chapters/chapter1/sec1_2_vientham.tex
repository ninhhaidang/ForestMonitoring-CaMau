\section{Viễn thám}

\subsection{Nguyên lý viễn thám}

Viễn thám là khoa học và kỹ thuật thu thập thông tin về một đối tượng hoặc khu vực từ xa, thường thông qua việc ghi nhận bức xạ điện từ phản xạ hoặc phát ra từ bề mặt Trái Đất \cite{lillesand2015}. Nguyên lý cơ bản của viễn thám dựa trên tương tác giữa bức xạ điện từ và các đối tượng trên bề mặt. Dựa vào nguồn năng lượng sử dụng, viễn thám được chia thành hai loại chính: viễn thám bị động (passive remote sensing) và viễn thám chủ động (active remote sensing).

Trong hệ thống viễn thám bị động, nguồn năng lượng chính là bức xạ từ Mặt Trời. Khi các sóng này truyền qua khí quyển, một phần năng lượng bị hấp thụ hoặc tán xạ. Sau đó bức xạ tương tác với bề mặt, chịu các quá trình phản xạ, hấp thụ hoặc truyền qua tùy theo đặc tính vật liệu. Tín hiệu phản xạ được vệ tinh ghi nhận bởi cảm biến và được xử lý, truyền về trạm mặt đất để phục vụ phân tích. Nguyên lý cân bằng năng lượng cho thấy năng lượng tới bằng tổng năng lượng phản xạ, hấp thụ và truyền qua.

Khác với viễn thám bị động, hệ thống viễn thám chủ động tự phát ra nguồn năng lượng điện từ hướng về phía mục tiêu và ghi nhận tín hiệu phản xạ ngược từ bề mặt. Ưu điểm chính của viễn thám chủ động là khả năng hoạt động độc lập với ánh sáng Mặt Trời, cho phép thu thập dữ liệu cả ngày lẫn đêm và trong mọi điều kiện thời tiết, kể cả khi có mây che phủ. Radar khẩu độ tổng hợp là ví dụ điển hình của công nghệ viễn thám chủ động, sử dụng sóng vi ba có khả năng xuyên qua mây.

\begin{figure}[H]
    \centering
    \includegraphics[width=0.95\textwidth]{img/chapter2/Vien-tham.png}
    \caption{Nguyên lý viễn thám bị động và chủ động}
    \label{fig:remote_sensing_principle}
\end{figure}

Mỗi loại viễn thám có những ưu nhược điểm riêng. Viễn thám bị động, tiêu biểu như Sentinel-2, cung cấp ảnh quang học đa phổ với độ phân giải không gian và phổ cao, phù hợp cho phân loại lớp phủ đất chi tiết, nhưng bị hạn chế bởi mây và điều kiện chiếu sáng. Trong khi đó, viễn thám chủ động như Sentinel-1 hoạt động trong mọi điều kiện thời tiết, cung cấp thông tin về cấu trúc và độ ẩm bề mặt, tuy nhiên dữ liệu khó diễn giải hơn so với ảnh quang học. Sự kết hợp cả hai loại dữ liệu cho phép tận dụng ưu điểm của từng nguồn, đặc biệt quan trọng trong giám sát rừng nhiệt đới nơi mây che phủ thường xuyên.

\subsection{Chương trình Copernicus và vệ tinh Sentinel}

Chương trình Copernicus của Liên minh Châu Âu là hệ thống quan sát Trái Đất toàn diện, cung cấp dữ liệu viễn thám miễn phí phục vụ nghiên cứu khoa học và giám sát môi trường. Trong nghiên cứu này, dữ liệu từ hai vệ tinh Sentinel-1 và Sentinel-2 được sử dụng nhờ tính chất bổ sung giữa cảm biến ra-đa và quang học.

Sentinel-1 là vệ tinh mang cảm biến ra-đa khẩu độ tổng hợp hoạt động ở dải sóng C-band với tần số 5.405 GHz (bước sóng 5.55 cm). Trong chế độ IW, Sentinel-1 thu nhận dữ liệu ở hai chế độ phân cực VV và VH với độ phân giải không gian 10m và độ rộng dải quét 250 km \cite{esa2024s1}. Nguyên lý hoạt động của ra-đa dựa trên việc phát xung sóng vi ba và ghi nhận tín hiệu tán xạ ngược từ bề mặt. Cường độ tán xạ ngược phụ thuộc vào độ nhám bề mặt, hằng số điện môi và cấu trúc đối tượng. Đối với thảm thực vật, phân cực VV nhạy với tán xạ bề mặt liên quan đến độ ẩm đất, trong khi VH nhạy với tán xạ thể tích (volume scattering) từ cấu trúc tán lá \cite{torres2012}. Ưu điểm của ra-đa là khả năng xuyên qua mây và hoạt động độc lập với điều kiện chiếu sáng.

Sentinel-2 mang cảm biến quang học đa phổ thu nhận ảnh ở 13 dải phổ từ vùng nhìn thấy đến hồng ngoại sóng ngắn (443--2190 nm). Độ phân giải không gian thay đổi theo dải phổ: 10m cho các dải B2, B3, B4, B8; 20m cho B5, B6, B7, B8a, B11, B12; và 60m cho B1, B9, B10 \cite{esa2024s2}. Với 13 dải phổ bao phủ nhiều vùng bước sóng khác nhau, có thể phân biệt các loại lớp phủ đất dựa trên đặc tính phản xạ riêng biệt của từng đối tượng \cite{khatami2016}. Dữ liệu đa phổ này đặc biệt phù hợp để tính toán các chỉ số thực vật phục vụ giám sát biến động rừng. Bảng \ref{tab:sentinel_comparison} tổng hợp các thông số kỹ thuật chính của hai vệ tinh này.

\begin{table}[H]
\centering
\caption{So sánh thông số kỹ thuật giữa Sentinel-1 và Sentinel-2}
\label{tab:sentinel_comparison}
\begin{tabular}{|l|l|l|}
\hline
\textbf{Thông số} & \textbf{Sentinel-1} & \textbf{Sentinel-2} \\
\hline
Loại cảm biến & Ra-đa (chủ động) & Quang học (bị động) \\
\hline
Dải sóng & C-band (5.55 cm) & 443--2190 nm \\
\hline
Số kênh/phân cực & 2 (VV, VH) & 13 dải phổ \\
\hline
Độ phân giải không gian & 10m (IW mode) & 10/20/60m \\
\hline
Độ rộng dải quét & 250 km & 290 km \\
\hline
Chu kỳ quay lại & 6--12 ngày & 5--10 ngày \\
\hline
Hoạt động qua mây & Có & Không \\
\hline
Thông tin thu nhận & Cấu trúc, độ ẩm, độ nhám & Phản xạ phổ, chỉ số thực vật \\
\hline
\end{tabular}
\end{table}

Sự kết hợp dữ liệu từ hai nguồn cảm biến mang lại lợi thế quan trọng trong giám sát rừng nhiệt đới. SAR cung cấp dữ liệu liên tục trong điều kiện mây che phủ thường xuyên, đồng thời nhạy với biến đổi cấu trúc rừng như chặt phá cây. Trong khi đó, dữ liệu quang học cung cấp thông tin quang phổ phong phú để tính toán các chỉ số thực vật và đánh giá sức khỏe thảm thực vật. Các nghiên cứu đã chứng minh việc tích hợp SAR và quang học giúp cải thiện độ chính xác phân loại từ 5--15\% so với sử dụng đơn nguồn dữ liệu \cite{ienco2019, hu2020}.

\subsection{Các chỉ số thực vật viễn thám}

Các chỉ số thực vật (Vegetation Indices) là các công thức toán học kết hợp giá trị phản xạ từ các kênh phổ khác nhau nhằm tăng cường thông tin về thực vật và giảm nhiễu từ các yếu tố khác như đất và khí quyển. Nghiên cứu này sử dụng ba chỉ số chính là NDVI, NBR và NDMI, mỗi chỉ số khai thác các đặc tính phổ khác nhau để cung cấp thông tin bổ sung về trạng thái thực vật.

Chỉ số thực vật chuẩn hóa NDVI (Normalized Difference Vegetation Index) là chỉ số được sử dụng phổ biến nhất để đánh giá sức khỏe và mật độ thực vật \citeen{rouse1974}, được tính theo công thức:
\begin{equation}
NDVI = \frac{NIR - Red}{NIR + Red} = \frac{B8 - B4}{B8 + B4}
\label{eq:ndvi}
\end{equation}
Chỉ số này sử dụng giá trị phản xạ ở dải cận hồng ngoại $NIR$ (B8) và dải đỏ $Red$ (B4), khai thác sự khác biệt giữa phản xạ cao ở NIR và phản xạ thấp ở Red của thực vật xanh để định lượng mật độ và sức khỏe thực vật. Giá trị NDVI cao trong khoảng 0.3 đến 0.8 chỉ ra thực vật xanh khỏe mạnh, giá trị thấp từ 0.1 đến 0.2 tương ứng với đất trống, còn giá trị âm thường là nước.

Chỉ số cháy chuẩn hóa NBR (Normalized Burn Ratio) được thiết kế ban đầu để phát hiện khu vực cháy rừng, nhưng cũng cho thấy hiệu quả trong phát hiện mất rừng \citeen{key2006}. Công thức tính NBR như sau:
\begin{equation}
NBR = \frac{NIR - SWIR2}{NIR + SWIR2} = \frac{B8 - B12}{B8 + B12}
\label{eq:nbr}
\end{equation}
Chỉ số này sử dụng giá trị phản xạ ở dải cận hồng ngoại $NIR$ (B8) và dải hồng ngoại sóng ngắn 2 $SWIR2$ (B12, 2190 nm). Kênh SWIR nhạy với độ ẩm và cấu trúc thực vật, do đó khi rừng bị phá hoặc cháy, NIR giảm do mất lá xanh trong khi SWIR tăng do bề mặt khô hơn. NBR cao trong khoảng 0.3 đến 0.8 chỉ ra rừng khỏe mạnh, còn NBR thấp hoặc âm chỉ ra khu vực bị tác động.

Chỉ số độ ẩm chuẩn hóa NDMI (Normalized Difference Moisture Index) đo lường hàm lượng nước trong tán lá thực vật và là chỉ số quan trọng để phát hiện stress thực vật \citeen{gao1996}. NDMI được tính theo công thức:
\begin{equation}
NDMI = \frac{NIR - SWIR1}{NIR + SWIR1} = \frac{B8 - B11}{B8 + B11}
\label{eq:ndmi}
\end{equation}
Chỉ số này sử dụng giá trị phản xạ ở dải cận hồng ngoại $NIR$ (B8) và dải hồng ngoại sóng ngắn 1 $SWIR1$ (B11, 1610 nm). Do nước hấp thụ mạnh ở dải SWIR1, thực vật có hàm lượng nước cao sẽ cho phản xạ SWIR thấp. NDMI dương trong khoảng 0.2 đến 0.6 chỉ ra thực vật có hàm lượng nước cao, còn NDMI thấp hoặc âm chỉ ra thực vật bị stress hoặc khô. Đặc biệt, NDMI có thể phát hiện sự thay đổi độ ẩm trước khi NDVI phản ánh sự suy giảm sức khỏe thực vật, giúp cảnh báo sớm tình trạng suy thoái rừng.

\begin{table}[H]
\centering
\caption{Tổng hợp các chỉ số thực vật sử dụng trong nghiên cứu}
\label{tab:vegetation_indices}
\begin{tabular}{|l|c|l|l|}
\hline
\textbf{Chỉ số} & \textbf{Công thức} & \textbf{Phạm vi} & \textbf{Ý nghĩa} \\
\hline
NDVI & (B8-B4)/(B8+B4) & [-1, 1] & Mật độ, sức khỏe thực vật \\
\hline
NBR & (B8-B12)/(B8+B12) & [-1, 1] & Phát hiện cháy/mất rừng \\
\hline
NDMI & (B8-B11)/(B8+B11) & [-1, 1] & Độ ẩm tán lá \\
\hline
\end{tabular}
\end{table}

\subsection{Phát hiện biến động rừng và tích hợp đa nguồn dữ liệu}

Phát hiện biến động rừng (Forest Change Detection) là quá trình xác định sự thay đổi về diện tích, cấu trúc hoặc trạng thái của rừng giữa hai hoặc nhiều thời điểm khác nhau \citeen{huang2021}. Phương pháp này dựa trên việc so sánh các đặc trưng viễn thám thu được từ các thời điểm khác nhau.

Phương pháp phát hiện biến động dựa trên việc so sánh các đặc trưng giữa hai thời điểm, tính toán sự chênh lệch (delta) giữa ``after features'' và ``before features''. Temporal features bao gồm các ``before features'' thể hiện trạng thái rừng tại thời điểm $t_1$, các ``after features'' thể hiện trạng thái rừng tại thời điểm $t_2$, và các ``delta features'' biểu diễn biến đổi giữa hai thời điểm.

Ví dụ với NDVI, khi $\Delta NDVI$ (hiệu NDVI sau và trước) giảm mạnh thì đó là dấu hiệu mất rừng, khi $\Delta NDVI$ xấp xỉ 0 thì vùng được xem là rừng ổn định, và khi $\Delta NDVI$ tăng mạnh thì biểu hiện tái trồng rừng.

Việc tích hợp dữ liệu ra-đa khẩu độ tổng hợp từ Sentinel-1 và dữ liệu quang học đa phổ từ Sentinel-2 mang lại nhiều lợi ích quan trọng trong giám sát rừng. Hai nguồn dữ liệu này có tính chất bổ sung cho nhau: ra-đa cung cấp thông tin về cấu trúc, độ nhám bề mặt và độ ẩm, trong khi quang học cung cấp thông tin quang phổ và các chỉ số thực vật. Sự kết hợp này còn giúp khắc phục hạn chế của từng loại cảm biến, trong đó ra-đa hoạt động hiệu quả trong điều kiện mây mù thường xuyên của rừng nhiệt đới, còn quang học cung cấp dữ liệu trực quan dễ phiên giải hơn. Nhiều nghiên cứu đã chứng minh rằng việc kết hợp ra-đa và quang học giúp tăng độ chính xác từ 5 đến 10\% so với việc sử dụng mỗi nguồn dữ liệu riêng lẻ \citeen{ienco2019, hu2020}. Ngoài ra, sự tích hợp này cho phép phát hiện biến động đa chiều: ra-đa nhạy với biến đổi cấu trúc như chặt cây, trong khi quang học nhạy với biến đổi quang phổ phản ánh sức khỏe thực vật.
