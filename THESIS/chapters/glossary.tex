\phantomsection
\addcontentsline{toc}{chapter}{DANH MỤC THUẬT NGỮ VÀ TỪ VIẾT TẮT}
\unnumberedchapter{DANH MỤC THUẬT NGỮ VÀ TỪ VIẾT TẮT}

\begin{longtable}{p{5cm} c p{6cm}}
\centering\textbf{Từ} & \textbf{Viết tắt} & \centering\arraybackslash\textbf{Dịch - Nghĩa} \\[0.3cm]
Ablation Study & - & Nghiên cứu loại trừ (phân tích ảnh hưởng từng thành phần) \\
Accuracy & - & Độ chính xác tổng thể (tỷ lệ dự đoán đúng trên tổng số mẫu) \\
Activation Function & - & Hàm kích hoạt (hàm phi tuyến trong mạng nơ-ron) \\
AdamW & - & Thuật toán tối ưu Adam với phân rã trọng số tách rời \\
Artificial Intelligence & AI & Trí tuệ nhân tạo \\
Backscatter & - & Tán xạ ngược (năng lượng ra-đa phản hồi về cảm biến) \\
Batch Normalization & BN & Chuẩn hóa theo lô (chuẩn hóa đầu vào mỗi lớp theo mini-batch) \\
Bi-temporal & - & Hai thời điểm (phân tích so sánh hai thời kỳ) \\
C-band & - & Băng tần C (bước sóng ra-đa 4-8 GHz, xuyên qua mây) \\
Classification & - & Phân loại (gán nhãn cho đối tượng) \\
Compute Unified Device Architecture & CUDA & Nền tảng tính toán song song của NVIDIA trên GPU \\
Confusion Matrix & - & Ma trận nhầm lẫn (bảng so sánh dự đoán và thực tế) \\
Convolutional Neural Network & CNN & Mạng nơ-ron tích chập \\
Cross Validation & CV & Kiểm định chéo (đánh giá mô hình trên nhiều phần dữ liệu) \\
CrossEntropyLoss & - & Hàm mất mát entropy chéo (đo độ sai khác phân phối xác suất) \\
Deep Learning & DL & Học sâu (học máy với nhiều lớp ẩn) \\
Deforestation & - & Mất rừng (chuyển đổi từ rừng sang phi rừng) \\
Dropout & - & Loại bỏ ngẫu nhiên (tắt ngẫu nhiên một số nơ-ron khi huấn luyện) \\
Early Stopping & - & Dừng sớm (ngừng huấn luyện khi validation loss không giảm) \\
European Space Agency & ESA & Cơ quan Vũ trụ Châu Âu \\
European Union & EU & Liên minh Châu Âu \\
F1-Score & - & Điểm F1 (trung bình điều hòa của độ chuẩn xác và độ phủ) \\
False Negative & FN & Âm tính giả (dự đoán âm nhưng thực tế dương) \\
False Positive & FP & Dương tính giả (dự đoán dương nhưng thực tế âm) \\
False Positive Rate & FPR & Tỷ lệ dương tính giả \\
Feature & - & Đặc trưng (thuộc tính đầu vào của mô hình) \\
Feature Extraction & - & Trích xuất đặc trưng (rút trích thông tin từ dữ liệu thô) \\
Feature Map & - & Bản đồ đặc trưng (đầu ra của lớp tích chập) \\
Kernel & - & Bộ lọc/Nhân tích chập (ma trận trọng số trượt qua ảnh) \\
Fold & - & Phần gập (một phần dữ liệu trong Cross Validation) \\
Food and Agriculture Organization & FAO & Tổ chức Lương thực và Nông nghiệp Liên Hợp Quốc \\
Forest Change Detection & - & Phát hiện biến động rừng \\
Fully Connected Layer & FC & Lớp kết nối đầy đủ (mỗi nơ-ron kết nối với tất cả nơ-ron lớp trước) \\
Geographic Information System & GIS & Hệ thống thông tin địa lý \\
Global Average Pooling & GAP & Gộp trung bình toàn cục (lấy trung bình toàn bộ bản đồ đặc trưng) \\
Global Forest Watch & GFW & Giám sát rừng toàn cầu \\
Graphics Processing Unit & GPU & Bộ xử lý đồ họa (phần cứng tính toán song song) \\
Ground Truth & - & Dữ liệu mẫu (nhãn thực tế dùng để huấn luyện và đánh giá mô hình, thường thu thập từ khảo sát thực địa hoặc giải đoán ảnh) \\
Hyperparameter & - & Siêu tham số (tham số cấu hình trước khi huấn luyện) \\
Interferometric Wide & IW & Chế độ giao thoa rộng (chế độ chụp chính của Sentinel-1) \\
Intergovernmental Panel on Climate Change & IPCC & Ủy ban Liên chính phủ về Biến đổi Khí hậu \\
Learning Rate & LR & Tốc độ học (bước cập nhật trọng số mỗi lần lặp) \\
Lightweight & - & Kiến trúc nhẹ (mô hình có ít tham số, tính toán nhanh) \\
Logits & - & Giá trị logit (đầu ra thô của mạng trước khi áp dụng softmax) \\
Machine Learning & ML & Học máy \\
Mangrove Forest & - & Rừng ngập mặn \\
Multi-Layer Perceptron & MLP & Perceptron đa lớp (mạng nơ-ron nhiều lớp kết nối đầy đủ) \\
Multispectral & - & Đa phổ (ảnh chụp ở nhiều dải bước sóng) \\
Near-Infrared & NIR & Cận hồng ngoại (bước sóng 0.7--1.4 $\mu$m) \\
Non-forest & - & Phi rừng (vùng không có rừng che phủ) \\
Normalized Burn Ratio & NBR & Chỉ số cháy chuẩn hóa (phát hiện vùng cháy rừng) \\
Normalized Difference Moisture Index & NDMI & Chỉ số độ ẩm chuẩn hóa (đánh giá độ ẩm thực vật) \\
Normalized Difference Vegetation Index & NDVI & Chỉ số thực vật chuẩn hóa (đánh giá mức độ xanh tươi) \\
Optical & - & Quang học (ảnh vệ tinh dùng ánh sáng khả kiến và hồng ngoại) \\
Overfitting & - & Quá khớp (mô hình học thuộc dữ liệu huấn luyện, kém tổng quát) \\
Padding & - & Đệm viền (thêm pixel xung quanh ảnh khi tích chập) \\
Parameter & - & Tham số (trọng số học được trong quá trình huấn luyện) \\
Patch & - & Mảnh ảnh (vùng ảnh nhỏ trích xuất từ ảnh gốc) \\
Pixel & - & Điểm ảnh (đơn vị nhỏ nhất của ảnh số) \\
Polarization & - & Phân cực (hướng dao động của sóng ra-đa: VV, VH) \\
Pooling & - & Gộp (giảm kích thước không gian của feature map) \\
Precision & - & Độ chuẩn xác (tỷ lệ dự đoán dương đúng trên tổng dự đoán dương) \\
Prediction & - & Dự đoán (kết quả đầu ra của mô hình) \\
Raster & - & Dữ liệu raster (ảnh dạng lưới điểm ảnh) \\
Recall & - & Độ phủ (tỷ lệ phát hiện đúng trên tổng số thực tế dương) \\
Receiver Operating Characteristic & ROC & Đường cong ROC (biểu đồ đánh giá khả năng phân loại) \\
Rectified Linear Unit & ReLU & Hàm ReLU (hàm kích hoạt: f(x) = max(0, x)) \\
Reforestation & - & Phục hồi rừng (tái sinh hoặc trồng lại rừng) \\
Regularization & - & Điều chuẩn (kỹ thuật giảm overfitting) \\
Remote Sensing & - & Viễn thám (thu thập thông tin từ xa qua vệ tinh/máy bay) \\
ROC-AUC & - & Diện tích dưới đường cong ROC (đo khả năng phân biệt lớp) \\
Scheduler & - & Bộ điều chỉnh tốc độ học (thay đổi learning rate theo epoch) \\
Short-Wave Infrared & SWIR & Hồng ngoại sóng ngắn (bước sóng 1.4--3 $\mu$m) \\
Softmax & - & Hàm softmax (chuyển logits thành phân phối xác suất) \\
Spectral Signature & - & Phổ phản xạ đặc trưng (đặc điểm phản xạ theo bước sóng) \\
Stratified & - & Phân tầng (chia dữ liệu giữ nguyên tỷ lệ các lớp) \\
Synthetic Aperture Radar & SAR & Ra-đa khẩu độ tổng hợp (cảm biến chủ động, hoạt động mọi thời tiết) \\
Test Set & - & Tập kiểm tra (dữ liệu đánh giá cuối cùng, không dùng khi huấn luyện) \\
True Positive Rate & TPR & Tỷ lệ dương tính thật (độ nhạy/độ phủ) \\
Training & - & Huấn luyện (quá trình học tham số từ dữ liệu) \\
Universal Transverse Mercator & UTM & Hệ tọa độ UTM (phép chiếu bản đồ chia thành 60 múi) \\
Validation & - & Xác thực (đánh giá mô hình trong quá trình huấn luyện) \\
Weight Decay & - & Phân rã trọng số (thêm phạt L2 vào hàm mất mát) \\
World Geodetic System & WGS & Hệ trắc địa thế giới (hệ tọa độ toàn cầu, WGS84) \\
\end{longtable}
