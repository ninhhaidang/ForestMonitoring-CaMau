\section{Chuẩn bị mẫu huấn luyện}

\subsection{Chuẩn hóa dữ liệu}

Việc chuẩn hóa dữ liệu là bước quan trọng để đảm bảo các features có cùng phạm vi giá trị, giúp quá trình huấn luyện mô hình hội tụ nhanh và ổn định hơn. Nghiên cứu này áp dụng phương pháp chuẩn hóa Z-score:

\begin{equation}
x_{normalized} = \frac{x - \mu}{\sigma}
\end{equation}

trong đó $x$ là giá trị gốc, $\mu$ là giá trị trung bình và $\sigma$ là độ lệch chuẩn.

Để đảm bảo tính khoa học và tránh hiện tượng rò rỉ dữ liệu, các tham số chuẩn hóa ($\mu$ và $\sigma$) được tính toán chỉ trên tập huấn luyện theo quy trình bốn bước. Đầu tiên, stratified split được thực hiện để tách 20\% dữ liệu làm tập test cố định trước khi tính toán bất kỳ thống kê nào. Tiếp theo, mean và std được tính cho từng feature trên tập training. Sau đó, các tham số đã tính được sử dụng để chuẩn hóa cả tập training, validation và test. Cuối cùng, các tham số được lưu lại để áp dụng cho dữ liệu mới khi dự đoán.

\subsection{Phân chia dữ liệu}

Chiến lược chia dữ liệu được thiết kế theo khuyến nghị của Roberts et al. \citeen{roberts2017} về cross-validation cho dữ liệu không gian, đảm bảo đánh giá khách quan. Quy trình chia dữ liệu bao gồm bốn bước. Bước thứ nhất, tách 20\% dữ liệu làm tập test cố định (526 mẫu) — tập này không được sử dụng trong quá trình huấn luyện hay tinh chỉnh siêu tham số. Bước thứ hai, áp dụng 5-Fold Cross Validation trên 80\% còn lại (2,104 mẫu) để tìm kiếm siêu tham số tối ưu và đánh giá độ ổn định của mô hình. Bước thứ ba, huấn luyện mô hình cuối cùng trên toàn bộ 80\% dữ liệu training. Bước thứ tư, đánh giá mô hình cuối cùng trên 20\% tập test để báo cáo kết quả.

\begin{table}[H]
\centering
\caption{Phân bố dữ liệu theo tập huấn luyện và kiểm tra}
\label{tab:data_split_info}
\begin{tabular}{|l|c|c|}
\hline
\textbf{Tập dữ liệu} & \textbf{Số mẫu} & \textbf{Tỷ lệ} \\
\hline
Training + Validation (5-Fold CV) & 2,104 & 80\% \\
\hline
Test (cố định) & 526 & 20\% \\
\hline
\textbf{Tổng} & \textbf{2,630} & \textbf{100\%} \\
\hline
\end{tabular}
\end{table}

Việc sử dụng stratified sampling đảm bảo tỷ lệ các lớp (Rừng ổn định, Mất rừng, Phi rừng, Phục hồi rừng) được duy trì đồng đều trong cả tập training và test.
