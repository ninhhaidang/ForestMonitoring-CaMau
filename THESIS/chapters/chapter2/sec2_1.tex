\section{Công nghệ viễn thám và ảnh vệ tinh}

\subsection{Nguyên lý viễn thám}

Viễn thám (Remote Sensing) là khoa học và kỹ thuật thu thập thông tin về một đối tượng hoặc khu vực từ xa, thường thông qua việc ghi nhận bức xạ điện từ phản xạ hoặc phát ra từ bề mặt Trái Đất. Nguyên lý cơ bản của viễn thám dựa trên tương tác giữa bức xạ điện từ và các đối tượng trên bề mặt.

\textbf{Quá trình viễn thám bị động (Passive Remote Sensing):}

Trong hệ thống viễn thám bị động, nguồn năng lượng chính là bức xạ từ Mặt Trời. Khi các sóng này truyền qua khí quyển, một phần năng lượng bị hấp thụ hoặc tán xạ. Sau đó bức xạ tương tác với bề mặt, chịu các quá trình phản xạ, hấp thụ hoặc truyền qua tùy theo đặc tính vật liệu. Tín hiệu phản xạ được vệ tinh ghi nhận bởi cảm biến và được xử lý, truyền về trạm mặt đất để phục vụ phân tích.

\textbf{Phương trình cân bằng năng lượng:}
\begin{equation}
E_{incident} = E_{reflected} + E_{absorbed} + E_{transmitted}
\end{equation}

Trong đó, $E_{incident}$ là năng lượng tới từ Mặt Trời, $E_{reflected}$ là phần năng lượng phản xạ được cảm biến ghi nhận, $E_{absorbed}$ là phần năng lượng bị hấp thụ và chuyển thành nhiệt, và $E_{transmitted}$ là phần năng lượng truyền qua vật chất.

\begin{figure}[H]
    \centering
    \fbox{\parbox{0.8\textwidth}{\centering\vspace{2cm}\textbf{[PLACEHOLDER]}\\ Sơ đồ minh họa nguyên lý viễn thám\\ bị động và chủ động\vspace{2cm}}}
    \caption{Nguyên lý viễn thám bị động và chủ động}
    \label{fig:remote_sensing_principle}
\end{figure}

\subsection{Radar khẩu độ tổng hợp (SAR)}

\textbf{Nguyên lý hoạt động:}

Khác với viễn thám bị động, SAR là hệ thống chủ động (active remote sensing): anten phát xung sóng điện từ về phía Trái Đất, các sóng này tương tác với bề mặt và tạo hiện tượng phản xạ ngược (backscatter) với cường độ phụ thuộc vào nhiều yếu tố như độ nhám bề mặt, hàm lượng nước (độ ẩm), hằng số điện môi và góc tới.

\textbf{Hệ số Backscatter ($\sigma^0$):}
\begin{equation}
\sigma^0 (dB) = 10 \times \log_{10}(\sigma^0_{linear})
\end{equation}

Giá trị $\sigma^0$ phụ thuộc vào nhiều yếu tố. Về độ nhám bề mặt, bề mặt nhẵn như nước cho $\sigma^0$ thấp, trong khi bề mặt nhám như rừng cho $\sigma^0$ cao. Về hàm lượng nước, độ ẩm làm tăng $\sigma^0$ do hằng số điện môi lớn của nước. Về cấu trúc thực vật, khu vực rừng có cấu trúc phức tạp thường cho backscatter mạnh.

\textbf{Polarization:}

SAR có thể phát và thu theo các chế độ phân cực khác nhau: VV (phát V, thu V) nhạy với độ ẩm bề mặt, VH (phát V, thu H) thường nhạy với cấu trúc thực vật (volume scattering).

\subsection{Ảnh quang học đa phổ (Optical Multispectral)}

\textbf{Dải phổ điện từ:}

Ảnh quang học ghi nhận bức xạ phản xạ từ bề mặt Trái Đất ở các dải phổ khác nhau. Dải nhìn thấy (VIS) có bước sóng 400–700 nm, bao gồm Blue (450–520 nm), Green (520–600 nm) và Red (630–690 nm). Dải cận hồng ngoại (NIR) có bước sóng 700–1400 nm, với đặc trưng phản xạ cao ở thực vật xanh do chlorophyll. Dải hồng ngoại sóng ngắn (SWIR) có bước sóng 1400–3000 nm, nhạy với độ ẩm của thực vật và đất.

\textbf{Chữ ký phổ (Spectral Signature):}

Mỗi loại đối tượng có chữ ký phổ đặc trưng - mẫu phản xạ qua các dải phổ:
\begin{equation}
S = [\rho(\lambda_1), \rho(\lambda_2), ..., \rho(\lambda_n)]
\end{equation}

Ví dụ: thực vật xanh có phản xạ thấp ở dải Red (hấp thụ bởi chlorophyll) và phản xạ cao ở dải NIR; đất trống có phản xạ trung bình và tăng dần theo bước sóng; nước có phản xạ thấp ở hầu hết các dải, đặc biệt là NIR và SWIR.

\begin{table}[H]
\centering
\caption{Các dải phổ Sentinel-2 sử dụng trong nghiên cứu}
\label{tab:s2_bands}
\begin{tabular}{|c|l|c|c|l|}
\hline
\textbf{Band} & \textbf{Tên} & \textbf{Bước sóng (nm)} & \textbf{Độ phân giải (m)} & \textbf{Ứng dụng} \\
\hline
B4 & Red & 665 & 10 & Chlorophyll absorption \\
\hline
B8 & NIR & 842 & 10 & Biomass, NDVI \\
\hline
B11 & SWIR1 & 1610 & 20 & Độ ẩm, NDMI \\
\hline
B12 & SWIR2 & 2190 & 20 & NBR \\
\hline
\end{tabular}
\end{table}

\subsection{Phát hiện biến động rừng}

\textbf{Change Detection Approach:}
\begin{equation}
\Delta Feature = Feature_{after} - Feature_{before}
\end{equation}

\textbf{Temporal Features:}

Temporal features bao gồm các ``before features'' thể hiện trạng thái rừng tại thời điểm $t_1$, các ``after features'' thể hiện trạng thái rừng tại thời điểm $t_2$, và các ``delta features'' biểu diễn biến đổi giữa hai thời điểm ($t_2 - t_1$).

\textbf{Ví dụ với NDVI:}
\begin{equation}
\Delta NDVI = NDVI_{after} - NDVI_{before}
\end{equation}

Khi $\Delta NDVI$ giảm mạnh (rất nhỏ hơn 0) thì đó là dấu hiệu mất rừng; khi $\Delta NDVI$ xấp xỉ 0 thì vùng được xem là rừng ổn định; và khi $\Delta NDVI$ tăng mạnh (rất lớn hơn 0) thì biểu hiện tái trồng rừng.