\section{Công nghệ viễn thám và ảnh vệ tinh}

\subsection{Nguyên lý viễn thám và các loại dữ liệu vệ tinh}

Viễn thám (Remote Sensing) là khoa học và kỹ thuật thu thập thông tin về một đối tượng hoặc khu vực từ xa, thường thông qua việc ghi nhận bức xạ điện từ phản xạ hoặc phát ra từ bề mặt Trái Đất \citeen{lillesand2015}. Nguyên lý cơ bản của viễn thám dựa trên tương tác giữa bức xạ điện từ và các đối tượng trên bề mặt.

\textbf{Quá trình viễn thám bị động (Passive Remote Sensing):}

Trong hệ thống viễn thám bị động, nguồn năng lượng chính là bức xạ từ Mặt Trời. Khi các sóng này truyền qua khí quyển, một phần năng lượng bị hấp thụ hoặc tán xạ. Sau đó bức xạ tương tác với bề mặt, chịu các quá trình phản xạ, hấp thụ hoặc truyền qua tùy theo đặc tính vật liệu. Tín hiệu phản xạ được vệ tinh ghi nhận bởi cảm biến và được xử lý, truyền về trạm mặt đất để phục vụ phân tích. Nguyên lý cân bằng năng lượng cho thấy năng lượng tới bằng tổng năng lượng phản xạ, hấp thụ và truyền qua.

\textbf{Quá trình viễn thám chủ động (Active Remote Sensing):}

Khác với viễn thám bị động, hệ thống viễn thám chủ động tự phát ra nguồn năng lượng điện từ hướng về phía mục tiêu và ghi nhận tín hiệu phản xạ ngược (backscatter) từ bề mặt. Ưu điểm chính của viễn thám chủ động là khả năng hoạt động độc lập với ánh sáng Mặt Trời, cho phép thu thập dữ liệu cả ngày lẫn đêm và trong mọi điều kiện thời tiết (kể cả khi có mây che phủ). Radar khẩu độ tổng hợp (SAR) là ví dụ điển hình của công nghệ viễn thám chủ động, sử dụng sóng vi ba (microwave) có khả năng xuyên qua mây và mưa.

\textbf{So sánh viễn thám bị động và chủ động:}

Viễn thám bị động (như Sentinel-2) cung cấp ảnh quang học đa phổ với độ phân giải không gian và phổ cao, phù hợp cho phân loại lớp phủ đất chi tiết, nhưng bị hạn chế bởi mây và điều kiện chiếu sáng. Viễn thám chủ động (như Sentinel-1 SAR) hoạt động trong mọi điều kiện thời tiết, cung cấp thông tin về cấu trúc và độ ẩm bề mặt, nhưng khó diễn giải hơn. Sự kết hợp cả hai loại dữ liệu (data fusion) cho phép tận dụng ưu điểm của từng nguồn, đặc biệt quan trọng trong giám sát rừng nhiệt đới nơi mây che phủ thường xuyên.

\begin{figure}[H]
    \centering
    \includegraphics[width=0.95\textwidth]{img/chapter2/Vien-tham.png}
    \caption{Nguyên lý viễn thám bị động và chủ động}
    \label{fig:remote_sensing_principle}
\end{figure}

\textbf{Radar khẩu độ tổng hợp (SAR):}

SAR là công nghệ viễn thám chủ động đóng vai trò quan trọng trong giám sát rừng nhiệt đới nhờ khả năng quan sát trong mọi điều kiện thời tiết \citeen{esa2024s1, reiche2018}. SAR là hệ thống chủ động: anten phát xung sóng điện từ về phía Trái Đất, các sóng này tương tác với bề mặt và tạo hiện tượng phản xạ ngược (backscatter) với cường độ phụ thuộc vào nhiều yếu tố như độ nhám bề mặt, hàm lượng nước (độ ẩm), hằng số điện môi và góc tới. Hệ số Backscatter ($\sigma^0$) thường được biểu diễn theo đơn vị decibel (dB), với giá trị phụ thuộc vào nhiều yếu tố. Về độ nhám bề mặt, bề mặt nhẵn như nước cho $\sigma^0$ thấp, trong khi bề mặt nhám như rừng cho $\sigma^0$ cao. Về hàm lượng nước, độ ẩm làm tăng $\sigma^0$ do hằng số điện môi lớn của nước. Về cấu trúc thực vật, khu vực rừng có cấu trúc phức tạp thường cho backscatter mạnh. SAR có thể phát và thu theo các chế độ phân cực khác nhau: VV (phát V, thu V) nhạy với độ ẩm bề mặt, VH (phát V, thu H) thường nhạy với cấu trúc thực vật (volume scattering).

\textbf{Ảnh quang học đa phổ (Optical Multispectral):}

Ảnh quang học từ các vệ tinh như Sentinel-2 là nguồn dữ liệu quan trọng cho phân loại lớp phủ đất và giám sát thực vật \citeen{esa2024s2, khatami2016}. Cảm biến quang học ghi nhận bức xạ phản xạ từ bề mặt Trái Đất ở nhiều dải phổ khác nhau: dải nhìn thấy (VIS, 400–700 nm), dải cận hồng ngoại (NIR, 700–1400 nm) và dải hồng ngoại sóng ngắn (SWIR, 1400–3000 nm). Mỗi loại đối tượng có chữ ký phổ (spectral signature) đặc trưng --- thực vật xanh có phản xạ thấp ở dải đỏ và cao ở dải NIR, đất trống có phản xạ tăng dần theo bước sóng, còn nước có phản xạ thấp ở hầu hết các dải. Sự khác biệt này cho phép phân biệt các loại lớp phủ đất từ ảnh vệ tinh.

\subsection{Các chỉ số thực vật viễn thám}

Các chỉ số thực vật (Vegetation Indices) là các công thức toán học kết hợp giá trị phản xạ từ các kênh phổ khác nhau để tăng cường thông tin về thực vật và giảm nhiễu từ các yếu tố khác như đất, khí quyển. Nghiên cứu này sử dụng ba chỉ số chính:

\textbf{NDVI (Normalized Difference Vegetation Index):}

NDVI là chỉ số được sử dụng phổ biến nhất để đánh giá sức khỏe và mật độ thực vật \citeen{rouse1974}:
\begin{equation}
NDVI = \frac{NIR - Red}{NIR + Red} = \frac{B8 - B4}{B8 + B4}
\label{eq:ndvi}
\end{equation}

Chỉ số này sử dụng giá trị phản xạ ở dải cận hồng ngoại $NIR$ (B8) và dải đỏ $Red$ (B4), khai thác sự khác biệt giữa phản xạ cao ở NIR và phản xạ thấp ở Red của thực vật xanh để định lượng mật độ và sức khỏe thực vật. Giá trị NDVI cao (0.3--0.8) chỉ ra thực vật xanh khỏe mạnh, giá trị thấp (0.1--0.2) là đất trống, còn giá trị âm là nước.

\textbf{NBR (Normalized Burn Ratio):}

NBR được thiết kế ban đầu để phát hiện khu vực cháy rừng, nhưng cũng hiệu quả trong phát hiện mất rừng \citeen{key2006}:
\begin{equation}
NBR = \frac{NIR - SWIR2}{NIR + SWIR2} = \frac{B8 - B12}{B8 + B12}
\label{eq:nbr}
\end{equation}

Chỉ số này sử dụng giá trị phản xạ ở dải cận hồng ngoại $NIR$ (B8) và dải hồng ngoại sóng ngắn 2 $SWIR2$ (B12, 2190 nm). Kênh SWIR nhạy với độ ẩm và cấu trúc thực vật --- khi rừng bị phá hoặc cháy, NIR giảm do mất lá xanh trong khi SWIR tăng do bề mặt khô hơn. NBR cao (0.3--0.8) chỉ ra rừng khỏe mạnh, NBR thấp hoặc âm chỉ ra khu vực bị tác động.

\textbf{NDMI (Normalized Difference Moisture Index):}

NDMI đo lường hàm lượng nước trong tán lá thực vật, là chỉ số quan trọng để phát hiện stress thực vật \citeen{gao1996}:
\begin{equation}
NDMI = \frac{NIR - SWIR1}{NIR + SWIR1} = \frac{B8 - B11}{B8 + B11}
\label{eq:ndmi}
\end{equation}

Chỉ số này sử dụng giá trị phản xạ ở dải cận hồng ngoại $NIR$ (B8) và dải hồng ngoại sóng ngắn 1 $SWIR1$ (B11, 1610 nm). Kênh SWIR1 nhạy với hàm lượng nước trong lá --- nước hấp thụ mạnh ở dải này, nên thực vật có nhiều nước sẽ cho phản xạ SWIR thấp. NDMI dương (0.2--0.6) chỉ ra thực vật có hàm lượng nước cao, NDMI thấp hoặc âm chỉ ra thực vật bị stress hoặc khô. Đặc biệt, NDMI có thể phát hiện sự thay đổi độ ẩm trước khi NDVI phản ánh sự suy giảm sức khỏe thực vật.

\begin{table}[H]
\centering
\caption{Tổng hợp các chỉ số thực vật sử dụng trong nghiên cứu}
\label{tab:vegetation_indices}
\begin{tabular}{|l|c|l|l|}
\hline
\textbf{Chỉ số} & \textbf{Công thức} & \textbf{Phạm vi} & \textbf{Ý nghĩa} \\
\hline
NDVI & (B8-B4)/(B8+B4) & [-1, 1] & Mật độ, sức khỏe thực vật \\
\hline
NBR & (B8-B12)/(B8+B12) & [-1, 1] & Phát hiện cháy/mất rừng \\
\hline
NDMI & (B8-B11)/(B8+B11) & [-1, 1] & Độ ẩm tán lá \\
\hline
\end{tabular}
\end{table}

\subsection{Phát hiện biến động rừng}

Phát hiện biến động rừng (Forest Change Detection) là quá trình xác định sự thay đổi về diện tích, cấu trúc hoặc trạng thái của rừng giữa hai hoặc nhiều thời điểm khác nhau \citeen{huang2021}. Phương pháp này dựa trên việc so sánh các đặc trưng viễn thám thu được từ các thời điểm khác nhau.

Phương pháp phát hiện biến động dựa trên việc so sánh các đặc trưng giữa hai thời điểm, tính toán sự chênh lệch (delta) giữa ``after features'' và ``before features''. Temporal features bao gồm các ``before features'' thể hiện trạng thái rừng tại thời điểm $t_1$, các ``after features'' thể hiện trạng thái rừng tại thời điểm $t_2$, và các ``delta features'' biểu diễn biến đổi giữa hai thời điểm.

Ví dụ với NDVI: khi $\Delta NDVI$ (hiệu NDVI sau và trước) giảm mạnh thì đó là dấu hiệu mất rừng; khi $\Delta NDVI$ xấp xỉ 0 thì vùng được xem là rừng ổn định; và khi $\Delta NDVI$ tăng mạnh thì biểu hiện tái trồng rừng.