\section{Công nghệ viễn thám và ảnh vệ tinh}

\subsection{Nguyên lý viễn thám}

Viễn thám (Remote Sensing) là khoa học và kỹ thuật thu thập thông tin về một đối tượng hoặc khu vực từ xa, thường thông qua việc ghi nhận bức xạ điện từ phản xạ hoặc phát ra từ bề mặt Trái Đất \cite{lillesand2015}. Nguyên lý cơ bản của viễn thám dựa trên tương tác giữa bức xạ điện từ và các đối tượng trên bề mặt.

\textbf{Quá trình viễn thám bị động (Passive Remote Sensing):}

Trong hệ thống viễn thám bị động, nguồn năng lượng chính là bức xạ từ Mặt Trời. Khi các sóng này truyền qua khí quyển, một phần năng lượng bị hấp thụ hoặc tán xạ. Sau đó bức xạ tương tác với bề mặt, chịu các quá trình phản xạ, hấp thụ hoặc truyền qua tùy theo đặc tính vật liệu. Tín hiệu phản xạ được vệ tinh ghi nhận bởi cảm biến và được xử lý, truyền về trạm mặt đất để phục vụ phân tích.

\textbf{Phương trình cân bằng năng lượng:}
\begin{equation}
E_{incident} = E_{reflected} + E_{absorbed} + E_{transmitted}
\end{equation}

Trong đó, $E_{incident}$ là năng lượng tới từ Mặt Trời, $E_{reflected}$ là phần năng lượng phản xạ được cảm biến ghi nhận, $E_{absorbed}$ là phần năng lượng bị hấp thụ và chuyển thành nhiệt, và $E_{transmitted}$ là phần năng lượng truyền qua vật chất.

\textbf{Quá trình viễn thám chủ động (Active Remote Sensing):}

Khác với viễn thám bị động, hệ thống viễn thám chủ động tự phát ra nguồn năng lượng điện từ hướng về phía mục tiêu và ghi nhận tín hiệu phản xạ ngược (backscatter) từ bề mặt. Ưu điểm chính của viễn thám chủ động là khả năng hoạt động độc lập với ánh sáng Mặt Trời, cho phép thu thập dữ liệu cả ngày lẫn đêm và trong mọi điều kiện thời tiết (kể cả khi có mây che phủ). Radar khẩu độ tổng hợp (SAR) là ví dụ điển hình của công nghệ viễn thám chủ động, sử dụng sóng vi ba (microwave) có khả năng xuyên qua mây và mưa.

\textbf{So sánh viễn thám bị động và chủ động:}

Viễn thám bị động (như Sentinel-2) cung cấp ảnh quang học đa phổ với độ phân giải không gian và phổ cao, phù hợp cho phân loại lớp phủ đất chi tiết, nhưng bị hạn chế bởi mây và điều kiện chiếu sáng. Viễn thám chủ động (như Sentinel-1 SAR) hoạt động trong mọi điều kiện thời tiết, cung cấp thông tin về cấu trúc và độ ẩm bề mặt, nhưng khó diễn giải hơn. Sự kết hợp cả hai loại dữ liệu (data fusion) cho phép tận dụng ưu điểm của từng nguồn, đặc biệt quan trọng trong giám sát rừng nhiệt đới nơi mây che phủ thường xuyên.

\begin{figure}[H]
    \centering
    \includegraphics[width=0.9\textwidth]{img/chapter2/Vien-tham.png}
    \caption{Nguyên lý viễn thám bị động và chủ động}
    \label{fig:remote_sensing_principle}
\end{figure}

\subsection{Radar khẩu độ tổng hợp (SAR)}

SAR là công nghệ viễn thám chủ động đóng vai trò quan trọng trong giám sát rừng nhiệt đới nhờ khả năng quan sát trong mọi điều kiện thời tiết \cite{esa2024s1, reiche2018}.

\textbf{Nguyên lý hoạt động:}

Khác với viễn thám bị động, SAR là hệ thống chủ động (active remote sensing): anten phát xung sóng điện từ về phía Trái Đất, các sóng này tương tác với bề mặt và tạo hiện tượng phản xạ ngược (backscatter) với cường độ phụ thuộc vào nhiều yếu tố như độ nhám bề mặt, hàm lượng nước (độ ẩm), hằng số điện môi và góc tới.

\textbf{Hệ số Backscatter ($\sigma^0$):}
\begin{equation}
\sigma^0 (dB) = 10 \times \log_{10}(\sigma^0_{linear})
\end{equation}

Giá trị $\sigma^0$ phụ thuộc vào nhiều yếu tố. Về độ nhám bề mặt, bề mặt nhẵn như nước cho $\sigma^0$ thấp, trong khi bề mặt nhám như rừng cho $\sigma^0$ cao. Về hàm lượng nước, độ ẩm làm tăng $\sigma^0$ do hằng số điện môi lớn của nước. Về cấu trúc thực vật, khu vực rừng có cấu trúc phức tạp thường cho backscatter mạnh.

\textbf{Polarization:}

SAR có thể phát và thu theo các chế độ phân cực khác nhau: VV (phát V, thu V) nhạy với độ ẩm bề mặt, VH (phát V, thu H) thường nhạy với cấu trúc thực vật (volume scattering).

\subsection{Ảnh quang học đa phổ (Optical Multispectral)}

Ảnh quang học từ các vệ tinh như Sentinel-2 là nguồn dữ liệu quan trọng cho phân loại lớp phủ đất và giám sát thực vật \cite{esa2024s2, khatami2016}.

\textbf{Dải phổ điện từ:}

Ảnh quang học ghi nhận bức xạ phản xạ từ bề mặt Trái Đất ở các dải phổ khác nhau. Dải nhìn thấy (VIS) có bước sóng 400–700 nm, bao gồm Blue (450–520 nm), Green (520–600 nm) và Red (630–690 nm). Dải cận hồng ngoại (NIR) có bước sóng 700–1400 nm, với đặc trưng phản xạ cao ở thực vật xanh do chlorophyll. Dải hồng ngoại sóng ngắn (SWIR) có bước sóng 1400–3000 nm, nhạy với độ ẩm của thực vật và đất.

\textbf{Chữ ký phổ (Spectral Signature):}

Mỗi loại đối tượng có chữ ký phổ đặc trưng - mẫu phản xạ qua các dải phổ:
\begin{equation}
S = [\rho(\lambda_1), \rho(\lambda_2), ..., \rho(\lambda_n)]
\end{equation}

Ví dụ: thực vật xanh có phản xạ thấp ở dải Red (hấp thụ bởi chlorophyll) và phản xạ cao ở dải NIR; đất trống có phản xạ trung bình và tăng dần theo bước sóng; nước có phản xạ thấp ở hầu hết các dải, đặc biệt là NIR và SWIR.

\begin{table}[H]
\centering
\caption{Các dải phổ Sentinel-2 sử dụng trong nghiên cứu}
\label{tab:s2_bands}
\begin{tabular}{|c|l|c|c|l|}
\hline
\textbf{Band} & \textbf{Tên} & \textbf{Bước sóng (nm)} & \textbf{Độ phân giải (m)} & \textbf{Ứng dụng} \\
\hline
B4 & Red & 665 & 10 & Chlorophyll absorption \\
\hline
B8 & NIR & 842 & 10 & Biomass, NDVI \\
\hline
B11 & SWIR1 & 1610 & 20 & Độ ẩm, NDMI \\
\hline
B12 & SWIR2 & 2190 & 20 & NBR \\
\hline
\end{tabular}
\end{table}

\subsection{Các chỉ số thực vật viễn thám}

Các chỉ số thực vật (Vegetation Indices) là các công thức toán học kết hợp giá trị phản xạ từ các kênh phổ khác nhau để tăng cường thông tin về thực vật và giảm nhiễu từ các yếu tố khác như đất, khí quyển. Nghiên cứu này sử dụng ba chỉ số chính:

\textbf{NDVI (Normalized Difference Vegetation Index):}

NDVI là chỉ số được sử dụng phổ biến nhất để đánh giá sức khỏe và mật độ thực vật \cite{rouse1974}:
\begin{equation}
NDVI = \frac{NIR - Red}{NIR + Red} = \frac{B8 - B4}{B8 + B4}
\label{eq:ndvi}
\end{equation}

Giá trị NDVI nằm trong khoảng [-1, 1]. Thực vật xanh khỏe mạnh có NDVI từ 0.3 đến 0.8 do chlorophyll hấp thụ mạnh ánh sáng đỏ và phản xạ mạnh hồng ngoại gần. Đất trống có NDVI từ 0.1 đến 0.2, nước có NDVI âm (< 0) do phản xạ thấp ở cả hai kênh.

\textbf{NBR (Normalized Burn Ratio):}

NBR được thiết kế ban đầu để phát hiện khu vực cháy rừng, nhưng cũng hiệu quả trong phát hiện mất rừng do sử dụng kênh SWIR nhạy với độ ẩm và cấu trúc thực vật \cite{key2006}:
\begin{equation}
NBR = \frac{NIR - SWIR2}{NIR + SWIR2} = \frac{B8 - B12}{B8 + B12}
\label{eq:nbr}
\end{equation}

Rừng khỏe mạnh có NBR cao (0.3--0.8) do NIR cao và SWIR thấp. Khu vực mất rừng hoặc cháy rừng có NBR thấp hoặc âm do NIR giảm và SWIR tăng.

\textbf{NDMI (Normalized Difference Moisture Index):}

NDMI đo lường hàm lượng nước trong tán lá thực vật, là chỉ số quan trọng để phát hiện stress thực vật và thay đổi độ ẩm \cite{gao1996}:
\begin{equation}
NDMI = \frac{NIR - SWIR1}{NIR + SWIR1} = \frac{B8 - B11}{B8 + B11}
\label{eq:ndmi}
\end{equation}

Giá trị NDMI nằm trong khoảng [-1, 1]. Thực vật có hàm lượng nước cao có NDMI dương (0.2--0.6). Thực vật bị stress hoặc khô có NDMI thấp hoặc âm. NDMI nhạy với sự thay đổi độ ẩm trước khi NDVI phản ánh sự suy giảm sức khỏe thực vật.

\begin{table}[H]
\centering
\caption{Tổng hợp các chỉ số thực vật sử dụng trong nghiên cứu}
\label{tab:vegetation_indices}
\begin{tabular}{|l|c|l|l|}
\hline
\textbf{Chỉ số} & \textbf{Công thức} & \textbf{Phạm vi} & \textbf{Ý nghĩa} \\
\hline
NDVI & (B8-B4)/(B8+B4) & [-1, 1] & Mật độ, sức khỏe thực vật \\
\hline
NBR & (B8-B12)/(B8+B12) & [-1, 1] & Phát hiện cháy/mất rừng \\
\hline
NDMI & (B8-B11)/(B8+B11) & [-1, 1] & Độ ẩm tán lá \\
\hline
\end{tabular}
\end{table}

\subsection{Phát hiện biến động rừng}

Phát hiện biến động rừng (Forest Change Detection) là quá trình xác định sự thay đổi về diện tích, cấu trúc hoặc trạng thái của rừng giữa hai hoặc nhiều thời điểm khác nhau \cite{huang2021}. Phương pháp này dựa trên việc so sánh các đặc trưng viễn thám thu được từ các thời điểm khác nhau.

\textbf{Change Detection Approach:}
\begin{equation}
\Delta Feature = Feature_{after} - Feature_{before}
\end{equation}

\textbf{Temporal Features:}

Temporal features bao gồm các ``before features'' thể hiện trạng thái rừng tại thời điểm $t_1$, các ``after features'' thể hiện trạng thái rừng tại thời điểm $t_2$, và các ``delta features'' biểu diễn biến đổi giữa hai thời điểm ($t_2 - t_1$).

\textbf{Ví dụ với NDVI:}

Sử dụng chỉ số NDVI đã trình bày ở mục trước, ta có thể tính toán sự thay đổi thực vật theo thời gian:
\begin{equation}
\Delta NDVI = NDVI_{after} - NDVI_{before}
\end{equation}

Khi $\Delta NDVI$ giảm mạnh (rất nhỏ hơn 0) thì đó là dấu hiệu mất rừng; khi $\Delta NDVI$ xấp xỉ 0 thì vùng được xem là rừng ổn định; và khi $\Delta NDVI$ tăng mạnh (rất lớn hơn 0) thì biểu hiện tái trồng rừng.