\section{Khu vực và dữ liệu nghiên cứu}

\subsection{Khu vực nghiên cứu}

\textbf{Vị trí địa lý:}

Theo Nghị quyết số 1278/NQ-UBTVQH15 ngày 24/10/2024 của Ủy ban Thường vụ Quốc hội, kể từ ngày 01/07/2025, tỉnh Cà Mau và tỉnh Bạc Liêu được sáp nhập thành tỉnh Cà Mau mới. Tỉnh Cà Mau mới nằm ở cực Nam Tổ Quốc, thuộc vùng Đồng bằng sông Cửu Long; tọa độ địa lý nằm trong khoảng 8°36'–9°40' Bắc và 104°43'–105°50' Đông, diện tích tự nhiên là 7,942.38 km², dân số khoảng 2.6 triệu người, và chiều dài đường bờ biển khoảng 300 km (bao gồm cả bờ biển Bạc Liêu cũ).

\begin{figure}[H]
    \centering
    \fbox{\parbox{0.8\textwidth}{\centering\vspace{3cm}\textbf{[PLACEHOLDER]}\\ Bản đồ vị trí khu vực nghiên cứu:\\(a) Vị trí Cà Mau trong Việt Nam\\(b) Ranh giới tỉnh Cà Mau\\(c) Vùng rừng nghiên cứu với tọa độ UTM\vspace{3cm}}}
    \caption{Bản đồ vị trí khu vực nghiên cứu}
    \label{fig:study_area}
\end{figure}

\textbf{Vùng nghiên cứu:}

Đồ án tập trung vào toàn bộ vùng quy hoạch lâm nghiệp của tỉnh Cà Mau mới. Dữ liệu ranh giới quy hoạch lâm nghiệp được cung cấp bởi Công ty TNHH Tư vấn và Công nghệ Đồng Xanh — đối tác của Chi cục Kiểm lâm tỉnh Cà Mau.

\begin{itemize}
    \item \textbf{Tổng diện tích ranh giới quy hoạch:} 170,178.82 hecta (tương đương 1,701.79 km²), bao gồm 666 polygon trong file shapefile ranh giới.
    \item \textbf{Diện tích thực tế được phân loại:} 162,469.25 hecta (khoảng 95.5\% diện tích ranh giới). Phần còn lại (~7,710 ha, chiếm 4.5\%) bị loại do mây che phủ hoặc dữ liệu không hợp lệ (nodata) trong quá trình xử lý ảnh vệ tinh.
    \item \textbf{Kích thước raster:} 12,547 × 10,917 pixels (ở độ phân giải 10m).
    \item \textbf{Hệ quy chiếu:} EPSG:32648 (WGS 84 / UTM Zone 48N).
\end{itemize}

\subsection{Dữ liệu viễn thám}

\begin{table}[H]
\centering
\caption{Tổng quan dữ liệu viễn thám sử dụng}
\label{tab:data_overview}
\begin{tabular}{|l|c|c|c|c|}
\hline
\textbf{Nguồn dữ liệu} & \textbf{Độ phân giải} & \textbf{Kỳ ảnh} & \textbf{Số bands} & \textbf{Dung lượng} \\
\hline
Sentinel-2 Before & 10m & 30/01/2024 & 7 & ~850 MB \\
\hline
Sentinel-2 After & 10m & 28/02/2025 & 7 & ~850 MB \\
\hline
Sentinel-1 Before & 10m & 04/02/2024 & 2 & ~250 MB \\
\hline
Sentinel-1 After & 10m & 22/02/2025 & 2 & ~250 MB \\
\hline
Ground Truth & - & - & - & 2,630 points \\
\hline
Forest Boundary & Vector & - & - & Shapefile \\
\hline
\end{tabular}
\end{table}

\begin{figure}[H]
    \centering
    \fbox{\parbox{0.8\textwidth}{\centering\vspace{2cm}\textbf{[PLACEHOLDER]}\\ Ảnh tổ hợp màu Sentinel-2 (RGB)\\ của khu vực nghiên cứu ở 2 thời điểm\\ (before và after)\vspace{2cm}}}
    \caption{Ảnh Sentinel-2 khu vực nghiên cứu tại hai thời điểm}
    \label{fig:s2_images}
\end{figure}

\subsection{Ground Truth Data}

\begin{table}[H]
\centering
\caption{Thống kê Ground Truth}
\label{tab:ground_truth}
\begin{tabular}{|c|l|c|c|l|}
\hline
\textbf{Class} & \textbf{Tên} & \textbf{Số điểm} & \textbf{Tỷ lệ (\%)} & \textbf{Mô tả} \\
\hline
0 & Forest Stable & 656 & 24.9\% & Rừng ổn định (có rừng ở cả 2 kỳ) \\
\hline
1 & Deforestation & 650 & 24.7\% & Mất rừng (có rừng $\rightarrow$ không có rừng) \\
\hline
2 & Non-forest & 664 & 25.3\% & Không phải rừng (không có rừng ở cả 2 kỳ) \\
\hline
3 & Reforestation & 660 & 25.1\% & Tái trồng rừng (không có $\rightarrow$ có rừng) \\
\hline
\textbf{Tổng} & & \textbf{2,630} & \textbf{100\%} & Balanced distribution \\
\hline
\end{tabular}
\end{table}

\begin{figure}[H]
    \centering
    \fbox{\parbox{0.8\textwidth}{\centering\vspace{2cm}\textbf{[PLACEHOLDER]}\\ Bản đồ phân bố các điểm ground truth\\ theo từng lớp với màu sắc khác nhau\vspace{2cm}}}
    \caption{Phân bố không gian các điểm ground truth}
    \label{fig:ground_truth_distribution}
\end{figure}
