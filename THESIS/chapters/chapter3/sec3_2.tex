\section{Trích xuất đặc trưng}

\subsection{Xây dựng feature stack}

Việc kết hợp dữ liệu SAR và quang học (data fusion) đã được chứng minh là hiệu quả trong nhiều nghiên cứu phân loại lớp phủ đất \cite{ienco2019, hu2020}. Cách tiếp cận này tận dụng ưu điểm bổ sung của hai nguồn dữ liệu: SAR cung cấp thông tin về cấu trúc và độ ẩm bề mặt, trong khi quang học cung cấp thông tin về đặc tính quang phổ của thực vật.

Tổng cộng 27 features được xây dựng từ hai nguồn dữ liệu. \textbf{Sentinel-2} đóng góp 21 features, bao gồm 7 bands/chỉ số (B4, B8, B11, B12, NDVI, NBR, NDMI) cho kỳ trước, 7 bands/chỉ số tương ứng cho kỳ sau, và 7 giá trị delta (hiệu số giữa kỳ sau và kỳ trước). \textbf{Sentinel-1} đóng góp 6 features, bao gồm 2 bands (VV, VH) cho kỳ trước, 2 bands tương ứng cho kỳ sau, và 2 giá trị delta.

\begin{table}[H]
\centering
\caption{Chi tiết 27 đặc trưng sử dụng trong mô hình}
\label{tab:features_detail}
\begin{tabular}{|c|c|c|l|l|}
\hline
\textbf{Chỉ số} & \textbf{Nguồn} & \textbf{Thời kỳ} & \textbf{Đặc trưng} & \textbf{Mô tả} \\
\hline
0-6 & S2 & Kỳ trước & B4, B8, B11, B12, NDVI, NBR, NDMI & Quang phổ kỳ trước \\
\hline
7-13 & S2 & Kỳ sau & B4, B8, B11, B12, NDVI, NBR, NDMI & Quang phổ kỳ sau \\
\hline
14-20 & S2 & Biến đổi & $\Delta$B4, $\Delta$B8, ... & Biến đổi quang phổ \\
\hline
21-22 & S1 & Kỳ trước & VV, VH & SAR kỳ trước \\
\hline
23-24 & S1 & Kỳ sau & VV, VH & SAR kỳ sau \\
\hline
25-26 & S1 & Biến đổi & $\Delta$VV, $\Delta$VH & Biến đổi SAR \\
\hline
\end{tabular}
\end{table}

\subsection{Trích xuất patch 3×3}

Với mỗi điểm thực địa, một patch kích thước 3×3 pixels được trích xuất từ feature stack. Kích thước 3×3 được lựa chọn vì cho phép mô hình học được thông tin ngữ cảnh không gian xung quanh pixel trung tâm, phù hợp với độ phân giải 10m của Sentinel (mỗi patch tương đương vùng 30m × 30m), và giảm thiểu nhiễu từ các pixel lân cận không đồng nhất.

Kết quả là mỗi mẫu có kích thước (3, 3, 27) — tương ứng với chiều cao, chiều rộng và số kênh đặc trưng.
