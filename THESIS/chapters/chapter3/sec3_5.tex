\section{Áp dụng mô hình}

\subsection{Dự đoán toàn vùng nghiên cứu}

Sau khi huấn luyện, mô hình được áp dụng để phân loại toàn bộ vùng nghiên cứu với khoảng 16.2 triệu pixels hợp lệ. Quy trình dự đoán bắt đầu bằng việc tải feature stack 27 channels cho toàn vùng, sau đó trích xuất patch 3×3 cho mỗi pixel hợp lệ. Các patches được chuẩn hóa Z-score sử dụng mean và std từ tập training, rồi thực hiện forward pass qua mô hình và lấy argmax để xác định lớp phân loại. Kết quả cuối cùng được xuất dưới dạng GeoTIFF.

Do kích thước lớn của vùng nghiên cứu, việc dự đoán được thực hiện theo batch (10,000 pixels) với GPU inference và mixed precision (FP16) để tối ưu hóa bộ nhớ và tốc độ.

\subsection{Các độ đo đánh giá}

Hiệu suất của mô hình được đánh giá thông qua các độ đo chuẩn cho bài toán phân loại đa lớp. \textbf{Accuracy} đo tỷ lệ dự đoán đúng trên tổng số mẫu. \textbf{Precision} đo tỷ lệ dự đoán đúng trong số các mẫu được dự đoán thuộc mỗi lớp. \textbf{Recall} đo tỷ lệ phát hiện đúng trong số các mẫu thực sự thuộc mỗi lớp. \textbf{F1-Score} là trung bình điều hòa của Precision và Recall. \textbf{Macro-Average} là trung bình các metrics trên tất cả các lớp. \textbf{ROC-AUC} đo khả năng phân biệt giữa các lớp (One-vs-Rest). \textbf{Confusion Matrix} cho thấy chi tiết hiệu suất phân loại giữa các lớp.

Mô hình được đánh giá trên 20\% tập test cố định (526 mẫu) — hoàn toàn độc lập với quá trình huấn luyện. Kết quả chi tiết được trình bày trong Chương~\ref{chap:results}.
