\section{Sự phát triển của các phương pháp giám sát biến động rừng}

Giám sát biến động rừng đã trải qua nhiều giai đoạn phát triển, từ các phương pháp thủ công truyền thống đến các kỹ thuật học sâu hiện đại. Phần này trình bày quá trình phát triển của các phương pháp theo thời gian, làm cơ sở cho việc lựa chọn phương pháp trong nghiên cứu.

\subsection{Phương pháp truyền thống}

\textbf{Khảo sát thực địa (Field Survey):}

Trước khi có công nghệ viễn thám, giám sát rừng chủ yếu dựa vào khảo sát thực địa trực tiếp. Nhân viên kiểm lâm đi thực địa để đo đạc, ghi nhận trạng thái rừng và lập bản đồ thủ công. Phương pháp này có độ chính xác cao tại từng điểm khảo sát nhưng tốn kém về thời gian và nhân lực, không khả thi cho giám sát diện rộng và khó cập nhật thường xuyên.

\textbf{Ảnh hàng không (Aerial Photography):}

Từ giữa thế kỷ 20, ảnh hàng không bắt đầu được sử dụng để lập bản đồ rừng. Các chuyên gia giải đoán ảnh hàng không bằng mắt thường để xác định ranh giới rừng và phát hiện thay đổi. Phương pháp này cho phép quan sát diện tích lớn hơn khảo sát thực địa nhưng vẫn phụ thuộc nhiều vào kinh nghiệm của người giải đoán và chi phí bay chụp cao.

\subsection{Phương pháp viễn thám thế hệ đầu}

\textbf{Giải đoán ảnh trực quan (Visual Interpretation):}

Với sự ra đời của vệ tinh Landsat năm 1972, lần đầu tiên có thể quan sát bề mặt Trái Đất một cách hệ thống từ không gian. Giai đoạn đầu, việc phân tích ảnh vệ tinh chủ yếu dựa vào giải đoán trực quan - chuyên gia nhìn ảnh và vẽ ranh giới các vùng đất khác nhau. Phương pháp này tận dụng được kiến thức chuyên môn của người giải đoán nhưng mang tính chủ quan, không nhất quán giữa các chuyên gia và khó tái lập.

\textbf{Phân loại dựa trên ngưỡng (Threshold-based Classification):}

Các phương pháp đơn giản như phân ngưỡng chỉ số thực vật NDVI được sử dụng rộng rãi trong thập niên 1980-1990. Ví dụ, nếu NDVI < 0.2 thì phân loại là đất trống, nếu NDVI > 0.5 thì phân loại là rừng. Phương pháp này đơn giản, dễ triển khai nhưng thiếu linh hoạt, không xử lý tốt các trường hợp phức tạp và nhạy cảm với nhiễu khí quyển.

\subsection{Phương pháp học máy truyền thống}

\textbf{Phân loại không giám sát (Unsupervised Classification):}

Phân loại không giám sát không yêu cầu dữ liệu huấn luyện có nhãn, thay vào đó tự động nhóm các pixel có đặc trưng phổ tương tự thành các cụm. Hai thuật toán phổ biến nhất là K-means và ISODATA (Iterative Self-Organizing Data Analysis Technique) \citeen{jensen2015}. K-means phân chia dữ liệu thành k cụm dựa trên khoảng cách Euclidean đến tâm cụm. ISODATA mở rộng K-means bằng cách tự động tách, gộp hoặc loại bỏ cụm dựa trên các ngưỡng thống kê. Phương pháp không giám sát phù hợp khi không có dữ liệu thực địa hoặc khi cần khám phá cấu trúc dữ liệu ban đầu, nhưng có hạn chế là các cụm tạo ra không có ý nghĩa ngữ nghĩa rõ ràng và cần chuyên gia gán nhãn sau đó.

\textbf{Phân loại có giám sát (Supervised Classification):}

Từ thập niên 1990, các thuật toán phân loại thống kê bắt đầu được áp dụng rộng rãi. Maximum Likelihood Classification (MLC) giả định dữ liệu tuân theo phân phối Gaussian và phân loại pixel dựa trên xác suất. Phương pháp này đặt nền tảng cho phân loại ảnh viễn thám định lượng nhưng có hạn chế khi dữ liệu không tuân theo phân phối chuẩn \citeen{khatami2016}.

\textbf{Decision Trees và Random Forest:}

Hansen và cộng sự \citeen{hansen2013} đã sử dụng thuật toán Decision Trees để phát triển Global Forest Change dataset — bộ dữ liệu mất rừng toàn cầu đầu tiên ở độ phân giải 30m từ chuỗi thời gian Landsat (2000-2012). Công trình này đánh dấu bước tiến quan trọng trong giám sát rừng quy mô lớn, đạt accuracy khoảng 85\%. Random Forest, được giới thiệu bởi Breiman (2001), cải thiện Decision Trees bằng cách kết hợp nhiều cây quyết định và trở thành thuật toán phổ biến nhất trong phân loại ảnh viễn thám. Nguyen và cộng sự \citeen{nguyen2020} áp dụng Random Forest với Sentinel-2 đa thời gian để lập bản đồ sử dụng đất tại Đắk Nông, Việt Nam và đạt overall accuracy 91.2\%.

\textbf{Support Vector Machine (SVM):}

SVM được đề xuất cho phân loại ảnh viễn thám từ đầu những năm 2000 và nhanh chóng chứng minh hiệu quả vượt trội so với MLC, đặc biệt với dữ liệu đa chiều và bộ mẫu nhỏ. SVM tìm siêu phẳng tối ưu để phân tách các lớp trong không gian đặc trưng, phù hợp với dữ liệu viễn thám có số chiều cao (nhiều bands).

\subsection{Phương pháp học sâu}

\textbf{Convolutional Neural Networks (CNN):}

Từ năm 2012, với sự thành công của AlexNet trong ImageNet Competition, học sâu bắt đầu cách mạng hóa computer vision và nhanh chóng được áp dụng vào viễn thám. Zhu và cộng sự \citeen{zhu2017} tổng hợp các ứng dụng của deep learning trong viễn thám và chỉ ra tiềm năng to lớn của CNN trong phân loại ảnh vệ tinh. Zhang và cộng sự \citeen{zhang2016} giới thiệu các kiến trúc CNN phổ biến (AlexNet, VGGNet, ResNet) và ứng dụng của chúng trong viễn thám. Kussul và cộng sự \citeen{kussul2017} áp dụng CNN cho phân loại cây trồng từ Sentinel-2 và đạt accuracy 94.5\%, cao hơn đáng kể so với Random Forest.

\textbf{Kiến trúc U-Net và các biến thể:}

Ronneberger và cộng sự \citeen{ronneberger2015} đề xuất kiến trúc U-Net với cấu trúc encoder-decoder, ban đầu cho phân đoạn ảnh y sinh nhưng sau đó được áp dụng rộng rãi trong viễn thám nhờ khả năng phân đoạn ngữ nghĩa pixel-wise hiệu quả. Các biến thể như ResU-Net, Attention U-Net tiếp tục cải tiến hiệu suất cho các bài toán phân đoạn ảnh vệ tinh.

\textbf{Ứng dụng CNN trong giám sát rừng:}

Hethcoat và cộng sự \citeen{hethcoat2019} áp dụng CNN (kiến trúc ResNet) để phát hiện khai thác gỗ chọn lọc tại Amazon từ dữ liệu Sentinel-1 và Sentinel-2, đạt accuracy 94.3\%. Karra và cộng sự \citeen{karra2021} ứng dụng deep learning kết hợp Sentinel-2 để tạo bản đồ sử dụng đất toàn cầu ở độ phân giải 10m.

\subsection{Xu hướng tích hợp đa nguồn dữ liệu}

\textbf{Kết hợp SAR và Optical:}

Một xu hướng quan trọng trong giám sát rừng hiện đại là tích hợp dữ liệu radar (SAR) và quang học. Reiche và cộng sự \citeen{reiche2018} kết hợp Sentinel-1 (SAR) và Landsat (Optical) để phát hiện mất rừng near-real-time tại rừng nhiệt đới khô và đạt accuracy 93.8\%. Hu và cộng sự \citeen{hu2020} kết hợp Sentinel-1 và Sentinel-2 để phân loại rừng ở Madagascar và ghi nhận accuracy tăng từ 87\% (chỉ dùng optical) lên 92\% (kết hợp cả hai).

\textbf{Ưu điểm của tích hợp đa nguồn:}

Việc kết hợp SAR và Optical mang lại nhiều lợi ích. Dữ liệu SAR có khả năng quan sát xuyên mây, đặc biệt quan trọng ở vùng nhiệt đới thường xuyên có mây. Dữ liệu quang học cung cấp thông tin phong phú về phổ phản xạ của thực vật. Hai nguồn dữ liệu bổ sung cho nhau, giúp tăng độ chính xác và độ tin cậy của kết quả phân loại.

\textbf{Nghiên cứu tại Việt Nam:}

Pham và cộng sự \citeen{pham2016} đã sử dụng kết hợp ảnh QuickBird, LiDAR và chỉ số địa hình GIS để nhận dạng loài cây bản địa trong cảnh quan phức tạp. Vo và cộng sự \citeen{vo2020} đã xây dựng hệ thống giám sát biến động rừng ngập mặn hàng năm tại tỉnh Cà Mau sử dụng chuỗi thời gian Landsat-7 và Landsat-8, áp dụng phương pháp tối ưu hóa thời gian sau phân loại để tạo bản đồ rừng liên tục không có khoảng trống dữ liệu.

\subsection{Tổng hợp và so sánh các phương pháp}

\begin{table}[H]
\centering
\caption{So sánh các phương pháp giám sát biến động rừng qua các giai đoạn}
\label{tab:method_evolution}
\begin{tabular}{|l|c|l|l|}
\hline
\textbf{Giai đoạn} & \textbf{Thời kỳ} & \textbf{Phương pháp tiêu biểu} & \textbf{Đặc điểm} \\
\hline
Truyền thống & Trước 1970 & Khảo sát thực địa & Chính xác nhưng tốn kém \\
\hline
Viễn thám đầu & 1970-1990 & Giải đoán trực quan & Chủ quan, khó tái lập \\
\hline
ML truyền thống & 1990-2012 & MLC, Decision Tree, RF & Khách quan, tự động hóa \\
\hline
Học sâu & 2012-nay & CNN, U-Net & Học đặc trưng tự động \\
\hline
Tích hợp & 2015-nay & CNN + SAR + Optical & Bổ sung, tăng độ tin cậy \\
\hline
\end{tabular}
\end{table}

\begin{table}[H]
\centering
\caption{Tổng hợp các nghiên cứu tiêu biểu theo các phương pháp}
\label{tab:related_works}
\begin{tabular}{|l|c|l|l|c|}
\hline
\textbf{Tác giả} & \textbf{Năm} & \textbf{Phương pháp} & \textbf{Dữ liệu} & \textbf{Accuracy} \\
\hline
Hansen và cs. & 2013 & Decision Tree & Landsat & $\sim$85\% \\
\hline
Kussul và cs. & 2017 & CNN & Sentinel-2 & 94.5\% \\
\hline
Reiche và cs. & 2018 & Bayesian fusion & S1+Landsat & 93.8\% \\
\hline
Hethcoat và cs. & 2019 & CNN (ResNet) & S1+S2 & 94.3\% \\
\hline
Nguyen và cs. & 2020 & Random Forest & Sentinel-2 & 91.2\% \\
\hline
Hu và cs. & 2020 & ML + fusion & S1+S2 & 92.0\% \\
\hline
\end{tabular}
\end{table}

\textbf{Nhận xét:} Qua quá trình phát triển, các phương pháp giám sát rừng ngày càng tự động hóa, khách quan và chính xác hơn. Xu hướng hiện tại là kết hợp học sâu với dữ liệu đa nguồn (SAR + Optical) để tận dụng ưu điểm của cả hai. Nghiên cứu này kế thừa xu hướng đó, áp dụng CNN kết hợp dữ liệu Sentinel-1 (SAR) và Sentinel-2 (Optical) cho giám sát biến động rừng tại Cà Mau.
