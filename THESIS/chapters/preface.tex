\phantomsection
\addcontentsline{toc}{chapter}{MỞ ĐẦU}
\unnumberedchapter{MỞ ĐẦU}

\addcontentsline{toc}{section}{1. Đặt vấn đề}
\section*{1. Đặt vấn đề}

Tình trạng mất rừng đang diễn ra nghiêm trọng trên toàn cầu, đặc biệt tại các quốc gia đang phát triển. Theo báo cáo ``Global Forest Resources Assessment 2020'' của FAO \citeen{fao2020}, thế giới đã mất khoảng 178 triệu ha rừng trong giai đoạn 1990--2020, tương đương diện tích của Libya.

Tại Việt Nam, mặc dù độ che phủ rừng đã tăng từ 37\% (năm 2000) lên 42\% (năm 2020) nhờ các chương trình trồng rừng, nhưng tình trạng suy thoái và mất rừng tự nhiên vẫn đáng báo động, đặc biệt tại các tỉnh ven biển và đồng bằng sông Cửu Long. Tỉnh Cà Mau, nằm ở cực Nam Tổ Quốc, sở hữu hệ sinh thái rừng ngập mặn quan trọng nhưng đang phải đối mặt với áp lực từ nuôi trồng thủy sản, xâm nhập mặn và biến đổi khí hậu.

Phương pháp giám sát rừng truyền thống dựa trên điều tra thực địa tốn kém thời gian, chi phí và khó áp dụng cho diện tích rộng. Công nghệ viễn thám vệ tinh cung cấp giải pháp hiệu quả, cho phép giám sát liên tục, diện rộng với chi phí hợp lý. Chương trình Copernicus của Liên minh Châu Âu cung cấp dữ liệu miễn phí từ các vệ tinh Sentinel-1 và Sentinel-2 phù hợp cho giám sát rừng nhiệt đới.

Trong những năm gần đây, trí tuệ nhân tạo và học sâu đã đạt được những bước tiến vượt bậc trong xử lý ảnh và nhận dạng mẫu. Mạng nơ-ron tích chập đặc biệt hiệu quả trong phân loại ảnh nhờ khả năng tự động học đặc trưng không gian từ dữ liệu thô.

Xuất phát từ nhu cầu thực tiễn về giám sát rừng hiệu quả và xu hướng ứng dụng công nghệ AI tiên tiến, đồ án này lựa chọn đề tài \textbf{``Ứng dụng viễn thám và học sâu trong giám sát biến động rừng tỉnh Cà Mau''} nhằm phát triển mô hình phát hiện mất rừng với độ chính xác cao.

\addcontentsline{toc}{section}{2. Mục tiêu và nội dung nghiên cứu}
\section*{2. Mục tiêu và nội dung nghiên cứu}

Mục tiêu của nghiên cứu này là ứng dụng mô hình học sâu dựa trên kiến trúc mạng nơ-ron tích chập để phát hiện và phân loại biến động rừng tại khu vực quy hoạch lâm nghiệp tỉnh Cà Mau, một trong những vùng rừng ngập mặn quan trọng nhất của Việt Nam đang chịu áp lực từ nuôi trồng thủy sản và biến đổi khí hậu. Nghiên cứu tập trung vào việc tích hợp dữ liệu đa nguồn từ vệ tinh Sentinel-1 (ra-đa) và Sentinel-2 (quang học) để khai thác tối đa thông tin về trạng thái lớp phủ rừng qua hai thời kỳ. Thay vì sử dụng các phương pháp học máy truyền thống như Rừng ngẫu nhiên hay SVM vốn phổ biến trong các nghiên cứu tại Việt Nam, đồ án hướng đến thiết kế một kiến trúc CNN có khả năng hoạt động hiệu quả với bộ dữ liệu quy mô vừa phải. Giả thuyết được kiểm tra là mô hình CNN kết hợp đặc trưng từ cả hai nguồn dữ liệu ra-đa và quang học có thể đạt độ chính xác cao hơn so với việc chỉ sử dụng một nguồn dữ liệu đơn lẻ. Những kết quả của nghiên cứu này có thể được coi là công cụ hỗ trợ quan trọng cho công tác giám sát và quản lý rừng tại các tỉnh ven biển đồng bằng sông Cửu Long.

Để đạt được mục tiêu trên, đồ án thực hiện xây dựng bộ dữ liệu huấn luyện từ ảnh vệ tinh Sentinel-1 và Sentinel-2 đa thời gian, bao gồm trích xuất các đặc trưng phổ, chỉ số thực vật và tán xạ ngược cho khu vực nghiên cứu. Tiếp theo là thiết kế và tối ưu hóa kiến trúc CNN phù hợp cho bộ dữ liệu quy mô vừa phải, đánh giá ảnh hưởng của các tham số mô hình đến hiệu năng phân loại. Nghiên cứu cũng đánh giá hiệu quả của việc tích hợp dữ liệu đa nguồn (ra-đa và quang học) so với việc sử dụng từng nguồn dữ liệu riêng lẻ. Cuối cùng là áp dụng mô hình đã huấn luyện để phân loại biến động rừng toàn vùng quy hoạch lâm nghiệp tỉnh Cà Mau và tạo bản đồ biến động ở độ phân giải 10m.

\addcontentsline{toc}{section}{3. Đối tượng và phạm vi nghiên cứu}
\section*{3. Đối tượng và phạm vi nghiên cứu}

Đối tượng nghiên cứu của đồ án là biến động rừng tại khu vực ranh giới lâm nghiệp tỉnh Cà Mau. Các trạng thái biến động rừng được phân loại thành bốn nhóm: Rừng ổn định, Mất rừng, Phi rừng và Phục hồi rừng.

Về không gian, nghiên cứu được thực hiện trên khu vực ranh giới lâm nghiệp tỉnh Cà Mau (theo địa giới hành chính mới có hiệu lực từ ngày 01/07/2025, sau khi sáp nhập tỉnh Cà Mau cũ và tỉnh Bạc Liêu) với tổng diện tích 170,179 ha (tương đương 1,701.79 km²). Khu vực này bao gồm các loại hình rừng tự nhiên và rừng trồng, trong đó chủ yếu là rừng ngập mặn và rừng phòng hộ ven biển. Về thời gian, dữ liệu sử dụng bao gồm ảnh vệ tinh Sentinel-1 và Sentinel-2 trong giai đoạn từ tháng 01/2024 đến tháng 02/2025. Diện tích thực tế được phân loại là 162,469 ha (khoảng 95.5\% diện tích ranh giới, phần còn lại bị loại do mây che hoặc dữ liệu không hợp lệ).

\addcontentsline{toc}{section}{4. Quan điểm và phương pháp nghiên cứu}
\section*{4. Quan điểm và phương pháp nghiên cứu}

Đồ án được nghiên cứu trên cơ sở lý luận và thực nghiệm bằng cách thu thập dữ liệu ảnh vệ tinh Sentinel-1 và Sentinel-2 từ nền tảng Google Earth Engine, kết hợp với các nguồn dữ liệu quy hoạch lâm nghiệp sẵn có. Tiếp đó là kế thừa và tham khảo kết quả nghiên cứu về ứng dụng học sâu trong viễn thám của các tác giả trên thế giới và Việt Nam. Cuối cùng là thực nghiệm bài toán cụ thể bằng cách xây dựng mô hình mạng nơ-ron tích chập để phân loại biến động rừng, đánh giá hiệu năng thông qua kiểm định chéo 5 phần, và áp dụng mô hình đã tối ưu hóa để tạo bản đồ biến động rừng toàn vùng nghiên cứu.

\addcontentsline{toc}{section}{5. Ý nghĩa khoa học và thực tiễn của đề tài}
\section*{5. Ý nghĩa khoa học và thực tiễn của đề tài}

Về mặt khoa học, đồ án đề xuất kiến trúc CNN hiệu quả cho bài toán phân loại ảnh viễn thám với bộ dữ liệu quy mô vừa phải. Nghiên cứu chứng minh hiệu quả của việc tích hợp đa nguồn dữ liệu (ra-đa khẩu độ tổng hợp và quang học đa phổ) trong phát hiện biến động rừng, đồng thời đóng góp vào hướng nghiên cứu ứng dụng học sâu trong lĩnh vực viễn thám và giám sát môi trường.

Về ý nghĩa thực tiễn, đồ án cung cấp mô hình giám sát biến động rừng với độ chính xác cao (trên 98\%), giúp giảm đáng kể thời gian và chi phí so với phương pháp điều tra thực địa truyền thống. Kết quả nghiên cứu có thể hỗ trợ các cơ quan quản lý lâm nghiệp trong việc giám sát và bảo vệ rừng tại tỉnh Cà Mau. Ngoài ra, mô hình có thể mở rộng áp dụng cho các khu vực khác có điều kiện tương tự.

\addcontentsline{toc}{section}{6. Cấu trúc của đồ án}
\section*{6. Cấu trúc của đồ án}

Ngoài phần Mở đầu và Kết luận, nội dung chính của đồ án gồm 3 chương. Chương 1 trình bày cơ sở lý thuyết của bài toán giám sát biến động rừng, bao gồm tổng quan về tình hình mất rừng toàn cầu và tại Việt Nam, giới thiệu công nghệ viễn thám với các vệ tinh Sentinel-1 (ra-đa) và Sentinel-2 (quang học), các chỉ số thực vật (NDVI, NBR, NDMI), cơ sở lý thuyết về mạng nơ-ron tích chập, và tổng quan các nghiên cứu liên quan trong lĩnh vực giám sát rừng bằng học sâu. Chương 2 mô tả cơ sở dữ liệu và phương pháp nghiên cứu, bao gồm chi tiết nguồn dữ liệu viễn thám sử dụng, quy trình thu thập và tiền xử lý dữ liệu từ Google Earth Engine, phương pháp trích xuất 27 đặc trưng từ hai thời kỳ, thiết kế kiến trúc CNN với các kỹ thuật điều chuẩn, và quy trình huấn luyện với kiểm định chéo 5 phần. Chương 3 trình bày kết quả thực nghiệm, bao gồm thiết lập thực nghiệm, kết quả các thử nghiệm về lựa chọn kích thước patch và nghiên cứu loại trừ nguồn dữ liệu, đánh giá hiệu năng mô hình trên tập kiểm tra và kết quả phân loại biến động rừng toàn vùng quy hoạch lâm nghiệp tỉnh Cà Mau.