\phantomsection
\addcontentsline{toc}{chapter}{MỞ ĐẦU}
\unnumberedchapter{MỞ ĐẦU}

\section*{Đặt vấn đề}

Rừng đóng vai trò quan trọng trong việc duy trì cân bằng sinh thái, điều hòa khí hậu, lưu giữ carbon và bảo vệ đa dạng sinh học. Tuy nhiên, tình trạng mất rừng đang diễn ra nghiêm trọng trên toàn cầu, đặc biệt tại các quốc gia đang phát triển. Theo báo cáo ``Global Forest Resources Assessment 2020'' của Tổ chức Lương thực và Nông nghiệp Liên hợp quốc \cite{fao2020}, thế giới đã mất ròng (net loss) khoảng 178 triệu hecta rừng trong giai đoạn 1990-2020, tương đương diện tích của Libya.

Tại Việt Nam, mặc dù độ che phủ rừng đã tăng từ 37\% (năm 2000) lên 42\% (năm 2020) nhờ các chương trình trồng rừng, nhưng tình trạng suy thoái và mất rừng tự nhiên vẫn đáng báo động, đặc biệt tại các tỉnh ven biển và đồng bằng sông Cửu Long. Tỉnh Cà Mau, nằm ở cực Nam Tổ Quốc, sở hữu hệ sinh thái rừng ngập mặn quan trọng nhưng đang phải đối mặt với áp lực từ nuôi trồng thủy sản, xâm nhập mặn, và biến đổi khí hậu.

Phương pháp giám sát rừng truyền thống dựa trên điều tra thực địa tốn kém thời gian, chi phí và khó áp dụng cho diện tích rộng. Công nghệ viễn thám vệ tinh cung cấp giải pháp hiệu quả, cho phép giám sát liên tục, diện rộng với chi phí hợp lý. Chương trình Copernicus của Liên minh Châu Âu (EU) cung cấp dữ liệu miễn phí từ các vệ tinh Sentinel-1 (SAR) và Sentinel-2 (Optical) với độ phân giải không gian 10m và chu kỳ quay trở lại ngắn (5-6 ngày), phù hợp cho giám sát rừng nhiệt đới.

Trong những năm gần đây, trí tuệ nhân tạo (AI) và học sâu (Deep Learning) đã đạt được những bước tiến vượt bậc trong xử lý ảnh và nhận dạng mẫu. Mạng Neural Tích chập (Convolutional Neural Networks - CNN) đặc biệt hiệu quả trong phân loại ảnh nhờ khả năng tự động học đặc trưng không gian (spatial features) từ dữ liệu thô.

Xuất phát từ nhu cầu thực tiễn về giám sát rừng hiệu quả và xu hướng ứng dụng công nghệ AI tiên tiến, đồ án này lựa chọn đề tài \textbf{``Ứng dụng viễn thám và học sâu trong giám sát biến động rừng tỉnh Cà Mau''} nhằm phát triển mô hình phát hiện mất rừng với độ chính xác cao.

\section*{Mục tiêu và nội dung nghiên cứu}

Mục tiêu tổng quát của đồ án là phát triển mô hình học sâu dựa trên kiến trúc CNN để phát hiện và phân loại tự động các khu vực biến động rừng từ ảnh vệ tinh đa nguồn (Sentinel-1 SAR và Sentinel-2 Optical) tại tỉnh Cà Mau.

Để đạt được mục tiêu tổng quát, đề tài tập trung vào năm mục tiêu cụ thể. Thứ nhất, xây dựng bộ dữ liệu huấn luyện thông qua việc thu thập và xử lý dữ liệu ảnh vệ tinh Sentinel-1/2 đa thời gian, kết hợp với ground truth points để tạo bộ dữ liệu huấn luyện chất lượng cao. Thứ hai, thiết kế kiến trúc CNN tối ưu, đề xuất kiến trúc CNN nhẹ (lightweight) phù hợp với bộ dữ liệu có quy mô vừa phải (khoảng 2,600 mẫu). Thứ ba, phân chia dữ liệu khoa học được triển khai bằng phương pháp stratified random split kết hợp với 5-Fold Cross Validation. Thứ tư, huấn luyện và tối ưu hóa mô hình bằng cách áp dụng các kỹ thuật huấn luyện tiên tiến. Thứ năm, ứng dụng thực tế để phân loại toàn bộ khu vực rừng Cà Mau.

\section*{Đối tượng và phạm vi nghiên cứu}

Đối tượng nghiên cứu chính bao gồm các khu vực rừng tự nhiên và rừng trồng tại tỉnh Cà Mau (theo địa giới hành chính mới có hiệu lực từ ngày 01/07/2025, sau khi sáp nhập tỉnh Cà Mau cũ và tỉnh Bạc Liêu), bao gồm rừng ngập mặn và rừng phòng hộ ven biển. Các trạng thái biến động rừng được phân loại thành bốn nhóm: Forest Stable (Rừng ổn định), Deforestation (Mất rừng), Non-forest (Không phải rừng), và Reforestation (Tái trồng rừng).

Phạm vi nghiên cứu bao gồm toàn bộ vùng quy hoạch lâm nghiệp của tỉnh Cà Mau mới với tổng diện tích ranh giới 170,179 hecta (tương đương 1,701.79 km²), trong đó diện tích thực tế được phân loại là 162,469 hecta (khoảng 95.5\% diện tích ranh giới, phần còn lại bị loại do mây che hoặc dữ liệu không hợp lệ). Dữ liệu ranh giới quy hoạch lâm nghiệp được cung cấp bởi Công ty TNHH Tư vấn và Công nghệ Đồng Xanh — đối tác của Chi cục Kiểm lâm tỉnh Cà Mau. Thời gian nghiên cứu kéo dài từ tháng 01/2024 đến tháng 02/2025 (khoảng 13 tháng).

\section*{Ý nghĩa khoa học và thực tiễn của đề tài}

Về mặt khoa học, đồ án đề xuất kiến trúc CNN nhẹ và hiệu quả cho bài toán phân loại ảnh viễn thám với bộ dữ liệu nhỏ; chứng minh hiệu quả tích hợp đa nguồn bằng cách kết hợp dữ liệu SAR và Optical.

Về ý nghĩa thực tiễn, mô hình cung cấp công cụ tự động phát hiện mất rừng với độ chính xác cao (trên 98\%), giúp giảm đáng kể thời gian và chi phí so với phương pháp điều tra thực địa truyền thống.

\section*{Cấu trúc của đồ án}

Đồ án được tổ chức thành bốn chương chính:

\textbf{\textit{Chương 1: Tổng quan về vấn đề nghiên cứu}}

Trình bày tổng quan về vấn đề nghiên cứu, bao gồm bối cảnh mất rừng, công nghệ viễn thám, tổng quan các nghiên cứu liên quan và các khoảng trống nghiên cứu.

\textbf{\textit{Chương 2: Cơ sở lý thuyết}}

Giới thiệu chi tiết về công nghệ viễn thám (Sentinel-1/2), lý thuyết về mạng Neural Tích chập (CNN) và các phương pháp phân loại ảnh cùng những tiêu chí đánh giá mô hình.

\textbf{\textit{Chương 3: Phương pháp nghiên cứu}}

Mô tả khu vực nghiên cứu, dữ liệu sử dụng, quy trình xử lý, kiến trúc mô hình CNN đề xuất, phương pháp huấn luyện và đánh giá.

\textbf{\textit{Chương 4: Kết quả và thảo luận}}

Trình bày các kết quả huấn luyện, đánh giá mô hình, phân loại toàn vùng, phân tích lỗi và trực quan hóa. Đồng thời đưa ra kết luận và kiến nghị cho các nghiên cứu tiếp theo.
