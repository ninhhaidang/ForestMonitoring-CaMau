\subsection{Tua-bin gió trục ngang (HAWT)}

Tua-bin gió trục ngang, là thiết kế sử dụng phổ biến nhất trong các loại tua-bin gió với rô-to cánh quạt được kết nối với 1 trục ngang và bộ sản xuất điện năng. Hiện nay, HAWT ba cánh quạt được đánh giá là có hiệu suất cao nhất nhưng tuỳ thuộc vào mục đích thiết kế, số lượng cánh có thể thay đổi thành hai hoặc một. Tua-bin trục ngang sử dụng thiết kế cánh tương tự cánh máy bay và dựa vào lực nâng sinh ra trên cánh để quay và tạo ra công suất. Những đặc điểm trong thiết kế đòi hỏi HAWT phải đặt mặt phẳng cánh quạt đúng hướng gió, được lắp đặt cảm biến xác định hướng gió và hỗ trợ chuyển động quay. Tua-bin gió trục ngang được thiết kế với dải công suất đa dạng và được sử dụng phổ biến trong việc sản xuất điện năng từ gió. Với một số đặc trưng như kích thước lớn, cánh quạt dài, HAWT có thể hoạt động trên cao nhằm thu được lượng gió lớn, mạnh và ổn định. Do đó, HAWT thích hợp cho những cánh đồng gió cỡ lớn, thường được đặt ở những khu vực có luồng không khí có hướng ổn định, hạn chế nhiễu loạn.

\begin{figure}[H]
    \centering
    \includegraphics[width=0.75\linewidth]{img/chapter1/HAWT.png}
    \caption{Tua-bin gió trục ngang (HAWT) \cite{avaada_types_of_wind_turbines}}
    \label{fig:enter-label}
\end{figure}