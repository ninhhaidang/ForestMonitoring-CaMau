\section{Tiềm năng năng lượng gió tại Việt Nam}

Với nền tảng là một nền kinh tế hội nhập, những biến đổi phức tạp của khí hậu thế giới và những xu hướng toàn cầu mới luôn có tác động mạnh mẽ đối với Việt Nam, đặc biệt là đối với lĩnh vực năng lượng, vốn là nền tảng của nền kinh tế quốc dân. Theo một báo cáo của tổ chức Germanwatch về các chỉ số rủi ro khí hậu toàn cầu năm 2021, Việt Nam là một trong những nước chịu hậu quả nặng nề của biến đổi khí hậu. Trong giai đoạn từ năm 2000 đến 2019, Việt Nam đã phải hứng chịu trực tiếp các hiện tượng thời tiết cực đoan như bão nhiệt đới, lũ lụt, mực nước biển dâng, ảnh hưởng tới kinh tế xã hội và đặt áp lực lên nỗ lực hội nhập toàn cầu.\cite{germanwatch_cri_2024} Nhằm đối mặt với hiện trạng đó, trong Hội nghị Liên hợp quốc về biến đổi khí hậu lần thứ 26 (COP26), Việt Nam là một trong các nước đầu tiên ký cam kết và thể hiện quyết tâm hướng tới các biện pháp bền vững ứng phó với biến đổi khí hậu. Trong đó, nổi bật là cam kết từng bước từ bỏ điện than và chuyển đổi sử dụng điện từ các nguồn năng lượng tái tạo.

Trong số các nguồn năng lượng tái tạo phổ biến hiện nay, Việt Nam được đánh giá là nước có tiềm năng lớn để phát triển năng lượng gió nhờ nằm trong vùng gió mùa Châu Á mạnh và ổn định. Theo kết quả khảo sát của chương trình đánh giá về năng lượng cho Châu Á của Ngân hàng Thế giới (WB), Việt Nam có tiềm năng gió trung bình so với các nước trên thế giới và trong khu vực nhưng thuộc diện lớn nhất trong khu vực Đông Nam Á. Ước tính, hơn 39\% diện tích Việt Nam có tốc độ gió trung bình hàng năm trên 6 m/s ở độ cao 65 m và hơn 8\% diện tích đất liền có tốc độ trên 7 m/s. Nhờ vậy, tổng tiềm năng điện gió ước đạt 513,360 MW, lớn gấp 200 lần công suất của nhà máy thuỷ điện Sơn La và hơn 10 lần tổng công suất dự báo của ngành điện Việt Nam năm 2020. Trong đó, tiềm năng kỹ thuật của điện gió trên bờ vào khoảng 42 GW phù hợp với dự án điện gió quy mô lớn.\cite{nangluongvietnam_gio_2024} 

\begin{table}[H]
\caption{Tiềm năng năng lượng gió ở độ cao 65 m tại Việt Nam. \cite{nangluongvietnam_gio_tiem_nang_2024}}
\label{tab:my-table}
\begin{tabular}{|l|c|c|c|c|c|}
    \hline
    \multicolumn{1}{|l|}{\textbf{\begin{tabular}[l]{@{}l@{}}Tốc độ gió trung\\bình\end{tabular}}} & \textbf{\begin{tabular}[c]{@{}c@{}}Thấp\\< 6 m/s\end{tabular}} & \textbf{\begin{tabular}[c]{@{}c@{}}Trung bình\\6 – 7 m/s\end{tabular}} & \textbf{\begin{tabular}[c]{@{}c@{}}Tương đối cao\\7 – 8 m/s\end{tabular}} & \textbf{\begin{tabular}[c]{@{}c@{}}Cao\\8 – 9 m/s\end{tabular}} & \textbf{\begin{tabular}[c]{@{}c@{}}Rất cao\\> 9 m/s\end{tabular}} \\ \hline
    \textbf{Diện tích ($km^2$)}                             & 197,242                                                               & 100,367                                                                       & 25,679                                                                           & 2,178                                                                  & 111                                                                      \\ \hline
    \textbf{Tỷ lệ diện tích (\%)}                         & 60.6                                                                  & 30.8                                                                          & 7.9                                                                              & 0.7                                                                    & > 0                                                                      \\ \hline
    \textbf{Tiềm năng (MW)}                              & -                                                                     & 401,444                                                                       & 102,716                                                                          & 8,748                                                                  & 482                                                                      \\ \hline 
    \end{tabular}
\end{table} 

Với đường bờ biển dài hơn 3,000 km, tiềm năng gió ngoài khơi của Việt Nam được đánh giá là lớn hơn nhiều so với tiềm năng gió trên bờ. Các dự án điện gió ngoài khơi đang thu hút các chính sách đầu tư nhờ gió ngoài khơi thường có tốc độ cao, ổn định hơn, cũng như hạ tầng cho điện gió ngoài khơi và lưới điện cũng ít bị hạn chế bởi vấn đề sử dụng đất.

Theo Quy hoạch điện VII Điều chỉnh, được Thủ tướng Chính phủ phê duyệt năm 2016, Việt Nam sẽ phát triển 800 MW điện gió vào năm 2020, chiếm khoảng 0,8\% tổng nhu cầu điện khi đó. Mục tiêu là phát triển 2,000 MW điện gió vào năm 2025 và 6,000 MW vào năm 2030. Dự thảo Quy hoạch điện VIII (đang được hoàn thiện) đã đưa mục tiêu phát triển điện gió lên tới trên 11,000 MW vào năm 2025.\cite{nangluongvietnam_gio_2024} Đặc biệt, quy hoạch cũng bao gồm mục tiêu điện gió trên bờ đạt 21.8 GW vào năm 2030, tăng khoảng 5 GW so với công suất lắp đặt cuối năm 2023. Điện gió ngoài khơi đạt 6 GW đến năm 2030 so với trạng thái chưa lắp đặt hiện giờ, sau đó, tăng nhanh chóng lên 91 GW vào năm 2050.\cite{vir_wind_power_outlook_2024}

\begin{table}[H]
\caption{Tiềm năng năng lượng gió theo vùng tại Việt Nam.\cite{nangluongvietnam_gio_2024}}
\label{tab:my-table}
\centering
\begin{tabular}{|l|c|c|}
\hline
\multicolumn{1}{|c|}{\textbf{Khu vực}} & \textbf{\begin{tabular}[c]{@{}c@{}}Tiềm năng gió trên bờ   \\ (GW)\end{tabular}} & \textbf{\begin{tabular}[c]{@{}c@{}}Tiềm năng gió ngoài khơi \\ (GW)\end{tabular}} \\ \hline
\textbf{Đông Bắc}                      & 4.6                                                                              & 64.5                                                                              \\ \hline
\textbf{Tây Bắc}                       & 2.8                                                                              & -                                                                                 \\ \hline
\textbf{Đồng bằng Sông Hồng}           & 1.5                                                                              & 66.7                                                                              \\ \hline
\textbf{Bắc Trung Bộ}                  & 0.3                                                                              & 113.0                                                                             \\ \hline
\textbf{Nam Trung Bộ}                  & 16.8                                                                             & 78.8                                                                              \\ \hline
\textbf{Tây Nguyên}                    & 12.5                                                                             & -                                                                                 \\ \hline
\textbf{Đông Nam Bộ}                   & 3.3                                                                              & 27.1                                                                              \\ \hline
\textbf{Đồng bằng Sông Cửu Long}       & 0.2                                                                              & 259.7                                                                             \\ \hline
\textbf{Tổng cộng}                     & 42.0                                                                             & 609.8                                                                             \\ \hline
\end{tabular}
\end{table}

Để khai thác hiệu quả tiềm năng năng lượng gió ở Việt Nam, việc phát triển công nghệ tua-bin gió đóng vai trò vô cùng quan trọng. Với đặc điểm địa lý và khí hậu, Việt Nam có lợi thế lớn về nguồn năng lượng gió, đặc biệt là ở các khu vực ven biển và ngoài khơi. Tuy nhiên, để tận dụng tối đa nguồn tài nguyên này, cần tập trung vào việc cải tiến thiết kế và nâng cao hiệu suất của tua-bin gió, đặc biệt là các tua-bin gió trục đứng và trục ngang phù hợp với điều kiện gió đa dạng. Phát triển công nghệ tua-bin gió không chỉ giúp gia tăng hiệu quả chuyển đổi năng lượng mà còn giảm thiểu chi phí lắp đặt và bảo trì, từ đó tạo điều kiện thuận lợi cho việc mở rộng các dự án năng lượng gió trên khắp cả nước, góp phần quan trọng vào mục tiêu phát triển bền vững và bảo vệ môi trường.