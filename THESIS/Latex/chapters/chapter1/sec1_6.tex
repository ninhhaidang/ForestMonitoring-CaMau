\section{Mục tiêu đồ án}

Đồ án với đề tài "Nghiên cứu biên dạng cánh airfoil cho tua-bin gió Savonius" hướng đến việc nghiên cứu và cải tiến biên dạng cánh của tua-bin gió Savonius bằng cách kết hợp biên dạng cánh máy bay airfoil nhằm nâng cao hiệu suất hoạt động của tua-bin gió Savonius. Đồ án sẽ tiến hành thiết kế và mô phỏng biên dạng tua-bin mới, áp dụng phương pháp mô phỏng số động lực học chất lỏng (CFD). Quá trình này nhằm xác định hình dạng cánh cải tiến có thể cải thiện đáng kể hiệu suất khí động học của tua-bin. Các mô hình thử nghiệm sẽ được xây dựng để so sánh hiệu suất của các thiết kế mới với các thiết kế hiện tại, từ đó đánh giá những tiến bộ đạt được trong quá trình nghiên cứu.

Cuối cùng, đồ án sẽ tổng hợp và phân tích kết quả nghiên cứu, đưa ra các khuyến nghị và giải pháp thiết thực để cải tiến thiết kế biên dạng cánh airfoil cho tua-bin gió Savonius, góp phần nâng cao khả năng khai thác năng lượng gió trong các điều kiện khác nhau. Qua đó, nghiên cứu không chỉ làm rõ những tiềm năng của năng lượng gió ở Việt Nam mà còn đóng góp vào việc phát triển bền vững nguồn năng lượng tái tạo.

Cấu trúc nội dung của đồ án sẽ bao gồm:

\textbf{Chương 1:} Tổng quan về xu hướng chuyển dịch năng lượng toàn cầu và nghiên cứu cải thiện hiệu suất tua-bin gió Savonius

\textbf{Chương 2:} Lý thuyết công suất tua-bin gió và phương pháp số động lực học chất lỏng

\textbf{Chương 3:} Quá trình mô phỏng số dòng chảy qua tua-bin gió Savonius

\textbf{Chương 4:} Phân tích kết quả mô phỏng và thảo luận

\textbf{Chương 5:} Kết luận chung