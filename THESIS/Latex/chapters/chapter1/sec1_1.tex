\section{Xu hướng chuyển dịch năng lượng toàn cầu}

Hiện nay, với nhiều dẫn chứng khoa học và các công bố quốc tế, nhân loại đã chứng minh được sự tồn tại của biến đổi khí hậu. Đây là một trong những vấn đề cấp bách cần được giải quyết trong thế kỷ XXI nhằm hướng tới sự phát triển bền vững của nền kinh tế toàn cầu. Theo thống kê của Tổ chức Khí tượng Thế giới (WMO), năm 2023 đã chứng kiến những kỉ lục mới của các chỉ số biến đổi khí hậu. Cụ thể, mật độ khí nhà kính đã cao hơn đến 50\% so với thời kỳ tiền công nghiệp, đạt mức kỷ lục năm 2022 và đang tiếp tục gia tăng. Nồng độ khí $CO_2$ giữ nhiệt lâu dài trên bầu khí quyển dự đoán sự gia tăng nhiệt độ toàn cầu trong nhiều năm tiếp theo. Thực tế, năm 2023 ghi nhận mức tăng trung bình khoảng $1.45^oC$ so với giai đoạn tiền công nghiệp từ năm 1850 – 1900 và trở thành năm nóng nhất kể từ 2016.\cite{wmo_climate_indicators_2023}
\begin{figure}[H]
    \centering
    \includegraphics[width=0.75\linewidth]{img/chapter1/Temperature.jpg}
    \caption{Chênh lệch nhiệt độ trung bình toàn cầu hàng năm (so với giai đoạn 1850–1900) từ 1850 đến 2023. Dữ liệu được lấy từ sáu bộ dữ liệu. \cite{wmo_climate_indicators_2023}}
    \label{fig:enter-label}
\end{figure}

Các hiện tượng khí hậu cực đoan diễn ra trên toàn thế giới đang đặt ra thách thức lớn cho các quốc gia, đòi hỏi các nỗ lực khẩn thiết nhằm hướng tới sự hình thành các nền kinh tế xanh, các-bon thấp. Trong đó, nổi bật nhất là xu hướng chuyển dịch năng lượng sạch toàn cầu. Tại Hội nghị Liên hợp quốc về biến đổi khí hậu lần thứ 26 ở Paris (COP26), các quốc gia thành viên đã thống nhất được các thỏa thuận liên quan đến việc giảm lượng phát thải khí nhà kính, hạn chế sử dụng năng lượng hoá thạch, tăng cường ứng dụng công nghệ xanh và đặc biệt là hướng tới cam kết “NET ZERO” vào năm 2050.\cite{moit_cop26} Nhằm tăng cường khả năng cạnh tranh của nền kinh tế, các quốc gia đang thúc đẩy quá trình chuyển đổi sang nền kinh tế các-bon thấp, nền tảng là quá trình chuyển đổi năng lượng.

Cụ thể, đối với mỗi quốc gia, ngành năng lượng đóng vai trò cốt lõi của nền kinh tế, là nền tảng cơ bản của đời sống nhân loại. Từ xa xưa, con người đã biết sử dụng năng lượng để sưởi ấm, thắp sáng nhà cửa, đi lại và hoạt động sản xuất. Các nguồn năng lượng sơ cấp bao gồm nhiên liệu hoá thạch, năng lượng tái tạo và năng lượng hạt nhân. Các nguồn năng lượng thứ cấp bao gồm điện, nhiệt và nhiên liệu (xăng và các sản phẩm dầu tinh chế).\cite{neu_energy_transition} Trong đó, năng lượng hoá thạch là nguồn năng lượng truyền thống có giới hạn được khai thác từ các nguồn nhiên liệu hoá thạch dạng các-bon từ các mỏ, bao gồm than, dầu và khí đốt tự nhiên.\cite{ipcc_ar6_annexvii} Mặt khác, năng lượng tái tạo được cung cấp bởi các nguồn tài nguyên không giới hạn, được bổ sung tự nhiên như mặt trời, thuỷ triều và gió.\cite{us_department_energy_renewable} Năng lượng hạt nhân là một dạng năng lượng được giải phóng từ hạt nhân, lõi của các nguyên tử, được tạo thành từ các proton và neutron.\cite{iaea_nuclear_energy} Hiện nay, quá trình chuyển đổi năng lượng đang được thực hiện chủ yếu liên quan đến sự thay đổi nguồn năng lượng chủ đạo của mỗi quốc gia từ năng lượng hoá thạch sang các nguồn năng lượng bền vững (chủ yếu là năng lượng tái tạo).

Theo Báo cáo hiện trạng toàn cầu về năng lượng tái tạo của Mạng lưới chính sách năng lượng tái tạo cho thế kỷ 21 (REN21) năm 2023 – GSR2023, năm 2022 được đánh giá là một năm khó đoán của thị trường năng lượng. Sự hồi phục kinh tế sau đại dịch COVID-19 và cuộc xung đột Nga – Ukraine đã khiến giá cả năng lượng tăng nhanh chóng. Cụ thể, giá dầu thô và khí đốt tăng đã dẫn tới lạm phát trong giá năng lượng (bao gồm cả các nguồn năng lượng tái tạo), lương thực và các lĩnh vực thiết yếu khác.\cite{ren21_global_overview_2023} Tại Châu Âu, khu vực đang đối mặt với các cuộc khủng hoảng năng lượng và ngày càng trở nên trầm trọng do cuộc xung đột giữa Nga và Ukraine kể từ đầu năm 2022. Liên minh Châu Âu vốn phụ thuộc chủ yếu vào nguồn khí đốt nhập khẩu từ Nga, chiếm khoảng 40\% nguồn cung trong năm 2021 nhưng đã sụt giảm nghiêm trọng trong năm 2022. Trong tháng 7 và 8 năm 2022, lượng khí đốt nhập khẩu từ Nga đến Châu Âu đã giảm gần 70\% so với cùng kỳ năm 2021.\cite{ren21_europe_snapshot_2023} Để thích ứng với tình hình thực tiễn, Uỷ ban Châu Âu đã công bố kế hoạch REPowerEU nhằm nhanh chóng giảm bớt sự phụ thuộc vào nhiên liệu hoá thạch từ Nga và tiến tới chuyển đổi năng lượng sạch. Để đạt được mục tiêu giảm lượng tiêu thụ khí đốt lên đến 124 tỉ khối vào năm 2023, REPowerEU khuyến khích tăng mục tiêu năng lượng tái tạo chiếm tới 45\% (so với 38\%) tổng năng lượng tiêu thụ và thúc đẩy lượng năng lượng dự trữ lên 13\% (so với 9\%). Song song với đó, để giảm nhẹ các ảnh hưởng của lạm phát, chính phủ Hoa Kỳ đã đưa ra Đạo luật giảm lạm phát, trong đó bao gồm các gói trợ cấp bền vững nhằm thúc đẩy hiệu suất năng lượng nội địa và phát triển công nghệ kỹ thuật thích ứng với các nguồn năng lượng tái tạo.\cite{eu_repowereu_2024}\cite{epa_inflation_reduction_act_2024}

Xu hướng chuyển dịch năng lượng tái tạo đang diễn ra mạnh mẽ trên toàn cầu và có thể thấy rõ qua các kết quả đạt được. Trong năm 2022, tổng sản lượng điện toàn cầu đã tăng 2.3\%, đạt 29,165 TWh, và nguồn năng lượng tái tạo đóng góp 92\%, còn lại là nguồn năng lượng hạt nhân, khí và than hoá thạch. So sánh với năm 2021, sản lượng điện tăng chủ yếu đến từ các nguồn nhiên liệu hoá thạch, hạt nhân (chiếm khoảng 64\%) trong khi năng lượng tái tạo chỉ chiếm 32\%. Tuy nhiên, nhu cầu điện tăng không đồng đều tại các khu vực. Nhìn chung, tỷ trọng năng lượng tái tạo trong sản lượng điện toàn cầu đã tăng 8.1\% và  đạt 29.9\% trong năm 2022. Đặc biệt, tỷ trọng kết hợp giữa tổ hợp năng lượng gió và mặt trời đã tăng ổn định 12\% kể từ năm 2015. Chuyển dịch năng lượng tái tạo đã tập trung chủ yếu vào việc phát triển các công nghệ tiên tiến trong lĩnh vực năng lượng. Năng lượng gió và năng lượng mặt trời chiếm 23.9\% tổng công suất phát điện lắp đặt vào năm 2022, cao hơn 2.4 điểm phần trăm so với mức của năm 2021. Công suất lắp đặt điện mặt trời đạt 1,185 GW và điện gió 906 GW.\cite{ren21_global_overview_2023}  
