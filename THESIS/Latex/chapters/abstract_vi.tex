\begin{center}
\textbf{\large{Tóm tắt}}
\end{center}
\addcontentsline{toc}{chapter}{Tóm tắt}

Đề tài này nghiên cứu ảnh hưởng của nhiệt độ môi trường tới hiệu suất hoạt động của tua-bin Savonius, với mục tiêu nâng cao hiệu suất chuyển đổi năng lượng gió, đặc biệt ở tỉ tốc cao, phù hợp với điều kiện gió phức tạp, đa hướng và có dải tốc độ rộng trong môi trường đô thị. Biên dạng cánh cải tiến được thiết kế từ sự kết hợp với biên dạng cánh máy bay airfoil FX74-CL5-140. Đặc tính khí động lực học của biên dạng này được đánh giá thông qua phương pháp mô phỏng số 2D, sử dụng phần mềm CFD ANSYS Fluent và so sánh với biên dạng cánh nguyên bản. Kết quả mô phỏng cho thấy, ở tỉ tốc gió thấp, hệ số mô-men và công suất của rô-to với biên dạng mới chưa tạo ra sự cải thiện đáng kể so với biên dạng nguyên bản. Tuy nhiên, khi tỉ tốc gió vượt qua 0.8, các hệ số này bắt đầu cải thiện, đạt mức tối ưu tại tỉ tốc 1.1. Biên dạng cải tiến nổi bật với khả năng kiểm soát dòng chảy tốt hơn, đặc biệt tại vùng chồng cánh. Dòng chảy từ mặt lõm của cánh tiến được dẫn qua khe chồng cánh hẹp, tạo ra dòng qua khe, cải thiện sự chênh lệch áp suất trên cánh tiến, từ đó tăng cường lực khí động. Các kết quả này cho thấy biên dạng cải tiến có tiềm năng lớn trong ứng dụng thực tế của tua-bin gió Savonius tại các khu vực đô thị, đem lại hiệu suất vượt trội so với biên dạng nguyên bản, đồng thời vẫn giữ được các đặc tính nổi bật của thiết kế tua-bin này.

\textbf{\textit{Từ khóa:}} CFD, năng lượng gió, tua-bin gió trục đứng, tua-bin Savonius, tỉ tốc
