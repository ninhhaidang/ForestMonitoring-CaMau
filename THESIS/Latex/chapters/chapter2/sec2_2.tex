\section{Tính toán các tham số đặc trưng hiệu suất của tua-bin gió Savonius}

Đặc tính dòng chảy qua tua-bin Savonius có thể được mô tả qua sự phân bố của các đại lượng khí động học như áp suất, vận tốc và độ xoáy của dòng khí. Những yếu tố này đóng vai trò quan trọng trong việc xác định hiệu suất chuyển hóa năng lượng của tua-bin, tức là khả năng chuyển đổi năng lượng động học của gió thành năng lượng cơ học để sinh công. Để đánh giá hiệu suất của tua-bin Savonius, các thông số quan trọng như hệ số mô-men xoắn $C_T$, hệ số công suất $C_P$ và tỉ tốc gió $\lambda$ được sử dụng.

Hệ số công suất $C_P$ biểu thị tỷ lệ giữa công suất thực tế của tua-bin và công suất động năng của gió đi qua diện tích quét của rô-to. Đây là một chỉ số quan trọng để đánh giá hiệu quả chuyển đổi năng lượng của tua-bin. Hệ số mô-men $C_T$ phản ánh sự thay đổi mô-men quay trên các cánh quạt của tua-bin, và có mối liên hệ chặt chẽ với sự phân bố áp suất và lực khí động học tác động lên cánh quạt. Tỉ tốc gió $\lambda$ là tỷ số giữa vận tốc đầu cánh tua-bin $v$ và vận tốc gió đầu vào $U_0$ thể hiện khả năng tua-bin hoạt động hiệu quả ở các điều kiện gió khác nhau.

Các thông số này được tính toán qua các biểu thức:
\begin{equation}
    C_{T}=\frac{4T}{\rho AD^{2}U_{0}^{2}}
\end{equation}
\begin{equation}
    C_{P}=\frac{2P}{\rho AD^{2}U_{0}^{3}}=\lambda C_{T}
\end{equation}
\begin{equation}
    \lambda=\frac{\omega D}{2U_{0}}
\end{equation}

Trong đó, 
\begin{itemize}
    \item $P$ là công suất của tua-bin [W]
    \item $T$ là mô-men xoắn trên cánh của tua-bin [Nm]
    \item $A$ là diện tích quét của rô-to [m²]
    \item $U_0$ là tốc độ gió [m/s]
    \item $\omega$ là vận tốc góc [rad/s]
    \item $D$ là đường kính vùng quay của tua-bin [m]
    \item $\lambda$ là tỉ tốc gió
\end{itemize}
 
Mỗi thông số đóng góp vào việc đánh giá và tối ưu hóa thiết kế tua-bin, từ đó cải thiện khả năng khai thác năng lượng gió. Việc phân tích đặc tính dòng chảy và các chỉ số hiệu suất này giúp cải tiến thiết kế tua-bin Savonius, đặc biệt là tối ưu hóa biên dạng cánh và điều kiện vận hành. Điều này có ý nghĩa quan trọng trong việc tăng cường hiệu suất khí động học, giảm thiểu tổn thất năng lượng và phát huy tối đa tiềm năng của năng lượng gió trong thực tế.