\chapter*{Kết luận chung}
\addcontentsline{toc}{chapter}{Kết luận chung}

Năng lượng tái tạo, đặc biệt là năng lượng gió, đang ngày càng khẳng định vai trò quan trọng trong chiến lược phát triển bền vững toàn cầu, nhằm đối phó với các vấn đề về biến đổi khí hậu và sự cạn kiệt của các nguồn năng lượng hoá thạch. Tuy nhiên, hiệu quả khai thác năng lượng gió vẫn còn nhiều hạn chế, đặc biệt là đối với các tua-bin gió Savonius, vốn có thiết kế đơn giản nhưng hiệu suất chuyển đổi năng lượng còn thấp. Để nâng cao hiệu quả của các hệ thống tua-bin gió Savonius, việc cải tiến thiết kế cánh là một hướng nghiên cứu quan trọng. Đồ án này tập trung vào nghiên cứu và cải tiến biên dạng cánh tua-bin gió Savonius thông qua việc áp dụng các biên dạng cánh máy bay (airfoil) nhằm cải thiện hiệu suất khí động học và khả năng khai thác năng lượng gió hiệu quả hơn trong các điều kiện gió đa dạng của môi trường đô thị.

Trong \textbf{Chương 1}, đồ án đã trình bày tổng quan về xu hướng chuyển dịch năng lượng của thế giới, đặc biệt là sự phát triển của năng lượng gió tại Việt Nam. Cũng trong chương này, thiết kế tua-bin gió Savonius đã được giới thiệu, cùng với các nghiên cứu hiện tại về việc cải thiện hiệu suất khí động học của loại tua-bin này. Đề tài đã chỉ ra mục tiêu nghiên cứu là cải tiến hiệu suất tua-bin Savonius qua việc thay đổi biên dạng cánh, nhằm đáp ứng tốt hơn với các yêu cầu khai thác năng lượng gió.

\textbf{Chương 2} đã cung cấp lý thuyết về công suất của tua-bin gió và phương pháp mô phỏng số động lực học chất lỏng, trong đó giới thiệu định lý Betz và các phương pháp tính toán hiệu suất rô-to Savonius.

\textbf{Chương 3} mô tả quy trình mô phỏng số dòng chảy qua tua-bin gió Savonius, bao gồm việc thiết kế mô hình, chia lưới miền tính toán, và lựa chọn các điều kiện biên cho mô phỏng. Phương trình Navier-Stokes và mô hình rối để mô phỏng dòng chảy qua biên dạng cánh mới của tua-bin Savonius cũng được giới thiệu.

Trong \textbf{Chương 4}, kết quả mô phỏng dòng chảy qua các biên dạng cánh khác nhau, với sự phân tích chi tiết về hiệu suất của các cấu hình cánh được cải tiến so với cánh nguyên bản đã được trình bày.

Các kết quả mô phỏng cho thấy, ở tỉ tốc gió thấp ($\lambda =$ 0.5), biên dạng Ov01 tỏ ra ưu việt nhờ khả năng tạo mô-men xoắn lớn ở các góc quay thuận lợi. Điều này giúp biên dạng Ov01 hoạt động hiệu quả trong điều kiện gió yếu. Tuy nhiên, ở các tỉ số vận tốc gió trung bình và cao ($\lambda =$ 1.0 và 1.4), biên dạng Fx lại cho thấy hiệu suất vượt trội hơn nhờ thiết kế kết hợp đặc tính đa độ cong và độ dày của cánh.

Một trong những ưu điểm nổi bật của biên dạng Fx là khả năng kiểm soát dòng chảy tốt hơn, đặc biệt là trong vùng chồng cánh. Khi dòng chảy từ mặt lõm của cánh tiến được dẫn qua khe chồng cánh hẹp sang mặt lõm của cánh lùi, sự gia tăng vận tốc dòng chảy qua khe này giúp nâng cao chênh lệch áp suất trên cánh tiến. Kết quả là lực khí động tăng lên đáng kể. Trong khi đó, biên dạng Ov01, với khe chồng cánh rộng hơn, gặp khó khăn trong việc kiểm soát dòng chảy tại khu vực này, dẫn đến sự nhiễu loạn và tách dòng, làm giảm hiệu suất của tua-bin.

Ngoài ra, biên dạng Fx còn giúp giảm cường độ rối và cường độ xoáy, đặc biệt ở vùng chồng cánh và khu vực thoát dòng sau cánh. Các vùng xoáy trong biên dạng Fx nhỏ gọn và ít lan rộng, giúp ổn định dòng chảy và giảm thiểu tổn thất năng lượng khí động. Phân bố áp suất trên bề mặt cánh cũng cho thấy sự ổn định vượt trội của Fx so với Ov01, với lực khí động được duy trì ổn định trong suốt chu kỳ quay.

Đề tài này đã chứng minh rằng việc cải tiến biên dạng cánh tua-bin Savonius là một giải pháp tiềm năng để tăng hiệu suất khai thác năng lượng gió trong môi trường đô thị. Biên dạng cánh cải tiến mang lại những ưu điểm nổi bật như tăng hiệu suất hoạt động, giảm lực cản, và duy trì khả năng vận hành ổn định ở các tỉ tốc gió khác nhau. Ngoài ra, thiết kế cánh đơn giản, dễ chế tạo và bảo trì của tua-bin Savonius vẫn được giữ nguyên, làm cho nó trở thành một lựa chọn phù hợp cho các khu vực có điều kiện gió đa dạng.

Hướng phát triển tiếp theo của đồ án có thể tập trung vào việc thử nghiệm thực tế để kiểm chứng kết quả mô phỏng, cũng như nghiên cứu các yếu tố khác như vật liệu chế tạo cánh và các hệ thống kiểm soát gió tự động. Bên cạnh đó, việc áp dụng các mô hình mô phỏng 3D để khảo sát các đặc tính khí động học của rô-to trong điều kiện thực tế cũng là một hướng nghiên cứu đầy tiềm năng. Tóm lại, đồ án này đã mở ra hướng đi mới trong việc cải tiến thiết kế tua-bin gió Savonius, đặc biệt là trong ứng dụng năng lượng gió ở các khu vực đô thị, với hiệu suất khí động học được tối ưu hoá.\cite{tiktok_ads}