\section{Kết quả tính toán hiệu suất khí động trung bình}

Trong phần này, hiệu suất khí động trung bình của hai biên dạng tua-bin gió, Fx và Ov01, được phân tích dựa trên hai tham số đặc trưng hiệu suất chính là hệ số mô-men $C_T$ và hệ số công suất $C_P$ theo công thức (2.32) đến (2.34), nhằm đánh giá khả năng chuyển đổi năng lượng của từng thiết kế trong dải tỉ tốc từ 0.4 đến 1.4. 

Biểu đồ phân bố hệ số mô-men $C_T$ trong Hình \ref{fig:CtCp}a  cho thấy sự thay đổi của hiệu suất mô-men khi tỉ tốc $\lambda$ tăng dần, cung cấp một cái nhìn tổng quan về hiệu quả hoạt động của từng biên dạng tua-bin trong các điều kiện tốc độ quay khác nhau.

Kết quả phân tích từ biểu đồ cho thấy xu hướng giảm của hệ số mô-men $C_T$ khi tỉ tốc tăng từ 0.4 lên 1.4 đối với cả hai biên dạng Fx và Ov01. Ở mức tỉ tốc thấp, cụ thể là tại $\lambda$ = 0.4, cả hai thiết kế đều đạt giá trị $C_T$ cao nhất, phản ánh khả năng chuyển đổi năng lượng gió thành mô-men xoắn hiệu quả hơn khi tốc độ tua-bin thấp. Cũng tại các mức tỉ tốc này, biên dạng Fx vẫn chưa có sự cải thiện hơn về giá trị hệ số mô-men $C_T$ so với biên dạng nguyên bản. Khi tỉ tốc tăng lên, $C_T$ lại giảm dần, cho thấy sự suy giảm trong hiệu suất chuyển đổi năng lượng. Điều này là đặc điểm phổ biến của các loại tua-bin gió, khi tốc độ quay tăng lên so với tốc độ gió, lượng năng lượng thu nhận được cho một vòng quay sẽ giảm, kéo theo hệ số mô-men giảm theo. Đối với biên dạng Ov01, giá trị này giảm gần như tuyến tính theo tỉ tốc, trong khi biên dạng Fx có phần giảm chậm hơn. Điểm đáng chú ý trong phân tích này là biên dạng Fx luôn duy trì hệ số mô-men $C_T$ cao hơn so với biên dạng Ov01 trong toàn bộ dải tỉ tốc cao từ 0.9, cho thấy khả năng chuyển đổi năng lượng vượt trội của thiết kế này ở các điều kiện có tốc độ quay cao.

\begin{figure} [h]
    \centering
    \includegraphics[width=1\linewidth]{img/chapter4/Cm Cp.jpg}
    \caption{So sánh (a) hệ số mô-men $C_T$, (b) hệ số công suất $C_P$ giữa hai cấu hình Ov01 và Fx.}
    \label{fig:CtCp}
\end{figure}

Tương ứng với phân bố hệ số công suất $C_P$ trong Hình \ref{fig:CtCp}b, cả hai biên dạng đều có xu hướng tăng hệ số công suất $C_P$ khi tỉ tốc tăng từ 0.4 và đạt giá trị cực đại tại một khoảng tỉ tốc nhất định, sau đó giảm dần khi tỉ tốc tiếp tục tăng. Điều này phản ánh khả năng tối ưu chuyển đổi năng lượng từ gió thành công suất ở một dải tỉ tốc nhất định trước khi hiệu suất suy giảm do sự giảm dần của lực cản khí động và các yếu tố khác khi tua-bin quay nhanh hơn. Cụ thể, biên dạng Ov01 tạo ra các giá trị $C_P$ có xu hướng biến thiên theo đường parabol, tăng lên, đạt đỉnh tại tỉ tốc 0.9 và giảm dần ở dải tỉ tốc cao, như quan sát được trong thực nghiệm của nhóm nghiên cứu Blackwell.\cite{blackwell_wind_tunnel_1977}. Trong khi đó, cấu hình Fx mặc dù có xu hướng tương tự Ov01 ở dải tỉ tốc thấp hơn 0.8 nhưng vẫn duy trì đà tăng lên cho đến khi đạt đỉnh mới tại tỉ tốc 1.1, góp phần mở rộng dải tỉ tốc hoạt động cao của tua-bin gió Savonius.

Để phân tích sâu hơn, phân bố hệ số mô-men $C_T$ trong một vòng quay tại ba giá trị tỉ tốc đặc trưng cho ba vùng tỉ tốc thấp, trung bình và cao được xem xét trong biểu đồ cực Hình \ref{fig:Ct360}. Cách xác định góc quay được đo từ tâm của rô-to đến vị trí đầu cánh như mô tả trong Hình \ref{fig:Ct360}d.

\begin{figure} [H]
    \centering
    \includegraphics[width=1\linewidth]{img/chapter4/3 360T T.jpg}
    \caption{Phân bố hệ số mô-men xoắn $C_T$ trong một vòng quay của 2 cấu hình Ov01 và Fx tại ba tỉ tốc.}
    \label{fig:Ct360}
\end{figure}

Các kết quả tính toán số cho thấy khi tỉ tốc tăng lên hay khi tua-bin quay nhanh hơn, trên cánh xuất hiện mô-men âm, như quan sát được tại tỉ tốc 1.0 (\ref{fig:Ct360}b) và 1.4 (\ref{fig:Ct360}c). Giá trị mô-men âm càng lớn sẽ cản trở chuyển động quay, làm giảm hiệu suất của tua-bin.

Tại tỉ tốc 0.5, đại diện cho dải tỉ tốc thấp, biên dạng Ov01 có khả năng sinh ra giá trị hệ số mô-men $C_T$ cao hơn của biên dạng Fx tại hai dải góc quay từ 19$^\circ$ đến 129$^\circ$ và từ 195$^\circ$ đến 305$^\circ$. Sự chênh lệch giá trị này tạo nên hai vùng mô-men ưu thế của biên dạng Ov01 tại hai khoảng góc quay trên, như ký hiệu màu đen trong Hình \ref{fig:Ct360}a. Ngược lại, hai vùng mô-men ưu thế của biên dạng Fx nằm ở các góc còn lại. Với vùng ưu thế ở dải góc rộng và tổng diện tích lớn, biên dạng Ov01 đã chứng minh được khả năng sinh ra mô-men lớn và có hiệu suất cao hơn biên dạng Fx đến 4.89\% tại tỉ tốc 0.5, thể hiện trong Hình \ref{fig:CtCp}b.

Tuy nhiên, tại hai tỉ tốc trung bình và cao, vùng mô-men ưu thế của biên dạng Fx lại có sự cải thiện đáng kể so với Ov01. Cụ thể, biên dạng Fx có diện tích vùng mô-men âm nhỏ hơn của Ov01, đồng thời, vùng ưu thế hơn (màu đỏ) được mở rộng rõ trong dải góc 21 - 115$^\circ$, 200 - 294$^\circ$ tại tỉ tốc 1.0 và 1 - 159$^\circ$, 183 - 335$^\circ$ tại tỉ tốc 1.4. Những điều này dẫn đến sự cải thiện về $C_P$ của biên dạng Fx lên tới 3.29\% và 20.47\% so với cấu hình Ov01 tại các tỉ tốc này, như trong Hình \ref{fig:CtCp}b. 

Để hiểu rõ hơn về sự cải thiện về công suất $C_P$ của cấu hình Fx, đặc tính dòng chảy xung quanh cả hai cấu hình sẽ được phân tích sâu hơn trong phần 4.3 tiếp theo. Hai tỉ tốc 0.5 và 1.4 được xem xét do sự chênh lệch lớn về giá trị $C_P$ ghi nhận tại đây.
