\begin{center}
\textbf{\large{Mở đầu}}
\end{center}
\addcontentsline{toc}{chapter}{Mở đầu}

Trong bối cảnh toàn cầu đang đối mặt với các thách thức về môi trường và sự suy giảm của nguồn năng lượng hóa thạch, năng lượng tái tạo được xem là một giải pháp then chốt cho phát triển bền vững. Trong các nguồn năng lượng tái tạo, năng lượng gió đóng vai trò trọng tâm nhờ tiềm năng khai thác dồi dào và tính lâu dài. Mặc dù vậy, việc nâng cao hiệu suất chuyển đổi năng lượng của các hệ thống tua-bin gió là một yêu cầu tất yếu để tối ưu hóa việc khai thác nguồn tài nguyên này.

Tua-bin gió Savonius trục đứng là loại tua-bin gió được đánh giá cao nhờ cấu tạo đơn giản, chi phí sản xuất thấp và khả năng hoạt động hiệu quả trong các điều kiện gió biến đổi. Những đặc điểm này khiến nó trở thành một lựa chọn phù hợp để khai thác năng lượng gió tại các khu vực đô thị hoặc nông thôn, nơi tốc độ gió thường không ổn định. Tuy nhiên, hiệu suất của tua-bin Savonius vẫn phụ thuộc vào các yếu tố môi trường bên ngoài như tốc độ gió hay nhiệt độ. Vì thế, việc nghiên cứu ảnh hưởng của nhiệt độ môi trường tới hiệu suất hoạt động của tua-bin Savonius là một câu hỏi cần được giải đáp nhằm tìm kiếm phương pháp nâng cao hiệu suất sản xuất điện của tua-bin.

Đồ án sẽ tập trung vào việc nghiên cứu ảnh hưởng của một dải nhiệt độ tới hiệu suất tua-bin gió Savonius. Với mục tiêu chính là giải đáp câu hỏi nhiệt độ có ảnh hưởng tới hiệu suất của tua-bin gió như thế nào và ở nhiệt độ nào thì tua-bin gió có thể tạo ra sản lượng điện cao nhất, từ đó có thể áp dụng tua-bin gió ở những môi trường có nhiệt độ thích hợp để tạo ra giá trị cao đồng thời tìm những giải pháp cải tiến hiệu suất tua-bin gió để không quá phụ thuộc vào điều kiện môi trường.

Nội dung chính của đồ án bao gồm 5 chương như sau:

\textbf{\textit{Chương 1: Tổng quan về xu hướng chuyển dịch năng lượng toàn cầu và nghiên cứu cải thiện hiệu suất tua-bin gió Savonius}}

Chương này trình bày một cái nhìn tổng quan về sự chuyển dịch năng lượng toàn cầu, với trọng tâm là sự dịch chuyển từ các nguồn nhiên liệu hóa thạch sang năng lượng tái tạo nhằm ứng phó với biến đổi khí hậu. Các loại hình năng lượng tái tạo chính sẽ được giới thiệu, trong đó, năng lượng gió được phân tích sâu hơn về tiềm năng khai thác tại Việt Nam. Đồng thời, chương cũng đi sâu vào lịch sử phát triển của tua-bin gió, đặc biệt là thiết kế Savonius, và những tiềm năng ứng dụng của nó. Cuối cùng, dựa trên việc đánh giá các công trình nghiên cứu hiện có về ảnh hưởng của nhiệt độ môi trường đến hiệu suất khí động học của tua-bin Savonius, chương sẽ xác định và làm rõ mục tiêu nghiên cứu của đề tài.

\textbf{\textit{Chương 2: Lý thuyết công suất tua-bin gió và phương pháp mô phỏng số động lực học chất lỏng}}

Trong chương này, định luật Betz về hiệu suất tối đa của tua-bin gió sẽ được đề cập tới, một nguyên lý cơ bản xác định hiệu suất tối đa có thể đạt được của tuabin gió. Hiệu suất của các loại tuabin, đặc biệt là\textbf{ }rô-to Savonius, được đặc trưng bởi hệ số công suất (CP) và hệ số mô-men (CT). Các hệ số này được tính toán dựa trên thiết kế của rô-to để đánh giá hiệu suất hoạt động. Để nghiên cứu sâu hơn, phương pháp mô phỏng CFD\textbf{ }(Computational Fluid Dynamics) và phần mềm Ansys\textbf{ }Fluent sẽ được sử dụng để phân tích hiệu suất một cách chi tiết và chính xác. 

\textbf{\textit{Chương 3: Quá trình mô phỏng số dòng chảy qua tua-bin gió Savonius}}

Nghiên cứu này tập trung vào quy trình thiết kế và mô phỏng. Bắt đầu với việc xây dựng mô hình tính toán, sau đó là chia lưới miền tính toán để chuẩn bị cho quá trình mô phỏng. Các điều kiện biên được thiết lập cẩn thận, cùng với việc lựa chọn thuật toán giải phù hợp. Một phần quan trọng của quy trình là cấu hình mô phỏng, trong đó chúng tôi đặc biệt chú trọng đến việc theo dõi và đảm bảo độ hội tụ của các kết quả. Việc này giúp xác nhận tính ổn định và chính xác của các số liệu thu được.

\textbf{\textit{Chương 4: Phân tích kết quả mô phỏng và thảo luận}}

Trình bày kết quả mô phỏng, bao gồm hiệu suất trung bình của tua-bin, sự phân bố áp suất, vận tốc và xoáy dòng chảy với các cấu hình được khảo sát. Kết quả này sẽ được phân tích để rút ra những nhận xét về hiệu quả của biên dạng airfoil đã lựa chọn, đồng thời so sánh với cấu hình cánh nguyên bản của tua-bin Savonius.

\textbf{\textit{Chương 5: Kết luận chung}}

Tổng kết lại các kết quả chính đã đạt được, đánh giá tính khả thi của các giải pháp thiết kế mới, và đưa ra những hướng nghiên cứu tiếp theo nhằm nâng cao hiệu quả khai thác năng lượng của tua-bin gió Savonius.


