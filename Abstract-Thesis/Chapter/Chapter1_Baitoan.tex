\section{Giới thiệu bài toán và cơ sở lý thuyết}

Tỉnh Cà Mau sở hữu hệ sinh thái rừng ngập mặn có giá trị cao về sinh thái và kinh tế, đóng vai trò quan trọng trong việc chống xói mòn bờ biển và giảm thiểu tác động của biến đổi khí hậu. Tuy nhiên, rừng ngập mặn Cà Mau đang đối mặt với nhiều áp lực từ việc chuyển đổi đất rừng sang ao nuôi tôm và khai thác không bền vững. Phương pháp giám sát truyền thống dựa trên điều tra thực địa có nhiều hạn chế về chi phí, khả năng tiếp cận và tần suất cập nhật. Công nghệ viễn thám vệ tinh cung cấp giải pháp hiệu quả hơn, trong đó chương trình Copernicus cung cấp dữ liệu miễn phí từ Sentinel-1 (ra-đa SAR hoạt động trong mọi điều kiện thời tiết) và Sentinel-2 (ảnh quang học đa phổ độ phân giải 10m).

Mạng nơ-ron tích chập (CNN) đã chứng minh hiệu quả vượt trội trong xử lý ảnh viễn thám nhờ khả năng tự động học đặc trưng phân cấp từ dữ liệu thô và khai thác tốt cấu trúc không gian của ảnh. Xuất phát từ nhu cầu giám sát rừng hiệu quả, đồ án này phát triển mô hình CNN kết hợp dữ liệu Sentinel-1 và Sentinel-2 để phát hiện biến động rừng tại khu vực ranh giới lâm nghiệp tỉnh Cà Mau (170,179 ha). Các trạng thái biến động được phân thành bốn lớp: rừng ổn định, mất rừng, phi rừng, và phục hồi rừng. Dữ liệu sử dụng bao gồm ảnh từ tháng 01/2024 đến tháng 02/2025 trong mùa khô để giảm ảnh hưởng của mây.
