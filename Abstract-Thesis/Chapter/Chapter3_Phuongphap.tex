\section{Dữ liệu và phương pháp nghiên cứu}

Quy trình nghiên cứu được minh họa trong Hình~\ref{fig:methodology-flowchart}, bao gồm năm giai đoạn chính: thu thập dữ liệu, tiền xử lý, xây dựng bộ dữ liệu huấn luyện, thiết kế và huấn luyện mô hình CNN, và áp dụng mô hình cho phân loại toàn vùng.

\begin{figure}[H]
\centering
\includegraphics[width=0.8\textwidth]{chapter2/flowchart.png}
\caption{Sơ đồ quy trình phương pháp nghiên cứu}
\label{fig:methodology-flowchart}
\end{figure}

Nghiên cứu sử dụng dữ liệu từ hai nguồn chính được thu thập qua Google Earth Engine. Sentinel-2 Level-2A Surface Reflectance cung cấp 4 kênh phổ quang học: B4 (Red, 665nm), B8 (NIR, 842nm), B11 (SWIR1, 1610nm) và B12 (SWIR2, 2190nm), tất cả ở độ phân giải 10m. Từ các kênh này, ba chỉ số thực vật được tính toán: NDVI = (NIR - Red) / (NIR + Red) đo mật độ thực vật, NBR = (NIR - SWIR2) / (NIR + SWIR2) phát hiện biến động rừng, và NDMI = (NIR - SWIR1) / (NIR + SWIR1) đo độ ẩm thực vật. Sentinel-1 GRD cung cấp dữ liệu ra-đa với hai kênh phân cực VV và VH ở độ phân giải 10m, hoạt động độc lập với thời tiết và ánh sáng. Dữ liệu được thu thập cho hai thời kỳ: tháng 01/2024 (kỳ trước) và tháng 02/2025 (kỳ sau).

Tổng cộng 27 đặc trưng được xây dựng: 21 từ Sentinel-2 (7 kênh/chỉ số × 3 thời điểm) và 6 từ Sentinel-1 (2 kênh × 3 thời điểm). Ba thời điểm bao gồm: kỳ trước (T1), kỳ sau (T2), và delta ($\Delta$ = T2 - T1). Giá trị delta đóng vai trò quan trọng trong phát hiện biến động; ví dụ khi rừng bị chặt phá, $\Delta$NDVI sẽ giảm mạnh. Với mỗi điểm mẫu, một patch 3×3 điểm ảnh (30m × 30m) được trích xuất để cung cấp thông tin ngữ cảnh không gian.

Bộ dữ liệu thực địa gồm 2,630 điểm được thu thập bằng phương pháp lấy mẫu phân tầng ngẫu nhiên, phân bố cân bằng cho 4 lớp: Rừng ổn định (660), Mất rừng (660), Phi rừng (660), và Phục hồi rừng (650). Dữ liệu được chuẩn hóa Z-score và chia theo tỷ lệ 80/20 cho huấn luyện/kiểm tra, với kiểm định chéo 5 phần trên tập huấn luyện.

Kiến trúc CNN được thiết kế đặc biệt cho bộ dữ liệu quy mô vừa phải, minh họa trong Hình~\ref{fig:cnn_architecture}. Mô hình gồm: đầu vào tensor (N, 27, 3, 3); khối tích chập thứ nhất với 64 bộ lọc 3×3, Batch Norm, ReLU, Dropout2D 70\%; khối tích chập thứ hai với 32 bộ lọc 3×3, Batch Norm, ReLU, Dropout2D 70\%; Global Average Pooling; và lớp kết nối đầy đủ 32→64→4. Tổng số tham số: 36,676.

\begin{figure}[H]
\centering
\includegraphics[width=0.75\textwidth]{chapter2/CNN-architecture.png}
\caption{Kiến trúc mạng nơ-ron tích chập}
\label{fig:cnn_architecture}
\end{figure}

Mô hình được huấn luyện với AdamW optimizer, learning rate 0.001, weight decay $10^{-3}$, batch size 64, tối đa 200 epochs với early stopping (patience 15). Trọng số được khởi tạo theo phương pháp Kaiming/He. Sau huấn luyện, mô hình được áp dụng để phân loại toàn bộ 16.2 triệu điểm ảnh theo lô 10,000 điểm, sử dụng mixed precision FP16. Kết quả xuất dạng GeoTIFF với hệ quy chiếu EPSG:32648.
