\section{Kết luận và hướng phát triển}

Đồ án đã hoàn thành đầy đủ các mục tiêu nghiên cứu đề ra với kết quả vượt kỳ vọng ban đầu. Về xây dựng bộ dữ liệu, nghiên cứu đã xây dựng thành công bộ dữ liệu gồm 2,630 điểm mẫu với 27 đặc trưng, bao gồm 21 đặc trưng từ Sentinel-2 (4 kênh phổ quang học và 3 chỉ số thực vật) và 6 đặc trưng từ Sentinel-1 (2 kênh phân cực), mỗi đặc trưng được tính cho hai thời kỳ và giá trị delta. Bộ dữ liệu được phân bố cân bằng cho 4 lớp phân loại, đảm bảo tính đại diện cho toàn vùng nghiên cứu.

Về thiết kế kiến trúc CNN, mô hình với 36,676 tham số được thiết kế phù hợp với bộ dữ liệu quy mô vừa, sử dụng các kỹ thuật điều chuẩn hiệu quả (Batch Normalization, Dropout 70\%, weight decay) để ngăn ngừa quá khớp. Thông qua nghiên cứu loại trừ có hệ thống, kích thước patch 3×3 được xác định là tối ưu, đạt accuracy 98.86\% trên tập kiểm tra và 98.48\% ± 0.36\% qua kiểm định chéo 5 phần.

Về đánh giá tích hợp đa nguồn, nghiên cứu loại trừ toàn diện chứng minh việc kết hợp Sentinel-1 và Sentinel-2 cải thiện accuracy 5.44 điểm phần trăm so với chỉ sử dụng Sentinel-2 đơn lẻ (từ 93.42\% lên 98.86\%). Kết quả này khẳng định giả thuyết nghiên cứu rằng dữ liệu ra-đa và quang học có tính bổ sung cao, trong đó Sentinel-2 đóng vai trò chủ đạo cung cấp thông tin quang phổ chi tiết, còn Sentinel-1 bổ sung thông tin về cấu trúc và độ ẩm bề mặt.

Về áp dụng mô hình toàn vùng, mô hình được triển khai thành công để phân loại 162,469 ha (95.5\% diện tích ranh giới lâm nghiệp), phát hiện 7,282 ha mất rừng (4.48\%) và 4,941 ha phục hồi rừng (3.04\%). Diện tích mất rừng ròng trong 13 tháng là 2,341 ha (khoảng 1.44\%). Bản đồ phân loại với độ phân giải 10m cung cấp thông tin chi tiết về phân bố không gian các loại biến động, hỗ trợ trực tiếp công tác quản lý rừng.

Về đóng góp khoa học và thực tiễn, đồ án đề xuất quy trình tích hợp dữ liệu đa nguồn hiệu quả kết hợp ra-đa Sentinel-1 và quang học Sentinel-2, thiết kế kiến trúc CNN phù hợp cho bộ dữ liệu nhỏ tránh hiện tượng quá khớp, đề xuất cấu trúc vector đặc trưng 27 chiều tổng hợp thông tin từ hai nguồn và ba thời điểm, và tạo ra bản đồ biến động rừng độ phân giải cao hỗ trợ công tác quản lý lâm nghiệp tại tỉnh Cà Mau.

Hạn chế của nghiên cứu bao gồm: thời gian dự đoán toàn vùng còn dài (14.83 phút cho 16.2 triệu điểm ảnh), chưa đáp ứng yêu cầu xử lý thời gian thực; khả năng giải thích mô hình hạn chế do tính chất ``hộp đen'' của CNN; quy mô dữ liệu thực địa còn nhỏ và chưa có khảo sát độc lập để kiểm chứng; phân tích chỉ ở hai thời điểm, chưa khai thác chuỗi thời gian đầy đủ.

Hướng phát triển tiếp theo bao gồm: mở rộng phân tích đa thời gian sử dụng chuỗi 5--10 năm với các mô hình LSTM hoặc Transformer để phát hiện xu hướng và dự báo biến động; cải thiện mô hình qua transfer learning từ các mô hình pretrained và kỹ thuật ensemble; mở rộng ứng dụng cho các tỉnh khác trong Đồng bằng sông Cửu Long để đánh giá khả năng tổng quát hóa; tích hợp kỹ thuật Explainable AI như Grad-CAM và SHAP để tăng khả năng giải thích mô hình.
