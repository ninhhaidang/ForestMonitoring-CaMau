\section{Mục tiêu và nội dung nghiên cứu}

Mục tiêu tổng quát của nghiên cứu là ứng dụng mô hình học sâu dựa trên kiến trúc mạng nơ-ron tích chập để phát hiện và phân loại biến động rừng tại khu vực quy hoạch lâm nghiệp tỉnh Cà Mau với độ chính xác cao. Nghiên cứu tập trung vào việc tích hợp dữ liệu đa nguồn từ vệ tinh Sentinel-1 (ra-đa khẩu độ tổng hợp) và Sentinel-2 (quang học đa phổ) để khai thác tối đa thông tin về trạng thái lớp phủ rừng qua hai thời kỳ quan sát. Giả thuyết nghiên cứu được đặt ra là mô hình CNN kết hợp đặc trưng từ cả hai nguồn dữ liệu ra-đa và quang học có thể đạt độ chính xác cao hơn so với việc chỉ sử dụng một nguồn dữ liệu đơn lẻ.

Kết quả nghiên cứu không chỉ có ý nghĩa khoa học trong việc đề xuất kiến trúc CNN tối ưu cho bài toán phân loại ảnh viễn thám với dữ liệu hạn chế, mà còn có giá trị thực tiễn cao. Mô hình có thể được triển khai như một công cụ hỗ trợ quan trọng cho các cơ quan quản lý lâm nghiệp trong công tác giám sát và bảo vệ rừng, giúp phát hiện sớm các hoạt động mất rừng bất hợp pháp.

Để đạt được mục tiêu trên, đồ án thực hiện bốn nội dung nghiên cứu chính. Nội dung thứ nhất là xây dựng bộ dữ liệu huấn luyện, tập trung vào việc thu thập và xử lý dữ liệu viễn thám từ Sentinel-1 và Sentinel-2 thông qua nền tảng Google Earth Engine. Các công việc bao gồm: tiền xử lý dữ liệu (lọc mây cho Sentinel-2 với ngưỡng 50\%, tạo mosaic, chuẩn hóa hệ quy chiếu EPSG:32648 và độ phân giải 10m), trích xuất 27 đặc trưng (21 từ Sentinel-2 gồm 4 kênh phổ và 3 chỉ số thực vật, 6 từ Sentinel-1 gồm 2 kênh phân cực, mỗi đặc trưng có 3 giá trị: trước, sau, delta), và thu thập khoảng 2,600 điểm mẫu thực địa bằng phương pháp lấy mẫu phân tầng ngẫu nhiên, đảm bảo phân bố cân bằng cho 4 lớp phân loại.

Nội dung thứ hai là thiết kế và tối ưu hóa kiến trúc CNN phù hợp với đặc thù của bài toán và quy mô dữ liệu. Kiến trúc được thiết kế nhẹ (dưới 50,000 tham số) để phù hợp với bộ dữ liệu nhỏ và tránh quá khớp. Các kỹ thuật điều chuẩn được áp dụng bao gồm Batch Normalization, Dropout với tỷ lệ cao (60--70\%), và phân rã trọng số. Nghiên cứu thử nghiệm với các kích thước patch khác nhau (1×1, 3×3, 5×5, 7×7, 9×9) để xác định cấu hình tối ưu, và sử dụng kiểm định chéo 5 phần để tìm kiếm siêu tham số tối ưu.

Nội dung thứ ba là đánh giá hiệu quả tích hợp đa nguồn thông qua phương pháp nghiên cứu loại trừ (ablation study) với ba kịch bản: chỉ sử dụng Sentinel-2 (21 đặc trưng), chỉ sử dụng Sentinel-1 (6 đặc trưng), và kết hợp cả hai nguồn (27 đặc trưng). So sánh kết quả giữa ba kịch bản cho phép định lượng mức độ cải thiện khi tích hợp đa nguồn và xác định vai trò của từng nguồn dữ liệu.

Nội dung cuối cùng là áp dụng mô hình đã tối ưu hóa để tạo bản đồ biến động rừng cho toàn bộ khu vực nghiên cứu. Dữ liệu raster toàn vùng (khoảng 16 triệu điểm ảnh) được xử lý theo lô để tránh tràn bộ nhớ, áp dụng kỹ thuật mixed precision (FP16) để tăng tốc. Kết quả được xuất dưới dạng GeoTIFF với độ phân giải 10m, kèm theo báo cáo thống kê chi tiết về diện tích và phân bố không gian của từng loại biến động.
