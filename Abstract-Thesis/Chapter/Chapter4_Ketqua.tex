\section{Kết quả thực nghiệm và thảo luận}

Thử nghiệm với 5 kích thước patch khác nhau cho thấy patch 3×3 đạt accuracy cao nhất (98.86\%) với số lượng tham số hợp lý (36,676). Patch 1×1 đạt 97.34\%, thấp hơn do thiếu thông tin ngữ cảnh không gian từ các điểm ảnh lân cận. Các patch lớn hơn (5×5, 7×7, 9×9) có accuracy giảm dần (98.67\%, 98.29\%, 97.91\%) mặc dù số tham số tăng đáng kể, nguyên nhân là chứa nhiễu từ điểm ảnh xa trung tâm và tăng nguy cơ quá khớp. Do đó, patch 3×3 được lựa chọn làm cấu hình tối ưu.

Nghiên cứu loại trừ có hệ thống đánh giá đóng góp của từng nguồn dữ liệu. Kịch bản chỉ Sentinel-2 (21 đặc trưng) đạt 93.42\%, cho thấy dữ liệu quang học đóng vai trò chủ đạo với các chỉ số thực vật rất nhạy với sự thay đổi thực vật. Kịch bản chỉ Sentinel-1 (6 đặc trưng) đạt 73.38\%, cho thấy dữ liệu ra-đa đơn thuần không đủ để phân loại chính xác, nhưng vẫn cung cấp thông tin bổ sung về cấu trúc và độ ẩm bề mặt. Kịch bản tích hợp S1+S2 (27 đặc trưng) đạt 98.86\%, cải thiện 5.44 điểm phần trăm so với chỉ Sentinel-2. Kết quả này chứng minh Sentinel-1 và Sentinel-2 có tính bổ sung cao, trong đó Sentinel-2 đóng vai trò chính còn Sentinel-1 bổ sung thông tin cấu trúc, giúp phân biệt tốt hơn các lớp có quang phổ tương đồng.

Đánh giá trên tập kiểm tra cố định (526 mẫu) cho kết quả chi tiết: lớp Phi rừng và Phục hồi rừng đạt 100\% precision, recall và F1-score, cho thấy mô hình phân biệt rất tốt các lớp này. Lớp Rừng ổn định đạt precision 98.48\%, recall 97.73\%, F1 98.10\%. Lớp Mất rừng đạt precision 97.73\%, recall 96.97\%, F1 97.35\%. Lỗi phân loại chủ yếu xảy ra giữa hai lớp này do đặc trưng quang phổ tương đồng tại vùng ranh giới. Macro average đạt precision 99.05\%, recall 98.67\%, F1 98.86\%. Kiểm định chéo 5 phần cho accuracy 98.48\% ± 0.36\%, độ lệch chuẩn nhỏ chứng minh mô hình ổn định, không phụ thuộc vào cách chia dữ liệu.

Áp dụng mô hình phân loại toàn vùng 162,469 ha cho kết quả: Rừng ổn định 120,717 ha (74.30\%), tập trung ở Vườn Quốc gia Mũi Cà Mau và các khu rừng phòng hộ ven biển; Mất rừng 7,282 ha (4.48\%), phân tán chủ yếu ở vùng tiếp giáp ao nuôi tôm do chuyển đổi mục đích sử dụng đất; Phi rừng 29,529 ha (18.17\%), bao gồm khu dân cư, đất nông nghiệp và ao nuôi thủy sản; Phục hồi rừng 4,941 ha (3.04\%), kết quả của các chương trình trồng rừng. Mất rừng ròng trong 13 tháng là 2,341 ha (khoảng 1.44\%).

\begin{figure}[H]
\centering
\includegraphics[width=0.9\textwidth]{chapter3/Classification.png}
\caption{Bản đồ phân loại biến động rừng toàn vùng tỉnh Cà Mau}
\label{fig:classification_map}
\end{figure}

So sánh với các nghiên cứu trước về giám sát rừng, mô hình CNN kết hợp Sentinel-1/2 của nghiên cứu này đạt accuracy 98.86\%, vượt trội so với Hansen và cộng sự (2013) sử dụng Decision Tree trên Landsat 30m đạt khoảng 85\%, Ortega và cộng sự (2020) sử dụng U-Net trên Landsat đạt khoảng 94\%, và Fayaz và cộng sự (2024) tổng hợp các nghiên cứu U-Net đạt 94--97\%. Lý do chính cho kết quả vượt trội bao gồm: tích hợp đa nguồn dữ liệu SAR và quang học, độ phân giải không gian cao hơn (10m so với 30m), và kiến trúc CNN được tối ưu hóa cho bài toán cụ thể với bộ dữ liệu quy mô vừa.
