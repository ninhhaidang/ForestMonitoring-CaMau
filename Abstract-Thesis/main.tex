\documentclass[a4paper,13pt]{article}

% --------------------------------------------------------------------
% Compile instructions:
% - To compile with XeLaTeX (recommended for Times New Roman and Unicode):
%     latexmk -pdfxe thesis.tex
%   or
%     xelatex thesis.tex
%
% - To compile with LuaLaTeX, use:
%     latexmk -pdflua thesis.tex
%
% - To compile with pdfLaTeX (default behaviour):
%     latexmk -pdf thesis.tex
%   (This uses Latin Modern and T5 encoding for Vietnamese.)
% --------------------------------------------------------------------

% Encoding and Vietnamese support
% This file supports both pdfLaTeX and XeLaTeX/LuaLaTeX. The default
% behaviour below keeps the existing pdfLaTeX setup (Latin Modern + T5
% encoding for Vietnamese). If you switch to XeLaTeX/LuaLaTeX, the
% document will use system fonts via `fontspec` (set to Times New Roman).
\usepackage{iftex}
\ifPDFTeX
	% pdfLaTeX: keep older encoding packages and Latin Modern.
	\usepackage[utf8]{inputenc} % keep for pdfLaTeX; not used with XeLaTeX
	\usepackage[T5]{fontenc}    % T5 encoding (Vietnamese) - required for vntex/vietnam
	\usepackage[utf8]{vietnam}  % Vietnamese support (vntex)
	% If you want a Times-like font with pdfLaTeX you can enable the
	% following packages, but note: T5 encoding support and some
	% Vietnamese glyphs may not be complete with these fonts. Test the
	% output before enabling.
	% \usepackage{newtxtext,newtxmath} % Times-like fonts (for pdfLaTeX)
\else
	% XeLaTeX/LuaLaTeX: use fontspec and polyglossia for Unicode & Vietnamese
	% NOTE: If your source uses TeX accent macros (e.g., `\h{a}` or `\'a`)
	% instead of raw Unicode characters, you may need to either (A) keep
	% the `vietnam` package loaded or (B) convert source files to Unicode
	% characters for Vietnamese. Loading both `polyglossia` and `vietnam`
	% may cause conflicts; test carefully.
	\usepackage{fontspec}
	\usepackage{polyglossia}
	% If your source contains TeX accent macros rather than raw Unicode
	% characters (e.g. `\h{a}`), uncomment the following line to load
	% the `vietnam` package which defines such macros. Keep in mind
	% that `vietnam` is designed for T5 encoding and may sometimes
	% overlap with `polyglossia` macros.
	\usepackage[utf8]{vietnam}
	\setmainlanguage{vietnamese}
	% Set Times New Roman as the main font. You can change the name if
	% your OS has a differently named Times family (e.g., "Times").
	\setmainfont{Times New Roman}[
		Ligatures=TeX,
		Script=Latin
	]
	% If you want math fonts to match Times, you can add e.g.:
	% \usepackage{unicode-math}
	% \setmathfont{TeX Gyre Termes Math}
	% Optional: specify sans/mono fonts to better match system defaults
	\setsansfont{Arial}
	\setmonofont{Courier New}
\fi
\usepackage{amsmath, amsfonts, amssymb}
\usepackage{textcomp}

% Note: If you prefer to compile with XeLaTeX or LuaLaTeX (recommended for Unicode):
%  - remove or comment out the `inputenc` and `fontenc` lines above
%  - uncomment the fontspec lines below and set an appropriate system font
%  - compile with: `latexmk -pdfxe thesis.tex` or `latexmk -pdflua thesis.tex`
%
% %%% XeLaTeX / LuaTeX example (uncomment when using XeLaTeX/LuaTeX):
% \usepackage{fontspec}
% \setmainfont{Times New Roman}
% \usepackage{polyglossia}
% \setmainlanguage{vietnamese}

% set font, font size and spacing
% If compiling with pdfLaTeX, keep Latin Modern for T5 coverage. For
% XeLaTeX/LuaLaTeX the fontspec package (above) will set Times.
\ifPDFTeX
	\usepackage{lmodern}
\fi
\usepackage{microtype} % micro-typography
\usepackage{scrextend}
\changefontsizes{13pt}
\renewcommand{\baselinestretch}{1.3}

% Hình ảnh và đồ họa
\usepackage{graphicx}
\usepackage{tikz}
\usetikzlibrary{shapes.geometric, arrows, positioning, arrows.meta, calc, fit}
% PGFPlots: required for axis environment and plotting
\usepackage{pgfplots}
\pgfplotsset{compat=1.18}
\usepgfplotslibrary{fillbetween}
% Pie chart support
\usepackage{pgf-pie}
% Disable thousand separators by default for axis tick labels (e.g., 1990 shows as 1990, not 1,990)
\pgfplotsset{every axis/.append style={xticklabel style={/pgf/number format/1000 sep={}}}}

% Định dạng bảng và caption
\usepackage{array}
\usepackage{booktabs}
% Combine caption options to avoid duplicate package load and enable sub-captions
\usepackage[justification=centering,font=small]{caption}
% Sub-figures support (provides the `subfigure` environment used in chapters)
\usepackage{subcaption}
\usepackage{makecell}
\usepackage{multirow}

% Điều chỉnh lề và layout trang
\usepackage[left=3cm,right=2cm,top=2.5cm,bottom=3cm]{geometry}

% Hyperlinks và danh mục
\usepackage[unicode, breaklinks=true, hidelinks]{hyperref}
\usepackage{xurl} % Cho phép URL tự động xuống dòng

% Tùy chỉnh các phần khác
\usepackage{scrextend, enumerate, float, afterpage, sectsty, tocloft, calc, listings}

% Thêm dấu chấm (...) cho section trong mục lục
\renewcommand{\cftsecdotsep}{\cftdotsep}

% Định dạng tiêu đề section (phải đặt SAU sectsty và tocloft)
\usepackage{titlesec}
\titleformat{\section}[block]{\normalfont\fontsize{13pt}{16pt}\selectfont\bfseries}{\thesection.}{0.5em}{\MakeUppercase}
\titleformat{name=\section,numberless}[block]{\normalfont\fontsize{13pt}{16pt}\selectfont\bfseries}{}{0pt}{\MakeUppercase}
\titlespacing{\section}{0pt}{20pt}{10pt}

% Định dạng chung cho các tiêu đề không đánh số: chữ hoa, 13pt, đậm, căn giữa
\newcommand{\unnumberedchapter}[1]{%
  \vspace*{-20pt}%
  \begin{center}
    {\fontsize{13pt}{16pt}\selectfont\textbf{\MakeUppercase{#1}}}
  \end{center}
  \vspace{10pt}%
}


\usepackage{pdflscape} % Gói giúp xoay trang ngang trong PDF
\usepackage{graphicx}
\usepackage{float} % Để cố định vị trí bảng
\graphicspath{{Figure/}}
\usepackage{longtable} % Nếu bảng dài
\usepackage{lscape} % Để hỗ trợ xoay trang (tuỳ chọn)

% Gói hỗ trợ thuật toán
\usepackage[linesnumbered,ruled,vlined]{algorithm2e}
\usepackage{optidef}

% Điều chỉnh các thuật toán
\SetAlgorithmName{Thuật toán}{Thuật toán}{Danh sách các Thuật toán}
\SetKwInput{KwIn}{Đầu vào}
\SetKwInput{KwOut}{Đầu ra}

% Định dạng danh mục tài liệu tham khảo
\usepackage[numbers,sort&compress]{natbib}
\usepackage{multibib}
\newcites{vi,en}{{},{}}

% Loại bỏ chapter header của multibib
\makeatletter
\renewcommand{\bibsection}{}
\makeatother

% Định dạng đánh số section
\renewcommand{\thesection}{\arabic{section}}

% Điều chỉnh khoảng cách giữa các đoạn
\usepackage{indentfirst}
\setlength{\parskip}{6pt}
\setlength{\parindent}{1cm}

% Tùy chỉnh chú thích
\usepackage{perpage} 
\MakePerPage{footnote}

% Khoảng cách dòng
\renewcommand{\baselinestretch}{1.3}

% Tùy chỉnh mục lục
% (Không sử dụng chapter trong article class)

% Thêm tiền tố "Hình" vào danh mục hình ảnh
\renewcommand{\cftfigpresnum}{Hình }
\setlength{\cftfignumwidth}{4.5em}

% Thêm tiền tố "Bảng" vào danh mục bảng
\renewcommand{\cfttabpresnum}{Bảng }
\setlength{\cfttabnumwidth}{4.5em}

% Thiết lập tiêu đề và tác giả
\title{Ứng dụng viễn thám và học sâu trong giám sát biến động rừng tỉnh Cà Mau}
\author{Ninh Hải Đăng} 


\newcommand{\argmax}{\arg\!\max}

\begin{document}
% Đảm bảo thụt đầu dòng 1cm cho mọi đoạn văn (kể cả đoạn đầu tiên sau tiêu đề)
\setlength{\parindent}{1cm}

% Bìa (không đánh số trang)
\thispagestyle{empty}
\input{Cover/Sub_cover}
\newpage

% Đánh số trang kiểu Ả Rập bắt đầu từ trang 1
\pagenumbering{arabic}
\setcounter{page}{1}
\setcounter{secnumdepth}{2}

% Các section nội dung (không ngắt trang giữa các section)
\section{Giới thiệu bài toán và cơ sở lý thuyết}

Tỉnh Cà Mau sở hữu hệ sinh thái rừng ngập mặn có giá trị cao về sinh thái và kinh tế, đóng vai trò quan trọng trong việc chống xói mòn bờ biển và giảm thiểu tác động của biến đổi khí hậu. Tuy nhiên, rừng ngập mặn Cà Mau đang đối mặt với nhiều áp lực từ việc chuyển đổi đất rừng sang ao nuôi tôm và khai thác không bền vững. Phương pháp giám sát truyền thống dựa trên điều tra thực địa có nhiều hạn chế về chi phí, khả năng tiếp cận và tần suất cập nhật. Công nghệ viễn thám vệ tinh cung cấp giải pháp hiệu quả hơn, trong đó chương trình Copernicus cung cấp dữ liệu miễn phí từ Sentinel-1 (ra-đa SAR hoạt động trong mọi điều kiện thời tiết) và Sentinel-2 (ảnh quang học đa phổ độ phân giải 10m).

Mạng nơ-ron tích chập (CNN) đã chứng minh hiệu quả vượt trội trong xử lý ảnh viễn thám nhờ khả năng tự động học đặc trưng phân cấp từ dữ liệu thô và khai thác tốt cấu trúc không gian của ảnh. Xuất phát từ nhu cầu giám sát rừng hiệu quả, đồ án này phát triển mô hình CNN kết hợp dữ liệu Sentinel-1 và Sentinel-2 để phát hiện biến động rừng tại khu vực ranh giới lâm nghiệp tỉnh Cà Mau (170,179 ha). Các trạng thái biến động được phân thành bốn lớp: rừng ổn định, mất rừng, phi rừng, và phục hồi rừng. Dữ liệu sử dụng bao gồm ảnh từ tháng 01/2024 đến tháng 02/2025 trong mùa khô để giảm ảnh hưởng của mây.

\section{Mục tiêu và nội dung nghiên cứu}

Mục tiêu tổng quát của nghiên cứu là ứng dụng mô hình học sâu dựa trên kiến trúc mạng nơ-ron tích chập để phát hiện và phân loại biến động rừng tại khu vực quy hoạch lâm nghiệp tỉnh Cà Mau với độ chính xác cao. Nghiên cứu tập trung vào việc tích hợp dữ liệu đa nguồn từ vệ tinh Sentinel-1 (ra-đa khẩu độ tổng hợp) và Sentinel-2 (quang học đa phổ) để khai thác tối đa thông tin về trạng thái lớp phủ rừng qua hai thời kỳ quan sát. Giả thuyết nghiên cứu được đặt ra là mô hình CNN kết hợp đặc trưng từ cả hai nguồn dữ liệu ra-đa và quang học có thể đạt độ chính xác cao hơn so với việc chỉ sử dụng một nguồn dữ liệu đơn lẻ.

Kết quả nghiên cứu không chỉ có ý nghĩa khoa học trong việc đề xuất kiến trúc CNN tối ưu cho bài toán phân loại ảnh viễn thám với dữ liệu hạn chế, mà còn có giá trị thực tiễn cao. Mô hình có thể được triển khai như một công cụ hỗ trợ quan trọng cho các cơ quan quản lý lâm nghiệp trong công tác giám sát và bảo vệ rừng, giúp phát hiện sớm các hoạt động mất rừng bất hợp pháp.

Để đạt được mục tiêu trên, đồ án thực hiện bốn nội dung nghiên cứu chính. Nội dung thứ nhất là xây dựng bộ dữ liệu huấn luyện, tập trung vào việc thu thập và xử lý dữ liệu viễn thám từ Sentinel-1 và Sentinel-2 thông qua nền tảng Google Earth Engine. Các công việc bao gồm: tiền xử lý dữ liệu (lọc mây cho Sentinel-2 với ngưỡng 50\%, tạo mosaic, chuẩn hóa hệ quy chiếu EPSG:32648 và độ phân giải 10m), trích xuất 27 đặc trưng (21 từ Sentinel-2 gồm 4 kênh phổ và 3 chỉ số thực vật, 6 từ Sentinel-1 gồm 2 kênh phân cực, mỗi đặc trưng có 3 giá trị: trước, sau, delta), và thu thập khoảng 2,600 điểm mẫu thực địa bằng phương pháp lấy mẫu phân tầng ngẫu nhiên, đảm bảo phân bố cân bằng cho 4 lớp phân loại.

Nội dung thứ hai là thiết kế và tối ưu hóa kiến trúc CNN phù hợp với đặc thù của bài toán và quy mô dữ liệu. Kiến trúc được thiết kế nhẹ (dưới 50,000 tham số) để phù hợp với bộ dữ liệu nhỏ và tránh quá khớp. Các kỹ thuật điều chuẩn được áp dụng bao gồm Batch Normalization, Dropout với tỷ lệ cao (60--70\%), và phân rã trọng số. Nghiên cứu thử nghiệm với các kích thước patch khác nhau (1×1, 3×3, 5×5, 7×7, 9×9) để xác định cấu hình tối ưu, và sử dụng kiểm định chéo 5 phần để tìm kiếm siêu tham số tối ưu.

Nội dung thứ ba là đánh giá hiệu quả tích hợp đa nguồn thông qua phương pháp nghiên cứu loại trừ (ablation study) với ba kịch bản: chỉ sử dụng Sentinel-2 (21 đặc trưng), chỉ sử dụng Sentinel-1 (6 đặc trưng), và kết hợp cả hai nguồn (27 đặc trưng). So sánh kết quả giữa ba kịch bản cho phép định lượng mức độ cải thiện khi tích hợp đa nguồn và xác định vai trò của từng nguồn dữ liệu.

Nội dung cuối cùng là áp dụng mô hình đã tối ưu hóa để tạo bản đồ biến động rừng cho toàn bộ khu vực nghiên cứu. Dữ liệu raster toàn vùng (khoảng 16 triệu điểm ảnh) được xử lý theo lô để tránh tràn bộ nhớ, áp dụng kỹ thuật mixed precision (FP16) để tăng tốc. Kết quả được xuất dưới dạng GeoTIFF với độ phân giải 10m, kèm theo báo cáo thống kê chi tiết về diện tích và phân bố không gian của từng loại biến động.

\section{Dữ liệu và phương pháp nghiên cứu}

Quy trình nghiên cứu được minh họa trong Hình~\ref{fig:methodology-flowchart}, bao gồm năm giai đoạn chính: thu thập dữ liệu, tiền xử lý, xây dựng bộ dữ liệu huấn luyện, thiết kế và huấn luyện mô hình CNN, và áp dụng mô hình cho phân loại toàn vùng.

\begin{figure}[H]
\centering
\includegraphics[width=0.8\textwidth]{chapter2/flowchart.png}
\caption{Sơ đồ quy trình phương pháp nghiên cứu}
\label{fig:methodology-flowchart}
\end{figure}

Nghiên cứu sử dụng dữ liệu từ hai nguồn chính được thu thập qua Google Earth Engine. Sentinel-2 Level-2A Surface Reflectance cung cấp 4 kênh phổ quang học: B4 (Red, 665nm), B8 (NIR, 842nm), B11 (SWIR1, 1610nm) và B12 (SWIR2, 2190nm), tất cả ở độ phân giải 10m. Từ các kênh này, ba chỉ số thực vật được tính toán: NDVI = (NIR - Red) / (NIR + Red) đo mật độ thực vật, NBR = (NIR - SWIR2) / (NIR + SWIR2) phát hiện biến động rừng, và NDMI = (NIR - SWIR1) / (NIR + SWIR1) đo độ ẩm thực vật. Sentinel-1 GRD cung cấp dữ liệu ra-đa với hai kênh phân cực VV và VH ở độ phân giải 10m, hoạt động độc lập với thời tiết và ánh sáng. Dữ liệu được thu thập cho hai thời kỳ: tháng 01/2024 (kỳ trước) và tháng 02/2025 (kỳ sau).

Tổng cộng 27 đặc trưng được xây dựng: 21 từ Sentinel-2 (7 kênh/chỉ số × 3 thời điểm) và 6 từ Sentinel-1 (2 kênh × 3 thời điểm). Ba thời điểm bao gồm: kỳ trước (T1), kỳ sau (T2), và delta ($\Delta$ = T2 - T1). Giá trị delta đóng vai trò quan trọng trong phát hiện biến động; ví dụ khi rừng bị chặt phá, $\Delta$NDVI sẽ giảm mạnh. Với mỗi điểm mẫu, một patch 3×3 điểm ảnh (30m × 30m) được trích xuất để cung cấp thông tin ngữ cảnh không gian.

Bộ dữ liệu thực địa gồm 2,630 điểm được thu thập bằng phương pháp lấy mẫu phân tầng ngẫu nhiên, phân bố cân bằng cho 4 lớp: Rừng ổn định (660), Mất rừng (660), Phi rừng (660), và Phục hồi rừng (650). Dữ liệu được chuẩn hóa Z-score và chia theo tỷ lệ 80/20 cho huấn luyện/kiểm tra, với kiểm định chéo 5 phần trên tập huấn luyện.

Kiến trúc CNN được thiết kế đặc biệt cho bộ dữ liệu quy mô vừa phải, minh họa trong Hình~\ref{fig:cnn_architecture}. Mô hình gồm: đầu vào tensor (N, 27, 3, 3); khối tích chập thứ nhất với 64 bộ lọc 3×3, Batch Norm, ReLU, Dropout2D 70\%; khối tích chập thứ hai với 32 bộ lọc 3×3, Batch Norm, ReLU, Dropout2D 70\%; Global Average Pooling; và lớp kết nối đầy đủ 32→64→4. Tổng số tham số: 36,676.

\begin{figure}[H]
\centering
\includegraphics[width=0.75\textwidth]{chapter2/CNN-architecture.png}
\caption{Kiến trúc mạng nơ-ron tích chập}
\label{fig:cnn_architecture}
\end{figure}

Mô hình được huấn luyện với AdamW optimizer, learning rate 0.001, weight decay $10^{-3}$, batch size 64, tối đa 200 epochs với early stopping (patience 15). Trọng số được khởi tạo theo phương pháp Kaiming/He. Sau huấn luyện, mô hình được áp dụng để phân loại toàn bộ 16.2 triệu điểm ảnh theo lô 10,000 điểm, sử dụng mixed precision FP16. Kết quả xuất dạng GeoTIFF với hệ quy chiếu EPSG:32648.

\chapter{Kết quả}

\section{Lựa chọn kích thước patch tối ưu}

Thử nghiệm được thực hiện với 5 kích thước patch khác nhau để xác định cấu hình tối ưu. Kết quả được tổng hợp trong Bảng~\ref{tab:patch_size_comparison}.

\begin{table}[H]
\centering
\caption{So sánh hiệu suất với các kích thước patch khác nhau}
\label{tab:patch_size_comparison}
\begin{tabular}{|c|c|c|c|}
\hline
\textbf{Patch Size} & \textbf{Accuracy (\%)} & \textbf{Số tham số} & \textbf{Thời gian (s/epoch)} \\
\hline
1×1 & 97.34 & 12,324 & 8.2 \\
\hline
\textbf{3×3} & \textbf{98.86} & \textbf{36,676} & \textbf{12.5} \\
\hline
5×5 & 98.67 & 95,812 & 18.7 \\
\hline
7×7 & 98.29 & 184,548 & 28.3 \\
\hline
9×9 & 97.91 & 303,284 & 42.1 \\
\hline
\end{tabular}
\end{table}

\textbf{Phân tích kết quả:}

Patch 1×1 đạt 97.34\%, thấp hơn do thiếu thông tin ngữ cảnh không gian từ các điểm ảnh lân cận. Mặc dù có ít tham số nhất và thời gian huấn luyện nhanh, nhưng độ chính xác chưa đạt mục tiêu.

Patch 3×3 đạt accuracy cao nhất (98.86\%) với số lượng tham số hợp lý (36,676). Kích thước này cung cấp đủ thông tin ngữ cảnh (vùng 30m × 30m) mà không gây quá khớp. Thời gian huấn luyện chấp nhận được (12.5 giây/epoch).

Các patch lớn hơn (5×5, 7×7, 9×9) có accuracy thấp hơn mặc dù số tham số tăng đáng kể. Nguyên nhân là:  chứa nhiễu từ các điểm ảnh xa trung tâm, tăng nguy cơ quá khớp do tỷ lệ tham số/mẫu cao, và thời gian huấn luyện tăng đáng kể.

\textbf{Kết luận:} Patch 3×3 được lựa chọn làm cấu hình tối ưu cho các thử nghiệm tiếp theo.

\section{Nghiên cứu loại trừ nguồn dữ liệu}

Nghiên cứu loại trừ có hệ thống được thực hiện để đánh giá đóng góp của từng nguồn dữ liệu. Ba kịch bản được thử nghiệm với kết quả trong Bảng~\ref{tab:ablation_study}.

\begin{table}[H]
\centering
\caption{Kết quả nghiên cứu loại trừ nguồn dữ liệu}
\label{tab:ablation_study}
\begin{tabular}{|l|c|c|c|}
\hline
\textbf{Kịch bản} & \textbf{Số đặc trưng} & \textbf{Accuracy (\%)} & \textbf{Cải thiện} \\
\hline
Chỉ Sentinel-2 & 21 & 93.42 & Baseline \\
\hline
Chỉ Sentinel-1 & 6 & 73.38 & -20.04\% \\
\hline
\textbf{S1 + S2} & \textbf{27} & \textbf{98.86} & \textbf{+5.44\%} \\
\hline
\end{tabular}
\end{table}

\textbf{Phân tích chi tiết:}

\textbf{Kịch bản 1 - Chỉ Sentinel-2 (93.42\%):} Dữ liệu quang học đóng vai trò chủ đạo trong phân loại. Các chỉ số thực vật (NDVI, NBR, NDMI) rất nhạy với sự thay đổi thực vật. Tuy nhiên, vẫn còn nhầm lẫn giữa các lớp có đặc trưng quang phổ tương đồng.

\textbf{Kịch bản 2 - Chỉ Sentinel-1 (73.38\%):} Accuracy thấp cho thấy dữ liệu ra-đa đơn thuần không đủ để phân loại chính xác. Tuy nhiên, Sentinel-1 vẫn cung cấp thông tin bổ sung về cấu trúc và độ ẩm bề mặt.

\textbf{Kịch bản 3 - Tích hợp S1+S2 (98.86\%):} Kết hợp hai nguồn cải thiện accuracy 5.44 điểm phần trăm so với chỉ Sentinel-2. Điều này chứng minh:
\begin{itemize}
    \item Sentinel-1 và Sentinel-2 có tính bổ sung cao
    \item Sentinel-2 đóng vai trò chính, Sentinel-1 bổ sung thông tin cấu trúc
    \item Tích hợp đa nguồn giúp phân biệt tốt hơn các lớp có quang phổ tương đồng
\end{itemize}

\section{Kết quả phân loại và đánh giá}

\subsection{Hiệu suất trên tập kiểm tra}

Mô hình được đánh giá trên tập kiểm tra cố định (526 mẫu, 20\% dữ liệu). Kết quả chi tiết được trình bày trong Bảng~\ref{tab:classification_report}.

\begin{table}[H]
\centering
\caption{Báo cáo phân loại chi tiết trên tập kiểm tra}
\label{tab:classification_report}
\begin{tabular}{|l|c|c|c|c|}
\hline
\textbf{Lớp} & \textbf{Precision (\%)} & \textbf{Recall (\%)} & \textbf{F1-score (\%)} & \textbf{Support} \\
\hline
Rừng ổn định & 98.48 & 97.73 & 98.10 & 132 \\
\hline
Mất rừng & 97.73 & 96.97 & 97.35 & 132 \\
\hline
Phi rừng & 100.00 & 100.00 & 100.00 & 132 \\
\hline
Phục hồi rừng & 100.00 & 100.00 & 100.00 & 130 \\
\hline
\textbf{Macro avg} & \textbf{99.05} & \textbf{98.67} & \textbf{98.86} & \textbf{526} \\
\hline
\textbf{Weighted avg} & \textbf{99.05} & \textbf{98.86} & \textbf{98.86} & \textbf{526} \\
\hline
\end{tabular}
\end{table}

\textbf{Nhận xét:} Lớp Phi rừng và Phục hồi rừng đạt 100\% trên tất cả các chỉ số, cho thấy mô hình phân biệt rất tốt các lớp này. Lớp Rừng ổn định và Mất rừng có độ chính xác cao (>96\%) nhưng vẫn có một ít nhầm lẫn do đặc trưng quang phổ tương đồng tại vùng ranh giới.

\subsection{Độ ổn định qua kiểm định chéo}

Kiểm định chéo 5 phần cho kết quả: Accuracy trung bình 98.48\% với độ lệch chuẩn 0.36\%. Độ lệch chuẩn nhỏ (<1\%) chứng minh mô hình có độ ổn định cao, không phụ thuộc vào cách chia dữ liệu.

\subsection{Kết quả phân loại toàn vùng}

Mô hình được áp dụng để phân loại 162,469 ha với kết quả được trình bày trong Bảng~\ref{tab:area_statistics} và Hình~\ref{fig:classification_map}.

\begin{table}[H]
\centering
\caption{Thống kê diện tích phân loại toàn vùng}
\label{tab:area_statistics}
\begin{tabular}{|l|c|c|}
\hline
\textbf{Lớp biến động} & \textbf{Diện tích (ha)} & \textbf{Tỷ lệ (\%)} \\
\hline
Rừng ổn định & 120,717 & 74.30 \\
\hline
Mất rừng & 7,282 & 4.48 \\
\hline
Phi rừng & 29,529 & 18.17 \\
\hline
Phục hồi rừng & 4,941 & 3.04 \\
\hline
\textbf{Tổng} & \textbf{162,469} & \textbf{100.00} \\
\hline
\end{tabular}
\end{table}

\begin{figure}[H]
\centering
\includegraphics[width=0.95\textwidth]{chapter3/Classification.png}
\caption{Bản đồ phân loại biến động rừng toàn vùng tỉnh Cà Mau}
\label{fig:classification_map}
\end{figure}

\textbf{Phân tích kết quả:}

Rừng ổn định chiếm tỷ lệ lớn nhất (74.30\%, 120,717 ha), chủ yếu tập trung ở Vườn Quốc gia Mũi Cà Mau và các khu rừng phòng hộ ven biển được quản lý tốt.

Mất rừng chiếm 4.48\% (7,282 ha), phân tán chủ yếu ở các vùng tiếp giáp với ao nuôi tôm. Nguyên nhân chính là chuyển đổi sang nuôi trồng thủy sản do lợi nhuận kinh tế cao.

Phi rừng chiếm 18.17\% (29,529 ha), bao gồm khu dân cư, đất nông nghiệp, và ao nuôi trồng thủy sản hiện hữu.

Phục hồi rừng chiếm 3.04\% (4,941 ha), chủ yếu là kết quả của các chương trình trồng rừng và phục hồi sinh thái.

Mất rừng ròng trong giai đoạn nghiên cứu (13 tháng) là khoảng 1.44\% (2,341 ha = 7,282 - 4,941 ha).

\section{So sánh với các nghiên cứu trước}

Bảng~\ref{tab:comparison} so sánh kết quả nghiên cứu này với các nghiên cứu tương tự trên thế giới.

\begin{table}[H]
\centering
\caption{So sánh với các nghiên cứu trước về giám sát rừng}
\label{tab:comparison}
\begin{tabular}{|l|l|c|c|}
\hline
\textbf{Nghiên cứu} & \textbf{Phương pháp} & \textbf{Dữ liệu} & \textbf{Accuracy (\%)} \\
\hline
Hansen et al. (2013) & Decision Tree & Landsat 30m & ~85 \\
\hline
Ortega et al. (2020) & U-Net & Landsat 30m & ~94 \\
\hline
Fayaz et al. (2024) & U-Net overview & Landsat 30m & 94-97 \\
\hline
\textbf{Nghiên cứu này} & \textbf{CNN} & \textbf{S1+S2 10m} & \textbf{98.86} \\
\hline
\end{tabular}
\end{table}

Mô hình CNN kết hợp Sentinel-1/2 của nghiên cứu này đạt accuracy 98.86\%, vượt trội so với các nghiên cứu trước đây. Lý do chính: (1) Tích hợp đa nguồn dữ liệu (SAR + quang học), (2) Độ phân giải không gian cao hơn (10m vs 30m), (3) Kiến trúc CNN được tối ưu hóa cho bài toán cụ thể.

\section{Kết luận và hướng phát triển}

Đồ án đã hoàn thành đầy đủ các mục tiêu nghiên cứu đề ra với kết quả vượt kỳ vọng ban đầu. Về xây dựng bộ dữ liệu, nghiên cứu đã xây dựng thành công bộ dữ liệu gồm 2,630 điểm mẫu với 27 đặc trưng, bao gồm 21 đặc trưng từ Sentinel-2 (4 kênh phổ quang học và 3 chỉ số thực vật) và 6 đặc trưng từ Sentinel-1 (2 kênh phân cực), mỗi đặc trưng được tính cho hai thời kỳ và giá trị delta. Bộ dữ liệu được phân bố cân bằng cho 4 lớp phân loại, đảm bảo tính đại diện cho toàn vùng nghiên cứu.

Về thiết kế kiến trúc CNN, mô hình với 36,676 tham số được thiết kế phù hợp với bộ dữ liệu quy mô vừa, sử dụng các kỹ thuật điều chuẩn hiệu quả (Batch Normalization, Dropout 70\%, weight decay) để ngăn ngừa quá khớp. Thông qua nghiên cứu loại trừ có hệ thống, kích thước patch 3×3 được xác định là tối ưu, đạt accuracy 98.86\% trên tập kiểm tra và 98.48\% ± 0.36\% qua kiểm định chéo 5 phần.

Về đánh giá tích hợp đa nguồn, nghiên cứu loại trừ toàn diện chứng minh việc kết hợp Sentinel-1 và Sentinel-2 cải thiện accuracy 5.44 điểm phần trăm so với chỉ sử dụng Sentinel-2 đơn lẻ (từ 93.42\% lên 98.86\%). Kết quả này khẳng định giả thuyết nghiên cứu rằng dữ liệu ra-đa và quang học có tính bổ sung cao, trong đó Sentinel-2 đóng vai trò chủ đạo cung cấp thông tin quang phổ chi tiết, còn Sentinel-1 bổ sung thông tin về cấu trúc và độ ẩm bề mặt.

Về áp dụng mô hình toàn vùng, mô hình được triển khai thành công để phân loại 162,469 ha (95.5\% diện tích ranh giới lâm nghiệp), phát hiện 7,282 ha mất rừng (4.48\%) và 4,941 ha phục hồi rừng (3.04\%). Diện tích mất rừng ròng trong 13 tháng là 2,341 ha (khoảng 1.44\%). Bản đồ phân loại với độ phân giải 10m cung cấp thông tin chi tiết về phân bố không gian các loại biến động, hỗ trợ trực tiếp công tác quản lý rừng.

Về đóng góp khoa học và thực tiễn, đồ án đề xuất quy trình tích hợp dữ liệu đa nguồn hiệu quả kết hợp ra-đa Sentinel-1 và quang học Sentinel-2, thiết kế kiến trúc CNN phù hợp cho bộ dữ liệu nhỏ tránh hiện tượng quá khớp, đề xuất cấu trúc vector đặc trưng 27 chiều tổng hợp thông tin từ hai nguồn và ba thời điểm, và tạo ra bản đồ biến động rừng độ phân giải cao hỗ trợ công tác quản lý lâm nghiệp tại tỉnh Cà Mau.

Hạn chế của nghiên cứu bao gồm: thời gian dự đoán toàn vùng còn dài (14.83 phút cho 16.2 triệu điểm ảnh), chưa đáp ứng yêu cầu xử lý thời gian thực; khả năng giải thích mô hình hạn chế do tính chất ``hộp đen'' của CNN; quy mô dữ liệu thực địa còn nhỏ và chưa có khảo sát độc lập để kiểm chứng; phân tích chỉ ở hai thời điểm, chưa khai thác chuỗi thời gian đầy đủ.

Hướng phát triển tiếp theo bao gồm: mở rộng phân tích đa thời gian sử dụng chuỗi 5--10 năm với các mô hình LSTM hoặc Transformer để phát hiện xu hướng và dự báo biến động; cải thiện mô hình qua transfer learning từ các mô hình pretrained và kỹ thuật ensemble; mở rộng ứng dụng cho các tỉnh khác trong Đồng bằng sông Cửu Long để đánh giá khả năng tổng quát hóa; tích hợp kỹ thuật Explainable AI như Grad-CAM và SHAP để tăng khả năng giải thích mô hình.


\end{document}